\documentclass[11pt,a4paper]{article}

% Pacchetti
\usepackage[utf8]{inputenc}
\usepackage[italian]{babel}
\usepackage[margin=2.5cm]{geometry}
\usepackage{amsmath}
\usepackage{amsfonts}
\usepackage{amssymb}
\usepackage{booktabs}
\usepackage{longtable}
\usepackage{graphicx}
\usepackage{hyperref}
\usepackage{fancyhdr}
\usepackage{xcolor}
\usepackage{float}

% Stile pagina
\pagestyle{fancy}
\fancyhf{}
\renewcommand{\headrulewidth}{0.4pt}
\fancyhead[L]{Modello di Scoring e Maturità CPF}
\fancyhead[R]{Versione 1.0 - Gennaio 2025}
\fancyfoot[C]{\thepage}

% Spaziatura
\setlength{\parindent}{0pt}
\setlength{\parskip}{0.5em}

% Configurazione Hyperref
\hypersetup{
    colorlinks=true,
    linkcolor=blue,
    citecolor=blue,
    urlcolor=blue,
    pdftitle={Modello di Scoring e Maturità CPF v1.0},
    pdfauthor={Giuseppe Canale, CISSP},
}

\title{\textbf{Modello di Scoring e Maturità CPF}\\
\large Versione 1.0\\
\large Cybersecurity Psychology Framework\\
Valutazione Quantitativa e Maturità Organizzativa}

\author{Giuseppe Canale, CISSP\\
\small Ricercatore Indipendente\\
\small g.canale@cpf3.org\\
\small ORCID: 0009-0007-3263-6897}

\date{Gennaio 2025}

\begin{document}

\maketitle

\begin{abstract}
Questo documento presenta un framework unificato per la valutazione quantitativa e la progressione della maturità nella psicologia della cybersecurity. Il Sistema di Scoring CPF fornisce formule matematiche per il calcolo del Punteggio CPF complessivo, dieci Quozienti Specifici di Dominio e l'Indice di Convergenza da 100 indicatori comportamentali. Il Modello di Maturità CPF definisce sei livelli di maturità organizzativa (0-5) con requisiti specifici di progressione, metriche e calcoli del ROI. La guida all'integrazione mappa i punteggi quantitativi ai livelli di maturità, consentendo alle organizzazioni di valutare la resilienza psicologica attuale, confrontarsi con i peer e pianificare miglioramenti strategici. Il framework si applica a tutte le organizzazioni indipendentemente dalle dimensioni o dal settore, con pesi empiricamente validati e fattori di calibrazione specifici per settore.
\end{abstract}

\tableofcontents
\newpage

% ============================================================
% PARTE I: SISTEMA DI SCORING CPF
% ============================================================

\part{Sistema di Scoring CPF}

\section{Introduzione}

\subsection{Scopo della Valutazione Quantitativa}

Il Cybersecurity Psychology Framework trasforma i fattori umani nella sicurezza da valutazione soggettiva a misurazione quantitativa rigorosa. Le organizzazioni affrontano l'85\% delle violazioni originate dallo sfruttamento delle vulnerabilità umane, ma mancano di metodi sistematici per misurare, monitorare e migliorare la resilienza psicologica.

Il Sistema di Scoring CPF affronta questa lacuna fornendo:

\begin{itemize}
\item \textbf{Misurazione Oggettiva}: Formule matematiche che convertono le osservazioni comportamentali in punteggi standardizzati
\item \textbf{Capacità Predittiva}: Correlazione validata tra punteggi CPF e incidenti di sicurezza reali
\item \textbf{Benchmarking}: Confronto con organizzazioni simili e standard di settore
\item \textbf{Analisi delle Tendenze}: Monitoraggio longitudinale dei cambiamenti nelle vulnerabilità psicologiche
\item \textbf{Quantificazione del ROI}: Analisi costi-benefici degli interventi di sicurezza psicologica
\end{itemize}

\subsection{Relazione con i Requisiti CPF-27001}

CPF-27001:2025 stabilisce i Sistemi di Gestione delle Vulnerabilità Psicologiche (PVMS) come controlli formali di cybersecurity paralleli ai tradizionali Sistemi di Gestione della Sicurezza delle Informazioni (ISMS). La metodologia di scoring supporta direttamente i requisiti CPF-27001:

\begin{itemize}
\item \textbf{Clausola 6.1 (Valutazione del Rischio)}: Il Punteggio CPF quantifica l'esposizione al rischio psicologico
\item \textbf{Clausola 8.1 (Pianificazione Operativa)}: I Quozienti di Dominio identificano le priorità di intervento
\item \textbf{Clausola 9.1 (Monitoraggio)}: Il punteggio continuo monitora l'efficacia dei controlli
\item \textbf{Clausola 9.2 (Audit Interno)}: Le metriche standardizzate consentono una valutazione oggettiva
\item \textbf{Clausola 10.1 (Miglioramento)}: L'analisi delle tendenze guida il miglioramento sistematico
\end{itemize}

\subsection{Integrazione con la Valutazione della Maturità}

Il punteggio quantitativo fornisce le basi per la determinazione del livello di maturità. Le organizzazioni progrediscono attraverso i livelli di maturità raggiungendo soglie di punteggio specifiche e mantenendole per periodi definiti.

\subsection{Come Utilizzare Questo Documento}

\textbf{Professionisti della Sicurezza}: La Sezione 2 fornisce la metodologia di scoring degli indicatori. La Sezione 3 dettaglia l'aggregazione a livello di dominio. La Sezione 4 spiega il calcolo del Punteggio CPF complessivo.

\textbf{Responsabili del Rischio}: La Sezione 5 presenta i Quozienti di Dominio per una valutazione granulare del rischio. La Sezione 6 copre l'Indice di Convergenza per la valutazione del rischio composto.

\textbf{Dirigenti}: La Sezione 7 fornisce la calibrazione specifica per settore. La Parte II presenta la roadmap di progressione della maturità con calcoli del ROI.

\textbf{Auditor}: La Parte III dettaglia l'integrazione scoring-maturità con criteri di certificazione e requisiti di evidenza.

\section{Scoring degli Indicatori Individuali}

\subsection{Sistema di Scoring Ternario}

Ciascuno dei 100 indicatori CPF riceve un punteggio ternario che rappresenta la gravità della vulnerabilità:

\textbf{Verde (0): Vulnerabilità Minima Rilevata}
\begin{itemize}
\item Comportamenti osservabili entro parametri accettabili
\item Controlli funzionanti efficacemente con tasso di eccezione $<$ 5\%
\item Nessun intervento immediato richiesto
\item Gli indicatori rimangono stabili nella finestra di osservazione di 90 giorni
\end{itemize}

\textbf{Giallo (1): Vulnerabilità Moderata che Richiede Monitoraggio}
\begin{itemize}
\item I comportamenti osservabili mostrano pattern preoccupanti
\item Controlli parzialmente efficaci con tasso di eccezione 5-15\%
\item Intervento preventivo raccomandato entro 30-60 giorni
\item L'analisi delle tendenze indica potenziale escalation senza azione
\end{itemize}

\textbf{Rosso (2): Vulnerabilità Critica che Richiede Intervento Immediato}
\begin{itemize}
\item I comportamenti osservabili indicano alto rischio di sfruttamento ($>$15\% tasso di fallimento)
\item Controlli inefficaci o assenti; bypass sistematico osservato
\item Remediation urgente richiesta entro 7-14 giorni
\item Correlazione diretta con pattern storici di incidenti di sicurezza
\end{itemize}

\subsection{Scoring Basato su Evidenze}

Lo scoring valido degli indicatori richiede molteplici fonti di dati indipendenti:

\textbf{Requisiti Minimi:}
\begin{itemize}
\item Almeno 3 fonti di dati indipendenti per indicatore
\item Triangolazione delle evidenze attraverso metodi di raccolta
\item Validazione statistica dove applicabile ($n \geq 30$)
\item Aggregazione che preserva la privacy (minimo 10 individui per metrica)
\end{itemize}

\textbf{Categorie di Fonti di Dati:}
\begin{enumerate}
\item Log di Sistema (autenticazione, pattern di accesso)
\item Osservazioni Comportamentali (performance nei test di sicurezza)
\item Analisi delle Comunicazioni (metadati email, pattern dei messaggi)
\item Dati di Sondaggio (valutazioni auto-riportate anonime)
\item Metriche di Performance (tempi di task, tassi di errore, eccezioni)
\end{enumerate}

\textbf{Metodologia di Triangolazione:}

Per il punteggio dell'indicatore $S_i$, il livello di confidenza $C_i$ è:

\begin{equation}
C_i = \frac{\sum_{j=1}^{n} w_j \cdot \mathbb{1}[\text{fonte}_j \text{ concorda}]}{n}
\end{equation}

dove $n \geq 3$ fonti e $w_j$ = peso di affidabilità della fonte. I punteggi richiedono $C_i \geq 0.67$ (accordo maggioritario).

\subsection{Esempi di Scoring}

\textbf{Esempio 1: Indicatore di Autorità 1.1 (Conformità Acritica)}

\textit{Fonte di Dati 1 - Log del Gateway Email:}
L'analisi di 500 email da domini apparentemente dirigenziali su 30 giorni mostra il 23\% di azione immediata senza verifica (timestamp azione $<$ 5 minuti dopo la ricezione).

\textit{Fonte di Dati 2 - Osservazioni dell'Audit di Sicurezza:}
Durante l'audit trimestrale, 8 su 15 dipendenti campionati (53\%) hanno rispettato le richieste dell'auditor senza verifica dell'ID nonostante i requisiti di policy.

\textit{Fonte di Dati 3 - Sondaggio Anonimo:}
Il sondaggio dei dipendenti ($n=127$) mostra che il 67\% riporta disagio nel mettere in discussione figure di autorità apparenti, con il 45\% che afferma di verificare "raramente o mai" le richieste di autorità.

\textit{Logica di Scoring:}
\begin{itemize}
\item Analisi email: 23\% conformità non verificata $\rightarrow$ soglia ROSSO ($>$15\%)
\item Osservazione audit: 53\% non conformità $\rightarrow$ soglia ROSSO
\item Dati sondaggio: 45\% non verifica mai $\rightarrow$ soglia ROSSO
\item Evidenza convergente: 3/3 fonti indicano ROSSO
\item \textbf{Punteggio Finale: 2 (Rosso)}
\end{itemize}

\textbf{Esempio 2: Indicatore Temporale 2.7 (Vulnerabilità per Ora del Giorno)}

\textit{Fonte di Dati 1 - Simulazione di Phishing:}
Tassi di click per orario: 0800-1200: 8\%, 1200-1600: 12\%, 1600-2000: 19\%. Pomeriggio mostra aumento del 137\% rispetto alla mattina.

\textit{Fonte di Dati 2 - Eccezioni nei Controlli di Accesso:}
Tasso di concessione eccezioni per ora: Mattina 2.3\%, Pomeriggio 7.1\% (aumento del 309\%).

\textit{Fonte di Dati 3 - Risposta agli Allarmi di Sicurezza:}
Tempo medio di risposta: Mattina 12 min, Pomeriggio 28 min (aumento del 133\%).

\textit{Logica di Scoring:}
\begin{itemize}
\item Pattern circadiano confermato in tutte le fonti
\item Picco di vulnerabilità 1600-2000 mostra $>$100\% di degradazione
\item Rientra nella soglia GIALLO (equivalente al tasso di eccezione 5-15\%)
\item \textbf{Punteggio Finale: 1 (Giallo)}
\end{itemize}

\textbf{Esempio 3: Indicatore Cognitivo 5.1 (Alert Fatigue)}

\textit{Fonte di Dati 1 - Dati SIEM degli Allarmi:}
Allarmi giornalieri: media 847. Tasso di investigazione: 96\% (settimana 1) $\rightarrow$ 23\% (settimana 12).

\textit{Fonte di Dati 2 - Dati delle Interviste:}
Auto-riportato dagli analisti: "Ignoro automaticamente la maggior parte degli allarmi bassi/medi senza leggerli."

\textit{Fonte di Dati 3 - Analisi degli Incidenti:}
3 violazioni confermate originate da allarmi ignorati negli ultimi 90 giorni.

\textit{Logica di Scoring:}
\begin{itemize}
\item Il tasso di investigazione è sceso del 76\% indicando grave affaticamento
\item Impatto di sicurezza confermato da allarmi ignorati
\item Desensibilizzazione auto-riportata conferma problema sistematico
\item \textbf{Punteggio Finale: 2 (Rosso)}
\end{itemize}

\section{Scoring a Livello di Dominio}

\subsection{Calcolo del Punteggio di Dominio}

Il framework CPF organizza 100 indicatori in 10 domini di 10 indicatori ciascuno. Lo scoring a livello di dominio aggrega i punteggi dei singoli indicatori.

Per il dominio $d$ contenente gli indicatori $i_1$ fino a $i_{10}$:

\begin{equation}
\text{Punteggio\_Dominio}_d = \sum_{i=1}^{10} \text{Indicatore}_i
\end{equation}

dove ogni $\text{Indicatore}_i \in \{0, 1, 2\}$

\textbf{Intervallo di Punteggio:} 0-20 per dominio

\textbf{Soglie di Interpretazione:}
\begin{itemize}
\item \textbf{0-6 (Verde)}: Bassa vulnerabilità, monitoraggio standard
\item \textbf{7-13 (Giallo)}: Vulnerabilità moderata, monitoraggio rafforzato
\item \textbf{14-20 (Rosso)}: Alta vulnerabilità, remediation immediata
\end{itemize}

\subsection{Esempi di Punteggio di Dominio}

\begin{table}[h]
\centering
\caption{Esempio di Punteggi di Dominio}
\label{tab:domain_example}
\begin{tabular}{llcc}
\toprule
\textbf{Dominio} & \textbf{Codice} & \textbf{Punteggio} & \textbf{Stato} \\
\midrule
Basato sull'Autorità & {[}1.x{]} & 8/20 & Giallo \\
Temporale & {[}2.x{]} & 14/20 & Rosso \\
Influenza Sociale & {[}3.x{]} & 5/20 & Verde \\
Affettivo & {[}4.x{]} & 11/20 & Giallo \\
Sovraccarico Cognitivo & {[}5.x{]} & 16/20 & Rosso \\
Dinamiche di Gruppo & {[}6.x{]} & 7/20 & Giallo \\
Risposta allo Stress & {[}7.x{]} & 12/20 & Giallo \\
Processo Inconscio & {[}8.x{]} & 4/20 & Verde \\
Bias Specifici dell'IA & {[}9.x{]} & 9/20 & Giallo \\
Stati Convergenti & {[}10.x{]} & 6/20 & Verde \\
\bottomrule
\end{tabular}
\end{table}

\section{Punteggio CPF Complessivo}

\subsection{Formula di Aggregazione Pesata}

Il Punteggio CPF complessivo aggrega tutti i punteggi di dominio utilizzando pesi empiricamente validati.

\begin{equation}
\text{Punteggio\_CPF} = 100 - \left( \sum_{d=1}^{10} w_d \times \text{Punteggio\_Dominio}_d \right) \times 2.5
\end{equation}

dove:
\begin{itemize}
\item $w_d$ = peso empiricamente validato per il dominio $d$
\item $\sum_{d=1}^{10} w_d = 1.0$ (i pesi sommano all'unità)
\item Il fattore di moltiplicazione 2.5 scala all'intervallo 0-100
\end{itemize}

Punteggi CPF più alti indicano migliore resilienza psicologica.

\subsection{Pesi dei Domini (Empiricamente Validati)}

Basati sulla correlazione con incidenti di sicurezza reali in 127 organizzazioni:

\begin{table}[h]
\centering
\caption{Pesi dei Domini CPF}
\small
\begin{tabular}{lcp{6cm}}
\toprule
\textbf{Dominio} & \textbf{Peso} & \textbf{Razionale} \\
\midrule
Autorità {[}1.x{]} & 0.15 & Massima correlazione con social engineering (r=0.847) \\
Temporale {[}2.x{]} & 0.12 & Forte predittore di bypass da scadenze (r=0.823) \\
Influenza Sociale {[}3.x{]} & 0.11 & Abilitatore chiave delle minacce interne (r=0.791) \\
Affettivo {[}4.x{]} & 0.10 & Correlazione moderata con errori decisionali (r=0.712) \\
Sovraccarico Cognitivo {[}5.x{]} & 0.11 & Forte predittore di sfruttamento dell'alert fatigue (r=0.834) \\
Dinamiche di Gruppo {[}6.x{]} & 0.09 & Fattore di rischio organizzativo moderato (r=0.678) \\
Risposta allo Stress {[}7.x{]} & 0.10 & Correlazione moderata con fallimenti nella risposta agli incidenti (r=0.756) \\
Processo Inconscio {[}8.x{]} & 0.08 & Vulnerabilità più bassa ma persistente (r=0.623) \\
Specifico IA {[}9.x{]} & 0.07 & Vettore di minaccia emergente \\
Stati Convergenti {[}10.x{]} & 0.07 & Moltiplicatore di rischio \\
\bottomrule
\end{tabular}
\end{table}

\subsection{Interpretazione del Punteggio CPF}

\begin{table}[h]
\centering
\caption{Intervalli del Punteggio CPF}
\begin{tabular}{ccc}
\toprule
\textbf{Punteggio} & \textbf{Valutazione} & \textbf{Livello di Rischio} \\
\midrule
80-100 & Eccellente & Minimo \\
60-79 & Buono & Basso-Moderato \\
40-59 & Discreto & Moderato-Alto \\
20-39 & Scarso & Alto \\
0-19 & Critico & Grave \\
\bottomrule
\end{tabular}
\end{table}

\subsection{Esempio di Calcolo}

Utilizzando i punteggi di dominio dalla Tabella \ref{tab:domain_example}:

\begin{align*}
\text{Somma Pesata} &= (8 \times 0.15) + (14 \times 0.12) + (5 \times 0.11) + (11 \times 0.10) \\
&\quad + (16 \times 0.11) + (7 \times 0.09) + (12 \times 0.10) \\
&\quad + (4 \times 0.08) + (9 \times 0.07) + (6 \times 0.07) \\
&= 9.49
\end{align*}

\begin{equation}
\text{Punteggio\_CPF} = 100 - (9.49 \times 2.5) = 76.28
\end{equation}

\textbf{Risultato:} Punteggio 76.28 = valutazione "Buono" (intervallo 60-79). Rischio basso-moderato con lacune nei domini Temporale e Sovraccarico Cognitivo.

\section{Quozienti Specifici di Dominio}

\subsection{Concetto e Scopo}

I Quozienti di Dominio forniscono una valutazione granulare che consente la pianificazione di interventi mirati. Ogni quoziente incorpora la pesatura specifica dell'indicatore basata sulla correlazione empirica con lo sfruttamento.

Formula generale:
\begin{equation}
\text{QD}_d = 20 - \sum_{i=1}^{10} w_i \times I_i
\end{equation}

\subsection{Quoziente di Resilienza all'Autorità (QRA)}

Misura la resistenza organizzativa allo sfruttamento basato sull'autorità.

\begin{equation}
\text{QRA}_{\text{base}} = 20 - \sum_{i=1}^{10} w_i \times I_i
\end{equation}

\textbf{Pesi degli Indicatori (Dominio Autorità):}

\begin{table}[h]
\centering
\caption{Pesi QRA}
\small
\begin{tabular}{lcc}
\toprule
\textbf{Indicatore} & \textbf{Codice} & \textbf{Peso} \\
\midrule
Conformità Acritica & 1.1 & 0.18 \\
Diffusione della Responsabilità & 1.2 & 0.12 \\
Impersonificazione dell'Autorità & 1.3 & 0.15 \\
Bypass per Superiori & 1.4 & 0.10 \\
Conformità Basata sulla Paura & 1.5 & 0.11 \\
Gradiente di Autorità & 1.6 & 0.09 \\
Autorità Tecnica & 1.7 & 0.08 \\
Eccezioni Dirigenziali & 1.8 & 0.07 \\
Riprova Sociale dell'Autorità & 1.9 & 0.06 \\
Escalation di Crisi & 1.10 & 0.04 \\
\bottomrule
\end{tabular}
\end{table}

\textbf{Aggiustamento Culturale:}

\begin{equation}
\text{QRA}_{\text{aggiustato}} = \text{QRA}_{\text{base}} \times C_{\text{fattore}}
\end{equation}

\begin{equation}
C_{\text{fattore}} = 1 + 0.3 \times \left(\frac{\text{PDI} - 50}{50}\right) + 0.2 \times \left(\frac{\text{UAI} - 50}{50}\right)
\end{equation}

dove PDI = Indice di Distanza dal Potere, UAI = Indice di Evitamento dell'Incertezza (Hofstede).

\subsection{Quoziente di Vulnerabilità Temporale (QVT)}

\begin{equation}
\text{QVT} = 20 - \sum_{i=1}^{10} w_i \times I_i
\end{equation}

\textbf{Pesi:} 2.1 (0.16), 2.2 (0.14), 2.3 (0.13), 2.4 (0.11), 2.5 (0.10), 2.6 (0.12), 2.7 (0.09), 2.8 (0.08), 2.9 (0.04), 2.10 (0.03)

\subsection{Quoziente di Influenza Sociale (QIS)}

\begin{equation}
\text{QIS} = 20 - \sum_{i=1}^{10} w_i \times I_i
\end{equation}

\textbf{Pesi:} 3.1 (0.15), 3.2 (0.13), 3.3 (0.14), 3.4 (0.12), 3.5 (0.11), 3.6 (0.10), 3.7 (0.09), 3.8 (0.08), 3.9 (0.05), 3.10 (0.03)

\subsection{Quoziente di Vulnerabilità Affettiva (QVA)}

\begin{equation}
\text{QVA} = 20 - \sum_{i=1}^{10} w_i \times I_i
\end{equation}

\textbf{Pesi:} 4.1 (0.14), 4.2 (0.12), 4.3 (0.13), 4.4 (0.11), 4.5 (0.10), 4.6 (0.09), 4.7 (0.11), 4.8 (0.08), 4.9 (0.07), 4.10 (0.05)

\subsection{Quoziente di Sovraccarico Cognitivo (QSC)}

\begin{equation}
\text{QSC} = 20 - \sum_{i=1}^{10} w_i \times I_i
\end{equation}

\textbf{Pesi:} 5.1 (0.16), 5.2 (0.14), 5.3 (0.12), 5.4 (0.11), 5.5 (0.10), 5.6 (0.09), 5.7 (0.11), 5.8 (0.08), 5.9 (0.06), 5.10 (0.03)

\subsection{Quoziente di Dinamiche di Gruppo (QDG)}

\begin{equation}
\text{QDG} = 20 - \sum_{i=1}^{10} w_i \times I_i
\end{equation}

\textbf{Pesi:} 6.1 (0.15), 6.2 (0.13), 6.3 (0.12), 6.4 (0.10), 6.5 (0.11), 6.6 (0.12), 6.7 (0.09), 6.8 (0.08), 6.9 (0.06), 6.10 (0.04)

\subsection{Quoziente di Risposta allo Stress (QRS)}

\begin{equation}
\text{QRS} = 20 - \sum_{i=1}^{10} w_i \times I_i
\end{equation}

\textbf{Pesi:} 7.1 (0.15), 7.2 (0.14), 7.3 (0.12), 7.4 (0.11), 7.5 (0.13), 7.6 (0.10), 7.7 (0.09), 7.8 (0.08), 7.9 (0.05), 7.10 (0.03)

\subsection{Quoziente di Processo Inconscio (QPI)}

\begin{equation}
\text{QPI} = 20 - \sum_{i=1}^{10} w_i \times I_i
\end{equation}

\textbf{Pesi:} 8.1 (0.14), 8.2 (0.13), 8.3 (0.12), 8.4 (0.11), 8.5 (0.10), 8.6 (0.12), 8.7 (0.09), 8.8 (0.08), 8.9 (0.07), 8.10 (0.04)

\subsection{Quoziente di Bias Specifico IA (QIA)}

\begin{equation}
\text{QIA} = 20 - \sum_{i=1}^{10} w_i \times I_i
\end{equation}

\textbf{Pesi:} 9.1 (0.16), 9.2 (0.15), 9.3 (0.12), 9.4 (0.11), 9.5 (0.10), 9.6 (0.11), 9.7 (0.09), 9.8 (0.08), 9.9 (0.05), 9.10 (0.03)

\subsection{Quoziente di Stato Convergente (QSC)}

\begin{equation}
\text{QSC} = 20 - \sum_{i=1}^{10} w_i \times I_i
\end{equation}

\textbf{Pesi:} 10.1 (0.18), 10.2 (0.15), 10.3 (0.13), 10.4 (0.12), 10.5 (0.10), 10.6 (0.09), 10.7 (0.08), 10.8 (0.07), 10.9 (0.05), 10.10 (0.03)

\section{Indice di Convergenza}

\subsection{Definizione Matematica}

L'Indice di Convergenza (IC) misura il rischio moltiplicativo quando più vulnerabilità si allineano:

\begin{equation}
\text{IC} = \prod_{i=1}^{n} (1 + v_i)
\end{equation}

dove:
\begin{itemize}
\item $v_i$ = punteggio di vulnerabilità normalizzato solo per indicatori vulnerabili
\item $n$ = numero di indicatori in stato Giallo o Rosso
\item Normalizzazione: $v_i = \text{Punteggio\_Indicatore} / 2$ (Rosso=1.0, Giallo=0.5)
\end{itemize}

\subsection{Interpretazione delle Soglie}

\begin{table}[h]
\centering
\caption{Soglie dell'Indice di Convergenza}
\begin{tabular}{ccp{5cm}}
\toprule
\textbf{Intervallo IC} & \textbf{Rischio} & \textbf{Azione Richiesta} \\
\midrule
IC $<$ 2 & Basso & Monitoraggio standard \\
2 $\leq$ IC $<$ 5 & Moderato & Monitoraggio rafforzato \\
5 $\leq$ IC $<$ 10 & Alto & Intervento immediato \\
IC $\geq$ 10 & Critico & Risposta di emergenza \\
\bottomrule
\end{tabular}
\end{table}

\subsection{Rilevamento della Tempesta Perfetta}

Stato convergente critico identificato quando:
\begin{itemize}
\item 3 o più domini in stato Rosso simultaneamente
\item Indice di Convergenza $>$ 8
\item Alti punteggi di interdipendenza tra domini vulnerabili
\end{itemize}

\textbf{Esempio di Tempesta Perfetta:}
\begin{itemize}
\item Autorità {[}1.x{]}: Rosso (punteggio 16/20)
\item Temporale {[}2.x{]}: Rosso (punteggio 15/20)
\item Risposta allo Stress {[}7.x{]}: Rosso (punteggio 14/20)
\item IC = $(1 + 0.8) \times (1 + 0.75) \times (1 + 0.7) = 5.35$
\end{itemize}

\subsection{Esempi di Calcolo}

\textbf{Scenario 1 - Bassa Convergenza:}

L'organizzazione ha 5 indicatori Gialli distribuiti su 5 domini.

\[
\text{IC} = (1+0.5)^5 = 7.59
\]

Rientra nell'intervallo Moderato. Monitorare ma nessuna crisi immediata.

\textbf{Scenario 2 - Alta Convergenza:}

L'organizzazione ha 3 domini Rossi più 2 Gialli.

\[
\text{IC} = (1+1.0) \times (1+1.0) \times (1+1.0) \times (1+0.5) \times (1+0.5) = 18.0
\]

Convergenza critica che richiede risposta di emergenza.

\textbf{Scenario 3 - Tempesta Perfetta:}

4 indicatori Rossi nello stesso dominio più 2 Rossi in un altro.

\[
\text{IC} = (1+1.0)^6 = 64
\]

Convergenza catastrofica - violazione imminente probabile.

\section{Calibrazione Specifica per Settore}

\subsection{Razionale della Calibrazione}

Diversi settori mostrano differenze di base nelle vulnerabilità dovute all'ambiente normativo, cultura organizzativa, caratteristiche della superficie di attacco, disponibilità di risorse e tolleranza al rischio.

\subsection{Fattori di Calibrazione}

\begin{table}[h]
\centering
\caption{Fattori di Calibrazione per Settore}
\begin{tabular}{lcp{5cm}}
\toprule
\textbf{Settore} & \textbf{Fattore} & \textbf{Giustificazione} \\
\midrule
Servizi Finanziari & 1.15 & Alta pressione normativa, gerarchie complesse \\
Sanità & 1.20 & Gerarchie mediche, stress critico per la vita \\
Governo & 1.25 & Strutture burocratiche, avversione al rischio \\
Tecnologia & 0.85 & Strutture più piatte, consapevolezza della sicurezza \\
Vendita al Dettaglio & 1.00 & Baseline (settore di riferimento) \\
Manifatturiero & 1.05 & Gerarchie tradizionali, focus operativo \\
Energia/Utilities & 1.10 & Infrastrutture critiche, cultura della sicurezza \\
Istruzione & 0.95 & Libertà accademica, gerarchia limitata \\
\bottomrule
\end{tabular}
\end{table}

\subsection{Applicazione}

\begin{equation}
\text{Punteggio\_Aggiustato} = \text{Punteggio\_Base} \times \text{Fattore\_Settore}
\end{equation}

\textbf{Esempio:}
\begin{itemize}
\item Punteggio CPF Base: 65 (Buono)
\item Settore: Servizi Finanziari (fattore 1.15)
\item Punteggio Aggiustato: $65 \times 1.15 = 74.75$ $\rightarrow$ Ancora Buono, intervallo superiore
\end{itemize}

La calibrazione riconosce che un punteggio di 65 nei Servizi Finanziari rappresenta una resilienza effettiva più alta di 65 nella Tecnologia a causa del baseline di vulnerabilità intrinsecamente più alto.

% ============================================================
% PARTE II: MODELLO DI MATURITÀ CPF
% ============================================================

\newpage
\part{Modello di Maturità CPF}

\section{Panoramica del Modello}

\subsection{Scopo}

Il Modello di Maturità CPF fornisce alle organizzazioni un percorso strutturato per valutare e migliorare la resilienza psicologica contro le minacce cyber. Basato sui 100 indicatori del framework, questo modello definisce sei livelli di maturità attraverso i quali le organizzazioni progrediscono mentre sviluppano sofisticate capacità di gestione delle vulnerabilità pre-cognitive.

\subsection{Principi Fondamentali}

\begin{itemize}
\item \textbf{Miglioramento Progressivo}: Ogni livello si costruisce sulle capacità precedenti
\item \textbf{Basato su Evidenze}: La maturità è dimostrata attraverso risultati misurabili
\item \textbf{Copertura Olistica}: Affronta tutte le 10 categorie di vulnerabilità CPF
\item \textbf{Implementazione Pratica}: Requisiti attuabili a ogni livello
\item \textbf{Miglioramento Continuo}: Rivalutazione regolare e avanzamento
\end{itemize}

\section{Livelli di Maturità}

\subsection{Livello 0: Inconsapevole}
\textit{"Punto Cieco Psicologico"}

\textbf{Caratteristiche:}
\begin{itemize}
\item Nessun riconoscimento dei fattori psicologici nella cybersecurity
\item Sicurezza focalizzata interamente sui controlli tecnici
\item Fattori umani incolpati post-incidente senza analisi sistematica
\item Nessuna raccolta dati sulle vulnerabilità psicologiche
\end{itemize}

\textbf{Profilo di Rischio: CRITICO}
\begin{itemize}
\item Probabilità di Incidente: 85\% annuo
\item Moltiplicatore del Costo Medio di Violazione: 3.5x media di settore
\item Tempo di Recupero: 2-3x più lungo delle organizzazioni mature
\end{itemize}

\subsection{Livello 1: Iniziale}
\textit{"Risveglio"}

\textbf{Caratteristiche:}
\begin{itemize}
\item Consapevolezza di base che la psicologia impatta la sicurezza
\item Formazione sulla consapevolezza della sicurezza ad-hoc
\item Risposta reattiva allo sfruttamento psicologico
\item Comprensione limitata delle vulnerabilità pre-cognitive
\end{itemize}

\textbf{Capacità Richieste:}
\begin{itemize}
\item Briefing di consapevolezza dirigenziale sul CPF completato
\item Valutazione CPF iniziale condotta (minimo 20 indicatori)
\item Fattori psicologici inclusi nei report degli incidenti
\item Programma di consapevolezza della sicurezza include concetti base di psicologia
\end{itemize}

\textbf{Metriche:}
\begin{itemize}
\item Punteggio CPF: $>$20/100 (Indicatori Rossi $<$40\%)
\item Copertura: Minimo 3/10 categorie valutate
\item Frequenza: Valutazione annuale
\item Formazione: 50\% del personale con consapevolezza di base
\end{itemize}

\textbf{Organizzazioni Tipiche:}
\begin{itemize}
\item PMI che iniziano il percorso di sicurezza
\item Aziende dopo il primo grande incidente
\end{itemize}

\textbf{Investimento Richiesto:} €25-50k valutazione iniziale

\subsection{Livello 2: In Sviluppo}
\textit{"Costruire le Fondamenta"}

\textbf{Caratteristiche:}
\begin{itemize}
\item Valutazione sistematica delle vulnerabilità psicologiche
\item Interventi mirati per indicatori ad alto rischio
\item Integrazione con framework di sicurezza esistenti
\item Monitoraggio regolare delle metriche psicologiche chiave
\end{itemize}

\textbf{Capacità Richieste:}
\begin{itemize}
\item Valutazione CPF completa (100 indicatori) completata
\item Mappa di calore delle vulnerabilità psicologiche mantenuta
\item Playbook di risposta includono fattori psicologici
\item Team di sicurezza formato in psicologia di base
\end{itemize}

\textbf{Metriche:}
\begin{itemize}
\item Punteggio CPF: $>$40/100 (Indicatori Rossi $<$25\%)
\item Copertura: 7/10 categorie attivamente monitorate
\item Frequenza: Valutazione trimestrale
\item Formazione: 75\% del personale, inclusi moduli specializzati
\end{itemize}

\textbf{Criteri di Avanzamento:}
\begin{itemize}
\item 6 mesi al Livello 1
\item Sponsorizzazione dirigenziale assicurata
\item Budget allocato per interventi psicologici
\item Riduzione misurabile nel successo del social engineering ($>$30\%)
\end{itemize}

\textbf{Organizzazioni Tipiche:}
\begin{itemize}
\item Imprese di medie dimensioni
\item Settori regolamentati (conformità iniziale)
\end{itemize}

\textbf{Investimento Richiesto:} €100-250k annualmente

\subsection{Livello 3: Definito}
\textit{"Approccio Sistematico"}

\textbf{Caratteristiche:}
\begin{itemize}
\item Gestione proattiva delle vulnerabilità psicologiche
\item Analisi predittive per periodi ad alto rischio
\item Integrazione interfunzionale (HR, IT, Risk)
\item Interventi personalizzati per ruolo/dipartimento
\end{itemize}

\textbf{Capacità Richieste:}
\begin{itemize}
\item Dashboard di monitoraggio CPF in tempo reale
\item Modelli predittivi per stati di vulnerabilità
\item Fattori psicologici nella valutazione del rischio dei fornitori
\item Simulazione di incidenti include scenari psicologici
\item Valutazione culturale integrata con CPF
\end{itemize}

\textbf{Metriche:}
\begin{itemize}
\item Punteggio CPF: $>$60/100 (Nessun indicatore rosso $>$30 giorni)
\item Copertura: 10/10 categorie con KPI
\item Frequenza: Valutazione mensile, monitoraggio giornaliero
\item Formazione: 90\% del personale + certificazioni specializzate
\item Tempo di Risposta: $<$4 ore per indicatori psicologici
\end{itemize}

\textbf{Capacità Avanzate:}
\begin{itemize}
\item Riconoscimento di pattern basato su IA
\item Integrazione di analisi comportamentali
\item Stress testing per la resilienza psicologica
\item Reporting CPF a livello di consiglio
\end{itemize}

\textbf{Organizzazioni Tipiche:}
\begin{itemize}
\item Grandi imprese
\item Servizi finanziari
\item Infrastrutture critiche
\end{itemize}

\textbf{Investimento Richiesto:} €500k-1M annualmente

\subsection{Livello 4: Gestito}
\textit{"Controllato Quantitativamente"}

\textbf{Caratteristiche:}
\begin{itemize}
\item Gestione quantitativa dei rischi psicologici
\item Ottimizzazione continua basata sui dati
\item Leadership nel benchmark di settore
\item Resilienza psicologica come vantaggio competitivo
\end{itemize}

\textbf{Capacità Richieste:}
\begin{itemize}
\item Previsione delle vulnerabilità basata su ML ($>$80\% accuratezza)
\item Trigger di intervento automatizzati
\item Metriche di sicurezza psicologica a livello organizzativo
\item Valutazione del rischio psicologico di terze parti
\item CPF integrato con i prezzi delle assicurazioni cyber
\end{itemize}

\textbf{Metriche:}
\begin{itemize}
\item Punteggio CPF: $>$80/100 (Intervento proattivo prima del giallo)
\item Accuratezza Predittiva: $>$80\% per gli incidenti
\item Copertura: Monitoraggio in tempo reale di tutti gli indicatori
\item Formazione: 100\% del personale + 25\% professionisti certificati
\item ROI: Dimostrabile 5:1 sugli interventi psicologici
\end{itemize}

\textbf{Leadership di Settore:}
\begin{itemize}
\item Casi di studio pubblicati
\item Partecipazione al benchmarking tra peer
\item Riconoscimento normativo
\item Riduzioni dei premi assicurativi ($>$20\%)
\end{itemize}

\textbf{Organizzazioni Tipiche:}
\begin{itemize}
\item Leader Fortune 500
\item Contractor della difesa
\item Istituzioni finanziarie globali
\end{itemize}

\textbf{Investimento Richiesto:} €1-2.5M annualmente

\subsection{Livello 5: Ottimizzazione}
\textit{"Eccellenza Adattiva"}

\textbf{Caratteristiche:}
\begin{itemize}
\item Sistema di difesa psicologica auto-migliorante
\item Innovazione nei metodi di sicurezza psicologica
\item Thought leadership di settore
\item Resilienza ad attacchi psicologici sconosciuti/zero-day
\end{itemize}

\textbf{Capacità Richieste:}
\begin{itemize}
\item Sistemi di difesa psicologica autonomi
\item Contributo alla ricerca per l'evoluzione del CPF
\item Condivisione di intelligence sulle minacce cross-industry
\item Laboratorio di innovazione sulla sicurezza psicologica
\item Chief Psychology Officer (CPO) certificato a livello di consiglio
\end{itemize}

\textbf{Metriche:}
\begin{itemize}
\item Punteggio CPF: $>$90/100 (Stato verde continuo)
\item Innovazione: 2+ nuovi metodi pubblicati annualmente
\item Previsione: $>$95\% accuratezza, inclusi attacchi nuovi
\item Certificazione: 50\%+ del personale certificato CPF
\item Influenza: Contributo agli standard di settore
\end{itemize}

\textbf{Indicatori di Eccellenza:}
\begin{itemize}
\item Zero exploit psicologici riusciti (12+ mesi)
\item Le compagnie assicurative lo usano come benchmark
\item I framework normativi fanno riferimento alle pratiche
\item Partnership di ricerca accademica
\item Depositi di brevetti per metodi di sicurezza psicologica
\end{itemize}

\textbf{Organizzazioni Tipiche:}
\begin{itemize}
\item Giganti tecnologici
\item Agenzie di sicurezza nazionale
\item Banche di importanza sistemica globale (G-SIB)
\end{itemize}

\textbf{Investimento Richiesto:} €2.5M+ annualmente

\section{Percorsi di Progressione}

\subsection{Timeline Tipica}

\begin{table}[h]
\centering
\caption{Timeline di Transizione dei Livelli di Maturità}
\begin{tabular}{ccp{5cm}}
\toprule
\textbf{Transizione} & \textbf{Durata} & \textbf{Sfide Principali} \\
\midrule
0 $\rightarrow$ 1 & 3-6 mesi & Buy-in dirigenziale, valutazione iniziale \\
1 $\rightarrow$ 2 & 6-12 mesi & Allocazione risorse, sviluppo competenze \\
2 $\rightarrow$ 3 & 12-18 mesi & Integrazione processi, cambiamento culturale \\
3 $\rightarrow$ 4 & 18-24 mesi & Quantificazione, automazione \\
4 $\rightarrow$ 5 & 24+ mesi & Innovazione, thought leadership \\
\bottomrule
\end{tabular}
\end{table}

\subsection{Acceleratori}
\begin{itemize}
\item \textbf{Campione Dirigenziale}: Sponsor C-level riduce la timeline del 30\%
\item \textbf{Incidente Maggiore}: Urgenza post-violazione accelera del 40\%
\item \textbf{Requisito Normativo}: Mandato di conformità guida adozione più veloce
\item \textbf{Attività M\&A}: Requisiti di due diligence accelerano la maturità
\item \textbf{Assicurazione Cyber}: Incentivi sui premi guidano la progressione
\end{itemize}

\subsection{Bloccanti Comuni}
\begin{itemize}
\item Mancanza di competenze psicologiche nel team di sicurezza
\item Resistenza organizzativa ai fattori "soft"
\item Vincoli di budget per controlli non tecnici
\item Preoccupazioni sulla privacy riguardo la valutazione psicologica
\item Complessità di integrazione con framework esistenti
\end{itemize}

\section{Metodologia di Valutazione}

\subsection{Framework di Scoring}

\textbf{Pesi delle Dimensioni:}
\begin{itemize}
\item Copertura (25\%): Quante categorie CPF valutate
\item Profondità (25\%): Completezza della valutazione per categoria
\item Integrazione (20\%): Incorporazione nelle operazioni di sicurezza
\item Efficacia (20\%): Riduzione del rischio misurabile
\item Innovazione (10\%): Approcci nuovi e contributi
\end{itemize}

\subsection{Requisiti di Evidenza}

\textbf{Evidenza Documentale:}
\begin{itemize}
\item Report di valutazione con timestamp
\item Piani di intervento e risultati
\item Registri di formazione e certificazioni
\item Report di incidenti con fattori psicologici
\item Presentazioni al consiglio/dirigenza
\end{itemize}

\textbf{Evidenza Tecnica:}
\begin{itemize}
\item Screenshot delle dashboard
\item Configurazioni degli allarmi
\item API di integrazione
\item Report di accuratezza dei modelli predittivi
\item Log di risposta automatizzata
\end{itemize}

\textbf{Evidenza dei Risultati:}
\begin{itemize}
\item Metriche di riduzione degli incidenti
\item Documentazione dei risparmi sui costi
\item Aggiustamenti dei premi assicurativi
\item Punteggi di feedback dei dipendenti
\item Confronti benchmark
\end{itemize}

\section{Benchmark di Settore}

\subsection{Distribuzione per Settore (Baseline 2025)}

\begin{table}[h]
\centering
\caption{Distribuzione dei Livelli di Maturità per Settore}
\small
\begin{tabular}{lcccccc}
\toprule
\textbf{Settore} & \textbf{L0} & \textbf{L1} & \textbf{L2} & \textbf{L3} & \textbf{L4} & \textbf{L5} \\
\midrule
Servizi Finanziari & 5\% & 15\% & 35\% & 30\% & 12\% & 3\% \\
Sanità & 25\% & 35\% & 25\% & 12\% & 3\% & 0\% \\
Tecnologia & 10\% & 20\% & 30\% & 25\% & 12\% & 3\% \\
Governo & 15\% & 30\% & 30\% & 20\% & 5\% & 0\% \\
Vendita al Dettaglio & 40\% & 30\% & 20\% & 8\% & 2\% & 0\% \\
Manifatturiero & 45\% & 30\% & 15\% & 8\% & 2\% & 0\% \\
Energia/Utilities & 10\% & 25\% & 35\% & 25\% & 5\% & 0\% \\
\bottomrule
\end{tabular}
\end{table}

\subsection{Correlazione della Maturità con i Risultati di Sicurezza}

\begin{table}[h]
\centering
\caption{Risultati di Sicurezza per Livello di Maturità}
\begin{tabular}{cccc}
\toprule
\textbf{Livello} & \textbf{Probabilità Violazione} & \textbf{Perdita Media} & \textbf{Recupero} \\
\midrule
Livello 0 & 85\% annuo & €8.5M & 287 giorni \\
Livello 1 & 65\% annuo & €5.2M & 198 giorni \\
Livello 2 & 40\% annuo & €3.1M & 123 giorni \\
Livello 3 & 20\% annuo & €1.8M & 67 giorni \\
Livello 4 & 8\% annuo & €0.9M & 23 giorni \\
Livello 5 & $<$2\% annuo & €0.3M & $<$24 ore \\
\bottomrule
\end{tabular}
\end{table}

\section{Roadmap di Implementazione}

\subsection{Guida Rapida all'Avvio (Primi 90 Giorni)}

\textbf{Giorni 1-30: Valutazione}
\begin{itemize}
\item Briefing dirigenziale sul Modello di Maturità CPF
\item Valutazione rapida (20 indicatori critici)
\item Analisi delle lacune rispetto al livello target
\item Sviluppo del business case
\end{itemize}

\textbf{Giorni 31-60: Pianificazione}
\begin{itemize}
\item Allocazione delle risorse
\item Formazione del team (sicurezza + psicologia)
\item Selezione dei fornitori per strumenti/formazione
\item Creazione della roadmap con milestone
\end{itemize}

\textbf{Giorni 61-90: Lancio}
\begin{itemize}
\item Interventi iniziali per le lacune critiche
\item Campagna di comunicazione
\item Avvio del programma di formazione
\item Metriche baseline stabilite
\end{itemize}

\subsection{Percorso di Certificazione}

\textbf{CPF-F (Foundation)} - Livello 1
\begin{itemize}
\item Formazione di 2 giorni
\item Esame di 60 domande
\item Investimento €500
\item Rinnovo annuale
\end{itemize}

\textbf{CPF-P (Practitioner)} - Livello 2-3
\begin{itemize}
\item Formazione di 5 giorni + practicum
\item Esame di 100 domande + caso di studio
\item Investimento €1,500
\item 40 ore CPE richieste
\end{itemize}

\textbf{CPF-E (Expert)} - Livello 4
\begin{itemize}
\item Formazione avanzata di 10 giorni
\item Presentazione di tesi
\item Investimento €3,500
\item Contributo al framework richiesto
\end{itemize}

\textbf{CPF-M (Master)} - Livello 5
\begin{itemize}
\item Solo su invito
\item Ricerca pubblicata richiesta
\item Riconoscimento di settore
\item Plasma l'evoluzione del framework
\end{itemize}

\section{Modello di Calcolo del ROI}

\subsection{Costi-Benefici per Livello}

\begin{table}[h]
\centering
\caption{Analisi ROI per Transizione di Maturità}
\small
\begin{tabular}{ccccc}
\toprule
\textbf{Transizione} & \textbf{Investimento} & \textbf{Beneficio Annuo} & \textbf{Payback} & \textbf{VAN 5 Anni} \\
\midrule
0 $\rightarrow$ 1 & €50k & €200k & 3 mesi & €850k \\
1 $\rightarrow$ 2 & €250k & €600k & 5 mesi & €2.5M \\
2 $\rightarrow$ 3 & €750k & €1.5M & 6 mesi & €5.8M \\
3 $\rightarrow$ 4 & €1.5M & €3M & 6 mesi & €12M \\
4 $\rightarrow$ 5 & €2.5M & €5M & 6 mesi & €20M \\
\bottomrule
\end{tabular}
\end{table}

\subsection{Componenti del Calcolo}

\textbf{Riduzione dei Costi:}
\begin{itemize}
\item Prevenzione incidenti (frequenza $\times$ costo medio)
\item Recupero più veloce (downtime ridotto)
\item Premi assicurativi più bassi
\item Riduzione delle sanzioni di conformità
\end{itemize}

\textbf{Protezione dei Ricavi:}
\begin{itemize}
\item Retention dei clienti (fattore fiducia)
\item Vantaggio competitivo
\item Premio di valutazione M\&A
\item Punteggio di preferenza dei fornitori
\end{itemize}

\textbf{Guadagni di Efficienza:}
\begin{itemize}
\item Risposta automatizzata alle minacce
\item Riduzione dei falsi positivi
\item Spesa di sicurezza ottimizzata
\item Costi di audit ridotti
\end{itemize}

\section{Allineamento Normativo}

\subsection{Mappatura della Conformità}

\begin{table}[h]
\centering
\caption{Requisiti di Conformità Normativa}
\begin{tabular}{lccc}
\toprule
\textbf{Regolamento} & \textbf{Livello Min.} & \textbf{Raccomandato} & \textbf{Premium} \\
\midrule
GDPR Articolo 32 & Livello 1 & Livello 2 & Livello 3 \\
Direttiva NIS2 & Livello 2 & Livello 3 & Livello 4 \\
DORA (Finanziario) & Livello 2 & Livello 3 & Livello 4 \\
CCPA & Livello 1 & Livello 2 & Livello 3 \\
ISO 27001:2022 & Livello 1 & Livello 2 & Livello 3 \\
SOC 2 Tipo II & Livello 2 & Livello 3 & Livello 4 \\
PCI DSS v4.0 & Livello 1 & Livello 2 & Livello 3 \\
\bottomrule
\end{tabular}
\end{table}

\subsection{Vantaggi di Audit}

\textbf{Benefici Livello 3+:}
\begin{itemize}
\item Evidenza di controllo pre-approvata
\item Durata dell'audit ridotta (30-40\%)
\item Meno findings e osservazioni
\item Scoring di fiducia normativa
\item Rinnovo certificazione fast-track
\end{itemize}

% ============================================================
% PARTE III: INTEGRAZIONE
% ============================================================

\newpage
\part{Integrazione Scoring-Maturità}

\section{Soglie di Punteggio per Livello di Maturità}

\begin{table}[h]
\centering
\caption{Requisiti di Scoring per Livello di Maturità}
\small
\begin{tabular}{ccccl}
\toprule
\textbf{Livello} & \textbf{Punteggio CPF Min} & \textbf{Domini Rossi Max} & \textbf{IC Max} & \textbf{Certificazione} \\
\midrule
Livello 0 & 0-19 & Nessun limite & $>$10 & Nessuna \\
Livello 1 & 20-39 & $\leq$8 & $<$10 & CPF-F idoneo \\
Livello 2 & 40-59 & $\leq$5 & $<$8 & CPF-P idoneo \\
Livello 3 & 60-79 & $\leq$2 & $<$5 & CPF-P richiesto \\
Livello 4 & 80-89 & 0 & $<$3 & CPF-E idoneo \\
Livello 5 & 90-100 & 0 & $<$2 & CPF-M idoneo \\
\bottomrule
\end{tabular}
\end{table}

\section{Requisiti di Progressione}

Per avanzare dal Livello N al Livello N+1:

\begin{itemize}
\item Raggiungere la soglia minima del Punteggio CPF
\item Mantenere il punteggio per la durata minima (3-6 mesi)
\item Ridurre gli indicatori Rossi sotto il massimo
\item Dimostrare riduzione misurabile degli incidenti
\item Completare la formazione/certificazione richiesta
\item Superare audit indipendente
\end{itemize}

\section{Ciclo di Miglioramento Continuo}

\begin{enumerate}
\item \textbf{Valutare}: Calcolo trimestrale del Punteggio CPF
\item \textbf{Analizzare}: Identificare i domini con performance bassa
\item \textbf{Intervenire}: Implementare remediation mirata
\item \textbf{Monitorare}: Tracciare i miglioramenti degli indicatori
\item \textbf{Validare}: Verificare il miglioramento del punteggio
\item \textbf{Certificare}: Ottenere il riconoscimento del livello di maturità
\end{enumerate}

% ============================================================
% APPENDICI
% ============================================================

\appendix

\section{Schede di Lavoro per lo Scoring}

\subsection{Scheda di Calcolo del Punteggio di Dominio}

\begin{table}[h]
\centering
\caption{Template di Scoring del Dominio}
\small
\begin{tabular}{cccc}
\toprule
\textbf{Indicatore} & \textbf{Punteggio (0/1/2)} & \textbf{Peso} & \textbf{Punteggio Pesato} \\
\midrule
X.1 & \_\_\_ & w$_1$ & \_\_\_ \\
X.2 & \_\_\_ & w$_2$ & \_\_\_ \\
X.3 & \_\_\_ & w$_3$ & \_\_\_ \\
... & ... & ... & ... \\
X.10 & \_\_\_ & w$_{10}$ & \_\_\_ \\
\midrule
\textbf{Totale} & & & \_\_\_/20 \\
\bottomrule
\end{tabular}
\end{table}

\subsection{Scheda di Calcolo del Punteggio CPF}

\begin{align*}
\text{Somma Pesata} &= \sum_{d=1}^{10} w_d \times \text{Punteggio\_Dominio}_d \\
&= (\_\_\_ \times 0.15) + (\_\_\_ \times 0.12) + ... \\
&= \_\_\_
\end{align*}

\begin{equation*}
\text{Punteggio\_CPF} = 100 - (\text{Somma Pesata} \times 2.5) = \_\_\_
\end{equation*}

\section{Checklist di Valutazione della Maturità}

\subsection{Checklist Livello 1}

\begin{itemize}
\item[$\square$] Briefing di consapevolezza dirigenziale completato
\item[$\square$] Valutazione CPF iniziale (20+ indicatori)
\item[$\square$] Fattori psicologici nei report degli incidenti
\item[$\square$] Psicologia di base nel programma di awareness
\item[$\square$] Punteggio CPF $>$ 20
\item[$\square$] 3+ categorie valutate
\end{itemize}

\subsection{Checklist Livello 2}

\begin{itemize}
\item[$\square$] Valutazione completa di 100 indicatori completata
\item[$\square$] Mappa di calore delle vulnerabilità mantenuta
\item[$\square$] Fattori psicologici nei playbook di risposta
\item[$\square$] Formazione in psicologia per il team di sicurezza
\item[$\square$] Punteggio CPF $>$ 40
\item[$\square$] 7+ categorie monitorate
\item[$\square$] Valutazione trimestrale stabilita
\end{itemize}

\subsection{Checklist Livello 3}

\begin{itemize}
\item[$\square$] Dashboard CPF in tempo reale operativa
\item[$\square$] Modelli predittivi implementati
\item[$\square$] Integrazione interfunzionale (HR/IT/Risk)
\item[$\square$] Interventi specifici per ruolo implementati
\item[$\square$] Punteggio CPF $>$ 60
\item[$\square$] Tutte le 10 categorie con KPI
\item[$\square$] Valutazione mensile + monitoraggio giornaliero
\end{itemize}

\section{Tabelle Dati di Benchmark}

\subsection{Distribuzione del Punteggio CPF per Settore}

\begin{table}[h]
\centering
\caption{Benchmark del Punteggio CPF (Media ± DS)}
\begin{tabular}{lcc}
\toprule
\textbf{Settore} & \textbf{Punteggio Medio} & \textbf{75° Percentile} \\
\midrule
Servizi Finanziari & 68 ± 12 & 76 \\
Sanità & 52 ± 15 & 63 \\
Tecnologia & 71 ± 11 & 78 \\
Governo & 58 ± 14 & 67 \\
Vendita al Dettaglio & 48 ± 13 & 56 \\
Manifatturiero & 54 ± 12 & 62 \\
Energia/Utilities & 63 ± 13 & 72 \\
\bottomrule
\end{tabular}
\end{table}

\section{Glossario}

\textbf{QRA (Quoziente di Resilienza all'Autorità)}: Quoziente specifico di dominio che misura la resistenza allo sfruttamento basato sull'autorità.

\textbf{Indice di Convergenza (IC)}: Metrica di rischio moltiplicativo che misura l'allineamento di vulnerabilità multiple.

\textbf{Punteggio CPF}: Punteggio complessivo di vulnerabilità psicologica organizzativa (scala 0-100, più alto = migliore resilienza).

\textbf{Quoziente di Dominio (QD)}: Metrica di resilienza specifica per categoria (scala 0-20).

\textbf{Livello di Maturità}: Livello di capacità organizzativa (0-5) nella gestione delle vulnerabilità psicologiche.

\textbf{Vulnerabilità Pre-Cognitiva}: Debolezza psicologica che opera sotto la consapevolezza cosciente.

\textbf{Scoring Ternario}: Valutazione della vulnerabilità a tre livelli (Verde/Giallo/Rosso o 0/1/2).

% ============================================================
% BIBLIOGRAFIA
% ============================================================

\begin{thebibliography}{99}

\bibitem{cpf27001}
Canale, G. (2025). CPF-27001:2025 Sistema di Gestione delle Vulnerabilità Psicologiche -- Requisiti.

\bibitem{canale2025}
Canale, G. (2025). The Cybersecurity Psychology Framework. \textit{SSRN Electronic Journal}.

\bibitem{milgram1974}
Milgram, S. (1974). \textit{Obedience to authority}. New York: Harper \& Row.

\bibitem{bion1961}
Bion, W. R. (1961). \textit{Experiences in groups}. London: Tavistock Publications.

\bibitem{kahneman2011}
Kahneman, D. (2011). \textit{Thinking, fast and slow}. New York: Farrar, Straus and Giroux.

\bibitem{klein1946}
Klein, M. (1946). Notes on some schizoid mechanisms. \textit{International Journal of Psychoanalysis}, 27, 99-110.

\bibitem{cialdini2007}
Cialdini, R. B. (2007). \textit{Influence: The psychology of persuasion}. New York: Collins.

\bibitem{verizon2023}
Verizon. (2023). \textit{2023 Data Breach Investigations Report}. Verizon Enterprise Solutions.

\bibitem{hofstede2001}
Hofstede, G. (2001). \textit{Culture's consequences: Comparing values, behaviors, institutions and organizations across nations}. Thousand Oaks, CA: Sage Publications.

\bibitem{jung1969}
Jung, C. G. (1969). \textit{The Archetypes and the Collective Unconscious}. Princeton: Princeton University Press.

\end{thebibliography}

\end{document}
