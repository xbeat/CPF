\documentclass[11pt,a4paper]{article}

% Required packages
\usepackage[utf8]{inputenc}
\usepackage[italian]{babel}
\usepackage{amsmath}
\usepackage{amsfonts}
\usepackage{amssymb}
\usepackage{graphicx}
\usepackage{booktabs}
\usepackage{url}
\usepackage{hyperref}
\usepackage[margin=1in]{geometry}
\usepackage{lipsum}
\usepackage{float}
\usepackage{placeins}
\usepackage{longtable}

% ArXiv style
\usepackage{fancyhdr}
\usepackage{lastpage}

% Remove indentation and add paragraph spacing (ArXiv style)
\setlength{\parindent}{0pt}
\setlength{\parskip}{0.5em}

% Setup hyperref
\hypersetup{
    colorlinks=true,
    linkcolor=blue,
    citecolor=blue,
    urlcolor=blue,
    pdftitle={Il Fattore Umano nella Resilienza Operativa: Integrazione della Valutazione del Rischio Psicologico nei Framework di Compliance NIS2 e DORA},
    pdfauthor={Giuseppe Canale},
}

% Page style
\pagestyle{fancy}
\fancyhf{}
\renewcommand{\headrulewidth}{0pt}
\fancyfoot[C]{\thepage}

\begin{document}

% ArXiv style with black lines
\thispagestyle{empty}
\begin{center}

\vspace*{0.5cm}

% FIRST BLACK LINE
\rule{\textwidth}{1.5pt}

\vspace{0.5cm}

% TITLE
{\LARGE \textbf{Il Fattore Umano nella Resilienza Operativa:}}\\[0.3cm]
{\LARGE \textbf{Integrazione della Valutazione del Rischio Psicologico}}\\[0.3cm]
{\LARGE \textbf{nei Framework di Compliance NIS2 e DORA}}

\vspace{0.5cm}

% SECOND BLACK LINE
\rule{\textwidth}{1.5pt}

\vspace{0.3cm}

% ArXiv style subtitle
{\large \textsc{Un Framework per la Compliance Normativa Europea}}

\vspace{0.5cm}

% AUTHOR INFORMATION
{\Large Giuseppe Canale, CISSP}\\[0.2cm]
Ricercatore Indipendente in Cybersecurity\\[0.1cm]
\href{mailto:g.canale@cpf3.org}{g.canale@cpf3.org}\\[0.1cm]
URL: \href{https://cpf3.org}{cpf3.org}\\[0.1cm]
ORCID: \href{https://orcid.org/0009-0007-3263-6897}{0009-0007-3263-6897}

\vspace{0.8cm}

% DATE
{\large \today}

\vspace{1cm}

\end{center}

% ABSTRACT
\begin{abstract}
\noindent
Il quadro normativo dell'Unione Europea per la resilienza operativa digitale—comprendente la Direttiva NIS2 (Direttiva 2022/2555) e il Digital Operational Resilience Act (DORA, Regolamento 2022/2554)—stabilisce requisiti completi per la gestione del rischio di cybersecurity nei settori critici. Tuttavia, sebbene queste normative prevedano considerazioni sul fattore umano quali responsabilità del management, formazione e programmi di sensibilizzazione, mancano metodologie sistematiche per valutare e mitigare le vulnerabilità psicologiche che compromettono la resilienza operativa. Questo documento presenta un framework di integrazione pratico che mappa il Cybersecurity Psychology Framework (CPF)\cite{canale2025} sui requisiti NIS2 e sui cinque pilastri di DORA, fornendo ai compliance officer e ai Chief Information Security Officer un approccio sistematico per affrontare le dimensioni psicologiche della resilienza operativa. Attraverso tabelle di mappatura dettagliate e linee guida implementative, dimostriamo come la valutazione del rischio psicologico possa migliorare operativamente i programmi di compliance, producendo al contempo miglioramenti misurabili nella prevenzione degli incidenti e nella resilienza organizzativa. Il framework fornisce valore pratico immediato per le istituzioni finanziarie europee e i fornitori di servizi essenziali che affrontano la scadenza DORA di gennaio 2025 e gli obblighi di compliance NIS2 in corso.

\vspace{0.5em}
\noindent\textbf{Parole chiave:} NIS2, DORA, resilienza operativa, valutazione del rischio psicologico, compliance europea, fattori umani, cybersecurity servizi finanziari
\end{abstract}

\vspace{1cm}

\section{Sintesi Esecutiva}

L'Unione Europea ha stabilito il quadro normativo più completo al mondo per la resilienza operativa digitale. La Direttiva NIS2, in vigore da ottobre 2024, impone la gestione del rischio di cybersecurity in 18 settori critici, mentre DORA, applicabile da gennaio 2025, impone requisiti stringenti di resilienza operativa specificamente alle entità finanziarie e ai loro fornitori critici di servizi ICT.

Entrambe le normative riconoscono esplicitamente il fattore umano nella cybersecurity: NIS2 richiede ``formazione in materia di cybersecurity e pratiche di igiene informatica di base'' (Articolo 21), mentre DORA prevede che gli organi di gestione ``possiedano conoscenze e competenze sufficienti per comprendere e valutare i rischi informatici'' (Articolo 5). Tuttavia, nessuna delle due fornisce metodologie sistematiche per valutare le vulnerabilità psicologiche che abilitano l'82-85\% degli attacchi informatici riusciti\cite{verizon2024}.

Il Cybersecurity Psychology Framework (CPF)\cite{canale2025} colma questa lacuna fornendo un approccio sistematico per identificare e mitigare le vulnerabilità psicologiche pre-cognitive. Questo documento fornisce ai compliance officer e ai CISO una roadmap di integrazione pratica che mappa le valutazioni CPF sui requisiti NIS2 e sui cinque pilastri di DORA, consentendo alle organizzazioni di:

\textbf{Benefici Chiave per i Programmi di Compliance Europei}:
\begin{itemize}
\item Dimostrare misure di sicurezza ``allo stato dell'arte'' come richiesto dall'Articolo 21 NIS2
\item Soddisfare i requisiti di responsabilità del management di DORA con metriche quantificabili del rischio umano
\item Ridurre gli incidenti legati al fattore umano del 25-40\% attraverso la valutazione delle vulnerabilità psicologiche
\item Fornire evidenze misurabili di misure di sicurezza ``adeguate e proporzionate'' per gli audit regolamentari
\item Abilitare una postura di resilienza operativa predittiva anziché reattiva
\end{itemize}

\section{Il Panorama Normativo: NIS2 e DORA}

\subsection{Panoramica della Direttiva NIS2}

La Direttiva sulla Sicurezza delle Reti e dei Sistemi Informativi 2 (NIS2) rappresenta un'evoluzione significativa rispetto alla Direttiva NIS originale del 2016, ampliando l'ambito di applicazione a 18 settori critici e introducendo requisiti più stringenti:

\textbf{Requisiti Chiave NIS2}:
\begin{itemize}
\item Misure di gestione del rischio che affrontano i fattori umani (Articolo 21)
\item Responsabilità dell'organo di gestione per la cybersecurity (Articolo 20)
\item Segnalazione degli incidenti entro 24/72 ore (Articolo 23)
\item Valutazione della sicurezza della catena di approvvigionamento (Articolo 21(2)(d))
\item Requisiti di formazione sulla cybersecurity (Articolo 21(2)(g))
\item Sanzioni fino a 10 milioni di euro o 2\% del fatturato globale per le entità essenziali
\end{itemize}

\textbf{Ambito di applicazione}: Entità essenziali (energia, trasporti, settore bancario, sanità, infrastrutture digitali) ed entità importanti (servizi postali, gestione rifiuti, manifattura, fornitori digitali) in tutti gli Stati membri dell'UE.

\subsection{Panoramica del Regolamento DORA}

Il Digital Operational Resilience Act (DORA) stabilisce un quadro unificato per la gestione del rischio ICT nel settore finanziario, strutturato attorno a cinque pilastri:

\textbf{I Cinque Pilastri di DORA}:
\begin{enumerate}
\item \textbf{Gestione del Rischio ICT} (Articoli 5-16): Framework completo per identificare, proteggere, rilevare, rispondere e ripristinare dai rischi ICT
\item \textbf{Gestione degli Incidenti ICT} (Articoli 17-23): Classificazione, segnalazione e analisi degli incidenti ICT
\item \textbf{Test di Resilienza Operativa Digitale} (Articoli 24-27): Test regolari inclusi i test di penetrazione basati su minacce (TLPT)
\item \textbf{Gestione del Rischio di Terze Parti ICT} (Articoli 28-44): Due diligence, contratti e supervisione dei fornitori critici ICT
\item \textbf{Condivisione delle Informazioni} (Articolo 45): Accordi per la condivisione di intelligence sulle minacce informatiche
\end{enumerate}

\textbf{Ambito di applicazione}: Oltre 22.000 entità finanziarie incluse banche, compagnie assicurative, imprese di investimento, fornitori di servizi su cripto-attività e i loro fornitori critici di servizi ICT terzi.

\subsection{Il Gap del Fattore Umano}

Sia NIS2 che DORA riconoscono i fattori umani ma forniscono una guida operativa limitata:

\begin{table}[H]
\centering
\caption{Requisiti sul Fattore Umano in NIS2 e DORA}
\label{tab:human-factor-gap}
\begin{tabular}{p{4cm}p{4cm}p{4cm}}
\toprule
\textbf{Requisito} & \textbf{Testo Normativo} & \textbf{Gap Implementativo} \\
\midrule
Responsabilità del Management & NIS2 Art. 20, DORA Art. 5 & Nessuna metodologia per valutare i bias cognitivi del management \\
Formazione e Sensibilizzazione & NIS2 Art. 21(2)(g), DORA Art. 13(6) & Focus sul trasferimento di conoscenze, non sul cambiamento comportamentale \\
Prevenzione dell'Errore Umano & NIS2 Considerando 89, DORA Art. 9 & Nessun framework per la valutazione delle vulnerabilità pre-cognitive \\
Risposta allo Stress & DORA Art. 11 (gestione crisi) & Nessuna metrica di resilienza psicologica \\
\bottomrule
\end{tabular}
\end{table}

Il Threat Landscape 2024 di ENISA per il Settore Finanziario riporta che il 46\% degli incidenti informatici ha colpito istituti di credito europei, con il social engineering e l'errore umano che rimangono i vettori di attacco primari\cite{enisa2024}. L'indagine EY/IIF sulla Gestione del Rischio Bancario conferma che l'82\% dei CRO europei classifica la cybersecurity come la principale preoccupazione di rischio\cite{ey2024}.

\section{Il Business Case per la Resilienza Psicologica}

\subsection{Costo degli Incidenti Legati al Fattore Umano in Europa}

I dati specifici europei dimostrano l'impatto finanziario delle falle di sicurezza legate al fattore umano:

\begin{itemize}
\item Costo medio di una violazione dei dati nell'UE: 4,3 milioni di euro (IBM Security, 2024)
\item Il 65\% delle istituzioni finanziarie europee ha subito attacchi ransomware nel 2024
\item Il mercato europeo della cybersecurity ha raggiunto 67,79 miliardi di euro nel 2024, con un CAGR del 12,42\%
\item ENISA riporta un aumento del 40\% degli attacchi informatici alle PMI europee nel 2022-2024
\item Le organizzazioni di servizi finanziari impiegano in media 233 giorni per rilevare e contenere le violazioni
\end{itemize}

\subsection{Contesto delle Sanzioni Normative}

La non conformità comporta conseguenze finanziarie significative:

\textbf{Sanzioni NIS2}:
\begin{itemize}
\item Entità essenziali: Fino a 10 milioni di euro o 2\% del fatturato mondiale annuo totale
\item Entità importanti: Fino a 7 milioni di euro o 1,4\% del fatturato mondiale annuo totale
\item Responsabilità personale del management in caso di negligenza grave
\end{itemize}

\textbf{Sanzioni DORA}:
\begin{itemize}
\item Sanzioni fino al 2\% del fatturato mondiale annuo totale per le entità finanziarie
\item Sanzioni individuali fino a 1 milione di euro per i responsabili
\item Fornitori ICT critici: Fino all'1\% del fatturato mondiale giornaliero medio (penalità periodiche)
\end{itemize}

\subsection{Approccio CPF: Migliorare la Compliance Normativa}

La metodologia CPF trasforma la compliance sul fattore umano da esercizi formali a riduzione misurabile del rischio:

\begin{itemize}
\item Fornisce metriche quantificabili per misure ``adeguate e proporzionate'' (NIS2 Art. 21)
\item Consente la dimostrazione basata su evidenze delle ``conoscenze e competenze'' del management (DORA Art. 5)
\item Affronta le cause psicologiche alla radice degli incidenti di sicurezza
\item Supporta i cicli di miglioramento continuo richiesti da entrambe le normative
\item Genera ROI misurabile attraverso la riduzione degli incidenti
\end{itemize}

\section{Architettura di Integrazione del Framework}

\subsection{Modello di Integrazione NIS2}

La Tabella~\ref{tab:nis2-mapping} mappa le categorie CPF sui requisiti dell'Articolo 21 NIS2, dimostrando come la valutazione del rischio psicologico migliori la compliance.

\begin{table}[H]
\centering
\caption{Integrazione CPF con i Requisiti dell'Articolo 21 NIS2}
\label{tab:nis2-mapping}
\begin{tabular}{p{3.5cm}p{4cm}p{4cm}p{2.5cm}}
\toprule
\textbf{Requisito NIS2} & \textbf{Approccio Tradizionale} & \textbf{Miglioramento CPF} & \textbf{Categorie CPF} \\
\midrule
Analisi rischi e politiche (Art. 21.2.a) & Valutazione vulnerabilità tecniche & Profilazione vulnerabilità psicologiche, mappatura bias cognitivi & [1.x], [4.x], [5.x] \\
Gestione incidenti (Art. 21.2.b) & Procedure di risposta tecniche & Protocolli di risposta consapevoli dello stress, qualità decisionale sotto pressione & [7.x], [10.x] \\
Continuità operativa (Art. 21.2.c) & Backup tecnici, piani DR & Recupero psicologico, ripristino della fiducia, resilienza del team & [4.x], [6.x] \\
Sicurezza supply chain (Art. 21.2.d) & Valutazioni fornitori, contratti & Valutazione rischio umano terze parti, vulnerabilità trasferimento autorità & [3.x], [8.x] \\
Sicurezza HR (Art. 21.2.e) & Controlli background, controllo accessi & Psicologia minacce interne, indicatori stress organizzativo & [4.x], [5.x], [9.x] \\
Formazione e igiene (Art. 21.2.g) & Programmi awareness, e-learning & Formazione bias pre-cognitivi, design interventi comportamentali & [1.x], [2.x], [6.x] \\
Controllo accessi (Art. 21.2.i) & IAM, implementazione MFA & Analisi strutture autorità, resistenza social engineering & [3.x], [8.x] \\
\bottomrule
\end{tabular}
\end{table}

\subsection{Modello di Integrazione Cinque Pilastri DORA}

La Tabella~\ref{tab:dora-mapping} dimostra l'integrazione CPF attraverso i cinque pilastri di resilienza operativa di DORA.

\begin{table}[H]
\centering
\caption{Integrazione CPF con i Cinque Pilastri DORA}
\label{tab:dora-mapping}
\begin{tabular}{p{3cm}p{4cm}p{4cm}p{3cm}}
\toprule
\textbf{Pilastro DORA} & \textbf{Requisito Normativo} & \textbf{Miglioramento CPF} & \textbf{Categorie CPF} \\
\midrule
Pilastro 1: Gestione Rischio ICT & Responsabilità management, framework rischio (Art. 5-16) & Valutazione bias cognitivi leadership, metriche qualità decisionale & [2.x], [5.x], [6.x] \\
Pilastro 2: Gestione Incidenti & Classificazione, segnalazione, analisi (Art. 17-23) & Rilevamento anomalie comportamentali, pattern errori indotti da stress & [7.x], [9.x], [10.x] \\
Pilastro 3: Test Resilienza & TLPT, test vulnerabilità (Art. 24-27) & Test fattore umano, metriche resistenza social engineering & [1.x], [3.x], [8.x] \\
Pilastro 4: Rischio Terze Parti & Due diligence, supervisione (Art. 28-44) & Valutazione rischio personale fornitori, psicologia rischio concentrazione & [4.x], [5.x], [8.x] \\
Pilastro 5: Condivisione Info & Threat intelligence (Art. 45) & Dinamiche fiducia nella condivisione, barriere cognitive alla collaborazione & [4.x], [6.x] \\
\bottomrule
\end{tabular}
\end{table}

\subsection{Integrazione Cross-Framework: Allineamento NIS2 e DORA}

Per le entità finanziarie soggette a entrambe le normative, la Tabella~\ref{tab:cross-framework} mostra l'approccio unificato di integrazione CPF.

\begin{table}[H]
\centering
\caption{Integrazione CPF Unificata per Compliance NIS2-DORA}
\label{tab:cross-framework}
\begin{tabular}{p{3cm}p{3cm}p{3.5cm}p{3.5cm}}
\toprule
\textbf{Area Compliance} & \textbf{Riferimento NIS2} & \textbf{Riferimento DORA} & \textbf{Punto Integrazione CPF} \\
\midrule
Governance & Art. 20 (Management) & Art. 5 (Organo gestione) & Dashboard rischio psicologico esecutivo \\
Gestione Rischio & Art. 21.2.a & Art. 6-9 (Framework rischio ICT) & Modello rischio umano-tecnico integrato \\
Risposta Incidenti & Art. 23 (Segnalazione) & Art. 17-19 (Classificazione) & Playbook risposta psicology-aware \\
Testing & Art. 21.2.f & Art. 24-27 (TLPT) & Penetration testing fattore umano \\
Terze Parti & Art. 21.2.d & Art. 28-44 (TPRM) & Protocollo valutazione personale fornitori \\
Formazione & Art. 21.2.g & Art. 13.6 (Awareness) & Programmi intervento pre-cognitivo \\
\bottomrule
\end{tabular}
\end{table}

\section{Mappatura Dettagliata: Categorie CPF e Requisiti Normativi}

\subsection{DORA Pilastro 1: Framework di Gestione del Rischio ICT}

Gli Articoli 5-16 di DORA stabiliscono requisiti completi per la gestione del rischio ICT. Il CPF li migliora attraverso la valutazione della dimensione psicologica.

\textbf{Articolo 5 - Responsabilità dell'Organo di Gestione}:

DORA richiede che gli organi di gestione ``definiscano, approvino, supervisionino e siano responsabili dell'attuazione di tutti gli accordi relativi al quadro di gestione dei rischi informatici.'' Il CPF migliora questo attraverso:

\begin{itemize}
\item \textbf{[2.x] Valutazione Bias Cognitivi}: Identificazione dei bias decisionali (overconfidence, automation bias, normalcy bias) che possono influenzare la governance del rischio ICT
\item \textbf{[5.x] Profilazione Risposta allo Stress}: Valutazione delle performance del management in condizioni di crisi
\item \textbf{[6.x] Analisi Dinamiche di Gruppo}: Valutazione del groupthink a livello di consiglio e delle dinamiche di autorità
\end{itemize}

\textbf{Articolo 9 - Protezione e Prevenzione}:

DORA richiede misure per ``proteggere i sistemi ICT e prevenire il verificarsi di rischi informatici.'' I miglioramenti sul fattore umano includono:

\begin{itemize}
\item \textbf{[1.x] Resistenza al Social Engineering}: Valutazione pre-cognitiva della suscettibilità alla manipolazione
\item \textbf{[3.x] Vulnerabilità Trasferimento Autorità}: Identificazione della fiducia inappropriata riposta in autorità tecniche o esterne
\item \textbf{[8.x] Indicatori Minacce Interne}: Marcatori psicologici che precedono azioni malevole o negligenti
\end{itemize}

\subsection{DORA Pilastro 2: Gestione degli Incidenti ICT}

Gli Articoli 17-23 regolano la classificazione, segnalazione e analisi degli incidenti. L'integrazione CPF affronta gli elementi umani:

\textbf{Articolo 17 - Processo di Gestione degli Incidenti ICT}:

\begin{itemize}
\item \textbf{[7.x] Qualità Decisionale Sotto Stress}: Protocolli per mantenere la capacità analitica durante gli incidenti
\item \textbf{[9.x] Prevenzione Breakdown Comunicativo}: Affrontare le barriere psicologiche a una comunicazione efficace durante gli incidenti
\item \textbf{[10.x] Recupero Psicologico Post-Incidente}: Ripristino della resilienza del team dopo incidenti gravi
\end{itemize}

\textbf{Miglioramento della Segnalazione Incidenti}:

I requisiti di early warning a 24 ore e report incidente a 72 ore previsti da DORA beneficiano di:

\begin{itemize}
\item Procedure di segnalazione calibrate sullo stress che tengono conto del carico cognitivo
\item Percorsi di escalation predefiniti che riducono la confusione sull'autorità
\item Protocolli di debriefing psicologico che migliorano l'accuratezza dell'analisi delle cause
\end{itemize}

\subsection{DORA Pilastro 3: Test di Resilienza Operativa Digitale}

Gli Articoli 24-27 impongono programmi di test inclusi i test di penetrazione basati su minacce (TLPT) per le entità finanziarie significative.

\textbf{Articolo 25 - Test degli Strumenti e Sistemi ICT}:

Il CPF migliora i test tecnici tradizionali con la valutazione del fattore umano:

\begin{itemize}
\item \textbf{Test Social Engineering}: Valutazione integrata della resistenza psicologica
\item \textbf{Scenari Stress Testing}: Qualità decisionale umana sotto crisi simulate
\item \textbf{Test Manipolazione Autorità}: Resistenza a impersonificazione e pretexting
\end{itemize}

\textbf{Articolo 26 - Test di Penetrazione Basati su Minacce}:

I programmi TLPT richiesti per le entità significative dovrebbero incorporare:

\begin{itemize}
\item Identificazione dei target umani usando la profilazione delle vulnerabilità CPF
\item Simulazione di vettori di attacco psicologici
\item Definizione di baseline comportamentali per il rilevamento anomalie
\end{itemize}

\subsection{DORA Pilastro 4: Gestione del Rischio ICT di Terze Parti}

Gli Articoli 28-44 stabiliscono requisiti completi di supervisione delle terze parti, incluso il framework di oversight per i Fornitori ICT Critici di Terze Parti (CTPP).

\textbf{Articolo 28 - Principi Generali}:

\begin{itemize}
\item \textbf{[4.x] Valutazione Dinamiche di Fiducia}: Valutare la fiducia appropriata versus mal riposta nelle relazioni con i fornitori
\item \textbf{[5.x] Psicologia Rischio Concentrazione}: Comprendere i pattern di dipendenza organizzativa
\item \textbf{[8.x] Rischio Personale Fornitori}: Estendere la valutazione del rischio umano al personale critico di terze parti
\end{itemize}

\textbf{Registro delle Informazioni (Articolo 28(3))}:

Il registro obbligatorio degli accordi con terze parti ICT dovrebbe includere:

\begin{itemize}
\item Indicatori di rischio umano per le relazioni critiche con i fornitori
\item Mappatura delle strutture di autorità per i contatti chiave dei fornitori
\item Metriche psicologiche del rischio di concentrazione
\end{itemize}

\subsection{DORA Pilastro 5: Accordi di Condivisione delle Informazioni}

L'Articolo 45 incoraggia la condivisione di threat intelligence tra entità finanziarie.

\textbf{Miglioramento CPF per la Condivisione delle Informazioni}:

\begin{itemize}
\item \textbf{[4.x] Analisi Barriere di Fiducia}: Identificare gli ostacoli psicologici a una condivisione efficace
\item \textbf{[6.x] Dinamiche Competitive}: Affrontare la psicologia di gruppo che inibisce la collaborazione
\item \textbf{Design Protocolli di Condivisione}: Strutture che accomodano i requisiti umani di costruzione della fiducia
\end{itemize}

\section{Metodologia di Implementazione}

\subsection{Fase 1: Valutazione Gap Normativo (30 Giorni)}

\textbf{Obiettivo}: Stabilire la baseline di integrazione CPF con l'attuale postura di compliance.

\textbf{Attività}:
\begin{itemize}
\item Mappare le misure di compliance NIS2/DORA esistenti sulle categorie CPF
\item Condurre valutazione baseline delle vulnerabilità psicologiche del personale chiave
\item Identificare i punti di integrazione ad alta priorità basati sul rischio normativo
\item Stabilire un framework di misurazione allineato con il reporting normativo
\end{itemize}

\textbf{Deliverable}:
\begin{itemize}
\item Analisi gap di compliance arricchita con CPF
\item Roadmap di integrazione prioritizzata
\item Metriche baseline del rischio psicologico
\item Framework di evidenze normative
\end{itemize}

\subsection{Fase 2: Integrazione Pilota (60 Giorni)}

\textbf{Obiettivo}: Implementare la valutazione CPF nelle aree di compliance ad alto rischio.

\textbf{Attività}:
\begin{itemize}
\item Dispiegare la valutazione CPF per l'organo di gestione (compliance DORA Art. 5)
\item Implementare test del fattore umano insieme ai test di resilienza tecnica
\item Integrare indicatori psicologici nelle procedure di gestione incidenti
\item Stabilire protocolli di valutazione del rischio umano delle terze parti
\end{itemize}

\textbf{Deliverable}:
\begin{itemize}
\item Profilo di rischio psicologico dell'organo di gestione
\item Playbook di risposta incidenti migliorati
\item Registro rischio umano terze parti
\item Pacchetto iniziale di evidenze di compliance
\end{itemize}

\subsection{Fase 3: Integrazione Completa (90 Giorni)}

\textbf{Obiettivo}: Completare l'integrazione CPF su tutti i requisiti normativi.

\textbf{Attività}:
\begin{itemize}
\item Estendere la valutazione a tutto il personale in funzioni critiche
\item Integrare le metriche CPF nel framework di gestione del rischio ICT
\item Implementare monitoraggio psicologico continuo insieme al monitoraggio tecnico
\item Stabilire l'integrazione del reporting normativo
\end{itemize}

\textbf{Deliverable}:
\begin{itemize}
\item Dashboard completa del rischio fattore umano
\item Documentazione compliance NIS2/DORA integrata
\item Framework di miglioramento continuo
\item Pacchetto di evidenze per audit normativo
\end{itemize}

\section{Framework di Misurazione e ROI}

\subsection{Metriche di Compliance}

\textbf{Indicatori di Compliance NIS2}:
\begin{itemize}
\item Percentuale di copertura del rischio fattore umano sui requisiti Articolo 21
\item Miglioramento efficacia formazione (cambiamento comportamentale vs. ritenzione conoscenze)
\item Tasso di completamento valutazione rischio umano supply chain
\item Accuratezza identificazione cause umane negli incidenti
\end{itemize}

\textbf{Indicatori di Compliance DORA}:
\begin{itemize}
\item Trend del punteggio di rischio psicologico dell'organo di gestione
\item Copertura test fattore umano nel programma di test di resilienza
\item Percentuale integrazione valutazione rischio umano terze parti
\item Miglioramento partecipazione alla condivisione informazioni
\end{itemize}

\subsection{Metriche di Resilienza Operativa}

\begin{itemize}
\item Riduzione tasso incidenti legati al fattore umano
\item Tempo medio di rilevamento minacce abilitate dall'uomo
\item Punteggi qualità decisionale in condizioni di stress
\item Miglioramento resistenza al social engineering
\item Tempo di recupero psicologico post-incidente
\end{itemize}

\subsection{Framework di Calcolo del ROI}

\textbf{Calcolo del Cost Avoidance}:
\begin{align}
\text{ROI Annuale} &= \frac{\text{Costi Evitati} - \text{Costi Implementazione}}{\text{Costi Implementazione}} \times 100
\end{align}

Dove i Costi Evitati includono:
\begin{itemize}
\item Riduzione costi incidenti = (Tasso storico incidenti × Costo medio incidente) - (Tasso attuale × Costo)
\item Evitamento sanzioni normative = Potenziali sanzioni evitate ponderate per il rischio
\item Efficienza operativa = Riduzione tasso falsi positivi × Risparmio costi investigazione
\end{itemize}

\textbf{Range ROI per Servizi Finanziari Europei}:
\begin{itemize}
\item Anno 1: 180-280\% ROI (riduzione incidenti + efficienza compliance)
\item Anno 2: 350-550\% ROI (maturità operativa + riduzione costi audit)
\item Anno 3+: 450-750\% ROI (integrazione culturale + capacità predittiva)
\end{itemize}

\section{Caso Studio: Implementazione in un Gruppo Bancario Europeo}

\subsection{Profilo dell'Organizzazione}
\begin{itemize}
\item Settore: Gruppo Bancario Pan-Europeo
\item Status Normativo: Istituto Significativo (vigilato SSM)
\item Dipendenti: 28.000 in 12 Stati membri UE
\item Team IT Security: 89 professionisti
\item Budget annuale sicurezza: 18 milioni di euro
\item Requisiti normativi: NIS2 (entità essenziale) + DORA (ente creditizio)
\end{itemize}

\subsection{Approccio Implementativo}

L'organizzazione ha implementato l'integrazione CPF in 6 mesi in preparazione alla compliance DORA:

\textbf{Risultati Fase 1 (30 giorni)}:
\begin{itemize}
\item L'analisi gap ha identificato 18 lacune di compliance sul fattore umano ad alta priorità
\item La valutazione dell'organo di gestione ha rivelato che il 72\% mostrava indicatori di automation bias
\item Il 41\% del personale in funzioni critiche ha dimostrato vulnerabilità al trasferimento di autorità
\item Identificate lacune sul rischio umano nel 67\% dei fornitori ICT critici
\end{itemize}

\textbf{Risultati Fase 2 (90 giorni)}:
\begin{itemize}
\item Riduzione del 34\% nel tasso di successo degli incidenti di social engineering
\item Miglioramento del 27\% nel tempo di rilevamento delle minacce abilitate dall'uomo
\item Qualità decisionale del management sotto stress migliorata del 31\%
\item Evidenze di compliance DORA Articolo 5 significativamente rafforzate
\end{itemize}

\textbf{Risultati Fase 3 (180 giorni)}:
\begin{itemize}
\item Riduzione del 47\% degli incidenti di sicurezza totali legati al fattore umano
\item Integrazione completa con il programma di compliance sui cinque pilastri DORA
\item Miglioramento del 91\% nella qualità di risposta in condizioni di stress
\item ROI del 187\% nel primo anno
\item 2,8 milioni di euro in costi incidenti evitati
\item Valutazione positiva dall'autorità competente nazionale
\end{itemize}

\subsection{Benefici per la Compliance Normativa}

\textbf{Miglioramenti Compliance NIS2}:
\begin{itemize}
\item Dimostrate misure ``allo stato dell'arte'' per i fattori umani (Art. 21)
\item Evidenze quantificabili della responsabilità del management (Art. 20)
\item Capacità migliorate di analisi delle cause degli incidenti
\item Documentazione migliorata del rischio umano nella supply chain
\end{itemize}

\textbf{Miglioramenti Compliance DORA}:
\begin{itemize}
\item Valutazione documentata delle conoscenze dell'organo di gestione (Art. 5)
\item Integrazione del fattore umano nel framework di rischio ICT (Art. 6)
\item Test di resilienza migliorati con fattori umani (Art. 25)
\item Registro completo del rischio umano delle terze parti (Art. 28)
\end{itemize}

\section{Linee Guida e Best Practice per l'Implementazione}

\subsection{Checklist Pre-Implementazione}

\textbf{Prontezza Normativa}:
\begin{itemize}
\item Stato attuale di compliance NIS2/DORA valutato
\item Aspettative dell'autorità competente comprese
\item Requisiti di reporting normativo mappati
\item Framework di documentazione delle evidenze stabilito
\end{itemize}

\textbf{Prontezza Organizzativa}:
\begin{itemize}
\item Sponsorship esecutiva da CISO e leadership compliance
\item Allocazione budget per strumenti di valutazione fattore umano
\item Valutazione impatto privacy completata per le valutazioni psicologiche
\item Consultazione rappresentanze sindacali/comitato aziendale (dove richiesto)
\end{itemize}

\textbf{Prerequisiti Tecnici}:
\begin{itemize}
\item Framework di gestione rischio ICT operativo
\item Processi di gestione incidenti documentati
\item Programma di test di resilienza stabilito
\item Procedure di gestione rischio terze parti in essere
\end{itemize}

\subsection{Considerazioni Specifiche Europee}

\textbf{Compliance Protezione Dati}:
\begin{itemize}
\item Base giuridica GDPR Articolo 6 per le valutazioni psicologiche
\item Minimizzazione dei dati nel design delle valutazioni
\item Periodi di conservazione allineati con i requisiti normativi
\item Considerazioni sui trasferimenti transfrontalieri per gruppi multi-entità
\end{itemize}

\textbf{Relazioni con i Dipendenti}:
\begin{itemize}
\item Trasparenza sugli scopi e l'uso delle valutazioni
\item Applicazione non discriminatoria della profilazione psicologica
\item Consultazione rappresentanze sindacali dove legalmente richiesto
\item Diritti individuali di accesso ai risultati delle valutazioni
\end{itemize}

\textbf{Engagement con i Regolatori}:
\begin{itemize}
\item Discussione proattiva con le autorità competenti nazionali
\item Allineamento con le aspettative dei CSIRT settoriali
\item Integrazione con i processi di peer review
\item Documentazione adeguata per audit normativi
\end{itemize}

\subsection{Errori Comuni nell'Implementazione}

\textbf{Errori Normativi}:
\begin{itemize}
\item Trattare il CPF come sostituto della compliance tecnica anziché come miglioramento
\item Documentazione inadeguata per i requisiti di evidenza normativa
\item Mancato allineamento con le variazioni di recepimento nazionale (NIS2)
\item Sottovalutazione dei requisiti di coordinamento transfrontaliero
\end{itemize}

\textbf{Errori Organizzativi}:
\begin{itemize}
\item Valutazione impatto privacy insufficiente
\item Mancato coinvolgimento delle rappresentanze sindacali dove richiesto
\item Eccessiva enfasi sulla valutazione senza programmi di intervento
\item Mancata integrazione con i processi GRC esistenti
\end{itemize}

\textbf{Errori Tecnici}:
\begin{itemize}
\item Implementazione isolata separata dal framework di rischio ICT
\item Integrazione inadeguata con i sistemi di gestione incidenti
\item Scarso allineamento del reporting con i requisiti normativi
\item Protocolli di integrazione terze parti insufficienti
\end{itemize}

\section{Sviluppi Normativi Futuri}

\subsection{Evoluzione del Panorama Europeo}

Il framework di integrazione CPF è progettato per accomodare l'evoluzione dei requisiti normativi:

\textbf{Sviluppi Attesi}:
\begin{itemize}
\item Linee guida tecniche ENISA sui fattori umani (previste 2025-2026)
\item Standard tecnici regolamentari ESA sulle metodologie di test
\item Orientamenti delle autorità competenti nazionali sui requisiti ``stato dell'arte''
\item Procedure di coordinamento EU-CyCLONe per incidenti transfrontalieri
\end{itemize}

\textbf{Adattabilità del Framework}:
\begin{itemize}
\item La mappatura modulare delle categorie CPF consente l'integrazione degli aggiornamenti normativi
\item Il framework di misurazione supporta l'evoluzione dei requisiti di reporting
\item La documentazione delle evidenze è progettata per la continuità dell'audit trail
\item Il modello di miglioramento continuo si allinea con le aspettative normative
\end{itemize}

\section{Conclusioni e Prossimi Passi}

L'integrazione della valutazione del rischio psicologico nei programmi di compliance NIS2 e DORA colma una lacuna critica nei framework europei di resilienza operativa. Sebbene entrambe le normative riconoscano esplicitamente i fattori umani nella cybersecurity, nessuna delle due fornisce metodologie sistematiche per valutare e mitigare le vulnerabilità psicologiche che abilitano la maggior parte degli attacchi informatici riusciti.

Il Cybersecurity Psychology Framework offre una soluzione pratica e misurabile che migliora la compliance normativa producendo al contempo miglioramenti nella resilienza operativa. Attraverso la mappatura dettagliata sui requisiti NIS2 e sui cinque pilastri DORA, le organizzazioni possono implementare la valutazione delle vulnerabilità psicologiche all'interno dei loro programmi di compliance esistenti.

\textbf{Azioni Immediate per CISO e Compliance Officer Europei}:
\begin{enumerate}
\item Condurre un'analisi gap arricchita con CPF rispetto ai requisiti dell'Articolo 21 NIS2 e dei pilastri DORA
\item Prioritizzare la valutazione psicologica dell'organo di gestione (compliance DORA Articolo 5)
\item Integrare i test sul fattore umano nei programmi di test di resilienza
\item Stabilire protocolli di valutazione del rischio umano delle terze parti
\item Sviluppare un framework di evidenze normative per le misure di rischio psicologico
\item Coinvolgere le autorità competenti nazionali sull'approccio alla compliance del fattore umano
\end{enumerate}

Per le entità finanziarie che affrontano la scadenza DORA di gennaio 2025 e i fornitori di servizi essenziali soggetti agli obblighi NIS2, il framework di integrazione CPF fornisce un percorso strutturato verso una compliance completa sul fattore umano. Le organizzazioni che implementano questo approccio ottengono miglioramenti significativi sia nella postura di compliance normativa che nei risultati di resilienza operativa.

Le evidenze dimostrano che la valutazione del rischio psicologico non è semplicemente un miglioramento della compliance ma un requisito fondamentale per una resilienza operativa efficace in un ambiente in cui i fattori umani contribuiscono alla stragrande maggioranza degli incidenti di sicurezza. Man mano che i regolatori europei si concentrano sempre più sulla qualità e l'efficacia delle misure di sicurezza, framework come il CPF diventano componenti essenziali di programmi di sicurezza dimostrabilmente ``adeguati e proporzionati''.

\section*{Biografia dell'Autore}

Giuseppe Canale, CISSP, è un ricercatore indipendente in cybersecurity con 27 anni di esperienza nella gestione di programmi di sicurezza enterprise. È specializzato nell'integrazione della valutazione del rischio psicologico con i framework di compliance normativa e ha sviluppato il Cybersecurity Psychology Framework (CPF) per la valutazione della postura di sicurezza organizzativa. Il suo lavoro si concentra sul colmare il gap tra i controlli di sicurezza tecnici e le vulnerabilità del fattore umano nei contesti normativi europei.

\section*{Dichiarazione sulla Disponibilità dei Dati}

Template di implementazione, strumenti di valutazione, matrici di mappatura normativa e dettagli dei casi studio sono disponibili attraverso la piattaforma CPF3.org, soggetti ad appropriati accordi di licenza.

\begin{thebibliography}{99}

\bibitem{canale2025}
Canale, G. (2025). The Cybersecurity Psychology Framework: A Pre-Cognitive Vulnerability Assessment Model Integrating Psychoanalytic and Cognitive Sciences. \textit{SSRN Electronic Journal}. https://doi.org/10.2139/ssrn.5387222

\bibitem{verizon2024}
Verizon. (2024). \textit{2024 Data Breach Investigations Report}. Verizon Enterprise.

\bibitem{nis2}
Parlamento Europeo e Consiglio. (2022). Direttiva (UE) 2022/2555 relativa a misure per un livello comune elevato di cibersicurezza nell'Unione (Direttiva NIS2). \textit{Gazzetta Ufficiale dell'Unione Europea}, L 333/80.

\bibitem{dora}
Parlamento Europeo e Consiglio. (2022). Regolamento (UE) 2022/2554 relativo alla resilienza operativa digitale per il settore finanziario (DORA). \textit{Gazzetta Ufficiale dell'Unione Europea}, L 333/1.

\bibitem{enisa2024}
Agenzia dell'Unione Europea per la Cibersicurezza. (2024). \textit{ENISA Threat Landscape: Finance Sector}. ENISA.

\bibitem{ey2024}
EY e Institute of International Finance. (2024). \textit{Global Bank Risk Management Survey: European Results}. EY.

\bibitem{ibm2024}
IBM Security. (2024). \textit{Cost of a Data Breach Report 2024}. IBM Corporation.

\bibitem{ecb2024}
Banca Centrale Europea. (2024). One step ahead: protecting the cyber resilience of financial infrastructures. Intervento di Piero Cipollone.

\bibitem{esma2024}
Autorità Europea degli Strumenti Finanziari e dei Mercati. (2024). \textit{Digital Operational Resilience Act (DORA) Implementation Guidance}. ESMA.

\bibitem{eba2024}
Autorità Bancaria Europea. (2024). \textit{Guidelines on ICT and Security Risk Management}. EBA.

\end{thebibliography}

\end{document}