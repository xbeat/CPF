\documentclass[11pt,a4paper]{article}

\usepackage[utf8]{inputenc}
\usepackage[english]{babel}
\usepackage{amsmath}
\usepackage{amsfonts}
\usepackage{amssymb}
\usepackage{graphicx}
\usepackage{booktabs}
\usepackage{url}
\usepackage{hyperref}
\usepackage[margin=1in]{geometry}
\usepackage{fancyhdr}
\usepackage{lastpage}
\usepackage{float}
\usepackage{placeins}
\usepackage{enumitem}
\usepackage{longtable}
\usepackage{array}
\usepackage{tabularx}

\setlength{\parindent}{0pt}
\setlength{\parskip}{0.6em}

\hypersetup{
    colorlinks=true,
    linkcolor=blue,
    citecolor=blue,
    urlcolor=blue,
    pdftitle={El CPF Educational Framework: Un Currículum Universal para la Literacy en Ciberseguridad Psicológica},
    pdfauthor={Giuseppe Canale},
}

\pagestyle{fancy}
\fancyhf{}
\renewcommand{\headrulewidth}{0pt}
\fancyfoot[C]{\thepage}

\begin{document}

\thispagestyle{empty}
\begin{center}

\vspace*{0.5cm}

\rule{\textwidth}{1.5pt}

\vspace{0.5cm}

{\LARGE \textbf{El CPF Educational Framework:}}\\[0.3cm]
{\LARGE \textbf{Un Currículum Universal para la}}\\[0.3cm]
{\LARGE \textbf{Literacy en Ciberseguridad Psicológica}}

\vspace{0.5cm}

\rule{\textwidth}{1.5pt}

\vspace{0.3cm}

{\large \textsc{Companion Educativo al Cybersecurity Psychology Framework}}

\vspace{0.5cm}

{\Large Giuseppe Canale, CISSP}\\[0.2cm]
Investigador Independiente\\[0.1cm]
\href{mailto:g.canale@cpf3.org}{g.canale@cpf3.org}\\[0.1cm]
URL: \href{https://cpf3.org}{cpf3.org}\\[0.1cm]
ORCID: \href{https://orcid.org/0009-0007-3263-6897}{0009-0007-3263-6897}

\vspace{0.8cm}

{\large \today}

\vspace{1cm}

\end{center}

\begin{abstract}
\noindent
El Cybersecurity Psychology Framework (CPF) proporciona una rigurosa fundación teórica y operativa para comprender las vulnerabilidades humanas en los contextos de security. Sin embargo, la teoría sin pedagogía permanece inaccesible; los frameworks sin caminos educativos se convierten en artefactos en lugar de herramientas de cambio. Este paper presenta el CPF Educational Framework, un currículum estructurado diseñado para introducir, desarrollar y especializar a los estudiantes a través de todo el espectro de la literacy en ciberseguridad psicológica. A diferencia de los programas tradicionales de security awareness que asumen actores racionales modificables a través de la transferencia de informaciones, este enfoque educativo reconoce que las decisiones de security ocurren sustancialmente debajo de la conciencia consciente y que una educación eficaz debe involucrar los procesos pre-cognitivos, las dinámicas de grupo y la compleja interacción entre inteligencia humana y artificial. El framework comprende cuatro módulos universales---``No Decides Tú,'' ``Cómo Te Engañan,'' ``El Grupo Piensa Por Ti,'' y ``Tú y las Máquinas''---que forman un esqueleto conceptual invariante. Este esqueleto es luego modulado a través de cuatro niveles de desarrollo (Base, Intermedio, Avanzado, Especializado), cada uno calibrado sobre la complejidad apropiada, sobre los ejemplos contextuales y sobre la integración con la documentación técnica del CPF. El currículum posiciona los papers fundamentales del CPF como waypoints progresivos: la Taxonomy como mapa de referencia, el Dense Implementation Companion como especificación operativa, el Intervention Framework como metodología de remediation, y el paper Depth como mentor teórico que acompaña a los estudiantes durante todo su viaje. Esta arquitectura educativa habilita tanto iniciativas de literacy a gran escala como desarrollo profesional especializado, manteniendo la coherencia con el framework científico subyacente.

\vspace{0.5em}
\noindent\textbf{Keywords:} educación en ciberseguridad, literacy psicológica, curriculum design, factores humanos, procesos pre-cognitivos, security awareness, lifelong learning
\end{abstract}

\newpage
\tableofcontents
\newpage

%==============================================================================
\section{Introducción: El Imperativo Pedagógico}
%==============================================================================

\subsection{El Fracaso de la Educación Tradicional en Security}

La inversión global en training de cybersecurity awareness supera los \$5 mil millones anuales, sin embargo las métricas fundamentales de los incidentes de security relacionados con el factor humano no muestran ninguna mejora correspondiente \cite{verizon2023, sans2023}. Este fracaso persistente requiere una explicación. El Cybersecurity Psychology Framework ofrece una: la educación tradicional en security opera sobre un modelo fundamentalmente erróneo de la cognición y del comportamiento humano.

El paradigma educativo prevalente asume que los seres humanos son actores racionales que, cuando son informados sobre los riesgos y las consecuencias, modificarán su comportamiento en consecuencia. Esta asunción contradice décadas de investigación en neurociencias, economía del comportamiento y teoría psicoanalítica. Los experimentos fundamentales de Benjamin Libet han demostrado que las decisiones motoras ocurren 300-500 milisegundos antes de la conciencia consciente \cite{libet1983}. La teoría del dual-process de Daniel Kahneman revela que el System 1 (rápido, automático, emocional) domina el System 2 (lento, deliberado, racional) precisamente en los ambientes presionados por el tiempo y cognitivamente sobrecargados donde ocurren las decisiones de security \cite{kahneman2011}. La investigación sobre dinámicas de grupo de Wilfred Bion muestra que el comportamiento colectivo emerge de basic assumptions inconscientes que operan enteramente debajo de la conciencia consciente \cite{bion1961}.

Si las decisiones de security se toman antes de la conciencia consciente, si los procesos automáticos dominan los deliberados, si las dinámicas de grupo moldean el comportamiento individual a través de canales inconscientes---entonces la educación que apunta solo a los procesos conscientes, racionales e individuales fallará necesariamente. La pregunta no es si la educación tradicional en security está implementada mal, sino si sus asunciones fundamentales están equivocadas.

\subsection{Una Filosofía Educativa Diferente}

El CPF Educational Framework procede de asunciones diferentes. Asumimos que:

\begin{itemize}[leftmargin=2cm]
    \item \textbf{Los procesos pre-cognitivos determinan sustancialmente el comportamiento de security.} La educación debe por lo tanto involucrar estos procesos, no simplemente informar la conciencia consciente.

    \item \textbf{El aprendizaje no es transferencia de informaciones sino desarrollo del reconocimiento de patrones.} El objetivo no es llenar a los estudiantes de hechos sino desarrollar su capacidad de reconocer patrones de vulnerabilidad en sí mismos, en otros y en las organizaciones.

    \item \textbf{La educación es ignición, no completamiento.} En un dominio caracterizado por constante evolución y variación individual, la educación formal proporciona la chispa inicial; el desarrollo subsiguiente ocurre a través de la exploración autodirigida con las herramientas disponibles (incluyendo tutores AI, recursos de la comunidad y retorno a las estructuras formales cuando sea necesario).

    \item \textbf{El mismo esqueleto conceptual sirve a todos los estudiantes.} Lo que varía no son los insights fundamentales sino su aplicación contextual, la complejidad de los ejemplos y la profundidad del grounding teórico.

    \item \textbf{La vulnerabilidad psicológica es permanente y pervasiva.} A diferencia de las vulnerabilidades técnicas que pueden ser parcheadas, las vulnerabilidades psicológicas son intrínsecas a la cognición humana. La educación apunta no a la eliminación sino a la conciencia, al reconocimiento y al acomodamiento estratégico.
\end{itemize}

Estas asunciones producen un framework educativo fundamentalmente diferente de la security awareness tradicional. No enseñamos reglas a seguir sino patrones a reconocer. No asumimos que los estudiantes cambiarán su naturaleza sino que puedan comprenderla. No posicionamos la educación como una credencial completada sino como un viaje iniciado.

\subsection{El Viaje del Héroe: Una Metáfora Organizativa}

El monomito de Joseph Campbell---el viaje del héroe---proporciona una metáfora organizativa útil para la experiencia educativa del CPF \cite{campbell1949}. El estudiante inicia en el mundo ordinario de la confianza ingenua en la propia racionalidad y autonomía. La llamada a la aventura llega a través del reconocimiento de que ``no decides tú''---que los procesos pre-cognitivos moldean sustancialmente el comportamiento. El cruce del umbral ocurre cuando este reconocimiento se vuelve personal, cuando el estudiante ve estos patrones operar en la propia experiencia.

El viaje a través del mundo especial involucra un engagement progresivamente más profundo con los mecanismos de la vulnerabilidad: influencia social, dinámicas de grupo, respuestas al estrés, procesos inconscientes. Cada estadio revela nuevos aspectos de cómo la psicología humana crea patrones explotables. El estudiante encuentra aliados (compañeros de viaje, recursos educativos, tutores AI) y enemigos (sesgos cognitivos, resistencia defensiva, la atracción de las ilusiones confortables).

En esta metáfora, la documentación técnica del CPF sirve funciones narrativas específicas:

\begin{itemize}[leftmargin=2cm]
    \item \textbf{La Taxonomy} es el mapa del mundo especial---la enumeración sistemática de los territorios a explorar, de los peligros a reconocer, de los patrones a comprender.

    \item \textbf{El Dense Implementation Companion} sirve como manual técnico---las especificaciones operativas que traducen la comprensión conceptual en detection y response accionables.

    \item \textbf{El Intervention Framework} representa el don del retorno---la metodología que transforma la comprensión personal en capacidad de cambio organizacional.

    \item \textbf{El paper Depth} funciona como la figura del mentor que aparece durante todo el viaje, proporcionando grounding teórico cuando sea necesario, explicando por qué el mapa está dibujado como está, ofreciendo sabiduría que se profundiza en cada nuevo encuentro.
\end{itemize}

El viaje del héroe no termina. El retorno al mundo ordinario encuentra al estudiante transformado, viendo patrones previamente invisibles, reconociendo vulnerabilidades en sí y en el ambiente, equipado con frameworks para el desarrollo continuo. Pero el viaje continúa porque la vulnerabilidad psicológica continúa, porque el threat landscape evoluciona, porque la comprensión se profundiza con la experiencia.

\subsection{Estructura del Documento}

Este paper procede como sigue. La Sección 2 presenta el Marco Universal: los cuatro módulos que constituyen el esqueleto conceptual invariante aplicable a todos los niveles de desarrollo. La Sección 3 detalla la Modulación Contextual: cómo cada módulo se adapta a los niveles Base, Intermedio, Avanzado y Especializado manteniendo la integridad conceptual. La Sección 4 aborda la Arquitectura de Integración: cómo el framework educativo se conecta e incorpora progresivamente la documentación técnica del CPF. La Sección 5 proporciona una Guía de Implementación: consideraciones prácticas para el deployment de este currículum a través de los contextos educativos. La Sección 6 discute Assessment y Progresión: cómo se evalúa el desarrollo del estudiante y cómo se gestionan las transiciones entre niveles. La Sección 7 concluye con reflexiones sobre el futuro de la educación en ciberseguridad psicológica.

%==============================================================================
\section{El Marco Universal: Cuatro Módulos}
%==============================================================================

El esqueleto conceptual de la educación CPF comprende cuatro módulos, cada uno abordando un dominio fundamental de vulnerabilidad psicológica. Estos módulos son universales en el sentido de que sus insights core se aplican a todas las edades, contextos y niveles de desarrollo. Lo que varía no es el insight sino su elaboración, ejemplificación y profundidad teórica.

Los cuatro módulos son:

\begin{enumerate}[leftmargin=2cm]
    \item \textbf{No Decides Tú} --- Las neurociencias y la psicología del decision-making pre-consciente
    \item \textbf{Cómo Te Engañan} --- Los mecanismos de la influencia social y de la manipulación
    \item \textbf{El Grupo Piensa Por Ti} --- Las dinámicas colectivas y sus implicaciones para la security
    \item \textbf{Tú y las Máquinas} --- Las vulnerabilidades de la interacción humano-AI
\end{enumerate}

Cada módulo está diseñado para funcionar tanto independientemente como parte de la secuencia integrada. La secuencia cuenta: el Módulo 1 establece el reconocimiento fundamental de que el control consciente es más limitado de lo que la intuición sugiere; el Módulo 2 aplica este reconocimiento a la influencia interpersonal; el Módulo 3 se extiende a los fenómenos colectivos; el Módulo 4 introduce las complicaciones nuevas de los sistemas artificiales. Sin embargo, cualquier módulo puede servir como punto de entrada para estudiantes con intereses o necesidades específicas.

\subsection{Módulo 1: No Decides Tú}

\subsubsection{Insight Core}

El insight core del Módulo 1 es que las decisiones humanas ocurren a través de procesos sustancialmente fuera de la conciencia consciente, y que estos procesos pre-conscientes son tanto explotables como ampliamente no modificables a través del solo esfuerzo consciente.

Este insight contradice intuiciones profundas sobre autonomía y autocontrol. La mayoría de las personas experimentan sus propias decisiones como productos de la deliberación consciente---``piensan sobre ello'' y luego ``deciden.'' La evidencia neurocientífica y psicológica sugiere que esta experiencia es parcialmente ilusoria: la decisión a menudo ya ha sido tomada por procesos pre-conscientes, y la deliberación consciente es una narrativa post-hoc que acompaña en lugar de causar la decisión \cite{libet1983, soon2008}.

\subsubsection{Fundamentos Teóricos}

El Módulo 1 extrae de tres tradiciones teóricas primarias:

\textbf{Neurociencias del Decision-Making.}
\begin{itemize}[leftmargin=2cm]
    \item Los experimentos de Libet han demostrado que el potencial de preparación del cerebro---actividad eléctrica que indica preparación motora---precede la conciencia consciente de la intención de moverse por aproximadamente 350 milisegundos \cite{libet1983}
    \item Soon et al. han extendido este resultado, mostrando que los patrones de actividad cerebral podían predecir las decisiones hasta 10 segundos antes de la conciencia consciente \cite{soon2008}
    \item Estos resultados sugieren que la conciencia consciente de la decisión es efecto en lugar de causa
\end{itemize}

\textbf{Teoría del Dual-Process.}
\begin{itemize}[leftmargin=2cm]
    \item El framework System 1/System 2 de Kahneman proporciona un modelo accesible para comprender la relación entre elaboración automática y deliberada \cite{kahneman2011}
    \item El System 1 opera automáticamente, rápidamente, con poco sentido de control voluntario
    \item El System 2 asigna atención a las actividades mentales effortful, incluyendo los cálculos complejos
    \item Crucialmente, el System 2 a menudo sirve como racionalizador post-hoc de las conclusiones del System 1 en lugar de como evaluador independiente
\end{itemize}

\textbf{Hipótesis del Marcador Somático.}
\begin{itemize}[leftmargin=2cm]
    \item La investigación de Damasio demuestra que las emociones y los estados corporales influyen sustancialmente el decision-making a través de mecanismos que evitan la deliberación consciente \cite{damasio1994}
    \item El ``gut feeling'' no es metafórico sino que refleja estados somáticos reales que guían la elección a través de canales pre-conscientes
\end{itemize}

\subsubsection{Implicaciones para la Security}

Las implicaciones para la security del control consciente limitado son profundas:

\begin{itemize}[leftmargin=2cm]
    \item Las decisiones de security tomadas bajo presión temporal, carga cognitiva o activación emocional son dominadas por procesos pre-conscientes que podrían no alinearse con los intereses de security.

    \item El training que apunta solo al conocimiento consciente (``recuerda verificar la dirección del remitente'') podría fallar en influir el comportamiento real cuando los procesos pre-conscientes apuntan diferentemente.

    \item Los atacantes que pueden activar estados emocionales específicos o cargas cognitivas pueden predeciblemente desplazar el decision-making hacia patrones explotables.

    \item El auto-assessment de la vulnerabilidad es no confiable porque los procesos que crean vulnerabilidad operan debajo del umbral del acceso consciente.
\end{itemize}

\subsubsection{Objetivos de Aprendizaje del Módulo}

Completando el Módulo 1, los estudiantes serán capaces de:

\begin{enumerate}[leftmargin=2cm]
    \item Explicar la evidencia del decision-making pre-consciente y sus implicaciones para el comportamiento de security.

    \item Identificar situaciones en las que las propias decisiones son probablemente dominadas por la elaboración del System 1.

    \item Reconocer las condiciones (presión temporal, carga cognitiva, activación emocional) que desplazan el decision-making lejos del control deliberado.

    \item Articular por qué el training tradicional de security awareness tiene eficacia limitada.

    \item Describir la relación entre este módulo y las Categorías CPF 5 (Cognitive Overload), 7 (Stress Response) y 8 (Unconscious Processes).
\end{enumerate}

\subsubsection{Conexión a la Documentación CPF}

El Módulo 1 introduce conceptos que son sistemáticamente desarrollados en la Taxonomy CPF y teóricamente fundados en el paper Depth. Específicamente:

\begin{itemize}[leftmargin=2cm]
    \item La Categoría 5 de la Taxonomy (Cognitive Overload Vulnerabilities) operacionaliza las dinámicas System 1/System 2 en indicadores medibles.

    \item La Categoría 7 de la Taxonomy (Stress Response Vulnerabilities) mapea la respuesta neurobiológica al estrés sobre los comportamientos security-relevant.

    \item La Categoría 8 de la Taxonomy (Unconscious Process Vulnerabilities) extiende la fundación neurocientífica al territorio psicoanalítico.

    \item La sección del paper Depth sobre ``The Integration Problem'' explica cómo estas dispares tradiciones teóricas son reconciliadas dentro del framework CPF.
\end{itemize}

Los estudiantes en el nivel Base reciben estas conexiones como referencias hacia adelante---invitaciones a la exploración futura. Los estudiantes en los niveles Avanzado y Especializado se comprometen directamente con el material referenciado.

%------------------------------------------------------------------------------
\subsection{Módulo 2: Cómo Te Engañan}
%------------------------------------------------------------------------------

\subsubsection{Insight Core}

El insight core del Módulo 2 es que la cognición social humana evolucionó para la cooperación en pequeños grupos y es sistemáticamente explotable a través de mecanismos de influencia predecibles que operan ampliamente debajo de la conciencia consciente.

Los seres humanos son animales sociales cuya supervivencia dependía históricamente de la cooperación dentro de pequeños grupos de individuos conocidos. Los atajos cognitivos que han facilitado esta cooperación---reciprocidad, consistencia, social proof, deferencia a la autoridad, liking, respuesta a la escasez---permanecen activos en ambientes modernos para los cuales están escasamente adaptados. La comunicación digital remueve los indicios que históricamente señalaban confiabilidad o engaño. Las redes globalizadas conectan a los individuos con otros desconocidos que pueden explotar la programación social diseñada para la interacción a escala de aldea.

\subsubsection{Fundamentos Teóricos}

El Módulo 2 extrae primariamente del análisis sistemático de los principios de influencia de Robert Cialdini \cite{cialdini2007}, integrado de la psicología evolucionista y de las neurociencias sociales.

\textbf{Los Seis Principios de Influencia.} Cialdini ha identificado seis principios fundamentales a través de los cuales las personas son influenciadas:

\begin{enumerate}[leftmargin=2cm]
    \item \textbf{Reciprocidad}: Sentimos la obligación de devolver los favores, incluso aquellos no solicitados, incluso cuando el retorno excede el don original.

    \item \textbf{Commitment y Consistencia}: Una vez tomada una posición, experimentamos presión a comportarnos coherentemente con ese compromiso.

    \item \textbf{Social Proof}: Determinamos el comportamiento correcto observando qué hacen los otros, especialmente en situaciones ambiguas.

    \item \textbf{Autoridad}: Nos sometemos a las figuras de autoridad percibidas, a menudo sin evaluación consciente de su real competencia o legitimidad.

    \item \textbf{Liking}: Cumplimos más prontamente con personas que nos gustan, y el liking es influido por similaridad, cumplidos y mera familiaridad.

    \item \textbf{Escasez}: Evaluamos las cosas más cuando son raras o se están volviendo raras, y esta evaluación distorsiona el decision-making.
\end{enumerate}

\textbf{Contexto de Psicología Evolucionista.} Estos mecanismos de influencia no son arbitrarios sino que reflejan presiones evolutivas. La reciprocidad ha habilitado la cooperación más allá del parentesco. La consistencia señalaba confiabilidad a los potenciales cooperadores. El social proof proporcionaba información sobre los peligros y las oportunidades ambientales. La deferencia a la autoridad facilitaba la coordinación. El liking promovía la cohesión in-group. La respuesta a la escasez aseguraba atención a los recursos raros.

\textbf{Investigación sobre Autoridad de Milgram.} Los experimentos sobre obediencia de Stanley Milgram han demostrado que personas ordinarias administrarían choques eléctricos aparentemente peligrosos a víctimas inocentes cuando instruidas por una figura de autoridad \cite{milgram1974}. Esta investigación ha revelado la profundidad de la deferencia a la autoridad---un override pre-consciente de la ética y del juicio personales.

\subsubsection{Implicaciones para la Security}

Los mecanismos de influencia social se mapean directamente sobre los vectores de ataque:

\begin{itemize}[leftmargin=2cm]
    \item \textbf{Reciprocidad} habilita ataques quid pro quo: ``Te ayudé con ese problema técnico, ahora podrías solo...''

    \item \textbf{Escalation del commitment} habilita escalación gradual de las solicitudes: pequeña compliance inicial lleva a mayor compliance subsiguiente.

    \item \textbf{Social proof} habilita claims de acción colectiva: ``Tus colegas ya han proporcionado sus credenciales para la auditoría.''

    \item \textbf{Autoridad} habilita ataques de impersonation: CEO fraud, fake IT support, falsas afirmaciones regulatorias.

    \item \textbf{Liking} habilita manipulación basada en el rapport: establecer conexión personal antes de la explotación.

    \item \textbf{Escasez} habilita ataques de urgencia: ``Esta oferta expira en 10 minutos'' o ``Solo 3 lugares restantes.''
\end{itemize}

\subsubsection{Objetivos de Aprendizaje del Módulo}

Completando el Módulo 2, los estudiantes serán capaces de:

\begin{enumerate}[leftmargin=2cm]
    \item Identificar cada uno de los seis principios de influencia de Cialdini en ejemplos del mundo real.

    \item Reconocer cuando los principios de influencia son empleados contra ellos en las comunicaciones digitales.

    \item Explicar los orígenes evolutivos de la susceptibilidad a estos mecanismos de influencia.

    \item Describir tipos de ataque específicos (phishing, pretexting, social engineering) en términos de los principios de influencia que explotan.

    \item Articular estrategias defensivas que tomen en cuenta la naturaleza pre-consciente de la susceptibilidad a la influencia.

    \item Conectar este módulo a las Categorías CPF 1 (Authority-Based), 2 (Temporal) y 3 (Social Influence) vulnerabilities.
\end{enumerate}

\subsubsection{Conexión a la Documentación CPF}

El Módulo 2 introduce las categorías de vulnerabilidad que forman las primeras tres columnas de la Taxonomy CPF:

\begin{itemize}[leftmargin=2cm]
    \item La Categoría 1 (Authority-Based Vulnerabilities) mapea sistemáticamente los patrones de deferencia a la autoridad incluyendo compliance sin cuestionamiento, efectos del gradiente de autoridad y normalización de las excepciones executive.

    \item La Categoría 2 (Temporal Vulnerabilities) operacionaliza los mecanismos de escasez y urgencia incluyendo deadline-driven risk acceptance e hyperbolic discounting de las amenazas futuras.

    \item La Categoría 3 (Social Influence Vulnerabilities) proporciona la enumeración completa de los indicadores derivados de Cialdini incluyendo reciprocity exploitation, commitment escalation y social proof manipulation.
\end{itemize}

El Dense Implementation Companion especifica cómo estas vulnerabilidades se manifiestan en comportamientos observables y cómo la detection logic puede identificar los intentos de explotación. Los estudiantes avanzados se comprometen directamente con estas especificaciones.

%------------------------------------------------------------------------------
\subsection{Módulo 3: El Grupo Piensa Por Ti}
%------------------------------------------------------------------------------

\subsubsection{Insight Core}

El insight core del Módulo 3 es que el comportamiento colectivo emerge de dinámicas a nivel de grupo que no son reducibles a la suma de las psicologías individuales, y que estas dinámicas crean vulnerabilidades de security sistemáticas invisibles al análisis focalizado en el individuo.

Cuando los seres humanos se reúnen en grupos, ocurre algo que trasciende la cognición individual. Los grupos desarrollan sus propias asunciones, defensas y patrones de comportamiento. Los individuos dentro de los grupos se comportan diferentemente de cómo lo harían solos, a menudo sin conciencia de esta influencia. El grupo se convierte en una entidad psicológica con sus propias dinámicas, y estas dinámicas pueden crear blind spots de security, amplificar el risk-taking, difundir la responsabilidad y sobrescribir el juicio individual.

\subsubsection{Fundamentos Teóricos}

El Módulo 3 extrae primariamente de la teoría de las dinámicas de grupo de Wilfred Bion \cite{bion1961}, integrada de la investigación sobre groupthink, social loafing y comportamiento colectivo.

\textbf{Las Basic Assumptions de Bion.} Bion ha identificado tres basic assumptions que los grupos adoptan inconscientemente cuando confrontan la ansiedad:

\begin{enumerate}[leftmargin=2cm]
    \item \textbf{Dependency (baD)}: El grupo se comporta como si se hubiera reunido para ser protegido por un líder omnisciente, omnipotente. En los contextos de security, esto se manifiesta como over-reliance sobre los vendors de security, sobre la autoridad del CISO, o sobre los ``silver bullets'' tecnológicos.

    \item \textbf{Fight-Flight (baF)}: El grupo se comporta como si se hubiera reunido para combatir o huir de un enemigo. En los contextos de security, esto se manifiesta como defensa perimetral agresiva combinada con negación de las amenazas insider, o como evitación y minimización de los riesgos reconocidos.

    \item \textbf{Pairing (baP)}: El grupo se comporta como si se hubiera reunido para asistir al nacimiento de un nuevo líder o idea que los salvará. En los contextos de security, esto se manifiesta como adquisición continua de tools y esperanza en soluciones futuras mientras las vulnerabilidades fundamentales permanecen no abordadas.
\end{enumerate}

Estas basic assumptions operan inconscientemente. Los miembros del grupo no deciden adoptarlas; son atraídos a ellas por fuerzas a nivel de grupo. La basic assumption proporciona seguridad psicológica gestionando la ansiedad, pero lo hace al costo de un engagement realista con las amenazas reales.

\textbf{Groupthink.} El análisis de Irving Janis sobre los desastres de política exterior ha identificado el groupthink---una modalidad de razonamiento colectivo en la cual el deseo de armonía sobrepasa la evaluación realista \cite{janis1982}. Los síntomas del groupthink incluyen ilusión de invulnerabilidad, racionalización colectiva, creencia en la moralidad intrínseca, estereotipización de los outgroups, presión sobre los disidentes, autocensura, ilusión de unanimidad y mindguards autonombrados.

\textbf{Sistemas de Defensa Social.} La investigación de Isabel Menzies Lyth sobre los servicios de enfermería ha revelado que las organizaciones desarrollan ``sistemas de defensa social''---estructuras y prácticas que sirven funciones defensivas inconscientes contra la ansiedad \cite{menzies1960}. Estos sistemas aparecen irracionales desde una perspectiva de tarea pero son altamente racionales desde una perspectiva defensiva. Intervenir en los sistemas de defensa social sin abordar la ansiedad subyacente produce crisis psicológica en lugar de mejora.

\subsubsection{Implicaciones para la Security}

Las dinámicas de grupo crean vulnerabilidades de security distintivas:

\begin{itemize}[leftmargin=2cm]
    \item \textbf{Groupthink} produce blind spots de security donde la evaluación crítica es suprimida para mantener la cohesión de grupo.

    \item \textbf{Risky shift} (polarización de grupo) lleva a los equipos a aceptar riesgos que ningún miembro individual aceptaría solo.

    \item \textbf{Difusión de la responsabilidad} significa que las tareas de security poseídas por ``todos'' son efectivamente poseídas por nadie.

    \item \textbf{Social loafing} reduce el esfuerzo individual sobre las responsabilidades de security colectivas.

    \item \textbf{Bystander effect} paraliza el incident response cuando múltiples personas asisten a un evento de security.

    \item \textbf{Basic assumptions} distorsionan la percepción y la respuesta organizativa a las amenazas en modos predecibles.
\end{itemize}

\subsubsection{Objetivos de Aprendizaje del Módulo}

Completando el Módulo 3, los estudiantes serán capaces de:

\begin{enumerate}[leftmargin=2cm]
    \item Describir las tres basic assumptions de Bion e identificar sus manifestaciones en las posturas de security organizacional.

    \item Reconocer los síntomas del groupthink en los procesos de toma de decisiones de equipo.

    \item Explicar cómo difusión de la responsabilidad, social loafing y bystander effect comprometen las funciones de security.

    \item Articular por qué las intervenciones focalizadas en el individuo son insuficientes para las vulnerabilidades a nivel de grupo.

    \item Identificar indicadores de dinámicas de grupo no saludables en los propios equipos y organizaciones.

    \item Conectar este módulo a la Categoría CPF 6 (Group Dynamic Vulnerabilities) y a los indicadores correlacionados a través de otras categorías.
\end{enumerate}

\subsubsection{Conexión a la Documentación CPF}

El Módulo 3 proporciona la fundación conceptual para la Categoría 6 de la Taxonomy CPF, que incluye:

\begin{itemize}[leftmargin=2cm]
    \item Los Indicadores 6.1-6.5 abordan los fenómenos de grupo clásicos (groupthink, risky shift, difusión de la responsabilidad, social loafing, bystander effect)

    \item Los Indicadores 6.6-6.8 operacionalizan las basic assumptions de Bion (dependency, fight-flight, pairing)

    \item Los Indicadores 6.9-6.10 abordan los fenómenos a nivel organizacional (organizational splitting, mecanismos de defensa colectivos)
\end{itemize}

La sección del paper Depth sobre ``The Integration Problem'' explica cómo la teoría psicoanalítica de grupo de Bion es integrada con la psicología cognitiva y traducida en indicadores organizacionales medibles. El Intervention Framework proporciona guía específica para abordar las vulnerabilidades a nivel de grupo, extrayendo de la teoría del cambio organizacional y de la metodología de consultoría psicoanalítica.

%------------------------------------------------------------------------------
\subsection{Módulo 4: Tú y las Máquinas}
%------------------------------------------------------------------------------

\subsubsection{Insight Core}

El insight core del Módulo 4 es que la interacción humano-AI introduce vulnerabilidades psicológicas nuevas que combinan y transforman las vulnerabilidades abordadas en los módulos precedentes, creando una categoría emergente de riesgo de security que los frameworks existentes no abordan adecuadamente.

A medida que los sistemas de inteligencia artificial se vuelven integrales a las operaciones de security y a la vida cotidiana, los seres humanos interactúan con entidades que no son ni humanas ni tools tradicionales. Estas interacciones activan mecanismos psicológicos diseñados para contextos sociales humanos, produciendo distorsiones características: antropomorfización que atribuye intenciones humanas a procesos algorítmicos, automation bias que sobre-confía en las recomendaciones de las máquinas, algorithm aversion que paradójicamente rechaza la guía del AI incluso cuando es superior al juicio humano.

Estas vulnerabilidades no son simplemente items adicionales en una lista. Interactúan con y transforman las vulnerabilidades de los módulos precedentes. La deferencia a la autoridad se extiende a los sistemas AI percibidos como autoritativos. Las dinámicas de grupo ahora incluyen equipos humano-AI con comportamientos colectivos nuevos. El decision-making pre-consciente es influido por recomendaciones AI que evitan la evaluación deliberada.

\subsubsection{Fundamentos Teóricos}

El Módulo 4 representa una integración teórica nueva, ya que el CPF está entre los primeros frameworks en abordar sistemáticamente las vulnerabilidades psicológicas AI-specific en los contextos de security. La base teórica extrae de:

\textbf{Investigación sobre Antropomorfización.} Los seres humanos atribuyen prontamente estados mentales, intenciones y emociones a entidades no humanas, incluyendo los sistemas AI \cite{epley2007}. Esta antropomorfización no es meramente metafórica sino que influye el comportamiento real: las personas que perciben el AI como human-like son más propensas a confiar en sus recomendaciones, sentir conexión emocional y ser manipulables a través de la interfaz AI.

\textbf{Investigación sobre Automation Bias.} El automation bias se refiere a la tendencia a sobre-depender de los sistemas automatizados, incluso cuando la evidencia sugiere que el sistema está errando \cite{parasuraman1997}. Este sesgo produce errores característicos: errores de omisión (fallo en detectar problemas porque el sistema no ha alertado) y errores de comisión (seguir recomendaciones automatizadas incorrectas).

\textbf{Investigación sobre Algorithm Aversion.} Paradójicamente, los seres humanos a veces rechazan las recomendaciones algorítmicas incluso cuando los algoritmos demostrablemente superan el juicio humano \cite{dietvorst2015}. Esta algorithm aversion es particularmente activada cuando los seres humanos observan el algoritmo hacer errores, incluso si las tasas de error humano son más altas.

\textbf{Investigación sobre Human-AI Teaming.} La investigación emergente sobre la colaboración humano-AI revela que los equipos mixtos exhiben dinámicas nuevas que no pueden ser predichas de las solas dinámicas de grupo humano. La calibración de la confianza, la asignación de los roles y la atribución de la responsabilidad funcionan diferentemente cuando los miembros del equipo incluyen sistemas AI.

\subsubsection{Implicaciones para la Security}

Las vulnerabilidades AI-specific crean riesgos de security distintivos:

\begin{itemize}[leftmargin=2cm]
    \item \textbf{Antropomorfización} habilita la manipulación a través de interfaces AI: un atacante que compromete un AI assistant gana la relación de confianza que el humano ha desarrollado con ese assistant.

    \item \textbf{Automation bias} produce sobre-dependencia de los tools de security AI, vigilancia humana reducida y atrofia de las skills en los equipos de security.

    \item \textbf{Algorithm aversion} produce subutilización de las capacidades de security AI, particularmente después de que se observan errores del AI.

    \item \textbf{AI hallucination acceptance} lleva a los seres humanos a confiar en outputs AI confiados que son factualmente incorrectos.

    \item \textbf{Human-AI team dysfunction} produce modalidades de fallo nuevas en las operaciones de security que incluyen componentes AI.

    \item \textbf{Adversarial AI exploitation} usa los sesgos AI-related de los seres humanos como vectores de ataque.
\end{itemize}

\subsubsection{Objetivos de Aprendizaje del Módulo}

Completando el Módulo 4, los estudiantes serán capaces de:

\begin{enumerate}[leftmargin=2cm]
    \item Explicar antropomorfización, automation bias y algorithm aversion, con ejemplos de contextos de security.

    \item Reconocer las propias tendencias hacia sesgos AI-related en las interacciones con sistemas AI.

    \item Describir cómo las vulnerabilidades AI-specific interactúan con y transforman las vulnerabilidades de los módulos precedentes.

    \item Articular estrategias de calibración de la confianza apropiadas para los tools de security AI.

    \item Identificar indicadores de dinámicas de equipo humano-AI no saludables.

    \item Conectar este módulo a la Categoría CPF 9 (AI-Specific Bias Vulnerabilities) y comprender su interacción con otras categorías.
\end{enumerate}

\subsubsection{Conexión a la Documentación CPF}

El Módulo 4 proporciona la fundación conceptual para la Categoría 9 de la Taxonomy CPF, que incluye:

\begin{itemize}[leftmargin=2cm]
    \item Los Indicadores 9.1-9.3 abordan los sesgos AI core (antropomorfización, automation bias, algorithm aversion)

    \item Los Indicadores 9.4-9.6 abordan las dinámicas de autoridad y confianza AI (AI authority transfer, efectos uncanny valley, ML opacity trust)

    \item Los Indicadores 9.7-9.10 abordan las modalidades de fallo AI-specific (hallucination acceptance, human-AI team dysfunction, AI emotional manipulation, algorithmic fairness blindness)
\end{itemize}

El Dense Implementation Companion proporciona especificaciones operativas para detectar las vulnerabilidades AI-specific, incluyendo la cuantificación de la antropomorfización a través del análisis del uso de los pronombres y del lenguaje emocional, y la medición del automation bias a través del tracking del override rate.

%==============================================================================
\section{Modulación Contextual: Cuatro Niveles de Desarrollo}
%==============================================================================

Los cuatro módulos descritos arriba constituyen el esqueleto conceptual invariante de la educación CPF. Este esqueleto es modulado a través de cuatro niveles de desarrollo, cada uno calibrado sobre:

\begin{itemize}[leftmargin=2cm]
    \item \textbf{Complejidad}: Profundidad teórica y sofisticación técnica
    \item \textbf{Contexto}: Ejemplos, escenarios y aplicaciones relevantes para la situación del estudiante
    \item \textbf{Integración}: Conexión a la documentación técnica CPF
    \item \textbf{Outcome}: Capacidades esperadas al completamiento
\end{itemize}

Los cuatro niveles son:

\begin{enumerate}[leftmargin=2cm]
    \item \textbf{Nivel Base} (edad 14-16, población general)
    \item \textbf{Nivel Intermedio} (edad 16-19, pre-profesional)
    \item \textbf{Nivel Avanzado} (universidad, inicio de carrera)
    \item \textbf{Nivel Especializado} (profesionales de la security)
\end{enumerate}

Estos niveles no son rígidas franjas de edad sino estadios de desarrollo que los estudiantes atraviesan a su propio ritmo. Un joven de catorce años con particular aptitud podría progresar rápidamente al nivel Intermedio; un profesional que encuentra el CPF por primera vez inicia del nivel Base independientemente de la edad. Los niveles describen gradientes de complejidad, no categorías demográficas.

\subsection{Nivel Base: Ignición}

\subsubsection{Target Audience}

El Nivel Base está diseñado para estudiantes sin exposición precedente a los conceptos de ciberseguridad psicológica. El target primario son los adolescentes (edad 14-16) en la instrucción secundaria, pero el nivel es igualmente apropiado para adultos que buscan una orientación inicial.

\subsubsection{Filosofía Educativa}

Al Nivel Base, la filosofía educativa enfatiza la \textit{ignición respecto al completamiento}. El objetivo no es una cobertura comprensiva sino un engagement suficiente para activar la exploración continua. El Nivel Base debería dejar a los estudiantes con:

\begin{itemize}[leftmargin=2cm]
    \item Reconocimiento de que sus decisiones son menos autónomas de lo que asumían
    \item Conciencia de técnicas de manipulación específicas que podrían encontrar
    \item Vocabulario para discutir las vulnerabilidades psicológicas
    \item Curiosidad hacia una comprensión más profunda
    \item Conocimiento de que existen recursos más profundos (la documentación CPF)
\end{itemize}

\subsubsection{Ejemplos Contextuales}

Los ejemplos del Nivel Base extraen de contextos familiares al target:

\begin{itemize}[leftmargin=2cm]
    \item \textbf{Manipulación en los social media}: Cómo las plataformas explotan los sesgos cognitivos para maximizar el engagement
    \item \textbf{Psicología del gaming}: Loot boxes, mecánicas FOMO, presión social en los ambientes multiplayer
    \item \textbf{Estafas online}: Phishing, romance scams, fake giveaways que toman de mira a los jóvenes
    \item \textbf{Influencia de los pares}: Cómo social proof y conformidad operan en los contextos sociales adolescentes
    \item \textbf{Asistentes AI}: Antropomorfización de Siri, Alexa, ChatGPT; calibración apropiada de la confianza
\end{itemize}

\subsubsection{Adaptaciones de los Módulos}

\textbf{Módulo 1 (No Decides Tú) al Nivel Base:}

Las neurociencias son simplificadas en demostraciones accesibles. Los estudiantes experimentan en lugar de estudiar la elaboración pre-consciente a través de:

\begin{itemize}[leftmargin=2cm]
    \item Demostraciones del efecto Stroop que muestran la elaboración automática
    \item Ilusiones ópticas que demuestran gaps percepción-cognición
    \item Simples experimentos de tiempo de reacción que revelan retrasos de elaboración
    \item Discusión de los ``gut feelings'' y de la intuición en el decision-making
\end{itemize}

El framework System 1/System 2 es introducido a través de ejemplos cotidianos (juicios instantáneos sobre las personas, matemática intuitiva versus calculada) antes de la aplicación a los contextos de security.

\textbf{Módulo 2 (Cómo Te Engañan) al Nivel Base:}

Los principios de influencia son enseñados a través de ejercicios de reconocimiento usando ejemplos reales:

\begin{itemize}[leftmargin=2cm]
    \item Análisis de emails de phishing para identificar urgencia (escasez), claims de autoridad y social proof
    \item Examen de publicidad en los social media para explotación de reciprocidad y liking
    \item Revisión del influencer marketing para mecanismos de autoridad y social proof
    \item Discusión de experiencias personales de intentos de manipulación
\end{itemize}

El objetivo es el reconocimiento de los patrones, no la teoría comprensiva. Los estudiantes deberían ser capaces de decir ``esto es un juego de escasez'' o ``están usando la autoridad'' cuando encuentran manipulación.

\textbf{Módulo 3 (El Grupo Piensa Por Ti) al Nivel Base:}

Las dinámicas de grupo son introducidas a través de escenarios relacionables:

\begin{itemize}[leftmargin=2cm]
    \item Por qué las personas comparten información no verificada cuando ``todos'' la comparten
    \item Cómo los chats de grupo crean presión a conformarse
    \item Por qué los bystanders no intervienen en el harassment online
    \item Cómo los clanes de gaming y las comunidades online desarrollan su propio ``groupthink''
\end{itemize}

Las basic assumptions de Bion son simplificadas en conceptos accesibles: ``buscar un salvador'' (dependency), ``nosotros contra ellos'' (fight-flight), ``esperar la próxima gran cosa'' (pairing).

\textbf{Módulo 4 (Tú y las Máquinas) al Nivel Base:}

Las vulnerabilidades AI son introducidas a través de experiencia directa:

\begin{itemize}[leftmargin=2cm]
    \item Ejercicios con AI chatbots para demostrar tendencias a la antropomorfización
    \item Discusión de cuando las recomendaciones AI deberían y no deberían ser confiadas
    \item Examen de contenido AI-generated (imágenes, texto) y riesgos de hallucination
    \item Consideración de las implicaciones privacy de las interacciones con asistentes AI
\end{itemize}

\subsubsection{Integración con la Documentación CPF}

Al Nivel Base, la documentación CPF es referenciada pero no asignada. La Taxonomy es mencionada como ``un mapa comprensivo de 100 modos diferentes en los que estas vulnerabilidades se manifiestan en las organizaciones.'' A los estudiantes se les dice que una exploración más profunda está disponible cuando estén listos, pero no se asume que la perseguirán.

La función de la referencia a la documentación en este nivel es de:

\begin{itemize}[leftmargin=2cm]
    \item Señalar que hay más por aprender (estimulación de la curiosidad)
    \item Proporcionar un landmark para la exploración autodirigida futura
    \item Establecer el CPF como un cuerpo de conocimiento coherente, no lecciones aisladas
\end{itemize}

\subsubsection{Assessment}

El assessment del Nivel Base enfatiza el reconocimiento respecto al recall:

\begin{itemize}[leftmargin=2cm]
    \item Dados escenarios, identificar qué vulnerabilidades psicológicas están siendo explotadas
    \item Dados ejemplos, clasificar las técnicas de manipulación por principio de influencia
    \item Ejercicios de reflexión sobre las experiencias personales con los fenómenos discutidos
    \item Ningún requisito de producir contenido técnico o comprometerse con documentación formal
\end{itemize}

\subsubsection{Duración y Formato}

El Nivel Base comprende cuatro sesiones de 90-120 minutos cada una, para un total de aproximadamente 8 horas de instrucción. El formato puede ser instrucción en aula, workshops o aprendizaje online self-paced. Cada sesión corresponde a un módulo pero incluye componentes interactivos y experienciales sustanciales.

%------------------------------------------------------------------------------
\subsection{Nivel Intermedio: Fundamento}
%------------------------------------------------------------------------------

\subsubsection{Target Audience}

El Nivel Intermedio sirve a estudiantes que han completado el Nivel Base (o exposición equivalente) y buscan una comprensión más profunda. El target primario son adolescentes más grandes (edad 16-19) que se preparan a la vida profesional, pero el nivel es apropiado para cualquier estudiante listo a comprometerse con material más complejo.

\subsubsection{Filosofía Educativa}

Al Nivel Intermedio, la filosofía educativa se desplaza de la ignición a la \textit{construcción de los fundamentos}. Los estudiantes desarrollan:

\begin{itemize}[leftmargin=2cm]
    \item Comprensión sistemática de las categorías de vulnerabilidad
    \item Capacidad de analizar incidentes del mundo real a través de la lente CPF
    \item Familiaridad con la Taxonomy como recurso de referencia
    \item Competencia inicial en aplicar frameworks a situaciones nuevas
    \item Conciencia de los caminos profesionales en la ciberseguridad psicológica
\end{itemize}

\subsubsection{Ejemplos Contextuales}

Los ejemplos del Nivel Intermedio se expanden para incluir contextos organizacionales y profesionales:

\begin{itemize}[leftmargin=2cm]
    \item \textbf{Escenarios workplace}: Situaciones del primer trabajo, contextos de pasantía, desafíos profesionales entry-level
    \item \textbf{Case studies}: Incidentes de security documentados analizados a través de lente psicológica
    \item \textbf{Dinámicas organizacionales}: Cómo las jerarquías workplace crean vulnerabilidades a la autoridad
    \item \textbf{Comunicación profesional}: Vectores de manipulación email, messaging y video calls
    \item \textbf{Implicaciones de carrera}: Cómo el conocimiento de ciberseguridad psicológica se aplica a varias profesiones
\end{itemize}

\subsubsection{Adaptaciones de los Módulos}

\textbf{Módulo 1 (No Decides Tú) al Nivel Intermedio:}

El fundamento teórico es profundizado:

\begin{itemize}[leftmargin=2cm]
    \item Los experimentos de Libet son explicados en detalle, incluyendo consideraciones metodológicas
    \item System 1/System 2 es conectado a sesgos cognitivos específicos (availability, anchoring, affect heuristic)
    \item Es introducida la hipótesis del marcador somático
    \item Las implicaciones para el decision-making de security son sistemáticamente exploradas
\end{itemize}

Los estudiantes se comprometen con fuentes primarias (extractos de \textit{Thinking, Fast and Slow} de Kahneman) y análisis secundario.

\textbf{Módulo 2 (Cómo Te Engañan) al Nivel Intermedio:}

El framework de influencia se convierte en herramienta analítica:

\begin{itemize}[leftmargin=2cm]
    \item Cada uno de los principios de Cialdini es estudiado en profundidad con evidencia experimental
    \item Los experimentos sobre autoridad de Milgram son examinados, incluyendo consideraciones éticas
    \item Incidentes de security reales (Business Email Compromise, campañas de phishing mayores) son analizados
    \item Estrategias defensivas son desarrolladas y criticadas
\end{itemize}

Los estudiantes practican el análisis de los incidentes usando las Categorías 1-3 de la Taxonomy como referencia.

\textbf{Módulo 3 (El Grupo Piensa Por Ti) al Nivel Intermedio:}

La teoría de las dinámicas de grupo es introducida propiamente:

\begin{itemize}[leftmargin=2cm]
    \item Las basic assumptions de Bion son enseñadas con ejemplos clínicos y organizacionales
    \item El modelo de groupthink de Janis es aplicado a los fallos de security
    \item Es introducido el concepto de sistemas de defensa social de Menzies Lyth
    \item Casos de estudio organizacionales demuestran vulnerabilidades a nivel de grupo
\end{itemize}

Los estudiantes analizan las dinámicas de equipo en contextos familiares (proyectos escolares, equipos deportivos, guildas de gaming) usando frameworks de dinámicas de grupo.

\textbf{Módulo 4 (Tú y las Máquinas) al Nivel Intermedio:}

La psicología AI es conectada a la literatura de investigación:

\begin{itemize}[leftmargin=2cm]
    \item Es revisada la investigación sobre antropomorfización
    \item Son examinados los estudios sobre automation bias, incluyendo consecuencias del mundo real
    \item Son discutidos los desafíos del human-AI teaming
    \item Son consideradas las capacidades AI emergentes y sus implicaciones psicológicas
\end{itemize}

Los estudiantes evalúan críticamente los sistemas AI que usan, aplicando frameworks de calibración de la confianza.

\subsubsection{Integración con la Documentación CPF}

Al Nivel Intermedio, la Taxonomy se convierte en un referente de trabajo:

\begin{itemize}[leftmargin=2cm]
    \item Los estudiantes son introducidos a la matriz completa 10×10
    \item Indicadores específicos son referenciados en el contenido del módulo
    \item Los ejercicios requieren localizar y aplicar indicadores de la Taxonomy
    \item La estructura de la Taxonomy (categorías, indicadores, attack vector mapping) es explicada
\end{itemize}

El paper Depth es mencionado como el fundamento teórico subyacente a la estructura de la Taxonomy. Los estudiantes comprenden que un grounding teórico más profundo está disponible pero no están obligados a comprometerse con él.

\subsubsection{Assessment}

El assessment del Nivel Intermedio incluye componentes analíticos:

\begin{itemize}[leftmargin=2cm]
    \item Análisis de incidentes: Dada una descripción de incidente de security, identificar las vulnerabilidades psicológicas explotadas usando la terminología de la Taxonomy
    \item Construcción de escenarios: Crear escenarios de ataque realistas que exploten categorías de vulnerabilidad especificadas
    \item Papers de reflexión: Analizar experiencias personales u observadas usando frameworks CPF
    \item Navegación de la Taxonomy: Demostrar capacidad de localizar indicadores relevantes para situaciones dadas
\end{itemize}

\subsubsection{Duración y Formato}

El Nivel Intermedio comprende ocho sesiones de 90-120 minutos cada una, para un total de aproximadamente 16 horas de instrucción. Se espera tiempo adicional de estudio autónomo (aproximadamente 8 horas) para revisión de la documentación y completamiento de los assignments. El formato puede incluir instrucción en aula, discusión seminarial o aprendizaje online estructurado con interacción entre pares.

%------------------------------------------------------------------------------
\subsection{Nivel Avanzado: Elaboración}
%------------------------------------------------------------------------------

\subsubsection{Target Audience}

El Nivel Avanzado sirve a estudiantes que persiguen carreras profesionales o académicas que involucrarán la ciberseguridad psicológica. El target primario son estudiantes universitarios en campos relevantes (ciberseguridad, psicología, organizational behavior, human-computer interaction) y profesionales al inicio de carrera. El completamiento del Nivel Intermedio (o competencia equivalente demostrada) es prerequisito.

\subsubsection{Filosofía Educativa}

Al Nivel Avanzado, la filosofía educativa enfatiza \textit{elaboración y aplicación}. Los estudiantes desarrollan:

\begin{itemize}[leftmargin=2cm]
    \item Comprensión profunda de los fundamentos teóricos a través de todas las categorías CPF
    \item Competencia en aplicar frameworks a situaciones organizacionales complejas
    \item Familiaridad con las metodologías de implementación (paper Dense)
    \item Introducción a los enfoques de intervención (Intervention Framework)
    \item Capacidad de contribuir al assessment de la security organizacional
\end{itemize}

\subsubsection{Ejemplos Contextuales}

Los ejemplos del Nivel Avanzado se comprometen con complejidad a escala profesional:

\begin{itemize}[leftmargin=2cm]
    \item \textbf{Advanced Persistent Threats}: Ataques multi-estadio que explotan vulnerabilidades psicológicas en el tiempo
    \item \textbf{Operaciones nation-state}: Cyber warfare con componentes psicológicos
    \item \textbf{Insider threats}: Dinámicas motivacionales y organizacionales complejas
    \item \textbf{Transformación organizacional}: Iniciativas de cambio de la security culture
    \item \textbf{Regulatory compliance}: Factores psicológicos en los programas de compliance
    \item \textbf{Incident response}: Dimensiones psicológicas de la gestión de las crisis
\end{itemize}

\subsubsection{Adaptaciones de los Módulos}

Al Nivel Avanzado, los módulos se expanden más allá del esqueleto de los cuatro módulos para comprender todas las diez categorías CPF. Los cuatro módulos originales se convierten en unidades extendidas que incorporan categorías correlacionadas:

\textbf{Unidad 1: Vulnerabilidades Cognitivas Individuales}
\begin{itemize}[leftmargin=2cm]
    \item El contenido del Módulo 1 se expande al tratamiento completo de las Categorías 5 (Cognitive Overload) y 7 (Stress Response)
    \item La Categoría 8 (Unconscious Processes) es introducida con fundamentos psicoanalíticos del paper Depth
    \item La investigación neurocientífica es revisada en profundidad
    \item Son discutidos los principios de diseño de las herramientas de assessment
\end{itemize}

\textbf{Unidad 2: Mecanismos de Influencia Social}
\begin{itemize}[leftmargin=2cm]
    \item El contenido del Módulo 2 se expande al tratamiento sistemático de las Categorías 1 (Authority), 2 (Temporal) y 3 (Social Influence)
    \item El set completo de indicadores es revisado con definiciones operativas
    \item El attack vector mapping es examinado en detalle
    \item Son introducidas las especificaciones del paper Dense para la detection logic
\end{itemize}

\textbf{Unidad 3: Dinámicas Colectivas}
\begin{itemize}[leftmargin=2cm]
    \item El contenido del Módulo 3 se expande al tratamiento completo de la Categoría 6 (Group Dynamics)
    \item Es agregada la Categoría 4 (Affective Vulnerabilities), incluyendo las relaciones objetales kleinianas
    \item Es estudiada la psicodinámica organizacional (Menzies Lyth, Hirschhorn)
    \item Son introducidos los principios del Intervention Framework para la intervención a nivel de grupo
\end{itemize}

\textbf{Unidad 4: Vulnerabilidades Emergentes}
\begin{itemize}[leftmargin=2cm]
    \item El contenido del Módulo 4 se expande al tratamiento completo de la Categoría 9 (AI-Specific Biases)
    \item La Categoría 10 (Critical Convergent States) es introducida con fundación de systems theory
    \item Es explicado el interdependency modeling (redes bayesianas)
    \item Son discutidos los desafíos de integración a través de las categorías
\end{itemize}

\subsubsection{Integración con la Documentación CPF}

Al Nivel Avanzado, se espera un engagement completo con la documentación CPF:

\textbf{La Taxonomy} es el referente primario, con todos los 100 indicadores estudiados.

\textbf{El Dense Implementation Companion} es introducido para la especificación operativa:
\begin{itemize}[leftmargin=2cm]
    \item El esquema OFTLISRV es explicado y aplicado
    \item La matemática de la detection logic (distancia de Mahalanobis, modelación temporal) es revisada
    \item Son discutidos los pathways de integración SOC
    \item Es examinada la metodología de validación
\end{itemize}

\textbf{El Intervention Framework} es introducido para la metodología de remediation:
\begin{itemize}[leftmargin=2cm]
    \item Son estudiados los principios de intervention design
    \item Son explicadas las dinámicas de resistencia
    \item Es revisada la integración de la change theory (Lewin, Schein, Kotter)
    \item Son discutidas las consideraciones de scaling
\end{itemize}

\textbf{El paper Depth} sirve como referente teórico durante todo el curso:
\begin{itemize}[leftmargin=2cm]
    \item El análisis del problema de integración proporciona contexto para la estructura del framework
    \item La sección sobre arquitectura de assessment informa la comprensión de los desafíos de medición
    \item La sección sobre interdependency modeling funda el enfoque de redes bayesianas
    \item La sección sobre el imperativo de validación encuadra las oportunidades de investigación
\end{itemize}

\subsubsection{Assessment}

El assessment del Nivel Avanzado requiere competencia demostrada con la documentación completa:

\begin{itemize}[leftmargin=2cm]
    \item \textbf{Análisis comprensivo de incidentes}: Análisis CPF completo de incidentes de security complejos usando todas las categorías y la documentación relevantes
    \item \textbf{Diseño de assessment}: Desarrollar herramientas de assessment para categorías de vulnerabilidad especificadas siguiendo el esquema OFTLISRV
    \item \textbf{Propuesta de intervención}: Diseñar un enfoque de intervención para vulnerabilidad organizacional usando la metodología del Intervention Framework
    \item \textbf{Propuesta de investigación}: Identificar oportunidades de validación y diseñar enfoque de estudio
    \item \textbf{Presentación}: Comunicar conceptos y análisis CPF a un público no especializado
\end{itemize}

\subsubsection{Duración y Formato}

El Nivel Avanzado comprende un curso semestral completo (aproximadamente 45 horas de instrucción) más estudio independiente sustancial (aproximadamente 90 horas) para revisión de la documentación, completamiento de los assignments y trabajo de proyecto. El formato típicamente combina lecciones, seminarios, discusiones de casos de estudio y aprendizaje basado en proyectos.

%------------------------------------------------------------------------------
\subsection{Nivel Especializado: Maestría}
%------------------------------------------------------------------------------

\subsubsection{Target Audience}

El Nivel Especializado sirve a profesionales de la security que aplicarán el CPF en contextos operativos. El target incluye analistas SOC, consultores de security, psicólogos organizacionales que trabajan en contextos de security e investigadores que contribuyen al desarrollo del framework. El completamiento del Nivel Avanzado (o competencia equivalente demostrada) es prerequisito.

\subsubsection{Filosofía Educativa}

Al Nivel Especializado, la filosofía educativa enfatiza \textit{maestría y contribución}. Los estudiantes desarrollan:

\begin{itemize}[leftmargin=2cm]
    \item Competencia operativa en el assessment e intervención CPF
    \item Capacidad de implementar detection logic en ambientes SOC
    \item Expertise en la metodología de assessment organizacional
    \item Capacidad de conducir programas de intervención
    \item Potencial de contribuir a la extensión y validación del framework
\end{itemize}

\subsubsection{Ejemplos Contextuales}

El Nivel Especializado trabaja con realidades operativas:

\begin{itemize}[leftmargin=2cm]
    \item \textbf{Integración SOC live}: Implementación de los indicadores CPF en operaciones de security reales
    \item \textbf{Assessment organizacional}: Conducción de assessments CPF completos en las organizaciones
    \item \textbf{Implementación de intervenciones}: Gestión de programas de cambio que abordan vulnerabilidades psicológicas
    \item \textbf{Ejecución de investigación}: Diseño y conducción de estudios de validación
    \item \textbf{Extensión del framework}: Desarrollo de nuevos indicadores o refinamiento de los existentes
\end{itemize}

\subsubsection{Estructura del Currículum}

El Nivel Especializado va más allá de la estructura a módulos hacia el desarrollo basado en competencias en tres tracks:

\textbf{Track A: Detection y Monitoring}
\begin{itemize}[leftmargin=2cm]
    \item Maestría completa del Dense Implementation Companion
    \item Implementación de detection logic en sistemas operativos
    \item Modelación de redes bayesianas para análisis de las interdependencias
    \item Ejecución de la metodología de validación
    \item Integración del workflow SOC
\end{itemize}

\textbf{Track B: Assessment y Consultoría}
\begin{itemize}[leftmargin=2cm]
    \item Maestría completa de la arquitectura de assessment
    \item Metodología de assessment organizacional
    \item Implementación de la protección de la privacidad
    \item Interpretación y comunicación de los resultados
    \item Desarrollo de las skills de consultoría
\end{itemize}

\textbf{Track C: Intervención y Cambio}
\begin{itemize}[leftmargin=2cm]
    \item Maestría completa del Intervention Framework
    \item Implementación del change management
    \item Skills de navegación de la resistencia
    \item Metodología de scaling
    \item Evaluación de los outcomes
\end{itemize}

Los especialistas pueden focalizarse en un track o desarrollar competencia a través de múltiples tracks.

\subsubsection{Integración con la Documentación CPF}

Al Nivel Especializado, toda la documentación es referente operativo:

\begin{itemize}[leftmargin=2cm]
    \item \textbf{Taxonomy}: Memorización completa de los indicadores; capacidad de aplicar sin referencia
    \item \textbf{Paper Dense}: Implementación operativa de todas las especificaciones
    \item \textbf{Intervention Framework}: Aplicación práctica de todos los principios de intervención
    \item \textbf{Paper Depth}: Recurso teórico para situaciones complejas y extensión del framework
\end{itemize}

\subsubsection{Assessment}

El assessment del Nivel Especializado es basado en competencias y práctico:

\begin{itemize}[leftmargin=2cm]
    \item \textbf{Track A}: Implementar detection logic funcional para indicadores especificados; demostrar integración SOC operativa
    \item \textbf{Track B}: Conducir assessment organizacional; entregar reporte y presentación de calidad profesional
    \item \textbf{Track C}: Diseñar e iniciar programa de intervención; documentar metodología y resultados iniciales
    \item \textbf{Todos los tracks}: Contribuir al desarrollo del framework a través de investigación de validación, refinamiento de los indicadores o extensión de la documentación
\end{itemize}

\subsubsection{Duración y Formato}

El Nivel Especializado es desarrollo profesional continuo en lugar de curso delimitado. La especialización inicial requiere aproximadamente 100-200 horas de desarrollo focalizado más experiencia práctica supervisada. El desarrollo continuo ocurre a través de práctica, engagement con la comunidad y contribución a la evolución del framework.

%==============================================================================
\section{Arquitectura de Integración}
%==============================================================================

El CPF Educational Framework está diseñado para integrarse con la documentación técnica CPF a través de exposición progresiva y engagement que se profundiza. Esta sección detalla cómo los cuatro papers---Taxonomy, Dense Implementation Companion, Intervention Framework y Depth---funcionan dentro de la estructura educativa.

\subsection{Funciones de los Documentos en el Viaje de Aprendizaje}

Cada paper CPF sirve una función pedagógica distinta:

\subsubsection{La Taxonomy: El Mapa}

La Taxonomy proporciona la enumeración comprensiva de las vulnerabilidades psicológicas---100 indicadores a través de 10 categorías. En el viaje educativo, funciona como:

\begin{itemize}[leftmargin=2cm]
    \item \textbf{Al Nivel Base}: Un landmark distante---los estudiantes saben que existe y representa el territorio completo
    \item \textbf{Al Nivel Intermedio}: Un referente de trabajo---los estudiantes navegan secciones específicas y localizan indicadores relevantes
    \item \textbf{Al Nivel Avanzado}: Un framework comprensivo---los estudiantes dominan la estructura completa y comprenden las relaciones entre categorías
    \item \textbf{Al Nivel Especializado}: Una herramienta operativa---los practitioners aplican automáticamente los indicadores y contribuyen al refinamiento
\end{itemize}

\subsubsection{El Dense Implementation Companion: El Manual Técnico}

El paper Dense traduce los indicadores conceptuales en especificaciones operativas---detection logic, telemetry sources, response protocols. Funciona como:

\begin{itemize}[leftmargin=2cm]
    \item \textbf{A los Niveles Base e Intermedio}: No directamente comprometido; mencionado como existente para aplicación avanzada
    \item \textbf{Al Nivel Avanzado}: Introducido y estudiado; los estudiantes comprenden el esquema OFTLISRV y los fundamentos matemáticos
    \item \textbf{Al Nivel Especializado}: Referente operativo; los practitioners implementan las especificaciones en ambientes reales
\end{itemize}

\subsubsection{El Intervention Framework: El Don del Retorno}

El Intervention Framework proporciona metodología para abordar las vulnerabilidades identificadas---intervention design, navegación de la resistencia, scaling. Funciona como:

\begin{itemize}[leftmargin=2cm]
    \item \textbf{A los Niveles Base e Intermedio}: No directamente comprometido; mencionado como existente para la remediation
    \item \textbf{Al Nivel Avanzado}: Introducido y estudiado; los estudiantes comprenden los principios de intervención y la integración de la change theory
    \item \textbf{Al Nivel Especializado}: Guía práctica; los practitioners diseñan e implementan programas de intervención
\end{itemize}

\subsubsection{El Paper Depth: El Mentor}

El paper Depth proporciona fundamentos teóricos---desafíos de integración, arquitectura de assessment, interdependency modeling. En la metáfora del viaje del héroe, funciona como el mentor que:

\begin{itemize}[leftmargin=2cm]
    \item Aparece cuando es necesaria una comprensión más profunda
    \item Explica por qué el mapa está dibujado como está
    \item Proporciona sabiduría que se profundiza en cada encuentro
    \item Permanece disponible durante todo el viaje para guía
\end{itemize}

Educativamente:

\begin{itemize}[leftmargin=2cm]
    \item \textbf{Al Nivel Base}: No directamente comprometido; representa la ``profundidad debajo'' que espera exploración
    \item \textbf{Al Nivel Intermedio}: Extraído; secciones específicas iluminan puntos teóricos
    \item \textbf{Al Nivel Avanzado}: Estudiado; los estudiantes se comprometen con los desafíos de integración y los compromisos teóricos
    \item \textbf{Al Nivel Especializado}: Recurso de referencia; los practitioners retornan cuando confrontan situaciones complejas
\end{itemize}

\subsection{Engagement Progresivo con la Documentación}

La siguiente tabla resume el engagement con la documentación a través de los niveles:

\begin{table}[H]
\centering
\caption{Engagement con la Documentación por Nivel}
\label{tab:doc_engagement}
\begin{tabular}{lcccc}
\toprule
Documento & Base & Intermedio & Avanzado & Especializado \\
\midrule
Taxonomy & Referencia & Uso de trabajo & Maestría completa & Operativo \\
Dense & Mención & Mención & Estudio & Implementación \\
Intervention & Mención & Mención & Estudio & Aplicación \\
Depth & Insinuación & Extraído & Estudio & Referencia \\
\bottomrule
\end{tabular}
\end{table}

\subsection{Arquitectura de los Cross-References}

Dentro de cada módulo en cada nivel, cross-references explícitos a la documentación crean caminos para exploración más profunda:

\textbf{Ejemplo: Módulo 2 (Cómo Te Engañan)}

\begin{itemize}[leftmargin=2cm]
    \item \textbf{Nivel Base}: ``La lista completa de las vulnerabilidades a la autoridad está en la Taxonomy CPF, Categoría 1. Cuando estés listo para ir más en profundidad, es allí donde encontrarás indicadores como `Authority gradient inhibiting security reporting' y `Executive exception normalization.' ''

    \item \textbf{Nivel Intermedio}: ``Revisa los indicadores 1.1 hasta 1.10 de la Taxonomy. Para cada indicador, identifica un ejemplo del mundo real de tu experiencia o investigación. Presta particular atención a cómo estos indicadores podrían aparecer en tu futuro workplace.''

    \item \textbf{Nivel Avanzado}: ``El Dense Implementation Companion especifica detection logic para las vulnerabilidades authority-based usando funciones de compliance rate y Bayesian legitimacy assessment. Revisa la sección 3.1 y diseña un enfoque de detection para el indicador 1.1 adaptado a un contexto organizacional específico.''

    \item \textbf{Nivel Especializado}: ``Implementa la especificación OFTLISRV para los indicadores 1.1-1.3 en tu ambiente SOC. Documenta telemetry sources, proceso de calibración de los thresholds y metodología de validación.''
\end{itemize}

\subsection{El Pattern de Referencia a la Tríada}

Durante todo el framework educativo, un pattern consistente referencia los tres documentos operativos como tríada:

\begin{quote}
``El CPF proporciona tres recursos integrados: la \textit{Taxonomy} te dice \textbf{qué} buscar, el \textit{Dense Implementation Companion} te dice \textbf{cómo} detectarlo, y el \textit{Intervention Framework} te dice \textbf{qué hacer al respecto}. Estos tres documentos forman un loop cerrado de la identificación a través de la detección a la remediation.''
\end{quote}

Este referente a la tríada aparece en cada nivel, con especificidad creciente:

\begin{itemize}[leftmargin=2cm]
    \item \textbf{Nivel Base}: La tríada es mencionada como el sistema completo que espera exploración
    \item \textbf{Nivel Intermedio}: La estructura de la tríada es explicada y la Taxonomy es activamente usada
    \item \textbf{Nivel Avanzado}: Todos los tres documentos son estudiados; la integración es comprendida
    \item \textbf{Nivel Especializado}: Todos los tres documentos son aplicados; la integración es practicada
\end{itemize}

El paper Depth se mantiene aparte de la tríada como fundamento teórico subyacente a todos los tres. Es el ``por qué'' detrás del ``qué,'' ``cómo'' y ``qué hacer.''

%==============================================================================
\section{Guía de Implementación}
%==============================================================================

Esta sección proporciona guía práctica para implementar el CPF Educational Framework a través de varios contextos educativos.

\subsection{Implementación en la Instrucción Secundaria}

\subsubsection{Integración Curricular}

El contenido del Nivel Base puede ser integrado en los currículos de la instrucción secundaria existentes a través de:

\begin{itemize}[leftmargin=2cm]
    \item \textbf{Computer Science / Digital Literacy}: Casa natural para los Módulos 2 y 4
    \item \textbf{Psicología / Social Studies}: Casa natural para los Módulos 1 y 3
    \item \textbf{Educación para la Salud}: Conexión a estrés, manipulación y bienestar
    \item \textbf{Unidad Standalone}: Intensivo de cuatro semanas dentro de cualquier curso relevante
\end{itemize}

\subsubsection{Preparación de los Maestros}

Los maestros que implementan el Nivel Base deberían:

\begin{itemize}[leftmargin=2cm]
    \item Completar al menos el Nivel Intermedio ellos mismos
    \item Comprender el contexto CPF más amplio incluso si no lo enseñan
    \item Tener acceso a la documentación para preguntas de los estudiantes que excedan el Nivel Base
    \item Conectarse con la comunidad CPF para soporte y actualizaciones
\end{itemize}

\subsubsection{Requisitos de Recursos}

La implementación del Nivel Base requiere:

\begin{itemize}[leftmargin=2cm]
    \item Acceso a Internet para demostraciones y ejemplos
    \item Capacidad de proyección para contenido visual
    \item Ningún software especializado o equipo de laboratorio
    \item Recomendado: Acceso a asistentes AI para demostraciones del Módulo 4
\end{itemize}

\subsection{Implementación en la Instrucción Superior}

\subsubsection{Posicionamiento del Curso}

El contenido del Nivel Avanzado puede ser implementado como:

\begin{itemize}[leftmargin=2cm]
    \item \textbf{Curso Dedicado}: ``Psychological Cybersecurity'' o ``Human Factors in Security''
    \item \textbf{Componente de Curso}: Módulo dentro de cursos más amplios de ciberseguridad, psicología organizacional o HCI
    \item \textbf{Seminario Graduate}: Engagement focalizado en la investigación con validación y extensión del framework
    \item \textbf{Certificado Profesional}: Continuing education para profesionales de la security
\end{itemize}

\subsubsection{Consideraciones sobre los Prerequisitos}

El Nivel Avanzado asume:

\begin{itemize}[leftmargin=2cm]
    \item Familiaridad básica con conceptos psicológicos (o inscripción concurrente a cursos de psicología)
    \item Comprensión fundamental de la information security (o inscripción concurrente)
    \item Literacy estadística suficiente para comprender la matemática de la detection logic
    \item Literacy de investigación suficiente para comprometerse con literatura académica
\end{itemize}

El Nivel Intermedio puede ser ofrecido como curso puente para estudiantes carentes de prerequisitos.

\subsubsection{Alineamiento del Assessment}

La implementación en la instrucción superior debería alinearse con los requisitos de assessment institucionales:

\begin{itemize}[leftmargin=2cm]
    \item Exámenes escritos pueden evaluar conocimiento teórico
    \item Análisis de casos de estudio puede evaluar competencia de aplicación
    \item Trabajo de proyecto puede evaluar integración y síntesis
    \item Propuestas de investigación pueden evaluar potencial de contribución
\end{itemize}

\subsection{Implementación en el Training Profesional}

\subsubsection{Deployment Organizacional}

Las organizaciones que implementan la educación CPF deberían considerar:

\begin{itemize}[leftmargin=2cm]
    \item \textbf{Amplitud vs. Profundidad}: Nivel Base para todos los empleados; Avanzado/Especializado para los equipos de security
    \item \textbf{Integración con Training Existente}: Los módulos CPF pueden suplementar o sustituir los programas de awareness convencionales
    \item \textbf{Integración del Assessment}: La educación CPF puede conectarse a los programas de assessment CPF organizacionales
    \item \textbf{Consideraciones Culturales}: Los conceptos CPF deberían alinearse con los valores organizacionales y el estilo de comunicación
\end{itemize}

\subsubsection{Desarrollo de los Especialistas}

Las organizaciones que desarrollan especialistas CPF internos deberían:

\begin{itemize}[leftmargin=2cm]
    \item Identificar candidatos con background apropiado (security + interés en la psicología)
    \item Proporcionar desarrollo estructurado a través de todos los cuatro niveles
    \item Soportar la aplicación práctica con proyectos de assessment organizacional
    \item Conectar los especialistas con la comunidad CPF más amplia
\end{itemize}

\subsection{Aprendizaje Autodirigido}

\subsubsection{Camino del Estudiante Individual}

Los estudiantes autodirigidos pueden progresar a través del framework usando:

\begin{itemize}[leftmargin=2cm]
    \item Este paper como guía del currículum
    \item La documentación CPF como recursos primarios
    \item Tutores AI (como Claude o similares) para aprendizaje interactivo
    \item Comunidades online para interacción entre pares
    \item Aplicación práctica en los contextos disponibles (security personal, observación en el workplace)
\end{itemize}

\subsubsection{Aprendizaje Asistido por AI}

Los large language models pueden servir como recursos educativos:

\begin{itemize}[leftmargin=2cm]
    \item Explicando conceptos a niveles de complejidad apropiados
    \item Generando escenarios de práctica para el análisis
    \item Proporcionando feedback sobre los intentos de análisis del estudiante
    \item Respondiendo a preguntas sobre el contenido de la documentación
    \item Adaptando ritmo y foco a las necesidades individuales del estudiante
\end{itemize}

Este modelo de aprendizaje asistido por AI se alinea con la filosofía educativa de que la educación formal proporciona ignición mientras el desarrollo subsiguiente ocurre a través de exploración autodirigida con las herramientas disponibles.

%==============================================================================
\section{Assessment y Progresión}
%==============================================================================

\subsection{Framework de las Competencias}

La progresión del estudiante es evaluada contra competencias organizadas por módulo y nivel:

\subsubsection{Competencias del Módulo 1}

\begin{itemize}[leftmargin=2cm]
    \item \textbf{Base}: Puede explicar que las decisiones ocurren parcialmente fuera de la conciencia consciente; puede identificar contextos de toma de decisiones de alto riesgo
    \item \textbf{Intermedio}: Puede describir la teoría del dual-process y aplicarla a escenarios de security; puede identificar sesgos cognitivos en ejemplos
    \item \textbf{Avanzado}: Puede analizar vulnerabilidades del decision-making usando el framework completo de las Categorías 5/7/8; puede diseñar enfoques de assessment
    \item \textbf{Especializado}: Puede implementar detection logic para vulnerabilidades cognitivas; puede conducir assessment organizacional
\end{itemize}

\subsubsection{Competencias del Módulo 2}

\begin{itemize}[leftmargin=2cm]
    \item \textbf{Base}: Puede reconocer técnicas de influencia básicas en ejemplos; puede identificar manipulación en las comunicaciones personales
    \item \textbf{Intermedio}: Puede analizar incidentes usando el framework de influencia completo; puede diseñar enfoques defensivos
    \item \textbf{Avanzado}: Puede aplicar sistemáticamente los indicadores de las Categorías 1/2/3; puede diseñar metodologías de detection
    \item \textbf{Especializado}: Puede implementar detection de la influencia social en sistemas operativos; puede conducir assessment de la vulnerabilidad organizacional
\end{itemize}

\subsubsection{Competencias del Módulo 3}

\begin{itemize}[leftmargin=2cm]
    \item \textbf{Base}: Puede reconocer dinámicas de grupo básicas en contextos familiares; puede identificar presión a la conformidad
    \item \textbf{Intermedio}: Puede analizar dinámicas de equipo usando frameworks de Bion y groupthink; puede identificar patrones organizacionales
    \item \textbf{Avanzado}: Puede aplicar el framework completo de la Categoría 6; puede diseñar intervenciones a nivel de grupo
    \item \textbf{Especializado}: Puede evaluar dinámicas de grupo organizacionales; puede implementar programas de intervención
\end{itemize}

\subsubsection{Competencias del Módulo 4}

\begin{itemize}[leftmargin=2cm]
    \item \textbf{Base}: Puede reconocer la antropomorfización en sí y en otros; puede calibrar apropiadamente la confianza en el AI
    \item \textbf{Intermedio}: Puede analizar patrones de interacción humano-AI; puede identificar riesgos de automation bias
    \item \textbf{Avanzado}: Puede aplicar el framework completo de la Categoría 9; puede diseñar protocolos de interacción AI
    \item \textbf{Especializado}: Puede evaluar dinámicas de equipos humano-AI; puede implementar operaciones de security AI-aware
\end{itemize}

\subsection{Criterios de Progresión}

\subsubsection{De Base a Intermedio}

La progresión requiere demostración de:

\begin{itemize}[leftmargin=2cm]
    \item Competencia de reconocimiento a través de todos los cuatro módulos
    \item Curiosidad de engagement (deseo de aprender más)
    \item Maestría del vocabulario básico
    \item Ningún assessment formal requerido; auto-progresión aceptable
\end{itemize}

\subsubsection{De Intermedio a Avanzado}

La progresión requiere demostración de:

\begin{itemize}[leftmargin=2cm]
    \item Competencia analítica a través de todos los cuatro módulos
    \item Familiaridad con la Taxonomy (puede navegar y aplicar)
    \item Capacidad de análisis de los incidentes
    \item Recomendado: Assessment formal o revisión del portfolio
\end{itemize}

\subsubsection{De Avanzado a Especializado}

La progresión requiere demostración de:

\begin{itemize}[leftmargin=2cm]
    \item Maestría comprensiva del framework
    \item Fluencia en la documentación (puede trabajar con todos los cuatro papers)
    \item Experiencia de aplicación práctica
    \item Requerido: Assessment práctico supervisado o credencial profesional
\end{itemize}

\subsection{Desarrollo Continuo}

El CPF Educational Framework no termina al Nivel Especializado. El desarrollo continuo incluye:

\begin{itemize}[leftmargin=2cm]
    \item \textbf{Refinamiento de la práctica}: Mejorar la aplicación a través de la experiencia
    \item \textbf{Contribución al framework}: Extender la validación, refinar los indicadores, desarrollar aplicaciones
    \item \textbf{Engagement con la comunidad}: Compartir conocimiento, hacer mentoring a practitioners en desarrollo
    \item \textbf{Adaptación a la evolución}: Actualizar el conocimiento a medida que threat landscape y framework evolucionan
\end{itemize}

%==============================================================================
\section{Conclusión: La Educación como Viaje Continuo}
%==============================================================================

\subsection{Síntesis del Framework}

El CPF Educational Framework proporciona un enfoque estructurado al desarrollo de la literacy en ciberseguridad psicológica a través de todo el espectro de la conciencia inicial a la maestría profesional. Sus características clave incluyen:

\begin{itemize}[leftmargin=2cm]
    \item \textbf{Esqueleto universal}: Cuatro módulos que abordan dominios fundamentales de vulnerabilidad, aplicables a todos los niveles
    \item \textbf{Modulación contextual}: Adaptación de complejidad, ejemplos y engagement con la documentación al desarrollo del estudiante
    \item \textbf{Integración progresiva}: Incorporación sistemática de la documentación técnica CPF a medida que los estudiantes avanzan
    \item \textbf{Filosofía de la ignición}: Educación como chispa para el desarrollo autodirigido continuo en lugar de credencial completada
\end{itemize}

\subsection{El Viaje Continuo}

La metáfora del viaje del héroe permanece adecuada para describir la relación del estudiante con la educación CPF. No hay destinación final. El viaje continúa porque:

\begin{itemize}[leftmargin=2cm]
    \item \textbf{La vulnerabilidad psicológica es permanente}: A diferencia de las vulnerabilidades técnicas que pueden ser parcheadas, la arquitectura cognitiva humana permanece explotable
    \item \textbf{El threat landscape evoluciona}: Los atacantes desarrollan técnicas nuevas que explotan vulnerabilidades duraderas en modos nuevos
    \item \textbf{La comprensión se profundiza}: Cada retorno a los conceptos fundamentales revela nuevas implicaciones y aplicaciones
    \item \textbf{El framework se desarrolla}: El CPF mismo evoluciona a través de validación, refinamiento y extensión
\end{itemize}

El practitioner educado no es uno que ha ``completado'' el training CPF sino uno que ha interiorizado sus patrones de pensamiento, que ve vulnerabilidades psicológicas donde otros ven solo sistemas técnicos, que reconoce en sí mismo los mismos mecanismos que identifica en las organizaciones.

\subsection{La Visión Más Amplia}

El CPF Educational Framework sirve una visión más amplia del desarrollo profesional individual. Si la literacy en ciberseguridad psicológica se vuelve difusa---si los patrones enseñados en estos módulos se convierten en conocimiento común---el landscape de la security cambia fundamentalmente.

Consideren un mundo donde cada empleado reconoce la manipulación de la autoridad cuando la encuentra, donde cada equipo comprende cómo las dinámicas de grupo crean blind spots, donde cada organización diseña sistemas teniendo en cuenta las limitaciones cognitivas, donde cada interacción AI ocurre con apropiada calibración de la confianza. Este no es un mundo sin incidentes de security. La vulnerabilidad humana es permanente. Pero es un mundo donde la explotación es más difícil, donde las defensas están informadas por modelos precisos de la psicología humana, donde el fallo persistente de la security awareness a nivel consciente ha sido sustituido por una educación que involucra los mecanismos reales del decision-making humano.

El CPF Educational Framework es una contribución hacia ese mundo. El viaje inicia con el reconocimiento de que ``no decides tú''---que el sí mismo que lee estas palabras es menos autónomo de lo que la intuición sugiere. Continúa a través de la comprensión de cómo esta autonomía limitada es explotada, cómo los grupos amplifican las vulnerabilidades individuales, cómo los sistemas artificiales introducen complicaciones nuevas. No termina nunca, porque el territorio que mapea es el paisaje permanente de la cognición humana.

La profundidad debajo espera exploración. El viaje continúa.

%==============================================================================
\section*{Nota sobre la Composición Asistida por AI}
%==============================================================================

Este manuscrito presenta el framework educativo original y las contribuciones intelectuales del autor. En el proceso de composición, el autor ha utilizado un large language model como herramienta auxiliar para el refinamiento estilístico y la consistencia del formateo. Las ideas core, la arquitectura educativa, la metodología de integración y el análisis pedagógico son exclusivamente producto de la expertise del autor. El autor es enteramente responsable de la exactitud y de la integridad del contenido publicado.

%==============================================================================
\section*{Agradecimientos}
%==============================================================================

El autor reconoce el trabajo fundamental en la educación en ciberseguridad, en la investigación psicológica y en el desarrollo organizacional sobre el cual este framework educativo se construye. Un reconocimiento especial va a los investigadores cuyos aportes teóricos---Kahneman, Cialdini, Bion, Klein, Milgram y muchos otros---hacen posible esta integración.

%==============================================================================
\begin{thebibliography}{99}

\bibitem{bion1961}
Bion, W. R. (1961). \textit{Experiences in groups}. London: Tavistock Publications.

\bibitem{campbell1949}
Campbell, J. (1949). \textit{The hero with a thousand faces}. New York: Pantheon Books.

\bibitem{cialdini2007}
Cialdini, R. B. (2007). \textit{Influence: The psychology of persuasion}. New York: Collins.

\bibitem{damasio1994}
Damasio, A. (1994). \textit{Descartes' error: Emotion, reason, and the human brain}. New York: Putnam.

\bibitem{dietvorst2015}
Dietvorst, B. J., Simmons, J. P., \& Massey, C. (2015). Algorithm aversion: People erroneously avoid algorithms after seeing them err. \textit{Journal of Experimental Psychology: General}, 144(1), 114-126.

\bibitem{epley2007}
Epley, N., Waytz, A., \& Cacioppo, J. T. (2007). On seeing human: A three-factor theory of anthropomorphism. \textit{Psychological Review}, 114(4), 864-886.

\bibitem{hirschhorn1988}
Hirschhorn, L. (1988). \textit{The workplace within: Psychodynamics of organizational life}. Cambridge, MA: MIT Press.

\bibitem{janis1982}
Janis, I. L. (1982). \textit{Groupthink: Psychological studies of policy decisions and fiascoes}. Boston: Houghton Mifflin.

\bibitem{kahneman2011}
Kahneman, D. (2011). \textit{Thinking, fast and slow}. New York: Farrar, Straus and Giroux.

\bibitem{klein1946}
Klein, M. (1946). Notes on some schizoid mechanisms. \textit{International Journal of Psychoanalysis}, 27, 99-110.

\bibitem{kotter1996}
Kotter, J. P. (1996). \textit{Leading change}. Boston: Harvard Business School Press.

\bibitem{lewin1947}
Lewin, K. (1947). Frontiers in group dynamics: Concept, method and reality in social science. \textit{Human Relations}, 1(1), 5-41.

\bibitem{libet1983}
Libet, B., Gleason, C. A., Wright, E. W., \& Pearl, D. K. (1983). Time of conscious intention to act in relation to onset of cerebral activity. \textit{Brain}, 106(3), 623-642.

\bibitem{menzies1960}
Menzies Lyth, I. (1960). A case-study in the functioning of social systems as a defence against anxiety. \textit{Human Relations}, 13, 95-121.

\bibitem{milgram1974}
Milgram, S. (1974). \textit{Obedience to authority}. New York: Harper \& Row.

\bibitem{parasuraman1997}
Parasuraman, R., \& Riley, V. (1997). Humans and automation: Use, misuse, disuse, abuse. \textit{Human Factors}, 39(2), 230-253.

\bibitem{sans2023}
SANS Institute. (2023). \textit{Security Awareness Report 2023}. SANS Security Awareness.

\bibitem{schein2010}
Schein, E. H. (2010). \textit{Organizational culture and leadership} (4th ed.). San Francisco: Jossey-Bass.

\bibitem{soon2008}
Soon, C. S., Brass, M., Heinze, H. J., \& Haynes, J. D. (2008). Unconscious determinants of free decisions in the human brain. \textit{Nature Neuroscience}, 11(5), 543-545.

\bibitem{verizon2023}
Verizon. (2023). \textit{2023 Data Breach Investigations Report}. Verizon Enterprise.

\bibitem{winnicott1971}
Winnicott, D. W. (1971). \textit{Playing and reality}. London: Tavistock Publications.

\end{thebibliography}

\end{document}
