\documentclass[11pt,a4paper]{article}

\usepackage[utf8]{inputenc}
\usepackage[english]{babel}
\usepackage{amsmath}
\usepackage{amsfonts}
\usepackage{amssymb}
\usepackage{graphicx}
\usepackage{booktabs}
\usepackage{url}
\usepackage{hyperref}
\usepackage[margin=1in]{geometry}
\usepackage{fancyhdr}
\usepackage{lastpage}
\usepackage{float}
\usepackage{placeins}
\usepackage{enumitem}
\usepackage{longtable}
\usepackage{array}
\usepackage{tabularx}

\setlength{\parindent}{0pt}
\setlength{\parskip}{0.6em}

\hypersetup{
    colorlinks=true,
    linkcolor=blue,
    citecolor=blue,
    urlcolor=blue,
    pdftitle={El CPF Educational Framework: Un Currículum Universal para la Literacy en Ciberseguridad Psicológica},
    pdfauthor={Giuseppe Canale},
}

\pagestyle{fancy}
\fancyhf{}
\renewcommand{\headrulewidth}{0pt}
\fancyfoot[C]{\thepage}

\begin{document}

\thispagestyle{empty}
\begin{center}

\vspace*{0.5cm}

\rule{\textwidth}{1.5pt}

\vspace{0.5cm}

{\LARGE \textbf{El CPF Educational Framework:}}\\[0.3cm]
{\LARGE \textbf{Un Currículum Universal para la}}\\[0.3cm]
{\LARGE \textbf{Literacy en Ciberseguridad Psicológica}}

\vspace{0.5cm}

\rule{\textwidth}{1.5pt}

\vspace{0.3cm}

{\large \textsc{Companion Educativo al Cybersecurity Psychology Framework}}

\vspace{0.5cm}

{\Large Giuseppe Canale, CISSP}\\[0.2cm]
Investigador Independiente\\[0.1cm]
\href{mailto:g.canale@cpf3.org}{g.canale@cpf3.org}\\[0.1cm]
URL: \href{https://cpf3.org}{cpf3.org}\\[0.1cm]
ORCID: \href{https://orcid.org/0009-0007-3263-6897}{0009-0007-3263-6897}

\vspace{0.8cm}

{\large \today}

\vspace{1cm}

\end{center}

\begin{abstract}
\noindent
El Cybersecurity Psychology Framework (CPF) proporciona una rigurosa fundación teórica y operativa para comprender las vulnerabilidades humanas en los contextos de security. Sin embargo, la teoría sin pedagogía permanece inaccesible; los frameworks sin caminos educativos se convierten en artefactos en lugar de herramientas de cambio. Este paper presenta el CPF Educational Framework, un currículum estructurado diseñado para introducir, desarrollar y especializar a los estudiantes a través de todo el espectro de la literacy en ciberseguridad psicológica. A diferencia de los programas tradicionales de security awareness que asumen actores racionales modificables a través de la transferencia de informaciones, este enfoque educativo reconoce que las decisiones de security ocurren sustancialmente debajo de la conciencia consciente y que una educación eficaz debe involucrar los procesos pre-cognitivos, las dinámicas de grupo y la compleja interacción entre inteligencia humana y artificial. El framework comprende cuatro módulos universales---``No Decides Tú,'' ``Cómo Te Engañan,'' ``El Grupo Piensa Por Ti,'' y ``Tú y las Máquinas''---que forman un esqueleto conceptual invariante. Este esqueleto es luego modulado a través de cuatro niveles de desarrollo (Base, Intermedio, Avanzado, Especializado), cada uno calibrado sobre la complejidad apropiada, sobre los ejemplos contextuales y sobre la integración con la documentación técnica del CPF. El currículum posiciona los papers fundamentales del CPF como waypoints progresivos: la Taxonomy como mapa de referencia, el Dense Implementation Companion como especificación operativa, el Intervention Framework como metodología de remediation, y el paper Depth como mentor teórico que acompaña a los estudiantes durante todo su viaje. Esta arquitectura educativa habilita tanto iniciativas de literacy a gran escala como desarrollo profesional especializado, manteniendo la coherencia con el framework científico subyacente.

\vspace{0.5em}
\noindent\textbf{Keywords:} educación en ciberseguridad, literacy psicológica, curriculum design, factores humanos, procesos pre-cognitivos, security awareness, lifelong learning
\end{abstract}

\newpage
\tableofcontents
\newpage

%==============================================================================
\section{Introducción: El Imperativo Pedagógico}
%==============================================================================

\subsection{El Fracaso de la Educación Tradicional en Security}

La inversión global en training de cybersecurity awareness supera los \$5 mil millones anuales, sin embargo las métricas fundamentales de los incidentes de security relacionados con el factor humano no muestran ninguna mejora correspondiente \cite{verizon2023, sans2023}. Este fracaso persistente requiere una explicación. El Cybersecurity Psychology Framework ofrece una: la educación tradicional en security opera sobre un modelo fundamentalmente erróneo de la cognición y del comportamiento humano.

El paradigma educativo prevalente asume que los seres humanos son actores racionales que, cuando son informados sobre los riesgos y las consecuencias, modificarán su comportamiento en consecuencia. Esta asunción contradice décadas de investigación en neurociencias, economía del comportamiento y teoría psicoanalítica. Los experimentos fundamentales de Benjamin Libet han demostrado que las decisiones motoras ocurren 300-500 milisegundos antes de la conciencia consciente \cite{libet1983}. La teoría del dual-process de Daniel Kahneman revela que el System 1 (rápido, automático, emocional) domina el System 2 (lento, deliberado, racional) precisamente en los ambientes presionados por el tiempo y cognitivamente sobrecargados donde ocurren las decisiones de security \cite{kahneman2011}. La investigación sobre dinámicas de grupo de Wilfred Bion muestra que el comportamiento colectivo emerge de basic assumptions inconscientes que operan enteramente debajo de la conciencia consciente \cite{bion1961}.

Si las decisiones de security se toman antes de la conciencia consciente, si los procesos automáticos dominan los deliberados, si las dinámicas de grupo moldean el comportamiento individual a través de canales inconscientes---entonces la educación que apunta solo a los procesos conscientes, racionales e individuales fallará necesariamente. La pregunta no es si la educación tradicional en security está implementada mal, sino si sus asunciones fundamentales están equivocadas.

\subsection{Una Filosofía Educativa Diferente}

El CPF Educational Framework procede de asunciones diferentes. Asumimos que:

\begin{itemize}[leftmargin=2cm]
    \item \textbf{Los procesos pre-cognitivos determinan sustancialmente el comportamiento de security.} La educación debe por lo tanto involucrar estos procesos, no simplemente informar la conciencia consciente.

    \item \textbf{El aprendizaje no es transferencia de informaciones sino desarrollo del reconocimiento de patrones.} El objetivo no es llenar a los estudiantes de hechos sino desarrollar su capacidad de reconocer patrones de vulnerabilidad en sí mismos, en otros y en las organizaciones.

    \item \textbf{La educación es ignición, no completamiento.} En un dominio caracterizado por constante evolución y variación individual, la educación formal proporciona la chispa inicial; el desarrollo subsiguiente ocurre a través de la exploración autodirigida con las herramientas disponibles (incluyendo tutores AI, recursos de la comunidad y retorno a las estructuras formales cuando sea necesario).

    \item \textbf{El mismo esqueleto conceptual sirve a todos los estudiantes.} Lo que varía no son los insights fundamentales sino su aplicación contextual, la complejidad de los ejemplos y la profundidad del grounding teórico.

    \item \textbf{La vulnerabilidad psicológica es permanente y pervasiva.} A diferencia de las vulnerabilidades técnicas que pueden ser parcheadas, las vulnerabilidades psicológicas son intrínsecas a la cognición humana. La educación apunta no a la eliminación sino a la conciencia, al reconocimiento y al acomodamiento estratégico.
\end{itemize}

Estas asunciones producen un framework educativo fundamentalmente diferente de la security awareness tradicional. No enseñamos reglas a seguir sino patrones a reconocer. No asumimos que los estudiantes cambiarán su naturaleza sino que puedan comprenderla. No posicionamos la educación como una credencial completada sino como un viaje iniciado.

\subsection{El Viaje del Héroe: Una Metáfora Organizativa}

El monomito de Joseph Campbell---el viaje del héroe---proporciona una metáfora organizativa útil para la experiencia educativa del CPF \cite{campbell1949}. El estudiante inicia en el mundo ordinario de la confianza ingenua en la propia racionalidad y autonomía. La llamada a la aventura llega a través del reconocimiento de que ``no decides tú''---que los procesos pre-cognitivos moldean sustancialmente el comportamiento. El cruce del umbral ocurre cuando este reconocimiento se vuelve personal, cuando el estudiante ve estos patrones operar en la propia experiencia.

El viaje a través del mundo especial involucra un engagement progresivamente más profundo con los mecanismos de la vulnerabilidad: influencia social, dinámicas de grupo, respuestas al estrés, procesos inconscientes. Cada estadio revela nuevos aspectos de cómo la psicología humana crea patrones explotables. El estudiante encuentra aliados (compañeros de viaje, recursos educativos, tutores AI) y enemigos (sesgos cognitivos, resistencia defensiva, la atracción de las ilusiones confortables).

En esta metáfora, la documentación técnica del CPF sirve funciones narrativas específicas:

\begin{itemize}[leftmargin=2cm]
    \item \textbf{La Taxonomy} es el mapa del mundo especial---la enumeración sistemática de los territorios a explorar, de los peligros a reconocer, de los patrones a comprender.

    \item \textbf{El Dense Implementation Companion} sirve como manual técnico---las especificaciones operativas que traducen la comprensión conceptual en detection y response accionables.

    \item \textbf{El Intervention Framework} representa el don del retorno---la metodología que transforma la comprensión personal en capacidad de cambio organizacional.

    \item \textbf{El paper Depth} funciona como la figura del mentor que aparece durante todo el viaje, proporcionando grounding teórico cuando sea necesario, explicando por qué el mapa está dibujado como está, ofreciendo sabiduría que se profundiza en cada nuevo encuentro.
\end{itemize}

El viaje del héroe no termina. El retorno al mundo ordinario encuentra al estudiante transformado, viendo patrones previamente invisibles, reconociendo vulnerabilidades en sí y en el ambiente, equipado con frameworks para el desarrollo continuo. Pero el viaje continúa porque la vulnerabilidad psicológica continúa, porque el threat landscape evoluciona, porque la comprensión se profundiza con la experiencia.

\subsection{Estructura del Documento}

Este paper procede como sigue. La Sección 2 presenta el Marco Universal: los cuatro módulos que constituyen el esqueleto conceptual invariante aplicable a todos los niveles de desarrollo. La Sección 3 detalla la Modulación Contextual: cómo cada módulo se adapta a los niveles Base, Intermedio, Avanzado y Especializado manteniendo la integridad conceptual. La Sección 4 aborda la Arquitectura de Integración: cómo el framework educativo se conecta e incorpora progresivamente la documentación técnica del CPF. La Sección 5 proporciona una Guía de Implementación: consideraciones prácticas para el deployment de este currículum a través de los contextos educativos. La Sección 6 discute Assessment y Progresión: cómo se evalúa el desarrollo del estudiante y cómo se gestionan las transiciones entre niveles. La Sección 7 concluye con reflexiones sobre el futuro de la educación en ciberseguridad psicológica.

%==============================================================================
\section{El Marco Universal: Cuatro Módulos}
%==============================================================================

El esqueleto conceptual de la educación CPF comprende cuatro módulos, cada uno abordando un dominio fundamental de vulnerabilidad psicológica. Estos módulos son universales en el sentido de que sus insights core se aplican a todas las edades, contextos y niveles de desarrollo. Lo que varía no es el insight sino su elaboración, ejemplificación y profundidad teórica.

Los cuatro módulos son:

\begin{enumerate}[leftmargin=2cm]
    \item \textbf{No Decides Tú} --- Las neurociencias y la psicología del decision-making pre-consciente
    \item \textbf{Cómo Te Engañan} --- Los mecanismos de la influencia social y de la manipulación
    \item \textbf{El Grupo Piensa Por Ti} --- Las dinámicas colectivas y sus implicaciones para la security
    \item \textbf{Tú y las Máquinas} --- Las vulnerabilidades de la interacción humano-AI
\end{enumerate}

Cada módulo está diseñado para funcionar tanto independientemente como parte de la secuencia integrada. La secuencia cuenta: el Módulo 1 establece el reconocimiento fundamental de que el control consciente es más limitado de lo que la intuición sugiere; el Módulo 2 aplica este reconocimiento a la influencia interpersonal; el Módulo 3 se extiende a los fenómenos colectivos; el Módulo 4 introduce las complicaciones nuevas de los sistemas artificiales. Sin embargo, cualquier módulo puede servir como punto de entrada para estudiantes con intereses o necesidades específicas.

\subsection{Módulo 1: No Decides Tú}

\subsubsection{Insight Core}

El insight core del Módulo 1 es que las decisiones humanas ocurren a través de procesos sustancialmente fuera de la conciencia consciente, y que estos procesos pre-conscientes son tanto explotables como ampliamente no modificables a través del solo esfuerzo consciente.

Este insight contradice intuiciones profundas sobre autonomía y autocontrol. La mayoría de las personas experimentan sus propias decisiones como productos de la deliberación consciente---``piensan sobre ello'' y luego ``deciden.'' La evidencia neurocientífica y psicológica sugiere que esta experiencia es parcialmente ilusoria: la decisión a menudo ya ha sido tomada por procesos pre-conscientes, y la deliberación consciente es una narrativa post-hoc que acompaña en lugar de causar la decisión \cite{libet1983, soon2008}.

\subsubsection{Fundamentos Teóricos}

El Módulo 1 extrae de tres tradiciones teóricas primarias:

\textbf{Neurociencias del Decision-Making.}
\begin{itemize}[leftmargin=2cm]
    \item Los experimentos de Libet han demostrado que el potencial de preparación del cerebro---actividad eléctrica que indica preparación motora---precede la conciencia consciente de la intención de moverse por aproximadamente 350 milisegundos \cite{libet1983}
    \item Soon et al. han extendido este resultado, mostrando que los patrones de actividad cerebral podían predecir las decisiones hasta 10 segundos antes de la conciencia consciente \cite{soon2008}
    \item Estos resultados sugieren que la conciencia consciente de la decisión es efecto en lugar de causa
\end{itemize}

\textbf{Teoría del Dual-Process.}
\begin{itemize}[leftmargin=2cm]
    \item El framework System 1/System 2 de Kahneman proporciona un modelo accesible para comprender la relación entre elaboración automática y deliberada \cite{kahneman2011}
    \item El System 1 opera automáticamente, rápidamente, con poco sentido de control voluntario
    \item El System 2 asigna atención a las actividades mentales effortful, incluyendo los cálculos complejos
    \item Crucialmente, el System 2 a menudo sirve como racionalizador post-hoc de las conclusiones del System 1 en lugar de como evaluador independiente
\end{itemize}

\textbf{Hipótesis del Marcador Somático.}
\begin{itemize}[leftmargin=2cm]
    \item La investigación de Damasio demuestra que las emociones y los estados corporales influyen sustancialmente el decision-making a través de mecanismos que evitan la deliberación consciente \cite{damasio1994}
    \item El ``gut feeling'' no es metafórico sino que refleja estados somáticos reales que guían la elección a través de canales pre-conscientes
\end{itemize}

\subsubsection{Implicaciones para la Security}

Las implicaciones para la security del control consciente limitado son profundas:

\begin{itemize}[leftmargin=2cm]
    \item Las decisiones de security tomadas bajo presión temporal, carga cognitiva o activación emocional son dominadas por procesos pre-conscientes que podrían no alinearse con los intereses de security.

    \item El training que apunta solo al conocimiento consciente (``recuerda verificar la dirección del remitente'') podría fallar en influir el comportamiento real cuando los procesos pre-conscientes apuntan diferentemente.

    \item Los atacantes que pueden activar estados emocionales específicos o cargas cognitivas pueden predeciblemente desplazar el decision-making hacia patrones explotables.

    \item El auto-assessment de la vulnerabilidad es no confiable porque los procesos que crean vulnerabilidad operan debajo del umbral del acceso consciente.
\end{itemize}

\subsubsection{Objetivos de Aprendizaje del Módulo}

Completando el Módulo 1, los estudiantes serán capaces de:

\begin{enumerate}[leftmargin=2cm]
    \item Explicar la evidencia del decision-making pre-consciente y sus implicaciones para el comportamiento de security.

    \item Identificar situaciones en las que las propias decisiones son probablemente dominadas por la elaboración del System 1.

    \item Reconocer las condiciones (presión temporal, carga cognitiva, activación emocional) que desplazan el decision-making lejos del control deliberado.

    \item Articular por qué el training tradicional de security awareness tiene eficacia limitada.

    \item Describir la relación entre este módulo y las Categorías CPF 5 (Cognitive Overload), 7 (Stress Response) y 8 (Unconscious Processes).
\end{enumerate}

\subsubsection{Conexión a la Documentación CPF}

El Módulo 1 introduce conceptos que son sistemáticamente desarrollados en la Taxonomy CPF y teóricamente fundados en el paper Depth. Específicamente:

\begin{itemize}[leftmargin=2cm]
    \item La Categoría 5 de la Taxonomy (Cognitive Overload Vulnerabilities) operacionaliza las dinámicas System 1/System 2 en indicadores medibles.

    \item La Categoría 7 de la Taxonomy (Stress Response Vulnerabilities) mapea la respuesta neurobiológica al estrés sobre los comportamientos security-relevant.

    \item La Categoría 8 de la Taxonomy (Unconscious Process Vulnerabilities) extiende la fundación neurocientífica al territorio psicoanalítico.

    \item La sección del paper Depth sobre ``The Integration Problem'' explica cómo estas dispares tradiciones teóricas son reconciliadas dentro del framework CPF.
\end{itemize}

Los estudiantes en el nivel Base reciben estas conexiones como referencias hacia adelante---invitaciones a la exploración futura. Los estudiantes en los niveles Avanzado y Especializado se comprometen directamente con el material referenciado.
