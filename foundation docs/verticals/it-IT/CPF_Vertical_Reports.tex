\documentclass[11pt,a4paper]{article}
\usepackage[utf8]{inputenc}
\usepackage[T1]{fontenc}
\usepackage{geometry}
\usepackage{graphicx}
\usepackage{xcolor}
\usepackage{titlesec}
\usepackage{fancyhdr}
\usepackage{hyperref}
\usepackage{setspace}
\usepackage{tcolorbox}
\usepackage{enumitem}

% Page geometry
\geometry{
    a4paper,
    top=2.5cm,
    bottom=2.5cm,
    left=2.5cm,
    right=2.5cm
}

% Colors matching the pitch
\definecolor{darkblue}{RGB}{13,47,70}
\definecolor{midblue}{RGB}{28,78,112}
\definecolor{accentorange}{RGB}{255,165,67}

% Hyperlinks
\hypersetup{
    colorlinks=true,
    linkcolor=darkblue,
    urlcolor=midblue,
    citecolor=darkblue
}

% Section formatting
\titleformat{\section}
{\color{darkblue}\Large\bfseries}
{\thesection}{1em}{}

\titleformat{\subsection}
{\color{midblue}\large\bfseries}
{\thesubsection}{1em}{}

% Header and footer
\pagestyle{fancy}
\fancyhf{}
\fancyhead[L]{\small\textit{CPF3 - Analisi Verticali di Settore}}
\fancyhead[R]{\small\thepage}
\renewcommand{\headrulewidth}{0.4pt}

\begin{document}

% Title page
\begin{center}
\vspace*{2cm}

{\Huge\bfseries\color{darkblue} ANALISI VERTICALI\\[0.3cm] DI SETTORE\par}
\vspace{0.5cm}
{\Large\color{midblue} Technical Report Personalizzati CPF3\par}

\vspace{3cm}

{\large\color{darkblue}
\textit{Ogni settore presenta vulnerabilità psicologiche uniche.}\\[0.3cm]
\textit{Ogni organizzazione merita un'analisi su misura.}\\[0.3cm]
\textit{Ogni investitore riceve commitment concreti.}
\par}

\vfill

{\large
\textbf{The Cybersecurity Psychology Framework}\\[0.3cm]
Giuseppe Canale, CISSP\\[0.2cm]
\texttt{g.canale@cpf3.org}\\[0.2cm]
\texttt{cpf3.org}
\par}

\vspace{1cm}

\end{center}

\newpage

% Main content
\section*{Il Valore dell'Analisi Verticale}

Il Cybersecurity Psychology Framework (CPF3) rappresenta un'architettura teorica universale che identifica le dieci categorie fondamentali di vulnerabilità psicologica sfruttabili in contesti cyber. Tuttavia, la vera potenza del framework emerge quando viene calibrato sulle specificità di settori verticali, dove le vulnerabilità generiche si manifestano attraverso fenomeni domain-specific misurabili e actionable.

\subsection*{Dalla Teoria Universale all'Applicazione Verticale}

Ogni settore industriale presenta un profilo di rischio psicologico distintivo:

\begin{itemize}[leftmargin=*, itemsep=0.3em]
\item \textbf{Venture Capital \& Private Equity}: vulnerabilità cognitive che si traducono in misallocazione di capitale, dove FOMO, founder worship e groupthink distruggono miliardi attraverso decisioni di investimento compromesse
\item \textbf{Healthcare}: conflitto strutturale tra imperativo ippocratico e controlli di sicurezza, dove l'urgenza clinica e l'altruismo professionale creano superfici di attacco uniche
\item \textbf{Financial Services}: pressure temporale sistematica e conformità regolamentare che genera vulnerability patterns specifici nel trading e nella gestione del rischio
\item \textbf{Critical Infrastructure}: dipendenza da sistemi legacy e competenze specialistiche che amplificano authority-based vulnerabilities in ambienti operativi 24/7
\item \textbf{Government \& Defense}: classificazione delle informazioni e catene di comando rigide che creano dinamiche psicologiche exploitable da avversari nation-state
\end{itemize}

\subsection*{Struttura dei Technical Report Verticali}

Ogni Technical Report CPF verticale costituisce un documento scientifico rigoroso (tipicamente 25-40 pagine) che include:

\begin{enumerate}[leftmargin=*, itemsep=0.3em]
\item \textbf{Threat Landscape Settoriale}: analisi del modello di minaccia specifico, identificando come gli attaccanti sfruttano le vulnerabilità psicologiche nel contesto verticale

\item \textbf{Mappatura delle Manifestazioni}: dimostrazione formale di come i fenomeni sector-specific si mappano sulle Categorie Core 10 del CPF, preservando l'integrità matematica dell'architettura bayesiana

\item \textbf{Calibrazione degli Indicatori}: adattamento dei 100 indicatori CPF alle realtà operative del settore, con threshold e telemetry specification customizzati

\item \textbf{Case Study Documentati}: analisi forensi di incidenti reali nel settore che dimostrano come le vulnerabilità psicologiche abbiano abilitato compromissioni (es. Theranos per VC, ransomware Universitätsklinikum Düsseldorf per healthcare)

\item \textbf{Metodologia CPIF Adattata}: personalizzazione del framework di intervento per le strutture di governance e i processi decisionali specifici del settore

\item \textbf{Implementazione OFTLISRV Verticale}: specifiche tecniche per l'integrazione della telemetria psicologica nei sistemi esistenti del settore, rispettando vincoli operativi e compliance requirements

\item \textbf{Metriche di Validazione}: protocolli di assessment per misurare l'efficacia dell'implementazione CPF nel contesto verticale specifico
\end{enumerate}

\subsection*{Il Commitment: Analisi su Misura per il Vostro Settore}

Comprendiamo che ogni potenziale partner e investitore opera in un contesto industriale specifico con sfide uniche. Per questo motivo, \textbf{ci impegniamo a fornire un Technical Report verticale personalizzato} che applica l'intero framework CPF al vostro settore di interesse.

Questo commitment include:

\begin{itemize}[leftmargin=*, itemsep=0.3em]
\item \textbf{Analisi delle Vulnerabilità Specifiche}: identificazione dei fenomeni psicologici critici nel vostro dominio operativo
\item \textbf{Mappatura Formale}: dimostrazione matematica di come questi fenomeni si integrano nell'architettura CPF esistente
\item \textbf{ROI Quantificabile}: stima dell'impatto economico delle vulnerabilità psicologiche nel settore e del value protection abilitato dall'implementazione CPF
\item \textbf{Roadmap di Implementazione}: percorso concreto per il deployment del framework nelle vostre operation o nelle organizzazioni del vostro portfolio
\item \textbf{Competitive Intelligence}: analisi di come i competitor e i peer di settore affrontano (o ignorano) le dimensioni psicologiche della cybersecurity
\end{itemize}

\subsection*{Esempi Disponibili e Nuovi Verticali}

Abbiamo già sviluppato analisi complete per:
\begin{itemize}[leftmargin=*, itemsep=0.2em]
\item \textbf{VC-CPF v1.0}: Venture Capital and Private Equity Framework
\item \textbf{HS-CPF v1.0}: Healthcare Sector Framework
\end{itemize}

Siamo pronti a sviluppare analisi verticali per qualsiasi settore di interesse, inclusi (ma non limitati a):
\begin{itemize}[leftmargin=*, itemsep=0.2em]
\item Financial Services \& Banking
\item Energy \& Critical Infrastructure  
\item Manufacturing \& Industrial IoT
\item Government \& Public Sector
\item Education \& Research Institutions
\item Legal Services \& Law Firms
\item Media \& Entertainment
\item Aerospace \& Defense
\end{itemize}

\vspace{1em}

\begin{tcolorbox}[colback=accentorange!10, colframe=accentorange, title=\textbf{Come Richiedere la Vostra Analisi Verticale}]
Per discutere lo sviluppo di un Technical Report specifico per il vostro settore o per ricevere copie degli esempi esistenti, contattateci a:

\begin{center}
\texttt{g.canale@cpf3.org} | \texttt{cpf3.org}
\end{center}

Ogni analisi verticale rappresenta un deliverable concreto che dimostra l'applicabilità pratica del framework CPF3 al vostro contesto specifico, fornendo immediate actionable insights per la protezione degli asset più critici: i processi decisionali delle persone.
\end{tcolorbox}

\end{document}
