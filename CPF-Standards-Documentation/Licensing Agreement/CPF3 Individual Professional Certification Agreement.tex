\documentclass[11pt,a4paper]{article}

% Packages
\usepackage[utf8]{inputenc}
\usepackage[english]{babel}
\usepackage[margin=2.5cm]{geometry}
\usepackage{hyperref}
\usepackage{fancyhdr}
\usepackage{enumitem}
\usepackage{tabularx}
\usepackage{amssymb}

% Page style
\pagestyle{fancy}
\fancyhf{}
\renewcommand{\headrulewidth}{0.4pt}
\fancyhead[L]{CPF Individual Certification Agreement}
\fancyhead[R]{Candidate \#: \_\_\_\_\_\_\_\_}
\fancyfoot[C]{\thepage}

% Spacing
\setlength{\parindent}{0pt}
\setlength{\parskip}{0.8em}

% Title
\title{\textbf{CPF INDIVIDUAL PROFESSIONAL\\CERTIFICATION AGREEMENT}}
\author{}
\date{}

\begin{document}

\maketitle

\section*{PARTIES}

This Individual Professional Certification Agreement ("Agreement") is entered into as of the date of execution by the Candidate ("Effective Date"), by and between:

\textbf{[CERTIFICATION BODY NAME]} ("Certification Body" or "CB")\\
A [jurisdiction] [entity type]\\
Authorized CPF Certification Body\\
Principal Office: [Address]\\
Email: [Email]

AND

\textbf{[CANDIDATE NAME]} ("Candidate" or "Certified Professional" upon certification)\\
Address: [Address]\\
Email: [Email]\\
Phone: [Phone]

Collectively referred to as the "Parties" and individually as a "Party."

\section*{RECITALS}

WHEREAS, Certification Body is authorized by CPF3 to operate the CPF Certification Scheme and certify individuals as CPF Assessors, CPF Practitioners, or CPF Auditors;

WHEREAS, Candidate desires to obtain professional certification under the CPF Certification Scheme;

WHEREAS, Certification Body is willing to evaluate Candidate's qualifications and, if appropriate, grant certification subject to the terms and conditions set forth herein;

NOW, THEREFORE, in consideration of the mutual covenants and agreements contained herein, the Parties agree as follows:

\section{DEFINITIONS}

\textbf{1.1 "Certification"} means the formal attestation by Certification Body that Candidate has met the requirements for one of the following:
\begin{itemize}
\item CPF Certified Assessor
\item CPF Certified Practitioner
\item CPF Certified Auditor
\end{itemize}

\textbf{1.2 "Certification Mark"} means the trademark, logo, and designation associated with the specific certification granted to Candidate.

\textbf{1.3 "CPF Code of Ethics"} means the professional conduct standards established for certified CPF professionals, as may be amended from time to time.

\textbf{1.4 "CPE"} means Continuing Professional Education credits required for recertification.

\textbf{1.5 "Certification Period"} means the three (3) year period from date of initial certification or recertification.

\textbf{1.6 "Certification Body"} includes its authorized personnel, committees, and representatives.

\section{CERTIFICATION TYPE}

Candidate is applying for the following certification (check one):

\begin{itemize}
\item[$\square$] \textbf{CPF Certified Assessor}\\
Requirements: Bachelor's degree (Psychology or Cybersecurity with supplemental training), 2 years experience, 80 hours training (CPF-101 + CPF-201), written and practical examinations.

\item[$\square$] \textbf{CPF Certified Practitioner}\\
Requirements: Bachelor's degree in relevant field, 1 year CPF implementation experience, 40 hours training (CPF-101), written examination and portfolio review.

\item[$\square$] \textbf{CPF Certified Auditor}\\
Requirements: Current CPF Assessor certification, 1 year as Assessor, 10 completed assessments, 64 hours additional training (CPF-401 + ISO 19011), written and practical examinations.
\end{itemize}

\section{APPLICATION AND CERTIFICATION PROCESS}

\textbf{3.1 Application Submission.} Candidate shall:

\begin{enumerate}[label=\alph*)]
\item Complete the application form accurately and truthfully;
\item Submit all required documentation including:
\begin{itemize}
\item Official transcripts or degree certificates;
\item Experience verification letters or professional portfolio;
\item Training completion certificates;
\item Professional references (minimum 2);
\item Current resume/CV;
\item Government-issued photo identification;
\end{itemize}
\item Pay the non-refundable application review fee;
\item Provide electronic signature on the CPF Code of Ethics.
\end{enumerate}

\textbf{3.2 Application Review.} Certification Body shall:

\begin{enumerate}[label=\alph*)]
\item Review application for completeness and eligibility;
\item Verify education credentials with degree-granting institutions;
\item Verify employment and experience with listed employers or through portfolio review;
\item Contact professional references;
\item Verify training completion with approved training providers;
\item Complete review within fifteen (15) business days of receiving complete application;
\item Notify Candidate of eligibility determination or request additional information.
\end{enumerate}

\textbf{3.3 Examination.} Upon application approval:

\begin{enumerate}[label=\alph*)]
\item Candidate shall schedule and complete required examinations within twelve (12) months;
\item Written examination administered via secure testing platform;
\item Practical examination or portfolio review as applicable;
\item Results provided within five (5) business days;
\item Passing scores required: 70\% for Assessor/Practitioner, 75\% for Auditor.
\end{enumerate}

\textbf{3.4 Retake Policy.} If Candidate fails examination:

\begin{enumerate}[label=\alph*)]
\item May retake after thirty (30) day waiting period;
\item Maximum three (3) attempts within twelve (12) months;
\item Each retake requires payment of retake fee (50\% of examination fee);
\item After three failures, must complete additional training and wait six (6) months before reapplying;
\item Written and practical examinations may be retaken independently.
\end{enumerate}

\textbf{3.5 Certification Decision.} Certification Body shall:

\begin{enumerate}[label=\alph*)]
\item Make certification decision within ten (10) business days of examination completion;
\item Grant certification if all requirements met;
\item Deny certification with written explanation if requirements not met;
\item Provide appeal rights if certification denied;
\item Issue certificate and digital badge upon approval;
\item Add Certified Professional to public certification registry within five (5) business days.
\end{enumerate}

\section{CERTIFICATION GRANT AND RIGHTS}

\textbf{4.1 Certification Grant.} Upon successful completion of all requirements and payment of certification fee, Certification Body grants to Certified Professional:

\begin{enumerate}[label=\alph*)]
\item Professional certification in the applied-for category;
\item Right to use the applicable Certification Mark;
\item Entry in the public certification registry;
\item Access to certification holder benefits;
\item Certificate valid for three (3) years from date of issuance.
\end{enumerate}

\textbf{4.2 Use of Certification Mark.} Certified Professional may:

\begin{enumerate}[label=\alph*)]
\item Use the Certification Mark after their name (e.g., "CPF Certified Assessor");
\item Display the certification logo on business cards, letterhead, email signatures, and professional profiles;
\item Reference the certification in marketing materials and proposals;
\item Use digital badge on professional networking sites (LinkedIn, etc.);
\item State certification status in biographical information and presentations.
\end{enumerate}

\textbf{4.3 Restrictions on Certification Mark Use.} Certified Professional shall NOT:

\begin{enumerate}[label=\alph*)]
\item Modify, alter, or create derivative versions of the Certification Mark;
\item Use the Certification Mark in a manner suggesting certification of products, services, or organizations (unless separately certified);
\item Use the Certification Mark after certification expires, is suspended, or is revoked;
\item Transfer or sublicense the right to use the Certification Mark;
\item Use the Certification Mark in a manner that brings disrepute to CPF or Certification Body;
\item Represent that certification extends beyond the specific category granted.
\end{enumerate}

\textbf{4.4 Certification Benefits.} Certified Professional receives:

\begin{enumerate}[label=\alph*)]
\item Electronic and physical certificate;
\item Digital badge for online use;
\item Entry in public certification registry with profile page;
\item Access to online CPE tracking portal;
\item Invitations to CPF community events and webinars;
\item Access to exclusive resources and tools (as applicable);
\item Newsletter and updates on CPF developments;
\item Networking opportunities through certified professional community.
\end{enumerate}

\section{OBLIGATIONS OF CERTIFIED PROFESSIONAL}

\textbf{5.1 Code of Ethics Compliance.} Certified Professional shall:

\begin{enumerate}[label=\alph*)]
\item Adhere to the CPF Code of Ethics at all times;
\item Maintain integrity, objectivity, and professional conduct;
\item Practice only within areas of demonstrated competence;
\item Protect confidentiality of assessment data and client information;
\item Never use assessment data for individual profiling;
\item Implement privacy-preserving methodologies in all CPF work;
\item Report suspected ethics violations by other certified professionals.
\end{enumerate}

\textbf{5.2 Continuing Professional Education (CPE).} Certified Professional shall:

\begin{enumerate}[label=\alph*)]
\item Complete required CPE credits annually:
\begin{itemize}
\item CPF Assessor: 40 credits per year (120 over 3 years)
\item CPF Practitioner: 30 credits per year (90 over 3 years)
\item CPF Auditor: 50 credits per year (150 over 3 years)
\end{itemize}
\item Document all CPE activities in the online CPE portal;
\item Retain supporting documentation for five (5) years;
\item Submit to CPE audit if selected (random 10\% annually);
\item Ensure minimum ethics CPE credits completed annually.
\end{enumerate}

\textbf{5.3 Professional Practice Requirements.}

\textit{For CPF Assessors:}
\begin{itemize}
\item Conduct minimum five (5) CPF assessments during 3-year certification period;
\item Participate in assessor calibration activities;
\item Submit at least one assessment report for peer review;
\item Maintain current knowledge of CPF methodology updates.
\end{itemize}

\textit{For CPF Practitioners:}
\begin{itemize}
\item Maintain updated portfolio demonstrating continued practical application;
\item Document minimum three (3) implementation projects during certification period;
\item Participate in practitioner community of practice.
\end{itemize}

\textit{For CPF Auditors:}
\begin{itemize}
\item Conduct minimum fifteen (15) audit days per year (45 over 3 years);
\item Serve as lead auditor in minimum five (5) audits during certification period;
\item Submit audit reports for quality review;
\item Participate in auditor competence evaluation activities;
\item Maintain independence from consulting activities per ISO 19011.
\end{itemize}

\textbf{5.4 Notification Obligations.} Certified Professional shall immediately notify Certification Body of:

\begin{enumerate}[label=\alph*)]
\item Changes to contact information;
\item Criminal convictions or professional disciplinary actions;
\item Loss of underlying qualifications (degree revocation, license suspension);
\item Involvement in significant ethics complaints or investigations;
\item Bankruptcy or financial circumstances affecting professional reputation;
\item Any circumstances that may impact certification status or eligibility.
\end{enumerate}

\textbf{5.5 Cooperation with Investigations.} Certified Professional shall:

\begin{enumerate}[label=\alph*)]
\item Cooperate fully with ethics complaint investigations;
\item Respond to Certification Body inquiries within specified timeframes;
\item Provide requested documentation and information;
\item Participate in interviews if required;
\item Not retaliate against complainants or witnesses.
\end{enumerate}

\textbf{5.6 Accurate Representation.} Certified Professional shall:

\begin{enumerate}[label=\alph*)]
\item Accurately represent certification status and scope;
\item Not misrepresent qualifications or experience;
\item Clearly distinguish CPF services from other services offered;
\item Provide truthful information in marketing and proposals;
\item Correct any misrepresentations promptly when discovered.
\end{enumerate}

\section{RECERTIFICATION}

\textbf{6.1 Recertification Requirement.} Certification expires three (3) years from date of issuance. To maintain certification, Certified Professional must apply for recertification.

\textbf{6.2 Recertification Process.}

\begin{enumerate}[label=\alph*)]
\item Certification Body sends recertification notice 180 days before expiration;
\item Certified Professional submits recertification application 90 days before expiration;
\item Recertification application includes:
\begin{itemize}
\item Complete CPE records for 3-year period;
\item Documentation of professional practice requirements;
\item Updated professional references (if requested);
\item Ethics attestation;
\item Recertification fee payment;
\end{itemize}
\item Certification Body reviews submission within 60 days;
\item If approved, new certificate issued with updated expiration date;
\item If denied, Certified Professional receives written explanation and appeal rights.
\end{enumerate}

\textbf{6.3 Grace Period.} If recertification not completed by expiration:

\begin{enumerate}[label=\alph*)]
\item Ninety (90) day grace period applies;
\item Certification status changes to "Pending Recertification";
\item Use of Certification Mark restricted during grace period;
\item Late recertification fee applies (additional \$100);
\item After grace period, full recertification process required including examinations.
\end{enumerate}

\textbf{6.4 CPE Deficit Remediation.} If CPE requirements not met:

\begin{enumerate}[label=\alph*)]
\item Certified Professional may request up to 90-day extension to complete remaining CPE;
\item Extension granted at Certification Body's discretion;
\item Certification status changes to "Conditional" during extension;
\item If CPE not completed within extension, certification lapses;
\item Extension fee may apply.
\end{enumerate}

\section{FEES}

\textbf{7.1 Application Fee.}
\begin{itemize}
\item CPF Assessor: \$300 (non-refundable)
\item CPF Practitioner: \$200 (non-refundable)
\item CPF Auditor: \$400 (non-refundable)
\end{itemize}

\textbf{7.2 Examination Fees.}
\begin{itemize}
\item CPF Assessor Written: \$400
\item CPF Assessor Practical: \$600
\item CPF Practitioner Written: \$300
\item CPF Practitioner Portfolio Review: \$400
\item CPF Auditor Written: \$450
\item CPF Auditor Practical: \$800
\item Retake Fee: 50\% of original examination fee
\end{itemize}

\textbf{7.3 Certification Fee.} Upon successful completion of all requirements:
\begin{itemize}
\item CPF Assessor: \$200
\item CPF Practitioner: \$150
\item CPF Auditor: \$250
\end{itemize}

\textbf{7.4 Recertification Fees.}
\begin{itemize}
\item CPF Assessor: \$400
\item CPF Practitioner: \$300
\item CPF Auditor: \$500
\item Late Recertification (within 90-day grace period): Add \$100
\end{itemize}

\textbf{7.5 Other Fees.}
\begin{itemize}
\item CPE Extension Request: \$50
\item Duplicate Certificate: \$25
\item Certification Verification Letter: \$15
\item Appeal Fee: \$200 (refunded if appeal successful)
\end{itemize}

\textbf{7.6 Payment Terms.}
\begin{enumerate}[label=\alph*)]
\item All fees payable in USD;
\item Payment by credit card, bank transfer, or check;
\item Fees non-refundable except as specifically stated;
\item Services not provided until payment received;
\item Delinquent fees may result in suspension of certification.
\end{enumerate}

\section{SUSPENSION AND REVOCATION}

\textbf{8.1 Grounds for Suspension.} Certification Body may suspend certification for:

\begin{enumerate}[label=\alph*)]
\item Failure to complete required CPE within deadline;
\item Failure to pay required fees;
\item Ethics complaint under investigation;
\item Failure to meet professional practice requirements;
\item Failure to respond to Certification Body inquiries;
\item Loss of underlying qualifications pending investigation.
\end{enumerate}

\textbf{8.2 Suspension Process.}

\begin{enumerate}[label=\alph*)]
\item Written notice of suspension with specific grounds;
\item Immediate cessation of Certification Mark use;
\item Registry status changed to "Suspended";
\item Suspension period: Maximum 90 days;
\item Remediation plan required within 30 days;
\item Reinstatement upon successful remediation;
\item If not remediated within 90 days: Revocation proceedings initiated.
\end{enumerate}

\textbf{8.3 Grounds for Revocation.} Certification Body may revoke certification for:

\begin{enumerate}[label=\alph*)]
\item Severe ethics violations including:
\begin{itemize}
\item Fraud, misrepresentation, or dishonesty;
\item Confidentiality breach or improper data use;
\item Individual profiling using assessment data;
\item Criminal conviction related to professional conduct;
\end{itemize}
\item Failure to remediate suspension within 90 days;
\item Repeated or systematic violations of CPF Code of Ethics;
\item Loss of underlying qualifications (degree revocation);
\item Provision of false information in application or recertification;
\item Unauthorized sublicensing or transfer of certification;
\item Material breach of this Agreement.
\end{enumerate}

\textbf{8.4 Revocation Process.}

\begin{enumerate}[label=\alph*)]
\item Written notice of intent to revoke with specific grounds;
\item Opportunity to respond within 30 days;
\item Independent review by Certification Body ethics committee;
\item Final decision communicated within 45 days;
\item If revoked:
\begin{itemize}
\item Immediate cessation of all Certification Mark use;
\item Removal from certification registry;
\item Public notice of revocation;
\item Return of certificate to Certification Body;
\item Prohibition on reapplication for minimum 2 years (or permanent);
\end{itemize}
\item Right to appeal revocation decision.
\end{enumerate}

\textbf{8.5 Voluntary Surrender.} Certified Professional may voluntarily surrender certification by:

\begin{enumerate}[label=\alph*)]
\item Written notice to Certification Body;
\item Immediate cessation of Certification Mark use;
\item Return of certificate;
\item No refund of fees;
\item May reapply for certification at any time by completing full certification process.
\end{enumerate}

\section{APPEALS}

\textbf{9.1 Right to Appeal.} Certified Professional may appeal:

\begin{enumerate}[label=\alph*)]
\item Certification denial;
\item Examination failure due to procedural irregularities (not score);
\item Recertification denial;
\item Suspension decision;
\item Revocation decision;
\item Disciplinary actions.
\end{enumerate}

\textbf{9.2 Appeal Process.}

\begin{enumerate}[label=\alph*)]
\item Appeal submitted in writing within 30 days of decision;
\item Appeal fee payment (\$200);
\item Specification of grounds for appeal and supporting documentation;
\item Independent appeals panel assigned (no involvement in original decision);
\item Panel reviews all evidence and decision rationale;
\item Appellant may provide additional written information;
\item Panel renders decision within 30 days;
\item Decision options: Uphold, Modify, Reverse, or Remand for reconsideration;
\item Fee refunded if appeal successful;
\item Appeals panel decision is final and binding.
\end{enumerate}

\textbf{9.3 Appeals Panel Composition.}

\begin{itemize}
\item Three members: one certified CPF professional, one subject matter expert, one Certification Body representative not involved in original decision;
\item Panel members have no conflict of interest;
\item Decisions made by majority vote;
\item Panel deliberations confidential.
\end{itemize}

\section{CONFIDENTIALITY AND DATA PROTECTION}

\textbf{10.1 Confidentiality.} Certification Body shall:

\begin{enumerate}[label=\alph*)]
\item Maintain confidentiality of Candidate/Certified Professional information;
\item Limit access to information to personnel with need to know;
\item Protect examination responses and assessment data;
\item Not disclose confidential information without consent, except:
\begin{itemize}
\item Public registry information (name, certification type, status, expiration date);
\item As required by law or court order;
\item To CPF3 for quality oversight purposes;
\item To accreditation bodies during audits;
\item Investigation of ethics complaints as necessary;
\end{itemize}
\item Implement appropriate technical and organizational security measures;
\item Comply with applicable data protection laws (GDPR, CCPA, etc.).
\end{enumerate}

\textbf{10.2 Data Protection Rights.} Certified Professional has right to:

\begin{enumerate}[label=\alph*)]
\item Access personal data held by Certification Body;
\item Request correction of inaccurate information;
\item Request deletion of data (subject to record retention requirements);
\item Object to processing for certain purposes;
\item Receive data in portable format;
\item Lodge complaint with data protection authority.
\end{enumerate}

\textbf{10.3 Data Retention.} Certification Body shall:

\begin{enumerate}[label=\alph*)]
\item Retain certification records for seven (7) years after certification expiration or revocation;
\item Retain ethics investigation records for ten (10) years;
\item Securely destroy data after retention period unless legal hold applies;
\item Maintain audit trails for data access and modifications.
\end{enumerate}

\textbf{10.4 Data Breach Notification.} In event of data breach:

\begin{enumerate}[label=\alph*)]
\item Certification Body shall notify affected Certified Professionals within 72 hours;
\item Notification includes nature of breach, data affected, and mitigation steps;
\item Certification Body shall notify applicable data protection authorities as required by law;
\item Certification Body shall take steps to prevent further breaches.
\end{enumerate}

\section{LIMITATION OF LIABILITY}

\textbf{11.1 Disclaimer of Warranties.} CERTIFICATION BODY MAKES NO WARRANTIES, EXPRESS OR IMPLIED, REGARDING CERTIFICATION OUTCOMES, CAREER BENEFITS, OR INCOME POTENTIAL. CERTIFICATION IS PROVIDED "AS IS" WITHOUT WARRANTY OF MERCHANTABILITY OR FITNESS FOR A PARTICULAR PURPOSE.

\textbf{11.2 Limitation of Damages.} IN NO EVENT SHALL CERTIFICATION BODY BE LIABLE FOR INDIRECT, INCIDENTAL, CONSEQUENTIAL, SPECIAL, EXEMPLARY, OR PUNITIVE DAMAGES, INCLUDING LOST INCOME, LOST BUSINESS OPPORTUNITIES, OR REPUTATIONAL HARM, ARISING FROM CERTIFICATION OR DENIAL THEREOF.

\textbf{11.3 Cap on Liability.} CERTIFICATION BODY'S TOTAL LIABILITY UNDER THIS AGREEMENT SHALL NOT EXCEED THE TOTAL FEES PAID BY CERTIFIED PROFESSIONAL IN THE TWELVE (12) MONTHS PRECEDING THE CLAIM.

\textbf{11.4 Exceptions.} Limitations do not apply to:

\begin{enumerate}[label=\alph*)]
\item Certification Body's gross negligence or willful misconduct;
\item Breaches of confidentiality obligations;
\item Data protection violations;
\item Claims not permitted to be limited under applicable law.
\end{enumerate}

\section{INDEMNIFICATION}

\textbf{12.1 Indemnification by Certified Professional.} Certified Professional shall indemnify, defend, and hold harmless Certification Body from claims arising from:

\begin{enumerate}[label=\alph*)]
\item Certified Professional's professional services to third parties;
\item Certified Professional's negligence or misconduct;
\item Certified Professional's violation of this Agreement or CPF Code of Ethics;
\item Certified Professional's unauthorized use of Certification Marks;
\item False or misleading information provided in application or recertification.
\end{enumerate}

\textbf{12.2 Professional Liability Insurance.} Certified Professional providing CPF services professionally shall maintain:

\begin{itemize}
\item Professional liability insurance (errors and omissions) with minimum coverage appropriate to scope of practice;
\item General liability insurance as applicable;
\item Evidence of insurance provided to clients upon request;
\item Certification Body not responsible for verifying insurance coverage.
\end{itemize}

\section{GENERAL PROVISIONS}

\textbf{13.1 Governing Law.} This Agreement shall be governed by laws of [Jurisdiction], without regard to conflict of laws principles.

\textbf{13.2 Dispute Resolution.}

\begin{enumerate}[label=\alph*)]
\item Good faith negotiation required before formal dispute resolution;
\item Disputes not resolved through negotiation or appeals process shall be resolved through binding arbitration;
\item Arbitration conducted per rules of [Arbitration Service];
\item Arbitration in [City, Jurisdiction], English language;
\item Arbitrator's decision final and binding;
\item Each party bears own costs unless arbitrator determines otherwise.
\end{enumerate}

\textbf{13.3 Entire Agreement.} This Agreement, including incorporated CPF Code of Ethics and Certification Scheme requirements, constitutes entire agreement and supersedes all prior understandings.

\textbf{13.4 Amendment.} Certification Body may amend this Agreement or CPF Code of Ethics by providing 90 days written notice. Continued certification after effective date constitutes acceptance. If Certified Professional does not accept amendments, may voluntarily surrender certification.

\textbf{13.5 Assignment.} Certified Professional may not assign or transfer certification. Certification Body may assign this Agreement in connection with business transfer or merger.

\textbf{13.6 Notices.} All notices shall be sent to addresses stated above or as updated in writing. Email with confirmation receipt is acceptable for routine communications.

\textbf{13.7 Severability.} If any provision found invalid, remaining provisions continue in full effect.

\textbf{13.8 Waiver.} Failure to enforce any provision does not waive right to enforce later.

\textbf{13.9 Independent Contractor.} Certified Professional is independent contractor, not employee or agent of Certification Body.

\textbf{13.10 Survival.} Sections 5.1 (Ethics), 8 (Suspension/Revocation effects), 10 (Confidentiality), 11 (Limitation of Liability), 12 (Indemnification), and 13 (General Provisions) survive termination of certification.

\section{ACKNOWLEDGMENTS}

By signing below, Candidate acknowledges and agrees that:

\begin{enumerate}[label=\alph*)]
\item Has read and understands this Agreement in its entirety;
\item Has read and agrees to comply with the CPF Code of Ethics;
\item Has provided accurate and truthful information in application;
\item Understands certification requirements and ongoing obligations;
\item Understands fees are non-refundable;
\item Understands certification may be suspended or revoked for violations;
\item Understands must maintain CPE and practice requirements;
\item Authorizes Certification Body to verify information provided;
\item Authorizes publication of name and certification status in public registry;
\item Consents to processing of personal data as described;
\item Understands certification does not guarantee employment or income;
\item Agrees to resolve disputes through arbitration;
\item Will immediately cease use of Certification Mark if certification ends.
\end{enumerate}

\vspace{2em}

\section*{SIGNATURES}

\textbf{CERTIFICATION BODY: [NAME]}

\vspace{1.5em}

By: \underline{\hspace{6cm}} Date: \underline{\hspace{3cm}}

Name: \underline{\hspace{6cm}}

Title: \underline{\hspace{6cm}}

\vspace{2em}

\textbf{CANDIDATE/CERTIFIED PROFESSIONAL}

\vspace{1.5em}

Signature: \underline{\hspace{6cm}} Date: \underline{\hspace{3cm}}

Print Name: \underline{\hspace{6cm}}

\vspace{2em}

\section*{CERTIFICATION RECORD (For CB Use Only)}

\begin{tabular}{|l|p{10cm}|}
\hline
\textbf{Certification Type} & \hspace{8cm} \\
\hline
\textbf{Certificate Number} & \\
\hline
\textbf{Issue Date} & \\
\hline
\textbf{Expiration Date} & \\
\hline
\textbf{Issued By} & \\
\hline
\end{tabular}

\vspace{2em}

\begin{center}
\textit{End of Individual Professional Certification Agreement}
\end{center}

\end{document}