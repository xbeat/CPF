\documentclass[11pt,a4paper]{article}

% Pacchetti
\usepackage[utf8]{inputenc}
\usepackage[english]{babel}
\usepackage[margin=2.5cm]{geometry}
\usepackage{amsmath}
\usepackage{booktabs}
\usepackage{hyperref}
\usepackage{fancyhdr}
\usepackage{enumitem}
\usepackage{amssymb}

% Stile pagina
\pagestyle{fancy}
\fancyhf{}
\renewcommand{\headrulewidth}{0.4pt}
\fancyhead[L]{CPF Certification Scheme}
\fancyhead[R]{Version 1.0}
\fancyfoot[C]{\thepage}

% Spacing
\setlength{\parindent}{0pt}
\setlength{\parskip}{0.5em}

% Hyperref setup
\hypersetup{
    colorlinks=true,
    linkcolor=blue,
    citecolor=blue,
    urlcolor=blue,
    pdftitle={CPF Certification Scheme},
    pdfauthor={Giuseppe Canale, CISSP},
}

% Titolo
\title{\textbf{CPF Certification Scheme}\\
\large Version 1.0}
\author{CPF Certification Body\\
Giuseppe Canale, CISSP\\
\small Independent Researcher\\
\small g.canale@cpf3.org}
\date{January 2025}

\begin{document}

\maketitle

\begin{abstract}
This document defines the certification scheme for the Cybersecurity Psychology Framework (CPF), including requirements for individual certifications (CPF Assessor, CPF Practitioner, CPF Auditor) and organizational certifications (CPF Compliance Levels). The scheme is designed in accordance with ISO/IEC 17065:2012 requirements for certification bodies operating product, process, and service certification systems. This certification scheme enables systematic validation of competence in psychological vulnerability assessment and organizational capability in implementing CPF-based security controls. The framework addresses the critical gap between technical security controls and human factors, providing standardized pathways for professionals and organizations to demonstrate mastery of pre-cognitive vulnerability management.

\textbf{Keywords:} certification, competence assessment, ISO/IEC 17065, cybersecurity psychology, professional development, organizational maturity
\end{abstract}

\tableofcontents
\newpage

\section{Introduction}

\subsection{Purpose and Scope}

The CPF Certification Scheme establishes comprehensive requirements for certifying individuals and organizations in the systematic assessment and mitigation of psychological vulnerabilities within cybersecurity contexts. This scheme addresses the fundamental gap in current cybersecurity certification programs, which focus predominantly on technical competencies while neglecting the psychological dimensions that contribute to 82-85\% of security incidents.

The scope of this certification scheme includes:

\textbf{Individual Certifications:}
\begin{itemize}
\item CPF Assessor Certification: Validates competence in conducting systematic psychological vulnerability assessments using the CPF methodology
\item CPF Practitioner Certification: Confirms practical application of CPF principles in organizational settings
\item CPF Auditor Certification: Certifies capability to audit CPF implementation and compliance
\end{itemize}

\textbf{Organizational Certifications:}
\begin{itemize}
\item CPF Compliance Level 1-4: Validates organizational maturity in psychological vulnerability management across four progressive levels
\end{itemize}

This scheme does not replace existing cybersecurity certifications (CISSP, CISM, CEH, etc.) but complements them by addressing the psychological dimensions absent from technical certification programs.

\subsection{Certification Benefits}

CPF certification provides measurable value to individuals, organizations, and the broader cybersecurity community:

\textbf{For Individuals:}
\begin{itemize}
\item Differentiation in competitive cybersecurity job market
\item Recognition of interdisciplinary expertise bridging psychology and security
\item Career advancement through specialized competence validation
\item Professional development pathway in emerging discipline
\item Access to CPF professional community and continuing education
\end{itemize}

\textbf{For Organizations:}
\begin{itemize}
\item Reduced human-factor security incidents through systematic psychological vulnerability management
\item Enhanced security posture addressing the 82-85\% of breaches with human components
\item Competitive advantage demonstrating commitment to comprehensive security
\item Improved insurance posture through risk reduction validation
\item Regulatory compliance support for human-factor security requirements
\item Measurable ROI through incident reduction and risk mitigation
\end{itemize}

\textbf{For the Community:}
\begin{itemize}
\item Standardization of psychological vulnerability assessment practices
\item Advancement of cybersecurity psychology as recognized discipline
\item Knowledge sharing through certified practitioner network
\item Research advancement through validated implementation data
\item Industry maturation beyond purely technical approaches
\end{itemize}

\subsection{Relationship to ISO/IEC 17065}

This certification scheme is designed to be operated by certification bodies conforming to ISO/IEC 17065:2012, which specifies requirements for bodies certifying products, processes, and services. CPF certification constitutes process and service certification under ISO/IEC 17065 definitions.

Key ISO/IEC 17065 principles applied in this scheme:

\textbf{Impartiality:} Certification bodies must maintain independence from training providers, consultancies, and certified entities. Conflicts of interest are systematically identified and managed.

\textbf{Competence:} Certification body personnel must demonstrate competence in both cybersecurity and psychology domains, validated through education, training, and experience requirements.

\textbf{Consistency:} Certification decisions follow standardized evaluation criteria ensuring comparable outcomes across different assessors, organizations, and time periods.

\textbf{Confidentiality:} Assessment data, particularly psychological vulnerability information, receives heightened confidentiality protection beyond standard ISO/IEC 17065 requirements.

\textbf{Responsiveness to Complaints:} Certification bodies implement robust complaint investigation and resolution procedures with appeal mechanisms for certification decisions.

Certification bodies operating this scheme must maintain ISO/IEC 17065 accreditation through nationally recognized accreditation bodies. This ensures international recognition and mutual acceptance of CPF certifications across jurisdictions.

\section{Certification Levels}

\subsection{Individual Certifications}

\subsubsection{CPF Assessor Certification}

The CPF Assessor certification validates competence to conduct systematic psychological vulnerability assessments using the CPF methodology. Assessors must demonstrate both theoretical knowledge and practical capability to identify, score, and document psychological vulnerabilities across all ten CPF domains.

\textbf{Requirements:}

\textit{Education:}
\begin{itemize}
\item Bachelor's degree in Psychology, Behavioral Science, Organizational Psychology, OR
\item Bachelor's degree in Cybersecurity, Information Security, Computer Science PLUS 40 hours of documented CPF-specific training in psychological foundations
\end{itemize}

\textit{Experience:}
\begin{itemize}
\item Minimum 2 years professional experience in cybersecurity OR psychology
\item At least 6 months must involve security-relevant work
\item Documentation of experience through employer verification or professional portfolio
\end{itemize}

\textit{Training:}
\begin{itemize}
\item CPF-101: Framework Fundamentals (40 hours)
\item CPF-201: Assessment Methodology (40 hours)
\item Total: 80 hours of mandatory training
\item Training must be completed within 24 months of certification application
\end{itemize}

\textit{Examination:}
\begin{itemize}
\item Written Examination: 100 questions covering all ten CPF domains, assessment methodology, privacy protection, and ethical considerations
\item Duration: 3 hours
\item Passing Score: 70\% (70 correct responses)
\item Question Distribution: 60 multiple-choice, 30 scenario-based, 10 case analysis
\item Practical Assessment: Analysis of organizational case study with vulnerability identification, scoring justification, and intervention recommendations
\item Duration: 4 hours
\item Passing Requirement: Demonstrated competence across all evaluation criteria
\end{itemize}

\textit{Ethical Requirements:}
\begin{itemize}
\item Agreement to CPF Code of Ethics
\item Commitment to privacy-preserving assessment practices
\item Understanding of psychological vulnerability sensitivity
\item Prohibition on using assessment data for individual performance evaluation
\end{itemize}

\textbf{Certification Validity:} 3 years from date of issuance

\textbf{Recertification Requirements:}
\begin{itemize}
\item 40 Continuing Professional Education (CPE) credits per year (120 total over 3 years)
\item Minimum 5 documented assessments using CPF methodology
\item Ethics review and updated agreement
\item Continuing education in psychological and cybersecurity advances
\end{itemize}

\subsubsection{CPF Practitioner Certification}

The CPF Practitioner certification validates practical application of CPF principles within organizational contexts. Practitioners implement CPF-based interventions, translate assessment findings into actionable security improvements, and integrate psychological vulnerability management with existing security programs.

\textbf{Requirements:}

\textit{Education:}
\begin{itemize}
\item Bachelor's degree in relevant field (Psychology, Cybersecurity, Organizational Behavior, Business Administration with security focus)
\item Documented understanding of both security and organizational psychology principles
\end{itemize}

\textit{Experience:}
\begin{itemize}
\item Minimum 1 year using CPF methodology within organizational setting
\item Documentation of practical implementation through project portfolio
\item Demonstrated integration of CPF with existing security programs
\item Evidence of intervention design and implementation
\end{itemize}

\textit{Training:}
\begin{itemize}
\item CPF-101: Framework Fundamentals (40 hours)
\item No additional mandatory training beyond fundamentals
\item Recommended: CPF-201 Assessment Methodology for enhanced competence
\end{itemize}

\textit{Examination:}
\begin{itemize}
\item Written Examination: 75 questions focusing on practical application, intervention design, and organizational integration
\item Duration: 2.5 hours
\item Passing Score: 70\% (53 correct responses)
\item Question Distribution: 45 multiple-choice, 20 scenario-based, 10 application problems
\item Portfolio Review: Submission of practical work demonstrating CPF implementation
\item Evaluation: Evidence of systematic application, intervention effectiveness, organizational integration
\end{itemize}

\textit{Portfolio Requirements:}
\begin{itemize}
\item Minimum 3 implementation projects with documented outcomes
\item Evidence of assessment-to-intervention pipeline
\item Integration documentation with existing security controls
\item Demonstrated privacy protection in practice
\item Stakeholder engagement and communication examples
\end{itemize}

\textbf{Certification Validity:} 3 years from date of issuance

\textbf{Recertification Requirements:}
\begin{itemize}
\item 30 CPE credits per year (90 total over 3 years)
\item Updated portfolio demonstrating continued practical application
\item Ethics review and agreement renewal
\item Participation in CPF practitioner community of practice
\end{itemize}

\subsubsection{CPF Auditor Certification}

The CPF Auditor certification validates competence to audit organizational implementation of CPF methodology, assess compliance with CPF-27001:2025 requirements, and evaluate effectiveness of psychological vulnerability management systems.

\textbf{Requirements:}

\textit{Prerequisites:}
\begin{itemize}
\item Current CPF Assessor certification in good standing
\item Minimum 1 year experience as certified CPF Assessor
\item Documented completion of at least 10 CPF assessments
\end{itemize}

\textit{Education:}
\begin{itemize}
\item Existing education requirements satisfied through CPF Assessor certification
\item Additional training in audit methodology (ISO 19011:2018)
\item Understanding of management system auditing principles
\end{itemize}

\textit{Experience:}
\begin{itemize}
\item Participation in minimum 3 CPF audits under supervision of certified CPF Auditor
\item Minimum 20 audit days documented
\item Experience across different organizational types and sizes
\item Demonstrated competence in audit planning, execution, and reporting
\end{itemize}

\textit{Training:}
\begin{itemize}
\item CPF-401: Audit Techniques (40 hours)
\item ISO 19011:2018 Auditor Training (minimum 24 hours)
\item Integration of psychological assessment with management system auditing
\end{itemize}

\textit{Examination:}
\begin{itemize}
\item Written Examination: 80 questions covering audit methodology, CPF-27001 requirements, auditor competencies, and professional conduct
\item Duration: 3 hours
\item Passing Score: 75\% (60 correct responses)
\item Audit Scenario Examination: Conduct mock audit including planning, execution, finding documentation, and report generation
\item Duration: 8 hours (full-day practical examination)
\item Passing Requirement: Demonstrated audit competence across all evaluation criteria
\end{itemize}

\textit{Professional Conduct Requirements:}
\begin{itemize}
\item Adherence to ISO 19011 auditor principles (integrity, impartiality, confidentiality)
\item Enhanced privacy protection for psychological vulnerability data
\item Independence from consulting and implementation services
\item Objective evidence-based audit approach
\end{itemize}

\textbf{Certification Validity:} 3 years from date of issuance

\textbf{Recertification Requirements:}
\begin{itemize}
\item 50 CPE credits per year (150 total over 3 years), with minimum 30 credits in audit-specific topics
\item Minimum 15 audit days per year as lead or co-auditor (45 total over 3 years)
\item Demonstrated continued audit competence through audit report submissions
\item Ethics and professional conduct review
\item Participation in auditor calibration activities
\end{itemize}

\subsection{Organizational Certifications}

\subsubsection{CPF Compliance Levels}

Organizational certification validates systematic implementation of psychological vulnerability management according to CPF-27001:2025 requirements. Four progressive compliance levels reflect organizational maturity in addressing human-factor security risks.

\textbf{Scoring Foundation:}

Organizational compliance is based on aggregate CPF Scores derived from systematic assessment of all 100 CPF indicators across the organization's scope. The CPF Score ranges from 0-200, with lower scores indicating better security posture (fewer and less severe vulnerabilities).

Scoring methodology:
\begin{itemize}
\item Each indicator scored using ternary system: Green (0), Yellow (1), Red (2)
\item Category Score = Sum of 10 indicators per category (range 0-20)
\item CPF Score = Sum of 10 category scores (range 0-200)
\item Assessment conducted by certified CPF Assessor or Auditor
\item Minimum assessment scope: Representative sample ensuring privacy-preserving aggregation (minimum 10 individuals per aggregation unit)
\end{itemize}

\textbf{Level 1: Foundation (CPF Score 100-149)}

\textit{Maturity Characteristics:}
\begin{itemize}
\item Initial CPF implementation with basic psychological vulnerability awareness
\item Reactive approach to human-factor incidents
\item Limited integration with existing security programs
\item Basic privacy-preserving assessment practices
\end{itemize}

\textit{Minimum Requirements:}
\begin{itemize}
\item CPF Score between 100-149 from certified assessment
\item Documented CPF policy approved by senior management
\item Designated CPF Coordinator with defined responsibilities
\item Completion of CPF-101 training by security leadership
\item Basic assessment conducted covering all 10 domains
\item Documented risk treatment plan for Red indicators
\item Privacy protection procedures implemented
\item Integration plan with existing Information Security Management System (ISMS)
\end{itemize}

\textit{Surveillance Requirements:}
\begin{itemize}
\item Annual assessment by certified CPF Assessor
\item Quarterly reporting of Red indicator status
\item Annual management review of CPF program
\end{itemize}

\textbf{Level 2: Intermediate (CPF Score 70-99)}

\textit{Maturity Characteristics:}
\begin{itemize}
\item Systematic psychological vulnerability management processes
\item Proactive identification and treatment of vulnerabilities
\item Integrated approach combining psychological and technical controls
\item Established privacy protection framework
\item Demonstrable reduction in human-factor incidents
\end{itemize}

\textit{Minimum Requirements:}
\begin{itemize}
\item CPF Score between 70-99 from certified assessment
\item All Level 1 requirements maintained
\item Minimum one certified CPF Assessor on staff or on retainer
\item Quarterly assessment cycles covering all domains
\item Documented intervention effectiveness tracking
\item Integration with Security Operations Center (SOC) for continuous monitoring of critical indicators
\item Established Continuing Professional Education (CPE) program for security staff
\item Privacy-preserving dashboard for psychological vulnerability monitoring
\item Documented reduction in human-factor incidents (minimum 20\% year-over-year)
\end{itemize}

\textit{Surveillance Requirements:}
\begin{itemize}
\item Bi-annual comprehensive assessment by certified CPF Auditor
\item Quarterly self-assessment with certified Assessor validation
\item Monthly Red indicator reporting
\item Semi-annual management review
\end{itemize}

\textbf{Level 3: Advanced (CPF Score 40-69)}

\textit{Maturity Characteristics:}
\begin{itemize}
\item Mature psychological vulnerability management program
\item Predictive identification of convergent states
\item Sophisticated integration across all security domains
\item Leading privacy protection practices
\item Substantial reduction in human-factor breaches
\item Organizational culture of psychological security awareness
\end{itemize}

\textit{Minimum Requirements:}
\begin{itemize}
\item CPF Score between 40-69 from certified assessment
\item All Level 1 and Level 2 requirements maintained
\item Minimum two certified CPF Assessors on staff
\item Continuous monitoring of all 100 indicators integrated with SIEM
\item Predictive analytics for convergent state identification
\item Automated alerting for critical psychological vulnerability patterns
\item Documented 40\% reduction in human-factor incidents compared to baseline
\item Advanced privacy protection using differential privacy ($\epsilon \leq 0.1$)
\item Contribution to CPF research and community knowledge base
\item Integration with third-party risk management for supply chain psychological security
\item Psychological vulnerability considerations in all change management processes
\end{itemize}

\textit{Surveillance Requirements:}
\begin{itemize}
\item Annual comprehensive assessment by certified CPF Auditor
\item Continuous self-monitoring with quarterly validation
\item Real-time Red indicator alerting and response
\item Quarterly management review
\item Annual external audit of privacy protection practices
\end{itemize}

\textbf{Level 4: Exemplary (CPF Score 0-39)}

\textit{Maturity Characteristics:}
\begin{itemize}
\item World-class psychological vulnerability management
\item Predictive security posture preventing incidents before occurrence
\item Complete integration across all organizational functions
\item Industry-leading privacy protection and ethical practices
\item Near-elimination of preventable human-factor breaches
\item Organizational culture of psychological resilience
\item Contribution to industry advancement
\end{itemize}

\textit{Minimum Requirements:}
\begin{itemize}
\item CPF Score between 0-39 from certified assessment
\item All Level 1, 2, and 3 requirements maintained
\item Dedicated CPF team including multiple certified Assessors and at least one certified Auditor
\item Real-time psychological vulnerability monitoring with AI-enhanced predictive analytics
\item Zero Red indicators sustained for minimum 6 months
\item Maximum 10\% Yellow indicators with documented treatment plans
\item Documented 60\% reduction in human-factor incidents compared to baseline
\item Published research or case studies advancing CPF methodology
\item Active contribution to CPF community through knowledge sharing, training, or tool development
\item Integration of psychological vulnerability management across supply chain
\item Psychological security considerations embedded in enterprise risk management
\item Advanced privacy protection practices exceeding differential privacy requirements
\item Regular third-party validation of privacy and ethical practices
\end{itemize}

\textit{Surveillance Requirements:}
\begin{itemize}
\item Annual comprehensive assessment by external certified CPF Auditor
\item Continuous self-monitoring with monthly validation
\item Real-time convergent state monitoring with automated response
\item Monthly management review
\item Quarterly external audit of privacy and ethical practices
\item Bi-annual peer review by other Level 4 organizations
\end{itemize}

\section{Certification Requirements}

\subsection{Education Requirements}

Education requirements validate foundational knowledge necessary for CPF competence. The interdisciplinary nature of CPF requires understanding of both psychological and cybersecurity principles.

\textbf{Acceptable Degrees (Bachelor's or Higher):}

\textit{For Assessor Certification:}
\begin{itemize}
\item Psychology
\item Behavioral Science
\item Organizational Psychology
\item Industrial/Organizational Psychology
\item Cognitive Science
\item Cybersecurity (with required supplemental psychology training)
\item Information Security (with required supplemental psychology training)
\item Computer Science (with required supplemental psychology training)
\end{itemize}

\textit{For Practitioner Certification:}
\begin{itemize}
\item All degrees acceptable for Assessor certification
\item Business Administration with security focus
\item Human Resources with security or organizational psychology focus
\item Risk Management
\end{itemize}

\textbf{Degree Equivalency:}

Candidates without formal degrees may qualify through combination of:
\begin{itemize}
\item Relevant professional certifications (CISSP, CISM, CEH for security; Licensed Psychologist, SHRM-SCP for psychology)
\item Documented professional experience (5 years minimum)
\item Completion of all required CPF training
\item Passage of enhanced examination demonstrating knowledge equivalent to degree requirements
\end{itemize}

\textbf{International Degree Recognition:}

Degrees from non-US institutions evaluated using:
\begin{itemize}
\item National recognition in degree-granting country
\item Equivalency evaluation by credential evaluation service
\item Demonstration of English language proficiency for examinations
\end{itemize}

\subsection{Experience Requirements}

Experience requirements validate practical capability beyond theoretical knowledge. Experience must demonstrate security-relevant work and exposure to organizational human factors.

\textbf{Verification Methods:}
\begin{itemize}
\item Employer verification letters on official letterhead
\item Detailed professional portfolio documenting projects and responsibilities
\item Professional references from supervisors or clients
\item Documented project deliverables (with confidential information redacted)
\end{itemize}

\textbf{Qualifying Experience Categories:}

\textit{Security Experience:}
\begin{itemize}
\item Security operations and monitoring
\item Incident response and investigation
\item Security assessment and testing
\item Security program management
\item Risk assessment and management
\item Security awareness program development
\item Security policy development and implementation
\end{itemize}

\textit{Psychology Experience:}
\begin{itemize}
\item Organizational psychology consulting
\item Behavioral assessment and intervention
\item Human factors analysis
\item Organizational development
\item Change management
\item Training and development program design
\end{itemize}

\textit{Integrated Experience (Counts Double):}
\begin{itemize}
\item Security awareness program psychology
\item Human factors in security design
\item Social engineering testing and analysis
\item Insider threat program psychology
\item Security culture development
\end{itemize}

\subsection{Training Requirements}

Mandatory training ensures standardized understanding of CPF methodology, assessment techniques, and ethical practices. Training must be completed through CPF-approved training providers meeting quality and curriculum standards.

\textbf{CPF-101: Framework Fundamentals (40 hours)}

\textit{Course Objectives:}
\begin{itemize}
\item Understand theoretical foundations: psychoanalytic theory, cognitive psychology, group dynamics
\item Master CPF architecture: 10 domains, 100 indicators, ternary scoring
\item Apply privacy-preserving assessment principles
\item Integrate CPF with existing security frameworks (ISO 27001, NIST CSF)
\end{itemize}

\textit{Course Outline:}
\begin{enumerate}
\item Introduction to Cybersecurity Psychology (4 hours)
\begin{itemize}
\item Failure of conscious-level security interventions
\item Pre-cognitive processing and security decisions
\item Overview of CPF framework
\end{itemize}

\item Theoretical Foundations (8 hours)
\begin{itemize}
\item Psychoanalytic contributions: Bion, Klein, Jung, Winnicott
\item Cognitive psychology: Kahneman, Cialdini, Miller
\item Group dynamics and organizational unconscious
\item AI psychology and human-AI interaction
\end{itemize}

\item CPF Domain Deep-Dive (20 hours - 2 hours per domain)
\begin{itemize}
\item Authority-Based Vulnerabilities [1.x]
\item Temporal Vulnerabilities [2.x]
\item Social Influence Vulnerabilities [3.x]
\item Affective Vulnerabilities [4.x]
\item Cognitive Overload Vulnerabilities [5.x]
\item Group Dynamic Vulnerabilities [6.x]
\item Stress Response Vulnerabilities [7.x]
\item Unconscious Process Vulnerabilities [8.x]
\item AI-Specific Bias Vulnerabilities [9.x]
\item Critical Convergent States [10.x]
\end{itemize}

\item Privacy and Ethics (4 hours)
\begin{itemize}
\item Privacy-preserving assessment methodology
\item Differential privacy and aggregation requirements
\item Ethical considerations in psychological assessment
\item Prohibition on individual profiling
\end{itemize}

\item Integration and Application (4 hours)
\begin{itemize}
\item Integration with ISO 27001 and NIST CSF
\item Organizational implementation strategies
\item Case studies and practical applications
\end{itemize}
\end{enumerate}

\textbf{CPF-201: Assessment Methodology (40 hours)}

\textit{Course Objectives:}
\begin{itemize}
\item Master systematic assessment process for all 100 indicators
\item Develop data collection and analysis skills
\item Apply ternary scoring methodology consistently
\item Create actionable assessment reports
\end{itemize}

\textit{Course Outline:}
\begin{enumerate}
\item Assessment Planning (6 hours)
\item Data Collection Methods (8 hours)
\item Scoring and Analysis (12 hours)
\item Privacy-Preserving Techniques (6 hours)
\item Report Writing and Communication (8 hours)
\end{enumerate}

\textbf{CPF-301: Advanced Implementation (40 hours)}

\textit{Optional Advanced Training for Practitioners}

\textit{Course Objectives:}
\begin{itemize}
\item Design effective interventions for identified vulnerabilities
\item Implement continuous monitoring systems
\item Integrate psychological and technical controls
\item Measure intervention effectiveness
\end{itemize}

\textbf{CPF-401: Audit Techniques (40 hours)}

\textit{Required for Auditor Certification}

\textit{Course Objectives:}
\begin{itemize}
\item Apply ISO 19011 auditing principles to CPF context
\item Conduct CPF-27001 compliance audits
\item Evaluate psychological vulnerability management systems
\item Document audit findings and recommendations
\end{itemize}

\subsection{Examination Requirements}

Examinations validate knowledge acquisition and practical capability. All examinations are developed using psychometric principles ensuring validity, reliability, and fairness.

\textbf{Written Examination Structure:}

\textit{Question Development:}
\begin{itemize}
\item Item difficulty distribution: 30\% easy, 50\% moderate, 20\% difficult
\item Bloom's taxonomy coverage: 20\% knowledge, 40\% comprehension/application, 40\% analysis/synthesis
\item Pilot testing and validation before operational use
\item Regular statistical analysis and item improvement
\end{itemize}

\textit{Domain Coverage (All Certifications):}
\begin{itemize}
\item Authority-Based Vulnerabilities: 10\%
\item Temporal Vulnerabilities: 10\%
\item Social Influence Vulnerabilities: 10\%
\item Affective Vulnerabilities: 10\%
\item Cognitive Overload Vulnerabilities: 10\%
\item Group Dynamic Vulnerabilities: 10\%
\item Stress Response Vulnerabilities: 10\%
\item Unconscious Process Vulnerabilities: 10\%
\item AI-Specific Bias Vulnerabilities: 10\%
\item Critical Convergent States: 10\%
\end{itemize}

\textit{Exam Administration:}
\begin{itemize}
\item Computer-based testing at authorized test centers
\item Remote proctoring available with enhanced security
\item Closed-book examination with no reference materials
\item Immediate preliminary results (pending quality review)
\item Official results within 5 business days
\end{itemize}

\textit{Passing Standards:}
\begin{itemize}
\item Assessor/Practitioner Written: 70\% minimum
\item Auditor Written: 75\% minimum (higher standard reflecting advanced role)
\item No minimum score per domain, but comprehensive coverage required
\item Candidates failing may retake after 30-day waiting period
\item Maximum 3 attempts within 12-month period
\end{itemize}

\textbf{Practical Examination Structure:}

\textit{Assessor Practical:}
\begin{itemize}
\item Case study: Realistic organizational scenario with psychological vulnerability indicators
\item Task: Conduct assessment, apply ternary scoring, justify ratings, recommend interventions
\item Duration: 4 hours
\item Evaluation criteria: Accuracy, justification quality, privacy protection, communication clarity
\end{itemize}

\textit{Auditor Practical:}
\begin{itemize}
\item Mock audit: Full-day audit simulation including planning, interviews, document review, finding documentation, report generation
\item Duration: 8 hours (full business day)
\item Evaluation criteria: Audit methodology, evidence collection, finding quality, professional conduct, report clarity
\end{itemize}

\section{Certification Process}

\subsection{Application}

The application process ensures candidates meet eligibility requirements before examination registration.

\textbf{Application Steps:}

\begin{enumerate}
\item \textbf{Eligibility Self-Assessment}
\begin{itemize}
\item Review certification requirements
\item Verify education qualifications
\item Confirm experience requirements
\item Ensure training completion
\end{itemize}

\item \textbf{Documentation Preparation}
\begin{itemize}
\item Official transcripts or degree certificates
\item Experience verification letters or portfolio
\item Training completion certificates
\item Professional references (minimum 2)
\item Current resume/CV
\end{itemize}

\item \textbf{Application Submission}
\begin{itemize}
\item Complete online application form
\item Upload required documentation
\item Pay application review fee (non-refundable)
\item Electronic signature on Code of Ethics
\end{itemize}

\item \textbf{Application Review}
\begin{itemize}
\item Eligibility verification by certification body
\item Documentation completeness check
\item Reference contact (if needed)
\item Approval or request for additional information
\item Timeline: 10 business days from complete application
\end{itemize}
\end{enumerate}

\textbf{Application Fees:}
\begin{itemize}
\item CPF Assessor: \$300 (application review)
\item CPF Practitioner: \$200 (application review)
\item CPF Auditor: \$400 (application review)
\item Organizational Certification: \$500-\$2000 based on organization size and scope
\end{itemize}

\subsection{Verification}

Verification ensures authenticity and accuracy of submitted documentation.

\textbf{Education Verification:}
\begin{itemize}
\item Direct contact with degree-granting institution
\item Verification of degree type, major, and conferral date
\item International degree equivalency evaluation
\item Timeline: 5-10 business days
\end{itemize}

\textbf{Experience Verification:}
\begin{itemize}
\item Contact with listed employers or clients
\item Confirmation of employment dates and responsibilities
\item Portfolio review for self-employed candidates
\item Timeline: 10-15 business days
\end{itemize}

\textbf{Reference Checks:}
\begin{itemize}
\item Contact minimum 2 professional references
\item Verification of candidate's competence and professional conduct
\item Assessment of suitability for certification
\item Confidential feedback to certification body
\end{itemize}

\textbf{Training Verification:}
\begin{itemize}
\item Direct verification with approved training providers
\item Confirmation of course completion and dates
\item Verification of attendance and assessment results
\item Timeline: 3-5 business days
\end{itemize}

\subsection{Examination}

Examination validates knowledge and competence through standardized assessment.

\textbf{Scheduling:}
\begin{itemize}
\item Exam eligibility notification within 3 business days of verification completion
\item Candidate selects exam date and location from available options
\item Minimum 14 days advance scheduling required
\item Rescheduling permitted up to 48 hours before exam (fee may apply)
\end{itemize}

\textbf{Exam Delivery Options:}

\textit{In-Person Testing:}
\begin{itemize}
\item Authorized Pearson VUE or Prometric test centers
\item Secure testing environment with proctoring
\item Identity verification required
\item No personal items permitted in testing room
\end{itemize}

\textit{Online Proctored Testing:}
\begin{itemize}
\item Remote examination via secure platform
\item Live proctor monitoring via webcam
\item Environmental scan required
\item System requirements: Computer, webcam, microphone, stable internet
\item Identity verification via government-issued photo ID
\end{itemize}

\textbf{Examination Fees:}
\begin{itemize}
\item CPF Assessor Written: \$400
\item CPF Assessor Practical: \$600
\item CPF Practitioner Written: \$300
\item CPF Practitioner Portfolio Review: \$400
\item CPF Auditor Written: \$450
\item CPF Auditor Practical: \$800
\item Retake Fee: 50\% of original exam fee
\end{itemize}

\textbf{Retake Policy:}
\begin{itemize}
\item Failed candidates may retake after 30-day waiting period
\item Maximum 3 attempts within 12 months
\item After 3 failures, candidate must complete additional training and wait 6 months
\item Each retake requires new examination fee
\item Practical examination may be retaken independently of written examination
\end{itemize}

\textbf{Accommodations:}
\begin{itemize}
\item Reasonable accommodations provided for documented disabilities
\item Request must be submitted with application
\item Documentation from qualified professional required
\item Examples: Extended time, separate room, screen reader, breaks
\end{itemize}

\subsection{Certification Decision}

Certification decisions are made by qualified certification body personnel based on standardized criteria.

\textbf{Decision Criteria:}

\textit{Individual Certification:}
\begin{itemize}
\item Verification of all eligibility requirements
\item Passing score on written examination
\item Passing evaluation on practical examination/portfolio
\item Satisfactory reference checks
\item Agreement to Code of Ethics
\item Payment of all applicable fees
\end{itemize}

\textit{Organizational Certification:}
\begin{itemize}
\item Valid CPF assessment by certified Assessor/Auditor
\item CPF Score within targeted compliance level range
\item Documentation of required policies and procedures
\item Evidence of systematic implementation
\item Management commitment demonstrated
\item Surveillance requirements agreed
\end{itemize}

\textbf{Decision Timeline:}
\begin{itemize}
\item Individual Certification: 10 business days from examination completion
\item Organizational Certification: 15 business days from audit completion
\item Expedited review available for additional fee
\end{itemize}

\textbf{Decision Outcomes:}

\textit{Certification Granted:}
\begin{itemize}
\item Certificate issued electronically and in hardcopy
\item Entry in public certification registry
\item Access to certification holder benefits
\item Authorization to use certification marks
\end{itemize}

\textit{Certification Denied:}
\begin{itemize}
\item Written explanation of deficiencies
\item Guidance on remediation steps
\item Right to appeal decision
\item Option to reapply after addressing deficiencies
\end{itemize}

\textbf{Appeal Process:}
\begin{itemize}
\item Appeals must be submitted in writing within 30 days
\item Independent review by appeals panel (not involved in original decision)
\item Review of all evidence and decision rationale
\item Decision within 30 days of appeal submission
\item Appeal fee: \$200 (refunded if appeal successful)
\item Final decision binding, but candidate may reapply after addressing issues
\end{itemize}

\textbf{Certificate Issuance:}
\begin{itemize}
\item Electronic certificate (PDF) issued within 3 business days
\item Physical certificate mailed within 10 business days
\item Digital badge for online professional profiles
\item Entry in public certification registry within 5 business days
\item Certification wallet card for physical identification
\end{itemize}

\section{Maintaining Certification}

\subsection{Continuing Education}

Continuing Professional Education (CPE) ensures certified individuals maintain current knowledge as CPF methodology and cybersecurity landscape evolve.

\textbf{CPE Requirements:}

\textit{CPF Assessor:}
\begin{itemize}
\item 40 CPE credits per year (120 over 3-year cycle)
\item Minimum 20 credits in CPF-specific topics
\item Maximum 10 credits from single activity/source per year
\item Minimum 5 credits in ethics and privacy annually
\end{itemize}

\textit{CPF Practitioner:}
\begin{itemize}
\item 30 CPE credits per year (90 over 3-year cycle)
\item Minimum 15 credits in CPF-specific topics
\item Maximum 10 credits from single activity/source per year
\item Minimum 3 credits in ethics annually
\end{itemize}

\textit{CPF Auditor:}
\begin{itemize}
\item 50 CPE credits per year (150 over 3-year cycle)
\item Minimum 30 credits in audit-specific topics
\item Minimum 10 credits in CPF methodology updates
\item Maximum 10 credits from single activity/source per year
\item Minimum 5 credits in ethics and professional conduct annually
\end{itemize}

\textbf{Accepted CPE Activities:}

\textit{Category A: Formal Education (1 hour = 1 credit)}
\begin{itemize}
\item Approved CPF training courses
\item Academic courses in psychology or cybersecurity
\item Professional certification training (CISSP, CISM, etc.)
\item Webinars and virtual training
\end{itemize}

\textit{Category B: Professional Development (1 hour = 1 credit)}
\begin{itemize}
\item Conference attendance (cybersecurity or psychology)
\item Professional association meetings
\item CPF community of practice participation
\item Mentoring certified candidates (maximum 5 credits/year)
\end{itemize}

\textit{Category C: Self-Study (2 hours = 1 credit)}
\begin{itemize}
\item Reading professional journals and publications
\item Review of CPF methodology updates
\item Independent research in relevant topics
\item Online courses without assessment
\end{itemize}

\textit{Category D: Contributions (Special Credit)}
\begin{itemize}
\item Publishing CPF research or case studies: 10 credits
\item Developing CPF training materials: 15 credits
\item Speaking at conferences on CPF topics: 5 credits per presentation
\item Serving on CPF advisory committees: 10 credits/year
\item Contributing to CPF methodology development: 20 credits
\end{itemize}

\textbf{Documentation Requirements:}
\begin{itemize}
\item Certificate of completion for formal training
\item Attendance records for conferences
\item Reading logs with summaries for self-study
\item Publication citations for authored works
\item Verification from benefiting organizations for volunteer work
\item All documentation maintained for 5 years
\end{itemize}

\textbf{CPE Tracking:}
\begin{itemize}
\item Online CPE portal for activity logging
\item Automatic credit for approved activities
\item Upload capability for supporting documentation
\item Progress dashboard showing credit accumulation
\item Automated reminders for approaching deadlines
\end{itemize}

\textbf{CPE Audit:}
\begin{itemize}
\item Random audit of 10\% of certification holders annually
\item Request for documentation of claimed CPE activities
\item Verification of activity completion and credit calculation
\item 30-day response period for documentation submission
\item Non-compliance may result in certification suspension
\end{itemize}

\subsection{Recertification}

Recertification occurs every 3 years and validates continued competence and ethical practice.

\textbf{Recertification Requirements:}

\textit{All Individual Certifications:}
\begin{itemize}
\item Completion of all CPE requirements for 3-year cycle
\item Continued professional experience in relevant role
\item No substantiated ethics violations
\item Payment of recertification fee
\item Updated agreement to Code of Ethics
\item Demonstration of current competence
\end{itemize}

\textit{Additional Assessor Requirements:}
\begin{itemize}
\item Minimum 5 documented CPF assessments over 3-year period
\item Peer review of at least one assessment report
\item Participation in assessor calibration activities
\end{itemize}

\textit{Additional Auditor Requirements:}
\begin{itemize}
\item Minimum 45 audit days over 3-year period (15/year average)
\item Lead auditor role in minimum 5 audits
\item Audit report quality review
\item Participation in auditor competence evaluation
\end{itemize}

\textbf{Recertification Process:}

\begin{enumerate}
\item \textbf{Notification} (180 days before expiration)
\begin{itemize}
\item Certification body sends recertification notice
\item Candidate reviews requirements and current status
\item CPE deficit identification and remediation plan if needed
\end{itemize}

\item \textbf{Documentation Submission} (90 days before expiration)
\begin{itemize}
\item CPE records submitted via online portal
\item Experience documentation uploaded
\item Professional references provided (if required)
\item Ethics attestation completed
\end{itemize}

\item \textbf{Review and Verification} (60 days before expiration)
\begin{itemize}
\item Certification body reviews submission
\item CPE audit (if selected)
\item Experience verification
\item Ethics record check
\end{itemize}

\item \textbf{Recertification Decision} (30 days before expiration)
\begin{itemize}
\item Approval or request for additional information
\item New certificate issued upon approval
\item Updated certification period in registry
\end{itemize}
\end{enumerate}

\textbf{Recertification Fees:}
\begin{itemize}
\item CPF Assessor: \$400
\item CPF Practitioner: \$300
\item CPF Auditor: \$500
\item Late recertification (within 90 days after expiration): Additional \$100
\item Reinstatement (beyond 90 days after expiration): Full certification process required
\end{itemize}

\textbf{Grace Period:}
\begin{itemize}
\item 90-day grace period after expiration
\item Certification status changes to "Pending Recertification"
\item Use of certification marks restricted during grace period
\item Late fee applies for recertification during grace period
\item After grace period, full recertification process required
\end{itemize}

\textbf{Organizational Recertification:}
\begin{itemize}
\item Annual surveillance audits required
\item Full reassessment every 3 years
\item Continuous monitoring of CPF Score
\item Compliance level may be upgraded or downgraded based on performance
\item Significant organizational changes trigger reassessment
\end{itemize}

\subsection{Ethics and Professional Conduct}

Ethics requirements ensure certification holders maintain professional standards and protect stakeholder interests.

\textbf{Code of Ethics - Core Principles:}

\begin{enumerate}
\item \textbf{Integrity}
\begin{itemize}
\item Honest representation of qualifications and capabilities
\item Accurate reporting of assessment findings
\item Transparent communication with stakeholders
\item No falsification of documentation or data
\end{itemize}

\item \textbf{Objectivity}
\begin{itemize}
\item Unbiased assessment and evaluation
\item No conflicts of interest
\item Independence from commercial pressures
\item Evidence-based decision making
\end{itemize}

\item \textbf{Confidentiality}
\begin{itemize}
\item Protection of assessment data
\item Secure handling of psychological vulnerability information
\item No unauthorized disclosure
\item Enhanced privacy protection for sensitive data
\end{itemize}

\item \textbf{Competence}
\begin{itemize}
\item Practice within areas of demonstrated competence
\item Continuous professional development
\item Recognition of competence limitations
\item Referral when expertise insufficient
\end{itemize}

\item \textbf{Professional Responsibility}
\begin{itemize}
\item Adherence to CPF methodology standards
\item Compliance with applicable laws and regulations
\item Reporting of unethical behavior
\item Contribution to professional community
\end{itemize}
\end{enumerate}

\textbf{Specific Ethical Requirements:}

\textit{Privacy Protection:}
\begin{itemize}
\item Never use assessment data for individual profiling
\item Maintain minimum aggregation units (10 individuals)
\item Implement differential privacy protections
\item Secure storage and transmission of all data
\item Prohibition on secondary use without explicit consent
\item Time-delayed reporting (minimum 72 hours)
\item Role-based rather than individual analysis
\end{itemize}

\textit{Conflict of Interest:}
\begin{itemize}
\item Disclosure of all potential conflicts before engagement
\item No financial interest in assessment outcomes
\item Independence from training providers when assessing
\item No consulting and auditing for same organization simultaneously
\item Prohibition on accepting gifts or incentives
\end{itemize}

\textit{Scope of Practice:}
\begin{itemize}
\item CPF assessment is organizational, not clinical psychological assessment
\item No individual diagnosis or therapeutic interventions
\item Recognition that psychological vulnerabilities are normal human characteristics
\item No stigmatization or blame of individuals for vulnerabilities
\item Clear boundaries between CPF and clinical psychology
\end{itemize}

\textit{Data Handling:}
\begin{itemize}
\item Encryption of all assessment data at rest and in transit
\item Access controls limiting data to authorized personnel
\item Audit trails for all data access
\item Retention limits (maximum 5 years unless legally required)
\item Secure destruction of data after retention period
\item No cross-border data transfer without appropriate safeguards
\end{itemize}

\textbf{Disciplinary Process:}

\textit{Complaint Submission:}
\begin{itemize}
\item Anyone may file complaint against certified individual or organization
\item Complaint submitted in writing with specific allegations
\item Supporting evidence provided
\item Complainant identity protected (option for anonymous complaints)
\end{itemize}

\textit{Investigation:}
\begin{itemize}
\item Initial review within 10 business days
\item Investigation by ethics committee (independent from certification decisions)
\item Opportunity for accused party to respond
\item Evidence gathering and witness interviews
\item Investigation completed within 60 days (extendable if complex)
\end{itemize}

\textit{Findings and Sanctions:}

Finding: No Violation
\begin{itemize}
\item Complaint dismissed
\item No record on certification file
\item Parties notified of outcome
\end{itemize}

Finding: Minor Violation
\begin{itemize}
\item Written warning issued
\item Corrective action plan required
\item Enhanced CPE requirements
\item Progress monitoring
\end{itemize}

Finding: Significant Violation
\begin{itemize}
\item Certification suspension (6-12 months)
\item Remediation requirements before reinstatement
\item Probationary period after reinstatement
\item Public disclosure in certification registry
\end{itemize}

Finding: Severe Violation
\begin{itemize}
\item Certification revocation
\item Prohibition on reapplication (2-5 years or permanent)
\item Public disclosure
\item Notification to relevant authorities if legal violations involved
\end{itemize}

\textit{Appeal of Disciplinary Action:}
\begin{itemize}
\item Appeal must be filed within 30 days of decision
\item Independent appeals panel reviews case
\item No new evidence permitted (review of process and proportionality)
\item Decision within 45 days
\item Appeal fee: \$500 (refunded if appeal successful)
\end{itemize}

\section{Certification Bodies}

\subsection{Accreditation Requirements}

Certification bodies operating this scheme must maintain appropriate accreditation and demonstrate specific competencies.

\textbf{ISO/IEC 17065 Accreditation:}
\begin{itemize}
\item Accreditation from nationally recognized accreditation body
\item Accreditation scope includes process and service certification
\item Annual surveillance audits by accreditation body
\item Full reassessment every 4 years
\item Compliance with all ISO/IEC 17065 requirements
\end{itemize}

\textbf{CPF-Specific Competencies:}

\textit{Personnel Requirements:}
\begin{itemize}
\item Certification scheme manager with expertise in both cybersecurity and psychology
\item Minimum 2 technical experts with CPF Auditor certification
\item Access to subject matter experts in psychoanalytic theory and cognitive psychology
\item Examination development personnel with psychometric expertise
\item Privacy and ethics specialists
\end{itemize}

\textit{Technical Competence:}
\begin{itemize}
\item Understanding of all ten CPF domains and 100 indicators
\item Knowledge of privacy-preserving assessment methodologies
\item Familiarity with differential privacy and aggregation requirements
\item Integration knowledge for ISO 27001 and NIST CSF
\item Competence in psychological ethics and professional conduct
\end{itemize}

\textit{Infrastructure Requirements:}
\begin{itemize}
\item Secure examination development and storage systems
\item Encrypted candidate database with access controls
\item Online application and CPE tracking platforms
\item Secure communication channels for sensitive information
\item Backup and disaster recovery capabilities
\end{itemize}

\textbf{Quality Management System:}

\textit{Documentation:}
\begin{itemize}
\item Certification scheme procedures fully documented
\item Decision-making criteria clearly defined
\item Appeals and complaints procedures established
\item Records retention and management procedures
\item Confidentiality and impartiality procedures
\end{itemize}

\textit{Process Controls:}
\begin{itemize}
\item Standardized application review process
\item Consistent examination administration
\item Calibrated decision-making across personnel
\item Regular competence evaluation of staff
\item Internal audits of certification processes
\end{itemize}

\textit{Continuous Improvement:}
\begin{itemize}
\item Regular review of examination statistics
\item Analysis of appeals and complaints for systemic issues
\item Stakeholder feedback mechanisms
\item Benchmarking against other certification bodies
\item Implementation of corrective and preventive actions
\end{itemize}

\subsection{Quality Assurance}

Quality assurance ensures consistency, reliability, and credibility of certification decisions.

\textbf{Audits of Certification Bodies:}

\textit{Accreditation Body Audits:}
\begin{itemize}
\item Annual surveillance by ISO/IEC 17065 accreditation body
\item Review of certification files and decisions
\item Witness assessments and examinations
\item Evaluation of competence management
\item Full reassessment every 4 years
\end{itemize}

\textit{CPF Scheme Owner Audits:}
\begin{itemize}
\item Annual audit of CPF-specific requirements compliance
\item Review of examination quality and validity
\item Assessment of technical competence in CPF domains
\item Evaluation of privacy protection practices
\item Verification of ethics and disciplinary procedures
\end{itemize}

\textit{Peer Review:}
\begin{itemize}
\item Cross-audits between certification bodies
\item Sharing of best practices
\item Calibration of decision-making
\item Identification of improvement opportunities
\end{itemize}

\textbf{Complaint Handling:}

\textit{Types of Complaints:}
\begin{itemize}
\item Certification process complaints (delays, communication, fairness)
\item Examination complaints (quality, administration, fairness)
\item Certified personnel complaints (ethics, competence, conduct)
\item Organizational certification complaints (audit quality, decisions)
\end{itemize}

\textit{Complaint Process:}
\begin{enumerate}
\item Complaint submission (written, with details)
\item Acknowledgment within 3 business days
\item Investigation within 30 days
\item Resolution and response to complainant
\item Corrective action if warranted
\item Trend analysis for systemic issues
\end{enumerate}

\textit{Complaint Records:}
\begin{itemize}
\item All complaints logged in complaint register
\item Investigation documentation maintained
\item Resolution and corrective actions recorded
\item Regular review by management
\item Annual summary reporting to accreditation body
\end{itemize}

\textbf{Continuous Improvement:}

\textit{Performance Metrics:}
\begin{itemize}
\item Application processing time
\item Examination pass rates and statistics
\item Appeals and complaints rates
\item Certification holder retention rates
\item Stakeholder satisfaction scores
\end{itemize}

\textit{Improvement Mechanisms:}
\begin{itemize}
\item Regular management review of quality metrics
\item Analysis of examination statistics for validity
\item Review of appeals for decision consistency
\item Stakeholder surveys and feedback
\item Implementation of identified improvements
\item Sharing of lessons learned across certification bodies
\end{itemize}

\subsection{Appeals and Complaints}

Robust appeals and complaints procedures ensure fairness and provide recourse for stakeholders.

\textbf{Appeal Process:}

\textit{Appealable Decisions:}
\begin{itemize}
\item Certification denial
\item Examination failure (procedural issues only, not score)
\item Disciplinary actions
\item Recertification denial
\item Certification suspension or revocation
\end{itemize}

\textit{Appeal Procedure:}
\begin{enumerate}
\item Appeal submitted in writing within 30 days of decision
\item Appeal fee payment (\$200-\$500 based on decision type)
\item Specification of grounds for appeal
\item Supporting documentation provided

\item Independent appeals panel assigned (no involvement in original decision)
\item Review of all evidence and decision rationale
\item Opportunity for appellant to provide additional information
\item Panel deliberation and decision

\item Decision communicated within 30 days of appeal submission
\item Options: Uphold original decision, Modify decision, Reverse decision, Remand for reconsideration
\item Fee refunded if appeal successful
\item Decision is final (no further appeals)
\end{enumerate}

\textbf{Complaint Investigation:}

\textit{Investigation Process:}
\begin{enumerate}
\item Complaint submission with specific allegations
\item Initial review for completeness and jurisdiction
\item Assignment to investigator (independent from subject)
\item Notice to subject of complaint with opportunity to respond
\item Evidence gathering and witness interviews
\item Investigation report with findings
\item Decision on complaint validity and corrective actions
\item Communication to complainant and subject
\item Implementation of corrective actions
\item Follow-up to verify effectiveness
\end{enumerate}

\textbf{Resolution Procedures:}

\textit{Informal Resolution:}
\begin{itemize}
\item Mediation between parties
\item Clarification of misunderstandings
\item Corrective action by certification body
\item Withdrawal of complaint if resolved
\end{itemize}

\textit{Formal Resolution:}
\begin{itemize}
\item Official investigation findings
\item Corrective or disciplinary actions
\item Changes to certification body procedures
\item Compensation or remediation if warranted
\item Prevention measures to avoid recurrence
\end{itemize}

\section*{Appendices}

\appendix

\section{Sample Exam Questions}

\subsection{CPF Assessor Sample Questions}

\textbf{Multiple Choice Questions:}

\textbf{Question 1:} Which of the following best describes the primary purpose of CPF assessment?

a) To identify employees who pose security risks\\
b) To measure conscious security awareness levels\\
c) To identify organizational-level pre-cognitive psychological vulnerabilities\\
d) To evaluate individual psychological fitness for security roles

\textit{Correct Answer: c}

\textbf{Question 2:} According to CPF methodology, what is the minimum aggregation unit for reporting assessment data?

a) 5 individuals\\
b) 10 individuals\\
c) 25 individuals\\
d) 50 individuals

\textit{Correct Answer: b}

\textbf{Question 3:} In the ternary scoring system, a Yellow (1) indicator represents:

a) Minimal vulnerability with no action required\\
b) Moderate vulnerability requiring monitoring\\
c) Critical vulnerability requiring immediate intervention\\
d) Eliminated vulnerability

\textit{Correct Answer: b}

\textbf{Scenario-Based Questions:}

\textbf{Question 4:} An organization's finance department consistently processes urgent wire transfer requests from anyone with "CEO" in their email signature without additional verification. This behavior primarily indicates vulnerability in which CPF domain?

a) [2.x] Temporal Vulnerabilities\\
b) [1.x] Authority-Based Vulnerabilities\\
c) [3.x] Social Influence Vulnerabilities\\
d) [5.x] Cognitive Overload Vulnerabilities

\textit{Correct Answer: b - Authority-Based Vulnerabilities, specifically indicator 1.1 (Unquestioning compliance with apparent authority)}

\textbf{Question 5:} During end-of-quarter periods, security incident rates increase by 35\%, primarily involving employees bypassing approval processes to meet deadlines. Which convergent vulnerability state does this represent?

a) Pure [2.x] Temporal Vulnerability\\
b) [10.4] Swiss cheese alignment of temporal and authority vulnerabilities\\
c) [6.1] Groupthink security blind spots\\
d) [5.2] Decision fatigue errors

\textit{Correct Answer: b - This represents convergence of temporal pressure ([2.3] Deadline-driven risk acceptance) with organizational patterns creating perfect storm conditions}

\subsection{CPF Practitioner Sample Questions}

\textbf{Application Questions:}

\textbf{Question 6:} An assessment reveals high Red scores in Domain 5 (Cognitive Overload) with alert fatigue affecting 70\% of security team. Which intervention would be MOST effective according to CPF principles?

a) Additional security awareness training\\
b) Disciplinary action for ignored alerts\\
c) Reduction of false positive alerts and alert consolidation\\
d) Hiring additional security personnel

\textit{Correct Answer: c - Addresses root cause of cognitive overload rather than symptoms}

\textbf{Question 7:} When integrating CPF with an existing ISO 27001 ISMS, psychological vulnerability assessment results should be primarily incorporated into which ISMS component?

a) Asset inventory\\
b) Risk assessment process\\
c) Access control procedures\\
d) Incident response plan

\textit{Correct Answer: b - Psychological vulnerabilities are risk factors requiring systematic risk assessment and treatment}

\subsection{CPF Auditor Sample Questions}

\textbf{Audit Methodology Questions:}

\textbf{Question 8:} During a CPF-27001 compliance audit, you discover that assessment data includes individual identifiers that could enable profiling. This represents a nonconformity with which CPF-27001 requirement?

a) Section 7.3 (Awareness)\\
b) Section 8.2.3 (Privacy-Preserving Measures)\\
c) Section 9.1 (Monitoring, Measurement, Analysis and Evaluation)\\
d) Section 10.1 (Nonconformity and Corrective Action)

\textit{Correct Answer: b - Privacy-Preserving Measures explicitly prohibit individual profiling}

\textbf{Question 9:} An organization claims CPF Level 3 certification but CPF Score is 75. What is the appropriate audit conclusion?

a) Certification should be downgraded to Level 2 (70-99 range)\\
b) Certification should be maintained with surveillance\\
c) Certification should be suspended pending corrective action\\
d) Certification is appropriate as score is within Level 3 range

\textit{Correct Answer: a - CPF Score of 75 falls in Level 2 range (70-99), not Level 3 (40-69)}

\textbf{Professional Conduct Questions:}

\textbf{Question 10:} You are offered a consulting engagement to help an organization improve their CPF Score before your scheduled audit. What is the appropriate response?

a) Accept if consulting occurs 6+ months before audit\\
b) Accept but recuse yourself from audit\\
c) Decline due to conflict of interest\\
d) Accept but disclose to certification body

\textit{Correct Answer: c - Auditors must maintain independence and cannot provide consulting to organizations they audit}

\section{CPF Training Curriculum}

\subsection{CPF-101: Framework Fundamentals (40 hours)}

\textbf{Module 1: Introduction to Cybersecurity Psychology (4 hours)}
\begin{itemize}
\item 1.1 The Human Factor Gap in Cybersecurity
\item 1.2 Failure of Conscious-Level Interventions
\item 1.3 Pre-Cognitive Processing and Security Decisions
\item 1.4 Overview of CPF Framework
\item 1.5 CPF Integration with Security Frameworks
\end{itemize}

\textbf{Module 2: Psychoanalytic Foundations (4 hours)}
\begin{itemize}
\item 2.1 Bion's Basic Assumptions Theory
\item 2.2 Klein's Object Relations Theory
\item 2.3 Jung's Analytical Psychology
\item 2.4 Winnicott's Transitional Space
\item 2.5 Application to Organizational Security
\end{itemize}

\textbf{Module 3: Cognitive Psychology Foundations (4 hours)}
\begin{itemize}
\item 3.1 Kahneman's Dual-Process Theory
\item 3.2 Cialdini's Influence Principles
\item 3.3 Miller's Cognitive Load Theory
\item 3.4 Heuristics and Biases in Security
\item 3.5 Decision-Making Under Uncertainty
\end{itemize}

\textbf{Modules 4-13: CPF Domain Deep-Dives (20 hours, 2 hours each)}
\begin{itemize}
\item Module 4: Authority-Based Vulnerabilities [1.x]
\item Module 5: Temporal Vulnerabilities [2.x]
\item Module 6: Social Influence Vulnerabilities [3.x]
\item Module 7: Affective Vulnerabilities [4.x]
\item Module 8: Cognitive Overload Vulnerabilities [5.x]
\item Module 9: Group Dynamic Vulnerabilities [6.x]
\item Module 10: Stress Response Vulnerabilities [7.x]
\item Module 11: Unconscious Process Vulnerabilities [8.x]
\item Module 12: AI-Specific Bias Vulnerabilities [9.x]
\item Module 13: Critical Convergent States [10.x]
\end{itemize}

\textbf{Module 14: Privacy and Ethics (4 hours)}
\begin{itemize}
\item 14.1 Privacy-Preserving Assessment Principles
\item 14.2 Differential Privacy and Aggregation Requirements
\item 14.3 Ethical Considerations in Psychological Assessment
\item 14.4 Prohibition on Individual Profiling
\item 14.5 Professional Conduct and Boundaries
\item 14.6 Data Handling and Confidentiality
\end{itemize}

\textbf{Module 15: Integration and Application (4 hours)}
\begin{itemize}
\item 15.1 CPF and ISO/IEC 27001:2022 Integration
\item 15.2 CPF and NIST CSF 2.0 Integration
\item 15.3 Organizational Implementation Strategies
\item 15.4 Case Studies and Practical Applications
\item 15.5 Common Implementation Challenges
\item 15.6 Course Review and Assessment
\end{itemize}

\subsection{CPF-201: Assessment Methodology (40 hours)}

\textbf{Module 1: Assessment Planning (6 hours)}
\begin{itemize}
\item 1.1 Scope Definition and Boundaries
\item 1.2 Stakeholder Engagement
\item 1.3 Resource Planning
\item 1.4 Privacy Impact Assessment
\item 1.5 Assessment Schedule and Timeline
\item 1.6 Risk Assessment for Assessment Process
\end{itemize}

\textbf{Module 2: Data Collection Methods (8 hours)}
\begin{itemize}
\item 2.1 Behavioral Observation Techniques
\item 2.2 Interview Methodologies
\item 2.3 Document Review and Analysis
\item 2.4 Survey Design and Administration
\item 2.5 Technical Log Analysis
\item 2.6 Combining Multiple Data Sources
\item 2.7 Avoiding Hawthorne Effect
\item 2.8 Practical Exercise: Data Collection Planning
\end{itemize}

\textbf{Module 3: Scoring and Analysis (12 hours)}
\begin{itemize}
\item 3.1 Ternary Scoring Methodology
\item 3.2 Evidence-Based Rating Decisions
\item 3.3 Indicator-by-Indicator Assessment
\item 3.4 Category Score Calculation
\item 3.5 CPF Score Computation
\item 3.6 Convergence Index Analysis
\item 3.7 Statistical Analysis Techniques
\item 3.8 Trend Analysis and Historical Comparison
\item 3.9 Inter-Rater Reliability
\item 3.10 Practical Exercise: Scoring Case Studies
\end{itemize}

\textbf{Module 4: Privacy-Preserving Techniques (6 hours)}
\begin{itemize}
\item 4.1 Minimum Aggregation Units Implementation
\item 4.2 Differential Privacy Mathematics
\item 4.3 Temporal Delay Mechanisms
\item 4.4 Role-Based Analysis
\item 4.5 Data Anonymization Techniques
\item 4.6 Secure Data Storage and Transmission
\item 4.7 Practical Exercise: Privacy Implementation
\end{itemize}

\textbf{Module 5: Report Writing and Communication (8 hours)}
\begin{itemize}
\item 5.1 Executive Summary Development
\item 5.2 Technical Findings Documentation
\item 5.3 Visualization of Assessment Results
\item 5.4 Risk Treatment Recommendations
\item 5.5 Stakeholder-Specific Communication
\item 5.6 Presentation Skills
\item 5.7 Handling Sensitive Findings
\item 5.8 Final Assessment: Complete Report Development
\end{itemize}

\subsection{CPF-301: Advanced Implementation (40 hours)}

\textbf{Module 1: Intervention Design (10 hours)}
\begin{itemize}
\item 1.1 From Assessment to Action
\item 1.2 Evidence-Based Intervention Selection
\item 1.3 Psychological Intervention Principles
\item 1.4 Technical Control Integration
\item 1.5 Organizational Change Management
\item 1.6 Intervention Pilot Testing
\end{itemize}

\textbf{Module 2: Continuous Monitoring (10 hours)}
\begin{itemize}
\item 2.1 Real-Time Indicator Monitoring
\item 2.2 SIEM Integration Strategies
\item 2.3 Automated Alerting Systems
\item 2.4 Dashboard Design and Implementation
\item 2.5 Convergent State Detection
\item 2.6 Continuous Monitoring Privacy Protections
\end{itemize}

\textbf{Module 3: Integration Strategies (10 hours)}
\begin{itemize}
\item 3.1 Security Operations Integration
\item 3.2 Incident Response Enhancement
\item 3.3 Threat Intelligence Augmentation
\item 3.4 Security Architecture Considerations
\item 3.5 Governance and Compliance Integration
\item 3.6 Enterprise Risk Management Alignment
\end{itemize}

\textbf{Module 4: Effectiveness Measurement (10 hours)}
\begin{itemize}
\item 4.1 Metrics and KPIs
\item 4.2 ROI Calculation Methodologies
\item 4.3 Incident Reduction Analysis
\item 4.4 Before-After Comparison Studies
\item 4.5 Continuous Improvement Processes
\item 4.6 Capstone Project: Implementation Plan
\end{itemize}

\subsection{CPF-401: Audit Techniques (40 hours)}

\textbf{Module 1: Audit Fundamentals (8 hours)}
\begin{itemize}
\item 1.1 ISO 19011:2018 Principles
\item 1.2 CPF-27001:2025 Requirements Overview
\item 1.3 Audit Process Overview
\item 1.4 Auditor Competencies and Ethics
\item 1.5 Independence and Objectivity
\item 1.6 Professional Conduct Standards
\end{itemize}

\textbf{Module 2: Audit Planning (8 hours)}
\begin{itemize}
\item 2.1 Audit Scope and Objectives
\item 2.2 Risk-Based Audit Planning
\item 2.3 Audit Team Selection
\item 2.4 Resource Allocation
\item 2.5 Audit Plan Development
\item 2.6 Communication with Auditee
\end{itemize}

\textbf{Module 3: Audit Execution (12 hours)}
\begin{itemize}
\item 3.1 Opening Meeting Conduct
\item 3.2 Document Review Techniques
\item 3.3 Interview Methodologies
\item 3.4 Sampling Strategies
\item 3.5 Evidence Collection and Documentation
\item 3.6 Observation Techniques
\item 3.7 Finding Development
\item 3.8 Closing Meeting Conduct
\item 3.9 Practical Exercise: Mock Audit
\end{itemize}

\textbf{Module 4: Audit Reporting (6 hours)}
\begin{itemize}
\item 4.1 Nonconformity Classification
\item 4.2 Observation and Opportunity Documentation
\item 4.3 Audit Report Structure
\item 4.4 Clear and Objective Writing
\item 4.5 Corrective Action Recommendations
\item 4.6 Report Review and Quality Assurance
\end{itemize}

\textbf{Module 5: Follow-Up and Closure (6 hours)}
\begin{itemize}
\item 5.1 Corrective Action Plan Review
\item 5.2 Verification of Corrections
\item 5.3 Effectiveness Evaluation
\item 5.4 Audit Closure Criteria
\item 5.5 Continuous Improvement from Audit Findings
\item 5.6 Final Practical Examination Preparation
\end{itemize}

\section{Application Forms}

\subsection{Individual Certification Application Template}

\textbf{CPF Certification Application}

\textit{Certification Type (Select One):}
\begin{itemize}
\item[$\square$] CPF Assessor
\item[$\square$] CPF Practitioner
\item[$\square$] CPF Auditor
\end{itemize}

\textbf{Personal Information:}
\begin{itemize}
\item Full Legal Name: \underline{\hspace{10cm}}
\item Preferred Name: \underline{\hspace{10cm}}
\item Date of Birth: \underline{\hspace{4cm}}
\item Email Address: \underline{\hspace{10cm}}
\item Phone Number: \underline{\hspace{6cm}}
\item Mailing Address: \underline{\hspace{10cm}}
\item \underline{\hspace{12cm}}
\end{itemize}

\textbf{Education (Bachelor's or Higher):}
\begin{itemize}
\item Institution: \underline{\hspace{10cm}}
\item Degree Type: \underline{\hspace{6cm}} Major: \underline{\hspace{5cm}}
\item Date Conferred: \underline{\hspace{4cm}}
\item Official Transcript Attached: $\square$ Yes $\square$ No
\end{itemize}

\textbf{Professional Experience:}

\textit{Current/Most Recent Position:}
\begin{itemize}
\item Employer: \underline{\hspace{10cm}}
\item Position Title: \underline{\hspace{10cm}}
\item Dates: \underline{\hspace{3cm}} to \underline{\hspace{3cm}}
\item Relevant Responsibilities: \underline{\hspace{10cm}}
\item \underline{\hspace{12cm}}
\item \underline{\hspace{12cm}}
\end{itemize}

\textit{Additional Relevant Experience (attach additional pages if needed):}

\textbf{Training Completion:}
\begin{itemize}
\item[$\square$] CPF-101: Framework Fundamentals
\begin{itemize}
\item Date Completed: \underline{\hspace{4cm}}
\item Training Provider: \underline{\hspace{6cm}}
\item Certificate Number: \underline{\hspace{6cm}}
\end{itemize}
\item[$\square$] CPF-201: Assessment Methodology (Assessor/Auditor only)
\begin{itemize}
\item Date Completed: \underline{\hspace{4cm}}
\item Training Provider: \underline{\hspace{6cm}}
\item Certificate Number: \underline{\hspace{6cm}}
\end{itemize}
\item[$\square$] CPF-401: Audit Techniques (Auditor only)
\begin{itemize}
\item Date Completed: \underline{\hspace{4cm}}
\item Training Provider: \underline{\hspace{6cm}}
\item Certificate Number: \underline{\hspace{6cm}}
\end{itemize}
\item[$\square$] ISO 19011 Auditor Training (Auditor only)
\begin{itemize}
\item Date Completed: \underline{\hspace{4cm}}
\item Training Provider: \underline{\hspace{6cm}}
\end{itemize}
\end{itemize}

\textbf{Professional References (Minimum 2):}

\textit{Reference 1:}
\begin{itemize}
\item Name: \underline{\hspace{8cm}} Title: \underline{\hspace{5cm}}
\item Organization: \underline{\hspace{10cm}}
\item Email: \underline{\hspace{8cm}} Phone: \underline{\hspace{5cm}}
\item Relationship: \underline{\hspace{10cm}}
\end{itemize}

\textit{Reference 2:}
\begin{itemize}
\item Name: \underline{\hspace{8cm}} Title: \underline{\hspace{5cm}}
\item Organization: \underline{\hspace{10cm}}
\item Email: \underline{\hspace{8cm}} Phone: \underline{\hspace{5cm}}
\item Relationship: \underline{\hspace{10cm}}
\end{itemize}

\textbf{Code of Ethics Acknowledgment:}

I have read and agree to abide by the CPF Code of Ethics, including:
\begin{itemize}
\item[$\square$] Maintaining integrity and objectivity
\item[$\square$] Protecting confidentiality of assessment data
\item[$\square$] Practicing within competence boundaries
\item[$\square$] Implementing privacy-preserving methodologies
\item[$\square$] Never using assessment data for individual profiling
\item[$\square$] Adhering to all professional conduct requirements
\end{itemize}

\textbf{Declaration:}

I declare that the information provided in this application is true, complete, and accurate to the best of my knowledge. I understand that false or misleading information may result in denial of certification or revocation of certification if already granted.

Signature: \underline{\hspace{8cm}} Date: \underline{\hspace{4cm}}

\textbf{Required Attachments:}
\begin{itemize}
\item[$\square$] Official transcript(s) or degree certificate(s)
\item[$\square$] Experience verification letters or professional portfolio
\item[$\square$] Training completion certificates
\item[$\square$] Current resume/CV
\item[$\square$] Application fee payment confirmation
\item[$\square$] Government-issued photo ID (copy)
\end{itemize}

\textbf{Application Fee:}
\begin{itemize}
\item CPF Assessor: \$300
\item CPF Practitioner: \$200
\item CPF Auditor: \$400
\end{itemize}

Payment Method: $\square$ Credit Card $\square$ Bank Transfer $\square$ Check

\textit{Submit completed application with all required attachments to:}

CPF Certification Body\\
Certification Applications Department\\
Email: certification@cpf-cert.org\\
Web Portal: https://apply.cpf-cert.org

\subsection{Organizational Certification Application Template}

\textbf{CPF Organizational Certification Application}

\textit{Target Certification Level (Select One):}
\begin{itemize}
\item[$\square$] Level 1: Foundation (CPF Score 100-149)
\item[$\square$] Level 2: Intermediate (CPF Score 70-99)
\item[$\square$] Level 3: Advanced (CPF Score 40-69)
\item[$\square$] Level 4: Exemplary (CPF Score 0-39)
\end{itemize}

\textbf{Organization Information:}
\begin{itemize}
\item Legal Organization Name: \underline{\hspace{10cm}}
\item Operating Name (if different): \underline{\hspace{10cm}}
\item Industry Sector: \underline{\hspace{10cm}}
\item Organization Size: $\square$ 1-50 $\square$ 51-250 $\square$ 251-1000 $\square$ 1000+
\item Headquarters Location: \underline{\hspace{10cm}}
\item Website: \underline{\hspace{10cm}}
\end{itemize}

\textbf{Primary Contact:}
\begin{itemize}
\item Name: \underline{\hspace{8cm}} Title: \underline{\hspace{5cm}}
\item Email: \underline{\hspace{8cm}} Phone: \underline{\hspace{5cm}}
\end{itemize}

\textbf{CPF Coordinator:}
\begin{itemize}
\item Name: \underline{\hspace{8cm}} Title: \underline{\hspace{5cm}}
\item Email: \underline{\hspace{8cm}} Phone: \underline{\hspace{5cm}}
\item CPF Certification (if applicable): \underline{\hspace{6cm}}
\end{itemize}

\textbf{Certification Scope:}
\begin{itemize}
\item Locations Covered: \underline{\hspace{10cm}}
\item \underline{\hspace{12cm}}
\item Business Units Covered: \underline{\hspace{10cm}}
\item \underline{\hspace{12cm}}
\item Total Personnel in Scope: \underline{\hspace{4cm}}
\item Exclusions (if any): \underline{\hspace{10cm}}
\item \underline{\hspace{12cm}}
\end{itemize}

\textbf{CPF Assessment Information:}
\begin{itemize}
\item Assessment Date: \underline{\hspace{4cm}}
\item Certified Assessor/Auditor Name: \underline{\hspace{8cm}}
\item Certification Number: \underline{\hspace{6cm}}
\item CPF Score: \underline{\hspace{3cm}} (Range: 0-200)
\item Assessment Report Attached: $\square$ Yes $\square$ No
\end{itemize}

\textbf{Existing Certifications:}
\begin{itemize}
\item[$\square$] ISO/IEC 27001 - Certificate Number: \underline{\hspace{6cm}}
\item[$\square$] ISO 9001 - Certificate Number: \underline{\hspace{6cm}}
\item[$\square$] SOC 2 - Report Date: \underline{\hspace{6cm}}
\item[$\square$] Other: \underline{\hspace{10cm}}
\end{itemize}

\textbf{Implementation Status:}

\textit{CPF Program Elements (Check all implemented):}
\begin{itemize}
\item[$\square$] Documented CPF Policy
\item[$\square$] Designated CPF Coordinator
\item[$\square$] Privacy Protection Procedures
\item[$\square$] Risk Treatment Plans
\item[$\square$] Continuous Monitoring (if applicable)
\item[$\square$] Integration with ISMS
\item[$\square$] Management Review Process
\item[$\square$] CPE Program for Staff
\end{itemize}

\textbf{Management Commitment:}

I, as authorized representative of the organization, commit to:
\begin{itemize}
\item[$\square$] Maintaining systematic psychological vulnerability management
\item[$\square$] Providing necessary resources for CPF implementation
\item[$\square$] Complying with surveillance requirements
\item[$\square$] Implementing corrective actions as needed
\item[$\square$] Protecting privacy in all CPF activities
\item[$\square$] Participating in required management reviews
\end{itemize}

\textbf{Authorization:}

Name: \underline{\hspace{8cm}} Title: \underline{\hspace{5cm}}

Signature: \underline{\hspace{8cm}} Date: \underline{\hspace{4cm}}

\textbf{Required Attachments:}
\begin{itemize}
\item[$\square$] CPF Assessment Report (complete)
\item[$\square$] CPF Policy Document
\item[$\square$] Organizational Chart showing CPF roles
\item[$\square$] Privacy Protection Procedures
\item[$\square$] Risk Treatment Plans for Red Indicators
\item[$\square$] Evidence of ISMS integration (if applicable)
\item[$\square$] Application fee payment confirmation
\end{itemize}

\textbf{Application Fee (Based on Organization Size):}
\begin{itemize}
\item 1-50 employees: \$500
\item 51-250 employees: \$1,000
\item 251-1000 employees: \$1,500
\item 1000+ employees: \$2,000
\end{itemize}

Payment Method: $\square$ Credit Card $\square$ Bank Transfer $\square$ Check

\textit{Submit completed application with all required attachments to:}

CPF Certification Body\\
Organizational Certification Department\\
Email: org-certification@cpf-cert.org\\
Web Portal: https://apply.cpf-cert.org

\subsection{Recertification Application Template}

\textbf{CPF Recertification Application}

\textit{Current Certification:}
\begin{itemize}
\item Certification Type: \underline{\hspace{8cm}}
\item Certificate Number: \underline{\hspace{8cm}}
\item Original Certification Date: \underline{\hspace{4cm}}
\item Current Expiration Date: \underline{\hspace{4cm}}
\end{itemize}

\textbf{Personal Information:}
\begin{itemize}
\item Full Name: \underline{\hspace{10cm}}
\item Email: \underline{\hspace{8cm}} Phone: \underline{\hspace{5cm}}
\item Has your contact information changed? $\square$ Yes $\square$ No
\item If yes, provide updated information: \underline{\hspace{8cm}}
\end{itemize}

\textbf{Continuing Professional Education (CPE):}

\textit{CPE Summary (3-Year Cycle):}
\begin{itemize}
\item Year 1 Credits: \underline{\hspace{3cm}} (Required: \underline{\hspace{2cm}})
\item Year 2 Credits: \underline{\hspace{3cm}} (Required: \underline{\hspace{2cm}})
\item Year 3 Credits: \underline{\hspace{3cm}} (Required: \underline{\hspace{2cm}})
\item Total CPE Credits: \underline{\hspace{3cm}} (Required: \underline{\hspace{2cm}})
\end{itemize}

\textit{CPE Documentation:}
\begin{itemize}
\item[$\square$] Complete CPE log attached (with dates, activities, credits)
\item[$\square$] Supporting certificates/documentation attached
\item[$\square$] All activities comply with CPE policy
\end{itemize}

\textbf{Professional Experience (Past 3 Years):}

\textit{For Assessors:}
\begin{itemize}
\item Number of CPF Assessments Conducted: \underline{\hspace{3cm}}
\item Assessment Summary Attached: $\square$ Yes $\square$ No
\end{itemize}

\textit{For Practitioners:}
\begin{itemize}
\item Updated Portfolio Attached: $\square$ Yes $\square$ No
\item Number of Implementation Projects: \underline{\hspace{3cm}}
\end{itemize}

\textit{For Auditors:}
\begin{itemize}
\item Total Audit Days: \underline{\hspace{3cm}} (Required: 45)
\item Number of Lead Auditor Roles: \underline{\hspace{3cm}} (Required: 5)
\item Audit Summary Attached: $\square$ Yes $\square$ No
\end{itemize}

\textbf{Ethics Attestation:}

I attest that during the past certification period:
\begin{itemize}
\item[$\square$] I have complied with the CPF Code of Ethics
\item[$\square$] I have maintained confidentiality requirements
\item[$\square$] I have practiced within my competence boundaries
\item[$\square$] I have maintained privacy-preserving practices
\item[$\square$] I have no unresolved ethics complaints
\item[$\square$] I am not subject to any professional discipline
\end{itemize}

I agree to continue adhering to the CPF Code of Ethics for the next certification period.

Signature: \underline{\hspace{8cm}} Date: \underline{\hspace{4cm}}

\textbf{Recertification Fee:}
\begin{itemize}
\item CPF Assessor: \$400
\item CPF Practitioner: \$300
\item CPF Auditor: \$500
\item Late Recertification (within 90 days): Add \$100
\end{itemize}

Payment Method: $\square$ Credit Card $\square$ Bank Transfer $\square$ Check

\textbf{Required Attachments:}
\begin{itemize}
\item[$\square$] Complete CPE log with supporting documentation
\item[$\square$] Experience documentation (assessments, portfolio, or audits)
\item[$\square$] Professional references (if requested)
\item[$\square$] Recertification fee payment confirmation
\end{itemize}

\textit{Submit completed application with all required attachments to:}

CPF Certification Body\\
Recertification Department\\
Email: recertification@cpf-cert.org\\
Web Portal: https://recertify.cpf-cert.org

\textit{Note: Applications should be submitted 90-180 days before expiration to ensure timely processing.}

\section*{Document Control}

\textbf{Version History:}

\begin{tabular}{llp{8cm}}
\toprule
Version & Date & Changes \\
\midrule
1.0 & January 2025 & Initial release \\
\bottomrule
\end{tabular}

\vspace{1em}

\textbf{Review Schedule:}
\begin{itemize}
\item Annual review: January of each year
\item Major revision: As needed based on industry changes, research advances, or stakeholder feedback
\item Next scheduled review: January 2026
\end{itemize}

\textbf{Approval:}

Document Owner: CPF Certification Scheme Committee

Approved by: \underline{\hspace{8cm}} Date: \underline{\hspace{4cm}}

\textbf{Distribution:}
\begin{itemize}
\item All approved CPF certification bodies
\item CPF training providers
\item Public version available at: https://cpf3.org/certification-scheme
\end{itemize}

\textbf{Contact Information:}

CPF Certification Body\\
Website: https://cpf3.org\\
Email: info@cpf3.org\\
Certification Questions: certification@cpf-cert.org\\
Technical Support: support@cpf-cert.org

\vspace{2em}

\begin{center}
\textit{End of Document}
\end{center}

\end{document}