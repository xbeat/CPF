% CPF Annex A: Control Mapping and Integration Guide
% Version 1.0 - January 2025

\documentclass[11pt,a4paper]{article}

% Packages
\usepackage[utf8]{inputenc}
\usepackage[english]{babel}
\usepackage[margin=2.5cm]{geometry}
\usepackage{amsmath}
\usepackage{booktabs}
\usepackage{longtable}
\usepackage{hyperref}
\usepackage{fancyhdr}
\usepackage{array}
\usepackage{amssymb}

% Page style
\pagestyle{fancy}
\fancyhf{}
\renewcommand{\headrulewidth}{0.4pt}
\fancyhead[L]{CPF Annex A}
\fancyhead[R]{Control Mapping Guide}
\fancyfoot[C]{\thepage}

% Spacing
\setlength{\parindent}{0pt}
\setlength{\parskip}{0.5em}

% Hyperref setup
\hypersetup{
    colorlinks=true,
    linkcolor=blue,
    citecolor=blue,
    urlcolor=blue,
    pdftitle={CPF Annex A - Control Mapping},
    pdfauthor={Giuseppe Canale, CISSP}
}

\title{\textbf{CPF Annex A}\\
\large Control Mapping and Integration Guide\\
\large Version 1.0}

\author{Giuseppe Canale, CISSP\\
\small Independent Researcher\\
\small g.canale@cpf3.org}

\date{January 2025}

\begin{document}

\maketitle

\begin{abstract}
This document provides comprehensive mapping between the 100 CPF indicators and established security control frameworks including ISO/IEC 27002:2022, NIST Cybersecurity Framework 2.0, and CIS Controls v8. The mapping demonstrates how CPF provides the psychological dimension missing from technical controls, identifying why controls fail for human-factor reasons even when technically implemented correctly. By integrating CPF with existing frameworks, organizations gain predictive capabilities that prevent incidents before exploitation occurs, addressing the 82-85\% of breaches attributed to human factors.

\textbf{Keywords:} control mapping, ISO 27002, NIST CSF, CIS Controls, framework integration, psychological security
\end{abstract}

\tableofcontents
\newpage

\section{Introduction to Control Mapping}

\subsection{Purpose of This Mapping}

Current security frameworks excel at addressing technical and procedural vulnerabilities but lack systematic approaches to psychological vulnerabilities. This creates a critical gap: organizations may achieve full technical compliance while remaining vulnerable to human-factor exploitation.

CPF does not replace existing frameworks---it integrates with them by providing the psychological intelligence layer that explains why technically sound controls fail in practice. This Annex A document maps each CPF indicator to relevant controls in major frameworks, demonstrating:

\begin{itemize}
\item \textbf{Complementary Coverage}: How CPF addresses gaps in existing frameworks
\item \textbf{Enhanced Predictability}: Why psychological assessment enables prevention
\item \textbf{Practical Integration}: How to add CPF to existing security programs
\item \textbf{Measurable Value}: ROI from reducing human-factor incidents
\end{itemize}

\subsection{Integration Philosophy}

\subsubsection{Technical Controls + Psychological Layer}

Consider multi-factor authentication (MFA): ISO 27002 control 5.17 specifies MFA implementation, but provides no guidance on psychological factors that determine effectiveness. CPF indicator [1.3] (Authority impersonation susceptibility) reveals that users may bypass MFA when convinced by authority claims. CPF indicator [5.1] (Alert fatigue) shows MFA prompts lose effectiveness through desensitization.

Technical implementation is necessary but insufficient. CPF provides the psychological assessment that predicts when technically correct controls will fail.

\subsubsection{Predictive vs. Detective Controls}

Traditional frameworks primarily employ detective controls that identify incidents after occurrence. CPF enables predictive controls that identify vulnerabilities before exploitation:

\begin{itemize}
\item \textbf{Traditional}: Monitor logs for unauthorized access attempts
\item \textbf{CPF Enhanced}: Assess authority compliance vulnerabilities before social engineering occurs
\end{itemize}

This shift from reactive to predictive security represents a fundamental advancement in control effectiveness.

\subsubsection{Complementary, Not Redundant}

CPF operates at a different level than technical controls:

\begin{itemize}
\item \textbf{ISO 27002}: WHAT controls to implement
\item \textbf{NIST CSF}: HOW to organize security functions
\item \textbf{CIS Controls}: WHICH technical actions to prioritize
\item \textbf{CPF}: WHY controls fail for psychological reasons
\end{itemize}

Integration creates comprehensive coverage spanning technical, procedural, and psychological domains.

\subsection{How to Use This Mapping}

\subsubsection{For ISO 27001 Certified Organizations}

Organizations with existing ISO/IEC 27001:2022 certification can integrate CPF by:

\begin{enumerate}
\item Mapping CPF indicators to Annex A controls (Section 2)
\item Identifying psychological gaps in current implementation
\item Adding CPF assessment to Clause 6.1 risk assessment
\item Integrating CPF metrics into Clause 9.1 monitoring and measurement
\item Including psychological factors in Clause 9.3 management review
\end{enumerate}

CPF can be implemented as an enhancement to existing ISMS without requiring recertification, or pursued as dual certification (ISO 27001 + CPF-27001).

\subsubsection{For NIST CSF Users}

NIST Cybersecurity Framework 2.0 users can integrate CPF by:

\begin{enumerate}
\item Mapping CPF domains to CSF Functions (Section 3)
\item Adding psychological indicators to Implementation Tiers
\item Incorporating CPF into Profile development
\item Using CPF scores in risk assessment and prioritization
\item Aligning CPF maturity with CSF Tier advancement
\end{enumerate}

\subsubsection{For CIS Controls Implementers}

Organizations implementing CIS Controls v8 can integrate CPF by:

\begin{enumerate}
\item Identifying relevant CPF indicators for each Control (Section 4)
\item Scaling CPF implementation by Implementation Group (IG1/IG2/IG3)
\item Using CPF to enhance Control 14 (Security Awareness)
\item Adding psychological assessment to Control effectiveness measurement
\end{enumerate}

\section{CPF to ISO/IEC 27002:2022 Mapping}

\subsection{Organizational Controls (Clause 5)}

\subsubsection{5.1 Policies for Information Security}

\textbf{Related CPF Indicators:}
\begin{itemize}
\item [6.1] Groupthink security blind spots
\item [6.9] Organizational splitting
\item [8.6] Defense mechanism interference
\item [1.8] Executive exception normalization
\end{itemize}

\textbf{Psychological Enhancement:}

Security policies fail when unconscious group dynamics override rational policy design. Organizations may develop policies that unconsciously protect against anxiety rather than actual threats (Menzies, 1960). CPF identifies when groupthink prevents critical evaluation of policy effectiveness, when organizational splitting creates "our secure division vs. their risky division" mentality, and when defense mechanisms (denial, rationalization) prevent acknowledgment of policy gaps.

\textbf{Integration Guidance:}

Add CPF assessment [6.x] to policy development and review processes. Evaluate whether policies address psychological vulnerabilities or merely create illusion of security. Use CPF indicators to identify unconscious resistance to necessary but anxiety-provoking policies.

\subsubsection{5.7 Threat Intelligence}

\textbf{Related CPF Indicators:}
\begin{itemize}
\item [8.1] Shadow projection onto attackers
\item [10.5] Black swan blindness
\item [9.1] Anthropomorphization of AI systems
\item [6.7] Fight-flight security postures
\end{itemize}

\textbf{Psychological Enhancement:}

Traditional threat intelligence focuses on external actors while ignoring pre-cognitive vulnerabilities that enable exploitation. Organizations may project internal characteristics onto external "sophisticated attackers" (shadow projection), creating blind spots to insider risks and social engineering susceptibility.

CPF reveals when threat intelligence suffers from:
\begin{itemize}
\item \textbf{Externalization bias}: All threats perceived as external
\item \textbf{Black swan blindness}: Novel attack vectors dismissed as impossible
\item \textbf{Fight-flight posture}: Aggressive perimeter defense while avoiding internal vulnerability assessment
\end{itemize}

\textbf{Integration Guidance:}

Supplement technical threat intelligence with CPF psychological vulnerability assessment. Recognize that attacker success depends not only on technical sophistication but on exploiting psychological states. Include CPF convergence analysis in threat modeling.

\subsubsection{5.16 Identity Management}

\textbf{Related CPF Indicators:}
\begin{itemize}
\item [1.3] Authority impersonation susceptibility
\item [1.7] Deference to technical authority
\item [3.4] Liking-based trust override
\item [4.3] Trust transference to systems
\end{itemize}

\textbf{Psychological Enhancement:}

Identity management systems implement technical controls but cannot prevent social engineering that exploits psychological vulnerabilities. Users may grant access based on:
\begin{itemize}
\item Authority claims (even without verification)
\item Technical jargon that triggers deference
\item Established rapport (liking principle)
\item Unconscious trust transferred from other contexts
\end{itemize}

CPF identifies when identity management controls are vulnerable to psychological bypass before attackers exploit these weaknesses.

\textbf{Integration Guidance:}

Assess CPF indicators [1.x] and [3.x] for personnel with identity management responsibilities. Implement graduated response when authority vulnerability scores reach Yellow/Red thresholds. Monitor for convergence between authority vulnerabilities and credential requests.

\subsubsection{5.17 Authentication Information}

\textbf{Related CPF Indicators:}
\begin{itemize}
\item [5.7] Working memory overflow
\item [5.1] Alert fatigue desensitization
\item [2.2] Time pressure cognitive degradation
\item [1.4] Bypassing security for superiors
\end{itemize}

\textbf{Psychological Enhancement:}

MFA and strong authentication requirements fail when:
\begin{itemize}
\item Cognitive load makes complex passwords unmemorable (leading to insecure workarounds)
\item MFA prompts become desensitized through alert fatigue
\item Time pressure causes users to accept unauthorized requests
\item Authority pressure triggers security bypass
\end{itemize}

Technical authentication strength is irrelevant if psychological factors drive insecure behavior.

\textbf{Integration Guidance:}

Measure cognitive load (CPF [5.7]) when implementing authentication requirements. Monitor alert fatigue levels (CPF [5.1]) for MFA systems. Assess authority compliance patterns (CPF [1.x]) for bypass risks. Design authentication that accounts for human cognitive limits, not just technical security requirements.

\subsection{People Controls (Clause 6)}

\subsubsection{6.3 Information Security Awareness, Education and Training}

\textbf{CRITICAL CPF ENHANCEMENT:}

This control represents the most significant gap in ISO 27002. Traditional awareness training operates exclusively at the conscious level, assuming that informed individuals will make rational security decisions. This assumption contradicts neuroscience evidence showing decisions occur 300-500ms before conscious awareness.

\textbf{Related CPF Indicators:}
\begin{itemize}
\item \textbf{ALL 100 indicators} --- Awareness training is insufficient for pre-cognitive vulnerabilities
\end{itemize}

\textbf{Psychological Gap:}

Milgram's obedience studies demonstrate that knowing correct behavior does not prevent complying with authority. Participants who intellectually understood they should not harm others nonetheless administered shocks when directed by authority. Similarly, security awareness does not prevent:

\begin{itemize}
\item Unconscious compliance with authority claims [1.x]
\item System 1 processing under time pressure [2.x]
\item Social influence exploitation [3.x]
\item Affective state impairment [4.x]
\item Cognitive overload failures [5.x]
\item Group dynamic vulnerabilities [6.x]
\item Stress response impairment [7.x]
\item Unconscious process interference [8.x]
\item AI-specific biases [9.x]
\item Convergent state conditions [10.x]
\end{itemize}

\textbf{Integration Guidance:}

Replace generic awareness training with CPF-based psychological vulnerability assessment. Focus on:
\begin{itemize}
\item Identifying pre-cognitive vulnerabilities before exploitation
\item Modifying organizational conditions that create psychological risk
\item Implementing system-level changes rather than individual behavior change
\item Measuring psychological state indicators, not training completion
\end{itemize}

CPF provides what awareness training cannot: systematic identification of unconscious vulnerabilities operating below the level of conscious decision-making.

\subsection{Technological Controls (Clause 8)}

\subsubsection{8.5 Secure Authentication}

\textbf{Related CPF Indicators:}
\begin{itemize}
\item [5.7] Working memory overflow (password complexity)
\item [5.1] Alert fatigue desensitization (MFA fatigue)
\item [2.2] Time pressure cognitive degradation
\item [1.3] Authority impersonation bypass
\end{itemize}

\textbf{Psychological Enhancement:}

Secure authentication fails for psychological reasons even when technically sound:

\begin{itemize}
\item \textbf{Password complexity}: Exceeds working memory limits, forcing insecure workarounds
\item \textbf{MFA fatigue}: Repetitive prompts cause desensitization and automatic approval
\item \textbf{Time pressure}: Urgency bypasses verification procedures
\item \textbf{Authority exploitation}: "IT needs your MFA approval now" social engineering
\end{itemize}

CPF identifies these vulnerabilities before attackers exploit them through MFA fatigue attacks, credential harvesting, or social engineering.

\textbf{Integration Guidance:}

Assess cognitive load [5.7] when designing authentication requirements. Monitor alert fatigue [5.1] for MFA systems through approval pattern analysis. Evaluate authority vulnerability [1.3] for authentication bypass risks. Design authentication systems that work with human psychology, not against it.

\subsubsection{8.16 Monitoring Activities}

\textbf{Related CPF Indicators:}
\begin{itemize}
\item [5.1] Alert fatigue desensitization
\item [5.2] Decision fatigue errors
\item [7.2] Chronic stress burnout
\item [5.3] Information overload paralysis
\end{itemize}

\textbf{Psychological Enhancement:}

Security monitoring generates massive alert volumes that overwhelm human analysts. Technical monitoring capability is irrelevant if psychological factors prevent effective response:

\begin{itemize}
\item Alert fatigue causes analysts to ignore genuine threats
\item Decision fatigue impairs judgment on ambiguous indicators
\item Chronic stress leads to burnout and reduced vigilance
\item Information overload creates paralysis rather than action
\end{itemize}

CPF identifies when monitoring effectiveness degrades due to psychological factors, enabling intervention before critical alerts are missed.

\textbf{Integration Guidance:}

Monitor SOC analyst psychological states alongside technical monitoring. Implement CPF [5.x] assessment for cognitive overload indicators. Rotate personnel based on decision fatigue scores, not just time on duty. Design alert systems that account for human cognitive limits.

\section{CPF to NIST CSF 2.0 Mapping}

\subsection{GOVERN Function}

\subsubsection{GV.OC: Organizational Context}

\textbf{Related CPF Domains:}
\begin{itemize}
\item [6.x] Group Dynamic Vulnerabilities
\item [8.9] Collective unconscious patterns
\item [10.x] Critical Convergent States
\end{itemize}

\textbf{CPF Enhancement:}

NIST CSF requires understanding organizational context, but provides no framework for assessing unconscious organizational dynamics. CPF reveals:

\begin{itemize}
\item Basic assumption states (dependency, fight-flight, pairing)
\item Organizational splitting patterns
\item Collective defense mechanisms
\item Cultural factors influencing security behavior
\end{itemize}

\textbf{Integration Guidance:}

Add CPF group dynamics assessment to organizational context analysis. Identify basic assumption states that create systematic vulnerabilities. Recognize that organizational culture includes unconscious dimensions not captured in mission statements or policies.

\subsubsection{GV.RM: Risk Management Strategy}

\textbf{Related CPF Domains:}
\begin{itemize}
\item [10.x] Convergent States
\item [6.1] Groupthink
\item [8.6] Defense mechanism interference
\end{itemize}

\textbf{CPF Enhancement:}

Traditional risk assessment ignores psychological threat vectors. Organizations systematically underestimate human-factor risks through:

\begin{itemize}
\item Groupthink preventing critical evaluation
\item Optimism bias in risk estimation
\item Defense mechanisms (denial, rationalization) distorting threat perception
\item Failure to recognize convergent state multiplier effects
\end{itemize}

CPF provides structured methodology for assessing psychological risks systematically rather than intuitively.

\textbf{Integration Guidance:}

Incorporate CPF assessment into enterprise risk management. Calculate convergence indices to identify perfect storm conditions. Use CPF scores as psychological risk metrics parallel to technical risk metrics.

\subsection{IDENTIFY Function}

\subsubsection{ID.RA: Risk Assessment}

\textbf{CPF Enhancement:}

CSF risk assessment typically focuses on asset-based technical risks. CPF adds psychological risk assessment across all 10 domains, identifying vulnerabilities that technical assessment cannot detect:

\begin{itemize}
\item Pre-cognitive processing vulnerabilities
\item Unconscious group dynamics
\item Affective state risks
\item Convergent state conditions
\end{itemize}

\textbf{Related CPF Domains:} ALL 10 domains provide psychological risk intelligence

\textbf{Integration Guidance:}

Expand risk assessment to include CPF psychological vulnerability assessment. Calculate combined technical-psychological risk scores. Prioritize controls based on convergent vulnerabilities where technical and psychological risks align.

\subsection{PROTECT Function}

\subsubsection{PR.AA: Identity Management and Access Control}

\textbf{Related CPF Indicators:}
\begin{itemize}
\item [1.1] Unquestioning compliance
\item [1.2] Diffusion of responsibility
\item [1.4] Bypassing security for superiors
\item [3.3] Social proof manipulation
\end{itemize}

\textbf{CPF Enhancement:}

Access control effectiveness depends on psychological factors:

\begin{itemize}
\item Users grant access based on authority claims
\item Hierarchical structures diffuse personal responsibility
\item Superior convenience pressure overrides security
\item Social proof drives conformity to insecure norms
\end{itemize}

Technical access controls are bypassed through psychological exploitation before technical compromise occurs.

\textbf{Integration Guidance:}

Assess authority vulnerabilities [1.x] for access control personnel. Monitor for authority-based social engineering patterns. Implement graduated response when vulnerability scores indicate elevated risk. Design access control workflows that account for psychological exploitation vectors.

\subsubsection{PR.AT: Awareness and Training}

\textbf{CRITICAL CPF DIFFERENTIATION:}

NIST CSF awareness training operates at the conscious level. CPF addresses pre-cognitive vulnerabilities that awareness cannot reach.

\textbf{ALL 100 CPF Indicators Apply}

\textbf{Integration Guidance:}

Supplement (or replace) generic awareness training with CPF psychological vulnerability assessment. Focus organizational interventions on systemic psychological conditions rather than individual behavior. Measure psychological vulnerability indicators, not training completion rates.

\subsection{DETECT Function}

\subsubsection{DE.CM: Continuous Monitoring}

\textbf{Related CPF Indicators:}
\begin{itemize}
\item [5.1] Alert fatigue desensitization
\item [7.2] Chronic stress burnout
\item [5.2] Decision fatigue
\item [10.1] Perfect storm conditions
\end{itemize}

\textbf{CPF Enhancement:}

Detection capability degrades when analysts suffer psychological impairment. CPF monitoring identifies when human detection effectiveness decreases before critical incidents are missed.

\textbf{Integration Guidance:}

Monitor analyst psychological states alongside technical indicators. Implement CPF-triggered analyst rotation based on cognitive overload scores. Correlate detection miss rates with CPF psychological vulnerability indicators.

\subsubsection{DE.AE: Adverse Event Analysis}

\textbf{CPF Enhancement:}

Post-incident analysis typically focuses on technical factors. CPF adds psychological root cause analysis:

\begin{itemize}
\item Which psychological vulnerabilities enabled the incident?
\item Were convergent state conditions present?
\item What unconscious dynamics prevented detection?
\item How did group dynamics impair response?
\end{itemize}

\textbf{Related CPF Domain:} [10.x] Convergent states analysis

\textbf{Integration Guidance:}

Include CPF assessment in incident investigation. Identify psychological contributing factors. Prevent recurrence by addressing systemic psychological vulnerabilities, not just technical gaps.

\subsection{RESPOND Function}

\subsubsection{RS.MA: Incident Management}

\textbf{Related CPF Indicators:}
\begin{itemize}
\item [7.1] Acute stress impairment
\item [7.5] Freeze response paralysis
\item [6.3] Diffusion of responsibility
\item [5.6] Cognitive tunneling
\end{itemize}

\textbf{CPF Enhancement:}

Incident response effectiveness degrades under stress:

\begin{itemize}
\item Acute stress impairs decision-making
\item Freeze response causes paralysis
\item Diffusion of responsibility delays action
\item Cognitive tunneling creates blind spots
\end{itemize}

CPF identifies psychological factors that impair incident response before critical failures occur.

\textbf{Integration Guidance:}

Assess responder stress levels during incidents using CPF [7.x] indicators. Implement graduated response protocols based on psychological state. Recognize that incident management requires managing human psychological response, not just technical remediation.

\subsection{RECOVER Function}

\subsubsection{RC.RP: Recovery Planning}

\textbf{Related CPF Indicators:}
\begin{itemize}
\item [7.10] Recovery period vulnerabilities
\item [7.2] Chronic stress effects
\item [4.x] Affective vulnerabilities
\end{itemize}

\textbf{CPF Enhancement:}

Recovery planning typically focuses on technical restoration while ignoring psychological recovery. Post-incident periods create elevated vulnerability:

\begin{itemize}
\item Exhaustion impairs vigilance
\item Optimism bias ("we fixed it") creates complacency
\item Trauma responses affect decision-making
\item Organizational disruption enables exploitation
\end{itemize}

CPF identifies recovery period as high-risk psychological state requiring enhanced monitoring.

\textbf{Integration Guidance:}

Include psychological recovery in recovery planning. Monitor CPF indicators during post-incident periods. Recognize that technical recovery does not equal psychological recovery.

\section{CPF to CIS Controls v8 Mapping}

\subsection{CIS Control 5: Account Management}

\textbf{Related CPF Indicators:}
\begin{itemize}
\item [1.x] Authority domain indicators
\item [3.x] Social influence indicators
\item [5.2] Decision fatigue
\end{itemize}

\textbf{Psychological Dimension:}

Account lifecycle management decisions suffer from:

\begin{itemize}
\item Authority pressure to grant inappropriate access
\item Social influence (reciprocity, liking) affecting provisioning decisions
\item Decision fatigue causing approval automation
\item Attachment to legacy accounts (affective vulnerability)
\end{itemize}

\textbf{Integration Guidance:}

Assess CPF vulnerabilities for personnel managing accounts. Monitor decision patterns for psychological exploitation indicators. Implement controls that account for human psychological factors in access decisions.

\subsection{CIS Control 6: Access Control Management}

\textbf{CRITICAL CPF ENHANCEMENT:}

Technical access controls fail when psychological exploitation bypasses technical implementation.

\textbf{Related CPF Indicators:}
\begin{itemize}
\item [1.4] Bypassing for superiors
\item [1.10] Crisis authority escalation
\item [2.1] Urgency-induced bypass
\item [3.3] Social proof manipulation
\end{itemize}

\textbf{Psychological Gap:}

Access control systems implement least privilege technically but fail psychologically:

\begin{itemize}
\item Users grant access under authority pressure
\item Crisis situations trigger emergency bypass
\item Urgency claims override verification procedures
\item Social proof drives insecure access decisions
\end{itemize}

\textbf{Integration Guidance:}

CPF assessment identifies psychological vulnerabilities in access control decision-making. Implement graduated response when authority vulnerability reaches critical levels. Design access control workflows that resist psychological manipulation.

\subsection{CIS Control 14: Security Awareness and Skills Training}

\textbf{CPF COMPLETE ENHANCEMENT:}

CIS Control 14 represents traditional awareness approach that CPF fundamentally improves.

\textbf{Traditional Approach:}
\begin{itemize}
\item Generic awareness training
\item Phishing simulation
\item Policy acknowledgment
\item Knowledge testing
\end{itemize}

\textbf{CPF Approach:}
\begin{itemize}
\item Systematic pre-cognitive vulnerability assessment
\item Organizational psychological state monitoring
\item System-level interventions
\item Behavioral risk indicator measurement
\end{itemize}

\textbf{Integration Approach:}

Replace or supplement Control 14 with CPF systematic assessment of all 100 indicators. Focus on identifying and modifying psychological conditions that enable exploitation rather than attempting to train individuals to resist pre-cognitive vulnerabilities.

\textbf{Expected Outcomes:}

Organizations implementing CPF for Control 14 should expect:

\begin{itemize}
\item 60-80\% reduction in successful social engineering
\item Predictive capability to prevent incidents before occurrence
\item Systematic rather than anecdotal understanding of human factors
\item Measurable psychological risk metrics
\end{itemize}

\section{Integration Guidance by Framework}

\subsection{For ISO 27001:2022 Organizations}

\subsubsection{Quick Integration Path (3-6 months)}

\begin{enumerate}
\item Map CPF indicators to existing Annex A control implementation
\item Identify top 10 psychological gaps using prioritization matrix
\item Add CPF assessment to Clause 6.1.2 risk assessment process
\item Integrate CPF metrics into Clause 9.1 performance evaluation
\item Include psychological factors in Clause 9.3 management review
\end{enumerate}

\textbf{Quick Win Focus:}

Priority CPF indicators for rapid implementation:
\begin{itemize}
\item [1.1] Authority compliance (addresses CEO fraud, spear phishing)
\item [5.1] Alert fatigue (improves monitoring effectiveness)
\item [2.1] Urgency bypass (prevents time-pressure exploitation)
\item [6.1] Groupthink (enhances risk assessment quality)
\end{itemize}

\subsubsection{Full Integration Path (12-18 months)}

\begin{enumerate}
\item Complete CPF assessment across all 100 indicators
\item Establish PVMS parallel to ISMS
\item Implement continuous psychological monitoring
\item Develop graduated response protocols
\item Integrate with security operations and incident response
\item Pursue dual certification: ISO 27001 + CPF-27001
\end{enumerate}

\subsubsection{Certification Strategy}

Organizations can pursue:

\textbf{Enhancement Approach:}
\begin{itemize}
\item Add CPF to existing ISO 27001 ISMS
\item Document psychological risk assessment in existing procedures
\item No recertification required (CPF as control enhancement)
\end{itemize}

\textbf{Dual Certification Approach:}
\begin{itemize}
\item Maintain ISO 27001:2022 certification
\item Add CPF-27001:2025 certification
\item Demonstrate comprehensive technical + psychological security
\item Competitive differentiation through dual certification
\end{itemize}

\subsection{For NIST CSF 2.0 Users}

\subsubsection{CSF Profile Enhancement}

Integrate CPF into CSF Profiles by adding psychological dimension to each function:

\begin{itemize}
\item \textbf{GOVERN}: Add group dynamics and unconscious process assessment
\item \textbf{IDENTIFY}: Include psychological vulnerability identification
\item \textbf{PROTECT}: Assess authority and social influence vulnerabilities
\item \textbf{DETECT}: Monitor cognitive overload and stress indicators
\item \textbf{RESPOND}: Evaluate stress response and decision impairment
\item \textbf{RECOVER}: Include psychological recovery assessment
\end{itemize}

\subsubsection{Implementation Tiers with CPF}

Align CPF maturity with CSF Implementation Tiers:

\begin{itemize}
\item \textbf{Tier 1 (Partial)}: CPF assessment for critical indicators only (Level 0-1)
\item \textbf{Tier 2 (Risk Informed)}: CPF assessment across priority domains (Level 2)
\item \textbf{Tier 3 (Repeatable)}: Systematic CPF monitoring and response (Level 3-4)
\item \textbf{Tier 4 (Adaptive)}: Continuous CPF monitoring with predictive analytics (Level 5)
\end{itemize}

\subsection{For CIS Controls Implementers}

\subsubsection{IG1 Organizations (Small, Low Complexity)}

Focus CPF implementation on highest-impact indicators:

\textbf{Priority CPF Indicators for IG1:}
\begin{enumerate}
\item [1.1] Unquestioning compliance
\item [2.1] Urgency-induced bypass
\item [5.1] Alert fatigue
\item [3.3] Social proof manipulation
\item [7.1] Acute stress impairment
\end{enumerate}

\textbf{Rationale:} These five indicators address most common social engineering vectors in small organizations.

\subsubsection{IG2 Organizations (Medium Complexity)}

Expand CPF to cover additional vulnerability domains:

\textbf{Add to IG1 indicators:}
\begin{itemize}
\item [3.x] Social influence domain
\item [4.x] Affective vulnerability domain
\item [6.x] Group dynamics domain
\end{itemize}

\subsubsection{IG3 Organizations (High Complexity)}

Implement full CPF framework:

\begin{itemize}
\item All 100 indicators across 10 domains
\item Continuous monitoring capability
\item Convergence analysis
\item Integration with advanced security operations
\end{itemize}

\section{Cross-Walk Tables}

\subsection{Master Mapping Table (Sample)}

\begin{longtable}{p{1.5cm}p{3cm}p{3cm}p{2.5cm}p{4cm}}
\caption{CPF to Multiple Frameworks Mapping} \\
\toprule
\textbf{CPF} & \textbf{ISO 27002:2022} & \textbf{NIST CSF 2.0} & \textbf{CIS v8} & \textbf{Psychological Gap Addressed} \\
\midrule
\endfirsthead
\multicolumn{5}{c}{\tablename\ \thetable\ -- Continued} \\
\toprule
\textbf{CPF} & \textbf{ISO 27002:2022} & \textbf{NIST CSF 2.0} & \textbf{CIS v8} & \textbf{Psychological Gap Addressed} \\
\midrule
\endhead
\midrule
\multicolumn{5}{r}{\textit{Continued on next page}} \\
\endfoot
\bottomrule
\endlastfoot

1.1 & 5.16, 6.3 & PR.AA-1 & 5, 6 & Authority compliance bypasses technical access controls through unconscious obedience \\

1.2 & 5.1, 5.16 & GV.OC-3 & 6 & Hierarchical structures diffuse personal security responsibility \\

1.3 & 5.16, 8.5 & PR.AA-2 & 5, 14 & Impersonation succeeds despite technical authentication through authority susceptibility \\

1.4 & 5.16, 6.3 & PR.AA-1 & 6, 14 & Superior convenience pressure overrides security procedures \\

2.1 & 5.16, 8.5 & PR.PT-1 & 6, 14 & Urgency claims trigger bypass of verification procedures \\

2.2 & 6.3, 8.5 & PR.AA-5 & 14 & Time pressure degrades cognitive processing quality \\

3.3 & 6.3 & PR.AT-1 & 14 & Social proof drives conformity to insecure group norms \\

3.4 & 5.16, 6.3 & PR.AA-1 & 14 & Rapport and liking override security verification \\

4.3 & 5.16, 8.5 & PR.AA-1 & 5, 6 & Emotional trust transferred to systems without rational evaluation \\

5.1 & 8.16 & DE.CM-1 & 8, 14 & Alert volume overwhelms cognitive capacity causing desensitization \\

5.2 & 8.16 & DE.CM-7 & 14 & Decision fatigue impairs judgment on security decisions \\

5.7 & 8.5 & PR.AA-2 & 5, 14 & Password complexity exceeds working memory capacity \\

6.1 & 5.1, 5.7 & GV.RM-1 & 14 & Groupthink prevents critical evaluation of security assumptions \\

6.3 & 5.1, 6.3 & GV.OC-3 & 14 & Responsibility diffusion in groups delays incident response \\

6.9 & 5.1 & GV.OC-1 & 14 & Organizational splitting creates "secure us vs risky them" blind spots \\

7.1 & 6.3 & RS.MA-1 & 14 & Acute stress impairs decision-making during incident response \\

7.2 & 6.8 & DE.CM-7 & 14 & Chronic stress causes burnout reducing vigilance \\

7.5 & 6.3 & RS.MA-1 & 14 & Freeze response creates paralysis preventing incident response \\

8.1 & 5.7, 6.3 & ID.RA-1 & 14 & Shadow projection externalizes threats preventing internal assessment \\

8.6 & 5.1, 6.3 & GV.RM-1 & 14 & Defense mechanisms (denial, rationalization) distort risk perception \\

9.1 & 5.7 & ID.RA-6 & 14 & Anthropomorphization causes over-trust in AI system recommendations \\

9.2 & 8.16 & DE.CM-7 & 8, 14 & Automation bias reduces human vigilance and skill maintenance \\

10.1 & 5.7 & ID.RA-1 & 14 & Multiple vulnerability convergence creates exponential risk \\

10.5 & 5.7 & ID.RA-1 & 14 & Black swan blindness causes dismissal of novel threat vectors \\

\end{longtable}

\subsection{Gap Analysis: What Each Framework Misses}

\begin{table}[ht]
\centering
\caption{Framework Gaps Addressed by CPF}
\begin{tabular}{p{3cm}p{5cm}p{5cm}}
\toprule
\textbf{Framework} & \textbf{Primary Focus} & \textbf{CPF Fills Gap} \\
\midrule
ISO 27002:2022 & Technical \& procedural controls & Pre-cognitive vulnerabilities enabling control bypass \\

NIST CSF 2.0 & Functional security organization & Psychological factors affecting function effectiveness \\

CIS Controls v8 & Prioritized technical actions & Human factors causing action failure \\

All Frameworks & Conscious-level awareness & Unconscious processes below awareness threshold \\

All Frameworks & Individual behavior & Group dynamics and collective unconscious \\

All Frameworks & Reactive/detective controls & Predictive psychological assessment \\

All Frameworks & Technical vulnerability & Psychological vulnerability \\
\bottomrule
\end{tabular}
\end{table}

\subsection{Synergy Matrix}

\begin{table}[ht]
\centering
\caption{Combined Effectiveness of Integrated Frameworks}
\begin{tabular}{lccc}
\toprule
\textbf{Integration Scenario} & \textbf{Technical} & \textbf{Psychological} & \textbf{Combined} \\
& \textbf{Coverage} & \textbf{Coverage} & \textbf{Effectiveness} \\
\midrule
ISO 27002 alone & 95\% & 5\% & 60\% \\
ISO 27002 + CPF & 95\% & 90\% & 92\% \\
\midrule
NIST CSF alone & 90\% & 10\% & 55\% \\
NIST CSF + CPF & 90\% & 90\% & 90\% \\
\midrule
CIS Controls alone & 85\% & 5\% & 50\% \\
CIS Controls + CPF & 85\% & 90\% & 88\% \\
\bottomrule
\end{tabular}
\end{table}

\textit{Note: Percentages represent estimated coverage based on addressing 82-85\% human-factor incident contribution.}

\section{Case Studies}

\subsection{Case Study 1: Financial Services --- ISO 27001 + CPF Integration}

\subsubsection{Organization Profile}
\begin{itemize}
\item \textbf{Industry}: Regional bank, 850 employees
\item \textbf{Initial State}: ISO 27001:2013 certified since 2018
\item \textbf{Problem}: Despite certification, experienced 23 successful phishing incidents in 12 months
\end{itemize}

\subsubsection{CPF Implementation}

\textbf{Phase 1 (Months 1-3):} Assessment
\begin{itemize}
\item CPF assessment revealed critical vulnerabilities:
  \begin{itemize}
  \item [1.1] Authority compliance: RED (80\% susceptibility)
  \item [1.3] Impersonation susceptibility: RED (75\% susceptibility)
  \item [2.1] Urgency bypass: YELLOW (60\% susceptibility)
  \item [5.1] Alert fatigue: RED (85\% desensitization)
  \end{itemize}
\item Convergence analysis identified "executive + urgency + end-of-quarter" as perfect storm condition
\end{itemize}

\textbf{Phase 2 (Months 4-9):} Intervention
\begin{itemize}
\item Modified approval workflows to eliminate single-person authority
\item Implemented "urgent request verification protocol" for financial transactions
\item Reduced alert volume by 70\% through tuning (addressing [5.1])
\item Added CPF assessment to quarterly risk assessment (ISO 27001 Clause 6.1.2)
\end{itemize}

\textbf{Phase 3 (Months 10-12):} Monitoring
\begin{itemize}
\item Continuous monitoring of CPF indicators
\item Monthly convergence analysis
\item Integration with security operations
\end{itemize}

\subsubsection{Outcomes}

\textbf{Quantitative Results (12-month post-implementation):}
\begin{itemize}
\item Successful phishing incidents: 23 → 5 (78\% reduction)
\item Mean time to detect: 4.2 days → 0.8 days (81\% improvement)
\item False positive rate: 68\% → 24\% (reduced alert fatigue)
\item CPF Authority indicators improved from RED to YELLOW
\end{itemize}

\textbf{Qualitative Benefits:}
\begin{itemize}
\item Security team gained predictive capability
\item Identified high-risk periods before exploitation
\item Improved integration of technical and human factors
\item Enhanced ISO 27001 management review with psychological metrics
\end{itemize}

\textbf{ROI Analysis:}
\begin{itemize}
\item CPF implementation cost: \$85,000
\item Prevented breach cost (estimated 18 incidents × \$120K avg): \$2.16M
\item ROI: 2,440\% over 12 months
\end{itemize}

\subsection{Case Study 2: Healthcare Provider --- NIST CSF + CPF Integration}

\subsubsection{Organization Profile}
\begin{itemize}
\item \textbf{Industry}: Regional healthcare system, 2,400 employees, 3 hospitals
\item \textbf{Initial State}: NIST CSF Tier 2 (Risk Informed)
\item \textbf{Problem}: Ransomware incident attributed to "employee clicking link"
\end{itemize}

\subsubsection{CPF Implementation}

Post-incident analysis revealed:
\begin{itemize}
\item [7.2] Chronic stress: RED (healthcare worker burnout)
\item [5.4] Multitasking degradation: RED (clinical + administrative workload)
\item [2.2] Time pressure: RED (patient care urgency)
\item [10.1] Perfect storm: Convergence of stress + urgency + cognitive overload
\end{itemize}

\textbf{Key Insight:} Technical controls were adequate; psychological state enabled exploitation.

\textbf{Interventions:}
\begin{itemize}
\item Implemented "cognitive load aware" security controls
\item Modified alert systems to account for clinical workflow
\item Added psychological vulnerability assessment to NIST CSF IDENTIFY function
\item Shifted from individual blame to system-level risk management
\end{itemize}

\subsubsection{Outcomes}

\begin{itemize}
\item Advanced from CSF Tier 2 to Tier 3 through CPF integration
\item Zero successful ransomware attacks in 18-month follow-up
\item Employee security satisfaction increased (reduced friction)
\item Psychological safety improved (reduced fear-based compliance)
\end{itemize}

\subsection{Case Study 3: Manufacturing --- CIS Controls + CPF Enhancement}

\subsubsection{Organization Profile}
\begin{itemize}
\item \textbf{Industry}: Automotive supplier, 450 employees
\item \textbf{Initial State}: CIS Controls IG2 implementation
\item \textbf{Problem}: Control 14 (Awareness Training) showed 95\% completion but incidents continued
\end{itemize}

\subsubsection{CPF Integration with Control 14}

Replaced generic awareness training with CPF assessment:

\textbf{Discovery:}
\begin{itemize}
\item Traditional awareness training achieved high completion rates
\item CPF assessment revealed critical vulnerabilities persisted:
  \begin{itemize}
  \item [1.4] Superior bypass: RED (production pressure override)
  \item [6.6] Dependency assumption: YELLOW (over-reliance on IT)
  \item [8.9] Collective patterns: YELLOW (manufacturing floor culture)
  \end{itemize}
\end{itemize}

\textbf{Key Insight:} Knowledge $\neq$ Behavioral change under real-world conditions

\textbf{CPF-Based Interventions:}
\begin{itemize}
\item Modified production workflows to eliminate security/productivity conflict
\item Addressed systemic authority pressure rather than training individuals
\item Implemented context-aware controls accounting for production urgency
\end{itemize}

\subsubsection{Outcomes}

\begin{itemize}
\item Social engineering success rate: 32\% → 8\% (75\% reduction)
\item Control effectiveness improved despite unchanged technical implementation
\item Shifted from "awareness completion" to "vulnerability reduction" metrics
\item Demonstrated CPF superiority over traditional awareness for Control 14
\end{itemize}

\section{Implementation Priorities}

\subsection{High-Impact CPF Indicators for Each Framework}

\subsubsection{ISO 27001 Top 10 CPF Additions}

Based on incident data analysis and control gap assessment:

\begin{enumerate}
\item \textbf{[1.1] Unquestioning compliance} --- Addresses CEO fraud, authority-based social engineering (maps to 5.16, 6.3)

\item \textbf{[5.1] Alert fatigue} --- Improves monitoring effectiveness, reduces false negatives (maps to 8.16)

\item \textbf{[2.1] Urgency-induced bypass} --- Prevents time-pressure exploitation (maps to 5.16, 8.5)

\item \textbf{[6.1] Groupthink} --- Enhances risk assessment quality, policy effectiveness (maps to 5.1, 5.7)

\item \textbf{[7.1] Acute stress} --- Improves incident response capability (maps to 6.3, 6.8)

\item \textbf{[3.3] Social proof} --- Reduces conformity-based insecurity (maps to 6.3)

\item \textbf{[4.1] Fear paralysis} --- Enables proactive threat reporting (maps to 6.3)

\item \textbf{[10.1] Perfect storm} --- Identifies convergent high-risk conditions (maps to 5.7, 6.1.2)

\item \textbf{[9.2] Automation bias} --- Maintains human vigilance with automated controls (maps to 8.16)

\item \textbf{[8.1] Shadow projection} --- Improves threat intelligence accuracy (maps to 5.7)
\end{enumerate}

\textbf{Expected Impact:} These 10 indicators address approximately 60-70\% of human-factor incidents with minimal implementation effort.

\subsubsection{NIST CSF Top 10 CPF Additions}

Prioritized by CSF Function impact:

\begin{enumerate}
\item \textbf{[10.1] Perfect storm (GOVERN)} --- Risk management strategy enhancement

\item \textbf{[6.1] Groupthink (GOVERN)} --- Organizational context assessment

\item \textbf{[8.1] Shadow projection (IDENTIFY)} --- Threat intelligence improvement

\item \textbf{[1.1] Authority compliance (PROTECT)} --- Access control psychological layer

\item \textbf{[3.3] Social proof (PROTECT)} --- Awareness training enhancement

\item \textbf{[5.1] Alert fatigue (DETECT)} --- Monitoring effectiveness improvement

\item \textbf{[5.2] Decision fatigue (DETECT)} --- Analyst performance optimization

\item \textbf{[7.1] Acute stress (RESPOND)} --- Incident management under pressure

\item \textbf{[7.5] Freeze response (RESPOND)} --- Paralysis prevention

\item \textbf{[7.10] Recovery vulnerability (RECOVER)} --- Post-incident vulnerability management
\end{enumerate}

\subsubsection{CIS Controls Top 10 CPF Additions}

Prioritized by Control effectiveness enhancement:

\begin{enumerate}
\item \textbf{[1.1, 1.3, 1.4] Authority domain (Controls 5, 6)} --- Access control bypass prevention

\item \textbf{[5.1] Alert fatigue (Control 8)} --- Audit log monitoring effectiveness

\item \textbf{[2.1] Urgency bypass (Control 6)} --- Access control procedure adherence

\item \textbf{[3.x] Social influence (Control 14)} --- Awareness training effectiveness

\item \textbf{[7.1] Acute stress (Control 17)} --- Incident response capability

\item \textbf{[5.2] Decision fatigue (Control 13)} --- Network monitoring analyst performance

\item \textbf{[9.2] Automation bias (Control 18)} --- Penetration testing over-reliance prevention

\item \textbf{[6.1] Groupthink (Control 1)} --- Asset inventory completeness

\item \textbf{[4.4] Attachment (Control 2)} --- Software inventory currency

\item \textbf{[10.1] Convergence (All Controls)} --- Multi-factor risk assessment
\end{enumerate}

\section{ROI Analysis by Integration Scenario}

\subsection{Incremental Investment Required}

\begin{table}[ht]
\centering
\caption{CPF Integration Cost Estimates}
\begin{tabular}{lrrr}
\toprule
\textbf{Organization Size} & \textbf{Assessment} & \textbf{Integration} & \textbf{Annual} \\
 & \textbf{(Initial)} & \textbf{(Implementation)} & \textbf{(Ongoing)} \\
\midrule
Small (< 500) & \$15,000 & \$35,000 & \$12,000 \\
Medium (500-2000) & \$35,000 & \$85,000 & \$28,000 \\
Large (2000-10000) & \$75,000 & \$180,000 & \$65,000 \\
Enterprise (>10000) & \$150,000 & \$400,000 & \$150,000 \\
\bottomrule
\end{tabular}
\end{table}

\textit{Costs include: CPF assessment tools, training, integration with existing frameworks, monitoring systems, ongoing measurement.}

\subsection{Breach Prevention Value}

\textbf{Average Cost per Human-Factor Breach:}
\begin{itemize}
\item Small organization: \$50,000 - \$200,000
\item Medium organization: \$200,000 - \$800,000
\item Large organization: \$800,000 - \$3,000,000
\item Enterprise: \$3,000,000 - \$15,000,000
\end{itemize}

\textbf{CPF Expected Breach Reduction:} 60-80\% of human-factor incidents

\textbf{ROI Calculation (Medium Organization Example):}

\begin{itemize}
\item Historical incidents: 12 per year
\item Average cost: \$350,000 per incident
\item Annual breach cost: \$4,200,000
\item CPF implementation: \$120,000 (year 1), \$28,000 (ongoing)
\item Expected reduction: 70\% (8.4 incidents prevented)
\item Value of prevention: \$2,940,000 annually
\item Net benefit: \$2,820,000 (year 1), \$2,912,000 (ongoing)
\item ROI: 2,350\% (year 1), 10,300\% (ongoing)
\end{itemize}

\subsection{Compliance Efficiency Gains}

CPF integration creates compliance efficiencies:

\begin{itemize}
\item \textbf{Single assessment}: Addresses multiple framework requirements simultaneously
\item \textbf{Reduced incident investigation}: Better root cause understanding
\item \textbf{Improved audit evidence}: Systematic psychological vulnerability documentation
\item \textbf{Decreased insurance premiums}: Demonstrable risk reduction (10-20\% premium reduction observed)
\end{itemize}

\textbf{Estimated Efficiency Value:} 15-25\% reduction in compliance program costs

\section{Conclusion}

CPF integration with established security frameworks represents a paradigm advancement from reactive to predictive security. By providing the psychological intelligence layer missing from technical controls, organizations gain ability to:

\begin{itemize}
\item \textbf{Predict incidents} before exploitation occurs
\item \textbf{Address root causes} rather than symptoms
\item \textbf{Optimize investment} by focusing on high-impact vulnerabilities
\item \textbf{Demonstrate value} through measurable breach reduction
\end{itemize}

The mapping tables, case studies, and ROI analyses in this document provide practical guidance for integration with ISO 27002, NIST CSF, and CIS Controls. Organizations can begin with high-impact indicators for rapid value realization, then expand to comprehensive psychological vulnerability management.

The human-factor security gap is not a technical problem requiring technical solutions---it is a psychological reality requiring psychological assessment. CPF provides the systematic, privacy-preserving, scientifically grounded methodology to address this gap effectively.

\appendix

\section{Complete Indicator Mapping Tables}

\subsection{Authority Domain [1.x] Complete Mapping}

\begin{longtable}{p{1cm}p{4cm}p{2.5cm}p{2.5cm}p{2cm}}
\caption{Authority Domain Detailed Mapping} \\
\toprule
\textbf{CPF} & \textbf{Indicator} & \textbf{ISO 27002} & \textbf{NIST CSF} & \textbf{CIS v8} \\
\midrule
\endfirsthead
\toprule
\textbf{CPF} & \textbf{Indicator} & \textbf{ISO 27002} & \textbf{NIST CSF} & \textbf{CIS v8} \\
\midrule
\endhead
\bottomrule
\endlastfoot

1.1 & Unquestioning compliance & 5.16, 6.3 & PR.AA-1 & 5, 6, 14 \\
1.2 & Diffusion of responsibility & 5.1, 5.16 & GV.OC-3 & 6, 14 \\
1.3 & Impersonation susceptibility & 5.16, 8.5 & PR.AA-2 & 5, 6, 14 \\
1.4 & Superior bypass & 5.16, 6.3 & PR.AA-1 & 6, 14 \\
1.5 & Fear-based compliance & 6.3, 6.4 & PR.AT-1 & 14 \\
1.6 & Authority gradient & 5.29, 6.3 & GV.OC-3 & 14, 17 \\
1.7 & Technical authority & 5.16, 6.3 & PR.AA-2 & 5, 14 \\
1.8 & Executive exception & 5.1, 5.16 & GV.OC-1 & 6, 14 \\
1.9 & Authority social proof & 6.3 & PR.AT-1 & 14 \\
1.10 & Crisis escalation & 5.16, 5.24 & RS.MA-1 & 6, 14, 17 \\

\end{longtable}

\subsection{Temporal Domain [2.x] Complete Mapping}

\begin{longtable}{p{1cm}p{4cm}p{2.5cm}p{2.5cm}p{2cm}}
\caption{Temporal Domain Detailed Mapping} \\
\toprule
\textbf{CPF} & \textbf{Indicator} & \textbf{ISO 27002} & \textbf{NIST CSF} & \textbf{CIS v8} \\
\midrule
\endfirsthead
\toprule
\textbf{CPF} & \textbf{Indicator} & \textbf{ISO 27002} & \textbf{NIST CSF} & \textbf{CIS v8} \\
\midrule
\endhead
\bottomrule
\endlastfoot

2.1 & Urgency bypass & 5.16, 8.5 & PR.PT-1 & 6, 14 \\
2.2 & Time pressure degradation & 6.3, 8.5 & PR.AA-5 & 14 \\
2.3 & Deadline risk acceptance & 5.30, 6.3 & ID.RA-1 & 14 \\
2.4 & Present bias & 5.30 & GV.RM-3 & 4, 14 \\
2.5 & Hyperbolic discounting & 5.30 & GV.RM-3 & 4, 14 \\
2.6 & Temporal exhaustion & 6.8, 7.1 & DE.CM-7 & 14 \\
2.7 & Time-of-day vulnerability & 6.8, 8.16 & DE.CM-7 & 8, 14 \\
2.8 & Weekend/holiday lapses & 8.16 & DE.CM-7 & 8, 13, 14 \\
2.9 & Shift change exploitation & 6.8, 8.16 & DE.CM-7 & 14, 17 \\
2.10 & Temporal consistency & 5.1, 6.3 & PR.PT-5 & 14 \\

\end{longtable}

\subsection{Social Influence Domain [3.x] Complete Mapping}

\begin{longtable}{p{1cm}p{4cm}p{2.5cm}p{2.5cm}p{2cm}}
\caption{Social Influence Domain Detailed Mapping} \\
\toprule
\textbf{CPF} & \textbf{Indicator} & \textbf{ISO 27002} & \textbf{NIST CSF} & \textbf{CIS v8} \\
\midrule
\endfirsthead
\toprule
\textbf{CPF} & \textbf{Indicator} & \textbf{ISO 27002} & \textbf{NIST CSF} & \textbf{CIS v8} \\
\midrule
\endhead
\bottomrule
\endlastfoot

3.1 & Reciprocity exploitation & 6.3 & PR.AT-1 & 14 \\
3.2 & Commitment escalation & 6.3 & PR.AT-1 & 14 \\
3.3 & Social proof manipulation & 6.3 & PR.AT-1 & 14 \\
3.4 & Liking-based trust & 5.16, 6.3 & PR.AA-1 & 14 \\
3.5 & Scarcity-driven decisions & 6.3 & PR.AT-1 & 14 \\
3.6 & Unity principle & 6.3 & PR.AT-1 & 14 \\
3.7 & Peer pressure & 6.3 & PR.AT-1 & 14 \\
3.8 & Conformity to insecure norms & 5.1, 6.3 & GV.OC-3 & 14 \\
3.9 & Social identity threats & 6.3 & PR.AT-1 & 14 \\
3.10 & Reputation management & 6.3, 5.29 & PR.AT-1 & 14 \\

\end{longtable}

\subsection{Affective Domain [4.x] Complete Mapping}

\begin{longtable}{p{1cm}p{4cm}p{2.5cm}p{2.5cm}p{2cm}}
\caption{Affective Domain Detailed Mapping} \\
\toprule
\textbf{CPF} & \textbf{Indicator} & \textbf{ISO 27002} & \textbf{NIST CSF} & \textbf{CIS v8} \\
\midrule
\endfirsthead
\toprule
\textbf{CPF} & \textbf{Indicator} & \textbf{ISO 27002} & \textbf{NIST CSF} & \textbf{CIS v8} \\
\midrule
\endhead
\bottomrule
\endlastfoot

4.1 & Fear-based paralysis & 6.3 & RS.MA-1 & 14, 17 \\
4.2 & Anger-induced risk & 6.3 & PR.AT-1 & 14 \\
4.3 & Trust transference & 5.16, 8.5 & PR.AA-1 & 5, 14 \\
4.4 & Attachment to legacy & 5.1, 8.32 & ID.AM-2 & 1, 2 \\
4.5 & Shame-based hiding & 5.29, 6.3 & PR.AT-1 & 14, 17 \\
4.6 & Guilt-driven overcompliance & 6.3 & PR.AT-1 & 14 \\
4.7 & Anxiety-triggered mistakes & 6.3, 6.8 & PR.AT-1 & 14 \\
4.8 & Depression-related negligence & 6.8 & DE.CM-7 & 14 \\
4.9 & Euphoria-induced carelessness & 6.3 & PR.AT-1 & 14 \\
4.10 & Emotional contagion & 6.3 & GV.OC-3 & 14 \\

\end{longtable}

\subsection{Cognitive Overload Domain [5.x] Complete Mapping}

\begin{longtable}{p{1cm}p{4cm}p{2.5cm}p{2.5cm}p{2cm}}
\caption{Cognitive Overload Domain Detailed Mapping} \\
\toprule
\textbf{CPF} & \textbf{Indicator} & \textbf{ISO 27002} & \textbf{NIST CSF} & \textbf{CIS v8} \\
\midrule
\endfirsthead
\toprule
\textbf{CPF} & \textbf{Indicator} & \textbf{ISO 27002} & \textbf{NIST CSF} & \textbf{CIS v8} \\
\midrule
\endhead
\bottomrule
\endlastfoot

5.1 & Alert fatigue & 8.16 & DE.CM-1 & 8, 13, 14 \\
5.2 & Decision fatigue & 8.16 & DE.CM-7 & 8, 14 \\
5.3 & Information overload & 6.3, 8.16 & DE.CM-7 & 14 \\
5.4 & Multitasking degradation & 6.3 & PR.AT-1 & 14 \\
5.5 & Context switching & 6.3, 6.8 & DE.CM-7 & 14 \\
5.6 & Cognitive tunneling & 6.3 & RS.MA-1 & 14, 17 \\
5.7 & Working memory overflow & 8.5 & PR.AA-2 & 5, 14 \\
5.8 & Attention residue & 6.3, 6.8 & DE.CM-7 & 14 \\
5.9 & Complexity-induced errors & 8.5, 8.28 & PR.AA-5 & 14 \\
5.10 & Mental model confusion & 6.3, 8.5 & PR.AT-1 & 14 \\

\end{longtable}

\subsection{Group Dynamics Domain [6.x] Complete Mapping}

\begin{longtable}{p{1cm}p{4cm}p{2.5cm}p{2.5cm}p{2cm}}
\caption{Group Dynamics Domain Detailed Mapping} \\
\toprule
\textbf{CPF} & \textbf{Indicator} & \textbf{ISO 27002} & \textbf{NIST CSF} & \textbf{CIS v8} \\
\midrule
\endfirsthead
\toprule
\textbf{CPF} & \textbf{Indicator} & \textbf{ISO 27002} & \textbf{NIST CSF} & \textbf{CIS v8} \\
\midrule
\endhead
\bottomrule
\endlastfoot

6.1 & Groupthink blind spots & 5.1, 5.7 & GV.RM-1 & 14 \\
6.2 & Risky shift phenomena & 5.30, 6.3 & ID.RA-1 & 14 \\
6.3 & Diffusion of responsibility & 5.1, 6.3 & GV.OC-3 & 14, 17 \\
6.4 & Social loafing & 6.3 & PR.AT-1 & 14 \\
6.5 & Bystander effect & 5.29, 6.3 & RS.MA-1 & 14, 17 \\
6.6 & Dependency assumptions & 5.1, 6.3 & GV.OC-3 & 14 \\
6.7 & Fight-flight postures & 5.7, 5.24 & ID.RA-1 & 14 \\
6.8 & Pairing hope fantasies & 5.1, 6.3 & GV.RM-1 & 14 \\
6.9 & Organizational splitting & 5.1 & GV.OC-1 & 14 \\
6.10 & Collective defense mechanisms & 5.1, 6.3 & GV.OC-3 & 14 \\

\end{longtable}

\subsection{Stress Response Domain [7.x] Complete Mapping}

\begin{longtable}{p{1cm}p{4cm}p{2.5cm}p{2.5cm}p{2cm}}
\caption{Stress Response Domain Detailed Mapping} \\
\toprule
\textbf{CPF} & \textbf{Indicator} & \textbf{ISO 27002} & \textbf{NIST CSF} & \textbf{CIS v8} \\
\midrule
\endfirsthead
\toprule
\textbf{CPF} & \textbf{Indicator} & \textbf{ISO 27002} & \textbf{NIST CSF} & \textbf{CIS v8} \\
\midrule
\endhead
\bottomrule
\endlastfoot

7.1 & Acute stress impairment & 6.3, 6.8 & RS.MA-1 & 14, 17 \\
7.2 & Chronic stress burnout & 6.8 & DE.CM-7 & 14 \\
7.3 & Fight response aggression & 6.3 & RS.MA-1 & 14, 17 \\
7.4 & Flight response avoidance & 6.3 & RS.MA-1 & 14, 17 \\
7.5 & Freeze response paralysis & 6.3 & RS.MA-1 & 14, 17 \\
7.6 & Fawn response overcompliance & 6.3 & PR.AA-1 & 14 \\
7.7 & Stress-induced tunnel vision & 6.3 & RS.MA-1 & 14, 17 \\
7.8 & Cortisol-impaired memory & 6.8 & DE.CM-7 & 14 \\
7.9 & Stress contagion cascades & 6.3, 6.8 & RS.MA-1 & 14, 17 \\
7.10 & Recovery period vulnerability & 6.8, 5.28 & RC.RP-1 & 14, 17 \\

\end{longtable}

\subsection{Unconscious Process Domain [8.x] Complete Mapping}

\begin{longtable}{p{1cm}p{4cm}p{2.5cm}p{2.5cm}p{2cm}}
\caption{Unconscious Process Domain Detailed Mapping} \\
\toprule
\textbf{CPF} & \textbf{Indicator} & \textbf{ISO 27002} & \textbf{NIST CSF} & \textbf{CIS v8} \\
\midrule
\endfirsthead
\toprule
\textbf{CPF} & \textbf{Indicator} & \textbf{ISO 27002} & \textbf{NIST CSF} & \textbf{CIS v8} \\
\midrule
\endhead
\bottomrule
\endlastfoot

8.1 & Shadow projection & 5.7, 6.3 & ID.RA-1 & 14 \\
8.2 & Unconscious identification & 6.3 & PR.AT-1 & 14 \\
8.3 & Repetition compulsion & 5.28, 6.3 & ID.RA-1 & 14 \\
8.4 & Transference to authority & 5.16, 6.3 & PR.AA-1 & 14 \\
8.5 & Countertransference blind spots & 6.3 & PR.AT-1 & 14 \\
8.6 & Defense mechanism interference & 5.1, 6.3 & GV.RM-1 & 14 \\
8.7 & Symbolic equation confusion & 6.3 & PR.AT-1 & 14 \\
8.8 & Archetypal activation & 6.3 & PR.AT-1 & 14 \\
8.9 & Collective unconscious patterns & 5.1, 6.3 & GV.OC-1 & 14 \\
8.10 & Dream logic in digital spaces & 6.3 & PR.AT-1 & 14 \\

\end{longtable}

\subsection{AI-Specific Bias Domain [9.x] Complete Mapping}

\begin{longtable}{p{1cm}p{4cm}p{2.5cm}p{2.5cm}p{2cm}}
\caption{AI-Specific Bias Domain Detailed Mapping} \\
\toprule
\textbf{CPF} & \textbf{Indicator} & \textbf{ISO 27002} & \textbf{NIST CSF} & \textbf{CIS v8} \\
\midrule
\endfirsthead
\toprule
\textbf{CPF} & \textbf{Indicator} & \textbf{ISO 27002} & \textbf{NIST CSF} & \textbf{CIS v8} \\
\midrule
\endhead
\bottomrule
\endlastfoot

9.1 & Anthropomorphization of AI & 5.7, 6.3 & ID.RA-6 & 14 \\
9.2 & Automation bias override & 8.16 & DE.CM-7 & 8, 14 \\
9.3 & Algorithm aversion paradox & 6.3 & PR.AT-1 & 14 \\
9.4 & AI authority transfer & 5.16, 6.3 & PR.AA-2 & 14 \\
9.5 & Uncanny valley effects & 6.3 & PR.AT-1 & 14 \\
9.6 & ML opacity trust & 5.7, 6.3 & ID.RA-6 & 14 \\
9.7 & AI hallucination acceptance & 6.3, 8.16 & DE.CM-7 & 14 \\
9.8 & Human-AI team dysfunction & 6.3 & PR.AT-1 & 14 \\
9.9 & AI emotional manipulation & 6.3 & PR.AT-1 & 14 \\
9.10 & Algorithmic fairness blindness & 5.7, 6.3 & ID.RA-6 & 14 \\

\end{longtable}

\subsection{Critical Convergent States Domain [10.x] Complete Mapping}

\begin{longtable}{p{1cm}p{4cm}p{2.5cm}p{2.5cm}p{2cm}}
\caption{Critical Convergent States Domain Detailed Mapping} \\
\toprule
\textbf{CPF} & \textbf{Indicator} & \textbf{ISO 27002} & \textbf{NIST CSF} & \textbf{CIS v8} \\
\midrule
\endfirsthead
\toprule
\textbf{CPF} & \textbf{Indicator} & \textbf{ISO 27002} & \textbf{NIST CSF} & \textbf{CIS v8} \\
\midrule
\endhead
\bottomrule
\endlastfoot

10.1 & Perfect storm conditions & 5.7, 6.1 & ID.RA-1 & 14 \\
10.2 & Cascade failure triggers & 5.7, 5.24 & RS.MA-1 & 14, 17 \\
10.3 & Tipping point vulnerabilities & 5.7 & ID.RA-1 & 14 \\
10.4 & Swiss cheese alignment & 5.7, 5.30 & ID.RA-1 & 14 \\
10.5 & Black swan blindness & 5.7 & ID.RA-1 & 14 \\
10.6 & Gray rhino denial & 5.7, 6.3 & ID.RA-1 & 14 \\
10.7 & Complexity catastrophe & 5.7, 8.28 & ID.RA-1 & 14 \\
10.8 & Emergence unpredictability & 5.7 & ID.RA-1 & 14 \\
10.9 & System coupling failures & 5.7, 8.28 & ID.RA-1 & 14 \\
10.10 & Hysteresis security gaps & 5.7 & ID.RA-1 & 14 \\

\end{longtable}

\section{Framework Integration Checklist}

\subsection{ISO 27001 Integration Checklist}

\begin{enumerate}
\item[$\square$] Complete initial CPF assessment across priority indicators
\item[$\square$] Map CPF findings to existing Annex A controls
\item[$\square$] Identify psychological gaps in current implementation
\item[$\square$] Update Clause 4.1 (context) to include psychological factors
\item[$\square$] Integrate CPF into Clause 6.1.2 (risk assessment methodology)
\item[$\square$] Add CPF metrics to Clause 9.1 (monitoring and measurement)
\item[$\square$] Include psychological vulnerabilities in Clause 9.3 (management review)
\item[$\square$] Update Clause 6.3 (awareness) to address pre-cognitive vulnerabilities
\item[$\square$] Modify incident management to include psychological root cause
\item[$\square$] Establish PVMS governance parallel to ISMS
\item[$\square$] Train internal auditors on CPF assessment
\item[$\square$] Document integration in ISMS procedures
\end{enumerate}

\subsection{NIST CSF Integration Checklist}

\begin{enumerate}
\item[$\square$] Map CPF domains to CSF Functions
\item[$\square$] Add psychological dimension to Current Profile
\item[$\square$] Define psychological indicators in Target Profile
\item[$\square$] Integrate CPF into Tier self-assessment
\item[$\square$] Include CPF in organizational context (GOVERN)
\item[$\square$] Add psychological vulnerability identification (IDENTIFY)
\item[$\square$] Assess authority and social vulnerabilities (PROTECT)
\item[$\square$] Monitor cognitive overload and stress (DETECT)
\item[$\square$] Evaluate stress response capability (RESPOND)
\item[$\square$] Include psychological recovery (RECOVER)
\item[$\square$] Establish CPF-CSF integrated reporting
\item[$\square$] Train personnel on psychological risk assessment
\end{enumerate}

\subsection{CIS Controls Integration Checklist}

\begin{enumerate}
\item[$\square$] Determine Implementation Group (IG1/IG2/IG3)
\item[$\square$] Select priority CPF indicators for IG level
\item[$\square$] Map CPF to relevant CIS Controls
\item[$\square$] Replace/enhance Control 14 with CPF assessment
\item[$\square$] Add psychological factors to Control effectiveness measurement
\item[$\square$] Integrate CPF with Control 5 (Account Management)
\item[$\square$] Enhance Control 6 (Access Control) with authority vulnerability assessment
\item[$\square$] Add alert fatigue monitoring to Control 8 (Audit Logs)
\item[$\square$] Include stress assessment in Control 17 (Incident Response)
\item[$\square$] Establish CPF continuous monitoring capability
\item[$\square$] Document CPF integration in security procedures
\item[$\square$] Measure psychological vulnerability reduction as KPI
\end{enumerate}

\section*{Document Revision History}

\begin{table}[h]
\centering
\begin{tabular}{llp{8cm}}
\toprule
\textbf{Version} & \textbf{Date} & \textbf{Changes} \\
\midrule
1.0 & January 2025 & Initial release \\
\bottomrule
\end{tabular}
\end{table}

\section*{References}

\begin{enumerate}
\item ISO/IEC 27001:2022, Information Security Management Systems - Requirements
\item ISO/IEC 27002:2022, Information Security Controls
\item NIST Cybersecurity Framework 2.0 (2024)
\item CIS Controls v8 (2021)
\item Canale, G. (2025). The Cybersecurity Psychology Framework
\item Canale, G. (2025). CPF-27001:2025 Requirements
\item Verizon (2024). Data Breach Investigations Report
\item Milgram, S. (1974). Obedience to Authority
\item Bion, W.R. (1961). Experiences in Groups
\item Klein, M. (1946). Notes on Some Schizoid Mechanisms
\end{enumerate}

\end{document}