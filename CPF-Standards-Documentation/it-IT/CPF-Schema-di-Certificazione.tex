\documentclass[11pt,a4paper]{article}

% Pacchetti
\usepackage[utf8]{inputenc}
\usepackage[italian]{babel}
\usepackage[margin=2.5cm]{geometry}
\usepackage{amsmath}
\usepackage{booktabs}
\usepackage{hyperref}
\usepackage{fancyhdr}
\usepackage{enumitem}
\usepackage{amssymb}

% Stile pagina
\pagestyle{fancy}
\fancyhf{}
\renewcommand{\headrulewidth}{0.4pt}
\fancyhead[L]{Schema di Certificazione CPF}
\fancyhead[R]{Versione 1.0}
\fancyfoot[C]{\thepage}

% Spacing
\setlength{\parindent}{0pt}
\setlength{\parskip}{0.5em}

% Hyperref setup
\hypersetup{
    colorlinks=true,
    linkcolor=blue,
    citecolor=blue,
    urlcolor=blue,
    pdftitle={Schema di Certificazione CPF},
    pdfauthor={Giuseppe Canale, CISSP},
}

% Titolo
\title{\textbf{Schema di Certificazione CPF}\\
\large Versione 1.0}
\author{Organismo di Certificazione CPF\\
Giuseppe Canale, CISSP\\
\small Ricercatore Indipendente\\
\small g.canale@cpf3.org}
\date{Gennaio 2025}

\begin{document}

\maketitle

\begin{abstract}
Questo documento definisce lo schema di certificazione per il Cybersecurity Psychology Framework (CPF), includendo i requisiti per le certificazioni individuali (CPF Assessor, CPF Practitioner, CPF Auditor) e le certificazioni organizzative (Livelli di Conformità CPF). Lo schema è progettato in conformità ai requisiti ISO/IEC 17065:2012 per gli organismi di certificazione che gestiscono sistemi di certificazione di prodotti, processi e servizi. Questo schema di certificazione consente la validazione sistematica delle competenze nella valutazione delle vulnerabilità psicologiche e della capacità organizzativa nell'implementazione di controlli di sicurezza basati su CPF. Il framework affronta il divario critico tra controlli di sicurezza tecnici e fattori umani, fornendo percorsi standardizzati per professionisti e organizzazioni per dimostrare la padronanza della gestione delle vulnerabilità pre-cognitive.

\textbf{Parole chiave:} certificazione, valutazione delle competenze, ISO/IEC 17065, psicologia della cybersecurity, sviluppo professionale, maturità organizzativa
\end{abstract}

\tableofcontents
\newpage

\section{Introduzione}

\subsection{Scopo e Ambito}

Lo Schema di Certificazione CPF stabilisce requisiti completi per la certificazione di individui e organizzazioni nella valutazione e mitigazione sistematica delle vulnerabilità psicologiche nei contesti di cybersecurity. Questo schema affronta il divario fondamentale nei programmi di certificazione cybersecurity attuali, che si concentrano prevalentemente sulle competenze tecniche trascurando le dimensioni psicologiche che contribuiscono all'82-85\% degli incidenti di sicurezza.

L'ambito di questo schema di certificazione include:

\textbf{Certificazioni Individuali:}
\begin{itemize}
\item Certificazione CPF Assessor: Valida la competenza nella conduzione di valutazioni sistematiche delle vulnerabilità psicologiche utilizzando la metodologia CPF
\item Certificazione CPF Practitioner: Conferma l'applicazione pratica dei principi CPF in contesti organizzativi
\item Certificazione CPF Auditor: Certifica la capacità di verificare l'implementazione e la conformità CPF
\end{itemize}

\textbf{Certificazioni Organizzative:}
\begin{itemize}
\item Livello di Conformità CPF 1-4: Valida la maturità organizzativa nella gestione delle vulnerabilità psicologiche attraverso quattro livelli progressivi
\end{itemize}

Questo schema non sostituisce le certificazioni cybersecurity esistenti (CISSP, CISM, CEH, ecc.) ma le integra affrontando le dimensioni psicologiche assenti dai programmi di certificazione tecnica.

\subsection{Benefici della Certificazione}

La certificazione CPF fornisce valore misurabile a individui, organizzazioni e alla più ampia comunità di cybersecurity:

\textbf{Per gli Individui:}
\begin{itemize}
\item Differenziazione nel mercato del lavoro cybersecurity competitivo
\item Riconoscimento dell'expertise interdisciplinare che collega psicologia e sicurezza
\item Avanzamento di carriera attraverso la validazione di competenze specializzate
\item Percorso di sviluppo professionale in una disciplina emergente
\item Accesso alla comunità professionale CPF e alla formazione continua
\end{itemize}

\textbf{Per le Organizzazioni:}
\begin{itemize}
\item Riduzione degli incidenti di sicurezza legati al fattore umano attraverso la gestione sistematica delle vulnerabilità psicologiche
\item Postura di sicurezza migliorata affrontando l'82-85\% delle violazioni con componenti umane
\item Vantaggio competitivo dimostrando impegno per una sicurezza completa
\item Miglioramento della posizione assicurativa attraverso la validazione della riduzione del rischio
\item Supporto alla conformità normativa per i requisiti di sicurezza del fattore umano
\item ROI misurabile attraverso la riduzione degli incidenti e la mitigazione del rischio
\end{itemize}

\textbf{Per la Comunità:}
\begin{itemize}
\item Standardizzazione delle pratiche di valutazione delle vulnerabilità psicologiche
\item Avanzamento della psicologia della cybersecurity come disciplina riconosciuta
\item Condivisione della conoscenza attraverso la rete di professionisti certificati
\item Avanzamento della ricerca attraverso dati di implementazione validati
\item Maturazione del settore oltre gli approcci puramente tecnici
\end{itemize}

\subsection{Struttura Organizzativa}

L'ecosistema di certificazione CPF opera attraverso una gerarchia strutturata di entità, ciascuna con ruoli e responsabilità distinti.

\subsubsection{Organizzazione CPF3}

CPF3 svolge ruoli duali nell'ecosistema di certificazione:

\textbf{Proprietario dello Schema (Ruolo Permanente):}

\begin{itemize}
\item Detiene tutti i diritti di proprietà intellettuale sul framework e sulla metodologia CPF
\item Definisce e mantiene i requisiti e gli standard di certificazione
\item Sviluppa contenuti d'esame e curricula formativi
\item Autorizza e concede licenze agli organismi di certificazione per operare lo schema
\item Fornisce supervisione della qualità delle attività di certificazione
\item Mantiene il registro centrale delle certificazioni
\item Promuove la ricerca e l'evoluzione del framework
\item Protegge il marchio e i marchi registrati CPF
\end{itemize}

\textbf{Primo Organismo di Certificazione (Periodo di Grazia Iniziale):}

\begin{itemize}
\item Opera come organismo di certificazione primario durante la fase di sviluppo del mercato (periodo di grazia di 3-5 anni)
\item Certifica i primi gruppi di professionisti e organizzazioni
\item Sviluppa procedure operative e sistemi di qualità
\item Stabilisce la credibilità sul mercato
\item Trasferisce gradualmente le operazioni a organismi terzi autorizzati
\end{itemize}

\subsubsection{Ecosistema degli Organismi di Certificazione}

Man mano che il programma matura, più organismi di certificazione possono essere autorizzati:

\textbf{Requisiti:}
\begin{itemize}
\item Accreditamento ISO/IEC 17065:2012 da organismo riconosciuto a livello nazionale
\item Autorizzazione da CPF3 attraverso Accordo di Licenza dello Schema
\item Competenza tecnica in cybersecurity e psicologia
\item Personale qualificato inclusi CPF Auditor certificati
\item Infrastruttura sicura per esami e gestione dati
\end{itemize}

\textbf{Operazioni:}
\begin{itemize}
\item Certificare individui (Assessor, Practitioner, Auditor)
\item Certificare organizzazioni (Livelli 1-4)
\item Certificare Fornitori di Servizi Autorizzati
\item Condurre sorveglianza e ricertificazione
\item Riferire a CPF3 trimestralmente
\end{itemize}

\subsubsection{Modelli di Pratica Professionale}

I professionisti certificati possono operare attraverso quattro modelli:

\textbf{Pratica Indipendente:}
\begin{itemize}
\item Consulenti freelance che forniscono servizi direttamente
\item Responsabilità professionale individuale
\item Non possono usare la designazione "Fornitore di Servizi Autorizzato"
\item Piena autonomia sugli incarichi
\end{itemize}

\textbf{Impiego presso Organismo di Certificazione:}
\begin{itemize}
\item Tipicamente CPF Auditor
\item Conducono audit organizzativi
\item Devono mantenere indipendenza secondo ISO 19011
\item Non possono fare consulenza e audit per la stessa organizzazione
\end{itemize}

\textbf{Società di Consulenza Non Certificata:}
\begin{itemize}
\item Impiega professionisti certificati
\item Può dichiarare "Impieghiamo professionisti certificati CPF"
\item Non può usare il marchio "Fornitore di Servizi Autorizzato"
\item Barriera d'ingresso inferiore
\end{itemize}

\textbf{Fornitore di Servizi Autorizzato (Società Certificata):}
\begin{itemize}
\item Certificazione organizzativa per società di consulenza
\item Requisiti minimi di personale certificato (varia in base alla dimensione)
\item Sistema di gestione della qualità richiesto
\item Autorizzazione a usare il marchio "Fornitore di Servizi Autorizzato CPF"
\item Benefici e referral migliorati
\item Maggiore credibilità sul mercato
\end{itemize}

\subsubsection{Panoramica del Fornitore di Servizi Autorizzato}

Le società di consulenza possono opzionalmente perseguire la certificazione come Fornitore di Servizi Autorizzato:

\textbf{Riepilogo dei Requisiti:}
\begin{itemize}
\item Personale certificato minimo (2-20 in base alla dimensione della società)
\item Sistema di gestione della qualità
\item Framework di protezione della privacy con audit annuale
\item Assicurazione di responsabilità professionale (\$1M-\$5M in base alla dimensione)
\item Audit di sorveglianza semestrali
\end{itemize}

\textbf{Riepilogo dei Benefici:}
\begin{itemize}
\item Uso del marchio "Fornitore di Servizi Autorizzato CPF"
\item Inserimento in directory con posizionamento prioritario
\item Referral da CPF3
\item Accesso a strumenti e template esclusivi
\item Sconti volume sulle certificazioni (10-20\%)
\item Accesso anticipato agli aggiornamenti del framework
\item Opportunità di co-marketing
\end{itemize}

Vedere la Sezione 2.3 per i requisiti completi di certificazione del Fornitore di Servizi Autorizzato.

\subsection{Relazione con ISO/IEC 17065}

Questo schema di certificazione è progettato per essere gestito da organismi di certificazione conformi a ISO/IEC 17065:2012, che specifica i requisiti per gli organismi che certificano prodotti, processi e servizi. La certificazione CPF costituisce certificazione di processo e servizio secondo le definizioni ISO/IEC 17065.

Principi chiave ISO/IEC 17065 applicati in questo schema:

\textbf{Imparzialità:} Gli organismi di certificazione devono mantenere indipendenza dai fornitori di formazione, consulenze e entità certificate. I conflitti di interesse sono sistematicamente identificati e gestiti.

\textbf{Competenza:} Il personale dell'organismo di certificazione deve dimostrare competenza sia nel dominio della cybersecurity che della psicologia, validata attraverso requisiti di istruzione, formazione ed esperienza.

\textbf{Coerenza:} Le decisioni di certificazione seguono criteri di valutazione standardizzati assicurando risultati comparabili tra diversi valutatori, organizzazioni e periodi temporali.

\textbf{Riservatezza:} I dati di valutazione, in particolare le informazioni sulle vulnerabilità psicologiche, ricevono protezione della riservatezza elevata oltre i requisiti standard ISO/IEC 17065.

\textbf{Reattività ai Reclami:} Gli organismi di certificazione implementano procedure robuste di indagine e risoluzione dei reclami con meccanismi di appello per le decisioni di certificazione.

Gli organismi di certificazione che operano questo schema devono mantenere l'accreditamento ISO/IEC 17065 attraverso organismi di accreditamento riconosciuti a livello nazionale. Questo assicura il riconoscimento internazionale e l'accettazione reciproca delle certificazioni CPF tra le giurisdizioni.

\section{Livelli di Certificazione}

\subsection{Certificazioni Individuali}

\subsubsection{Certificazione CPF Assessor}

La certificazione CPF Assessor valida la competenza nel condurre valutazioni sistematiche delle vulnerabilità psicologiche utilizzando la metodologia CPF. Gli Assessor devono dimostrare sia conoscenza teorica che capacità pratica di identificare, valutare e documentare le vulnerabilità psicologiche in tutti i dieci domini CPF.

\textbf{Requisiti:}

\textit{Istruzione:}
\begin{itemize}
\item Laurea in Psicologia, Scienze Comportamentali, Psicologia Organizzativa, OPPURE
\item Laurea in Cybersecurity, Sicurezza Informatica, Informatica PIÙ 40 ore di formazione specifica CPF documentata sui fondamenti psicologici
\end{itemize}

\textit{Esperienza:}
\begin{itemize}
\item Minimo 2 anni di esperienza professionale in cybersecurity O psicologia
\item Almeno 6 mesi devono coinvolgere lavoro rilevante per la sicurezza
\item Documentazione dell'esperienza attraverso verifica del datore di lavoro o portfolio professionale
\end{itemize}

\textit{Formazione:}
\begin{itemize}
\item CPF-101: Fondamenti del Framework (40 ore)
\item CPF-201: Metodologia di Valutazione (40 ore)
\item Totale: 80 ore di formazione obbligatoria
\item La formazione deve essere completata entro 24 mesi dalla domanda di certificazione
\end{itemize}

\textit{Esame:}
\begin{itemize}
\item Esame Scritto: 100 domande che coprono tutti i dieci domini CPF, metodologia di valutazione, protezione della privacy e considerazioni etiche
\item Durata: 3 ore
\item Punteggio di Superamento: 70\% (70 risposte corrette)
\item Distribuzione Domande: 60 a scelta multipla, 30 basate su scenari, 10 di analisi casi
\item Valutazione Pratica: Analisi di caso studio organizzativo con identificazione vulnerabilità, giustificazione del punteggio e raccomandazioni di intervento
\item Durata: 4 ore
\item Requisito di Superamento: Competenza dimostrata in tutti i criteri di valutazione
\end{itemize}

\textit{Requisiti Etici:}
\begin{itemize}
\item Accordo al Codice Etico CPF
\item Impegno a pratiche di valutazione che preservano la privacy
\item Comprensione della sensibilità delle vulnerabilità psicologiche
\item Divieto di utilizzare dati di valutazione per valutazione delle prestazioni individuali
\end{itemize}

\textbf{Validità della Certificazione:} 3 anni dalla data di emissione

\textbf{Requisiti di Ricertificazione:}
\begin{itemize}
\item 40 crediti di Educazione Professionale Continua (CPE) per anno (120 totali in 3 anni)
\item Minimo 5 valutazioni documentate utilizzando la metodologia CPF
\item Revisione etica e accordo aggiornato
\item Formazione continua sui progressi psicologici e di cybersecurity
\end{itemize}

\subsubsection{Certificazione CPF Practitioner}

La certificazione CPF Practitioner valida l'applicazione pratica dei principi CPF nei contesti organizzativi. I Practitioner implementano interventi basati su CPF, traducono i risultati delle valutazioni in miglioramenti di sicurezza attuabili e integrano la gestione delle vulnerabilità psicologiche con i programmi di sicurezza esistenti.

\textbf{Requisiti:}

\textit{Istruzione:}
\begin{itemize}
\item Laurea in campo pertinente (Psicologia, Cybersecurity, Comportamento Organizzativo, Amministrazione Aziendale con focus sulla sicurezza)
\item Comprensione documentata sia dei principi di sicurezza che di psicologia organizzativa
\end{itemize}

\textit{Esperienza:}
\begin{itemize}
\item Minimo 1 anno di utilizzo della metodologia CPF in contesto organizzativo
\item Documentazione dell'implementazione pratica attraverso portfolio di progetti
\item Integrazione dimostrata di CPF con programmi di sicurezza esistenti
\item Evidenza di progettazione e implementazione di interventi
\end{itemize}

\textit{Formazione:}
\begin{itemize}
\item CPF-101: Fondamenti del Framework (40 ore)
\item Nessuna formazione obbligatoria aggiuntiva oltre ai fondamenti
\item Raccomandato: CPF-201 Metodologia di Valutazione per competenza migliorata
\end{itemize}

\textit{Esame:}
\begin{itemize}
\item Esame Scritto: 75 domande focalizzate su applicazione pratica, progettazione di interventi e integrazione organizzativa
\item Durata: 2,5 ore
\item Punteggio di Superamento: 70\% (53 risposte corrette)
\item Distribuzione Domande: 45 a scelta multipla, 20 basate su scenari, 10 problemi applicativi
\item Revisione Portfolio: Presentazione di lavoro pratico che dimostra implementazione CPF
\item Valutazione: Evidenza di applicazione sistematica, efficacia degli interventi, integrazione organizzativa
\end{itemize}

\textit{Requisiti del Portfolio:}
\begin{itemize}
\item Minimo 3 progetti di implementazione con risultati documentati
\item Evidenza del pipeline valutazione-intervento
\item Documentazione di integrazione con controlli di sicurezza esistenti
\item Protezione della privacy dimostrata nella pratica
\item Esempi di coinvolgimento e comunicazione con gli stakeholder
\end{itemize}

\textbf{Validità della Certificazione:} 3 anni dalla data di emissione

\textbf{Requisiti di Ricertificazione:}
\begin{itemize}
\item 30 crediti CPE per anno (90 totali in 3 anni)
\item Portfolio aggiornato che dimostra applicazione pratica continua
\item Revisione etica e rinnovo dell'accordo
\item Partecipazione alla comunità di pratica CPF practitioner
\end{itemize}

\subsubsection{Certificazione CPF Auditor}

La certificazione CPF Auditor valida la competenza nell'auditare l'implementazione organizzativa della metodologia CPF, valutare la conformità ai requisiti CPF-27001:2025 e valutare l'efficacia dei sistemi di gestione delle vulnerabilità psicologiche.

\textbf{Requisiti:}

\textit{Prerequisiti:}
\begin{itemize}
\item Certificazione CPF Assessor attuale in buono stato
\item Minimo 1 anno di esperienza come CPF Assessor certificato
\item Completamento documentato di almeno 10 valutazioni CPF
\end{itemize}

\textit{Istruzione:}
\begin{itemize}
\item Requisiti educativi esistenti soddisfatti attraverso la certificazione CPF Assessor
\item Formazione aggiuntiva in metodologia di audit (ISO 19011:2018)
\item Comprensione dei principi di audit dei sistemi di gestione
\end{itemize}

\textit{Esperienza:}
\begin{itemize}
\item Partecipazione a minimo 3 audit CPF sotto supervisione di CPF Auditor certificato
\item Minimo 20 giorni di audit documentati
\item Esperienza in diversi tipi e dimensioni organizzative
\item Competenza dimostrata in pianificazione, esecuzione e reporting di audit
\end{itemize}

\textit{Formazione:}
\begin{itemize}
\item CPF-401: Tecniche di Audit (40 ore)
\item Formazione Auditor ISO 19011:2018 (minimo 24 ore)
\item Integrazione della valutazione psicologica con l'auditing dei sistemi di gestione
\end{itemize}

\textit{Esame:}
\begin{itemize}
\item Esame Scritto: 80 domande che coprono metodologia di audit, requisiti CPF-27001, competenze dell'auditor e condotta professionale
\item Durata: 3 ore
\item Punteggio di Superamento: 75\% (60 risposte corrette)
\item Esame Scenari di Audit: Conduzione di audit simulato includendo pianificazione, esecuzione, documentazione dei risultati e generazione del report
\item Durata: 8 ore (esame pratico di una giornata intera)
\item Requisito di Superamento: Competenza di audit dimostrata in tutti i criteri di valutazione
\end{itemize}

\textit{Requisiti di Condotta Professionale:}
\begin{itemize}
\item Aderenza ai principi dell'auditor ISO 19011 (integrità, imparzialità, riservatezza)
\item Protezione della privacy migliorata per i dati sulle vulnerabilità psicologiche
\item Indipendenza dai servizi di consulenza e implementazione
\item Approccio di audit basato su evidenze oggettive
\end{itemize}

\textbf{Validità della Certificazione:} 3 anni dalla data di emissione

\textbf{Requisiti di Ricertificazione:}
\begin{itemize}
\item 50 crediti CPE per anno (150 totali in 3 anni), con minimo 30 crediti in argomenti specifici di audit
\item Minimo 15 giorni di audit per anno come lead o co-auditor (45 totali in 3 anni)
\item Competenza di audit continua dimostrata attraverso presentazione di report di audit
\item Revisione etica e condotta professionale
\item Partecipazione ad attività di calibrazione degli auditor
\end{itemize}

\subsection{Certificazioni Organizzative}

\subsubsection{Livelli di Conformità CPF}

La certificazione organizzativa valida l'implementazione sistematica della gestione delle vulnerabilità psicologiche secondo i requisiti CPF-27001:2025. Quattro livelli di conformità progressivi riflettono la maturità organizzativa nell'affrontare i rischi di sicurezza legati al fattore umano.

\textbf{Fondamento del Punteggio:}

La conformità organizzativa si basa sui Punteggi CPF aggregati derivati dalla valutazione sistematica di tutti i 100 indicatori CPF nell'ambito dell'organizzazione. Il Punteggio CPF varia da 0-200, con punteggi inferiori che indicano una migliore postura di sicurezza (vulnerabilità minori e meno gravi).

Metodologia di punteggio:
\begin{itemize}
\item Ogni indicatore valutato usando sistema ternario: Verde (0), Giallo (1), Rosso (2)
\item Punteggio Categoria = Somma di 10 indicatori per categoria (range 0-20)
\item Punteggio CPF = Somma di 10 punteggi di categoria (range 0-200)
\item Valutazione condotta da CPF Assessor o Auditor certificato
\item Ambito minimo di valutazione: Campione rappresentativo che assicura aggregazione che preserva la privacy (minimo 10 individui per unità di aggregazione)
\end{itemize}

\textbf{Livello 1: Fondazione (Punteggio CPF 100-149)}

\textit{Caratteristiche di Maturità:}
\begin{itemize}
\item Implementazione CPF iniziale con consapevolezza base delle vulnerabilità psicologiche
\item Approccio reattivo agli incidenti legati al fattore umano
\item Integrazione limitata con programmi di sicurezza esistenti
\item Pratiche base di valutazione che preservano la privacy
\end{itemize}

\textit{Requisiti Minimi:}
\begin{itemize}
\item Punteggio CPF tra 100-149 da valutazione certificata
\item Policy CPF documentata approvata dall'alta direzione
\item Coordinatore CPF designato con responsabilità definite
\item Completamento della formazione CPF-101 da parte della leadership di sicurezza
\item Valutazione base condotta che copre tutti i 10 domini
\item Piano documentato di trattamento del rischio per indicatori Rossi
\item Procedure di protezione della privacy implementate
\item Piano di integrazione con Sistema di Gestione della Sicurezza delle Informazioni (ISMS) esistente
\end{itemize}

\textit{Requisiti di Sorveglianza:}
\begin{itemize}
\item Valutazione annuale da CPF Assessor certificato
\item Reporting trimestrale dello stato degli indicatori Rossi
\item Riesame annuale della direzione del programma CPF
\end{itemize}

\textbf{Livello 2: Intermedio (Punteggio CPF 70-99)}

\textit{Caratteristiche di Maturità:}
\begin{itemize}
\item Processi sistematici di gestione delle vulnerabilità psicologiche
\item Identificazione e trattamento proattivo delle vulnerabilità
\item Approccio integrato che combina controlli psicologici e tecnici
\item Framework di protezione della privacy stabilito
\item Riduzione dimostrabile degli incidenti legati al fattore umano
\end{itemize}

\textit{Requisiti Minimi:}
\begin{itemize}
\item Punteggio CPF tra 70-99 da valutazione certificata
\item Tutti i requisiti del Livello 1 mantenuti
\item Minimo un CPF Assessor certificato nello staff o a contratto
\item Cicli di valutazione trimestrali che coprono tutti i domini
\item Tracciamento documentato dell'efficacia degli interventi
\item Integrazione con Security Operations Center (SOC) per monitoraggio continuo degli indicatori critici
\item Programma di Educazione Professionale Continua (CPE) stabilito per lo staff di sicurezza
\item Dashboard che preserva la privacy per il monitoraggio delle vulnerabilità psicologiche
\item Riduzione documentata degli incidenti legati al fattore umano (minimo 20\% anno su anno)
\end{itemize}

\textit{Requisiti di Sorveglianza:}
\begin{itemize}
\item Valutazione completa semestrale da CPF Auditor certificato
\item Auto-valutazione trimestrale con validazione dell'Assessor certificato
\item Reporting mensile degli indicatori Rossi
\item Riesame della direzione semestrale
\end{itemize}

\textbf{Livello 3: Avanzato (Punteggio CPF 40-69)}

\textit{Caratteristiche di Maturità:}
\begin{itemize}
\item Programma maturo di gestione delle vulnerabilità psicologiche
\item Identificazione predittiva degli stati convergenti
\item Integrazione sofisticata attraverso tutti i domini di sicurezza
\item Pratiche leader di protezione della privacy
\item Riduzione sostanziale delle violazioni legate al fattore umano
\item Cultura organizzativa di consapevolezza della sicurezza psicologica
\end{itemize}

\textit{Requisiti Minimi:}
\begin{itemize}
\item Punteggio CPF tra 40-69 da valutazione certificata
\item Tutti i requisiti del Livello 1 e Livello 2 mantenuti
\item Minimo due CPF Assessor certificati nello staff
\item Monitoraggio continuo di tutti i 100 indicatori integrato con SIEM
\item Analytics predittiva per identificazione degli stati convergenti
\item Alerting automatizzato per pattern critici di vulnerabilità psicologica
\item Riduzione documentata del 40\% degli incidenti legati al fattore umano rispetto alla baseline
\item Protezione della privacy avanzata usando privacy differenziale ($\epsilon \leq 0.1$)
\item Contributo alla ricerca CPF e alla base di conoscenza della comunità
\item Integrazione con gestione del rischio di terze parti per sicurezza psicologica della supply chain
\item Considerazioni sulle vulnerabilità psicologiche in tutti i processi di gestione del cambiamento
\end{itemize}

\textit{Requisiti di Sorveglianza:}
\begin{itemize}
\item Valutazione completa annuale da CPF Auditor certificato
\item Auto-monitoraggio continuo con validazione trimestrale
\item Alerting e risposta in tempo reale per indicatori Rossi
\item Riesame della direzione trimestrale
\item Audit esterno annuale delle pratiche di protezione della privacy
\end{itemize}

\textbf{Livello 4: Esemplare (Punteggio CPF 0-39)}

\textit{Caratteristiche di Maturità:}
\begin{itemize}
\item Gestione delle vulnerabilità psicologiche di classe mondiale
\item Postura di sicurezza predittiva che previene incidenti prima che si verifichino
\item Integrazione completa attraverso tutte le funzioni organizzative
\item Protezione della privacy e pratiche etiche leader nel settore
\item Quasi eliminazione delle violazioni prevenibili legate al fattore umano
\item Cultura organizzativa di resilienza psicologica
\item Contributo all'avanzamento del settore
\end{itemize}

\textit{Requisiti Minimi:}
\begin{itemize}
\item Punteggio CPF tra 0-39 da valutazione certificata
\item Tutti i requisiti del Livello 1, 2 e 3 mantenuti
\item Team CPF dedicato includendo multipli Assessor certificati e almeno un Auditor certificato
\item Monitoraggio delle vulnerabilità psicologiche in tempo reale con analytics predittiva potenziata da AI
\item Zero indicatori Rossi mantenuti per minimo 6 mesi
\item Massimo 10\% indicatori Gialli con piani di trattamento documentati
\item Riduzione documentata del 60\% degli incidenti legati al fattore umano rispetto alla baseline
\item Ricerca pubblicata o case study che avanzano la metodologia CPF
\item Contributo attivo alla comunità CPF attraverso condivisione di conoscenza, formazione o sviluppo di strumenti
\item Integrazione della gestione delle vulnerabilità psicologiche attraverso la supply chain
\item Considerazioni sulla sicurezza psicologica incorporate nella gestione del rischio aziendale
\item Pratiche avanzate di protezione della privacy che superano i requisiti di privacy differenziale
\item Validazione regolare da terze parti delle pratiche di privacy ed etiche
\end{itemize}

\subsection{Certificazione Fornitore di Servizi Autorizzato}

La certificazione Fornitore di Servizi Autorizzato valida le società di consulenza e le organizzazioni di servizi professionali che forniscono servizi relativi a CPF mantenendo standard specifici di qualità, personale ed etici.

\subsubsection{Scopo e Ambito}

Questa certificazione organizzativa opzionale fornisce differenziazione di mercato per le società che impiegano multipli professionisti CPF certificati e mantengono una gestione sistematica della qualità dei servizi CPF.

\textbf{Distinzione da Altre Certificazioni:}

\textit{vs. Certificazione Professionale Individuale:}
\begin{itemize}
\item La certificazione individuale valida la competenza personale
\item La certificazione ASP valida la capacità organizzativa
\item Le società devono impiegare professionisti certificati individualmente
\item Entrambe sono complementari
\end{itemize}

\textit{vs. Certificazione di Conformità Organizzativa (Livelli 1-4):}
\begin{itemize}
\item La certificazione di conformità valida la gestione delle proprie vulnerabilità dell'organizzazione
\item La certificazione ASP valida la capacità di fornire servizi ad altri
\item ASP può anche perseguire la certificazione di Conformità (raccomandata ma non richiesta)
\end{itemize}

\subsubsection{Requisiti di Certificazione}

\textbf{Requisiti di Personale per Dimensione della Società:}

\begin{tabular}{|l|c|p{6cm}|}
\hline
\textbf{Dimensione Società} & \textbf{Min. Certificati} & \textbf{Composizione} \\
\hline
Micro (1-10) & 2 & Qualsiasi combinazione \\
Piccola (11-50) & 5 & Min. 2 Assessor \\
Media (51-250) & 10 & Min. 5 Assessor + 1 Auditor \\
Grande (250+) & 20 & Min. 10 Assessor + 2 Auditor \\
\hline
\end{tabular}

Tutto il personale certificato deve:
\begin{itemize}
\item Detenere certificazioni CPF attuali e valide in buono stato
\item Essere impiegato o sotto contratto esclusivo (impegno minimo di 1 anno)
\item Essere attivamente coinvolto nella fornitura di servizi CPF (minimo 25\% del tempo)
\item Mantenere i crediti CPE richiesti
\end{itemize}

\textbf{Sistema di Gestione della Qualità:}

Le società devono stabilire un QMS documentato che copre:

\textit{Procedure Operative Standard:}
\begin{itemize}
\item Documentazione completa della metodologia di valutazione
\item Linee guida per l'applicazione del punteggio ternario
\item Protocolli di protezione della privacy
\item Metodi e strumenti di raccolta dati
\item Standard di scrittura dei report con template
\item Protocolli di comunicazione con il cliente
\item Procedure di gestione del progetto
\item Checklist di controllo qualità
\end{itemize}

\textit{Processo di Revisione della Qualità:}
\begin{itemize}
\item Revisione paritaria obbligatoria di tutti i report da secondo professionista certificato
\item Checklist di controllo qualità per ogni progetto
\item Survey di soddisfazione cliente per ogni incarico (scala a 5 punti)
\item Revisione trimestrale delle metriche di qualità
\item Analisi delle cause radice per i problemi
\item Sistema di azioni correttive e preventive
\item Database delle lezioni apprese
\end{itemize}

\textit{Metriche e Monitoraggio:}
\begin{itemize}
\item Punteggi di soddisfazione cliente (target: >4.0/5.0)
\item Puntualità nella consegna dei progetti
\item Accuratezza e coerenza delle valutazioni
\item Tassi di retention dei clienti
\item Tassi di reclami (target: <5\%)
\item Utilizzo dello staff
\item Tassi di completamento CPE
\end{itemize}

\textbf{Framework Privacy ed Etica:}

\textit{Infrastruttura di Protezione della Privacy:}
\begin{itemize}
\item Policy documentata di protezione della privacy
\item Implementazione della privacy differenziale ($\epsilon \leq 0.1$)
\item Unità di aggregazione minime applicate (10 individui)
\item Gestione sicura dei dati:
\begin{itemize}
\item Crittografia a riposo (AES-256)
\item Crittografia in transito (TLS 1.3 minimo)
\item Autenticazione multi-fattore
\item Controlli di accesso basati sui ruoli
\item Logging di audit
\item Backup sicuro
\end{itemize}
\item Policy di retention dei dati (massimo 5 anni)
\item Procedure di distruzione sicura
\item Reporting con ritardo temporale (minimo 72 ore)
\item Valutazioni annuali dell'impatto sulla privacy
\item Audit esterno annuale sulla privacy
\end{itemize}

\textit{Gestione Etica:}
\begin{itemize}
\item Adozione del Codice Etico CPF a livello organizzativo
\item Riconoscimento etico del personale
\item Formazione etica annuale (minimo 2 ore)
\item Procedure interne per reclami etici
\item Gestione dei conflitti di interesse
\item Protocolli di indipendenza
\item Divieto di utilizzo non autorizzato dei dati
\end{itemize}

\textbf{Requisiti Assicurativi:}

\textit{Responsabilità Professionale (Errori e Omissioni):}

\begin{tabular}{|l|c|c|}
\hline
\textbf{Dimensione Società} & \textbf{Per Evento} & \textbf{Aggregato} \\
\hline
Micro/Piccola & \$1.000.000 & \$2.000.000 \\
Media & \$2.000.000 & \$4.000.000 \\
Grande & \$5.000.000 & \$10.000.000 \\
\hline
\end{tabular}

\textit{Responsabilità Cyber:}
\begin{itemize}
\item Tutte le dimensioni: Copertura minima \$1.000.000
\item Deve coprire violazioni dei dati, interruzione dell'attività, estorsione cyber
\end{itemize}

L'Organismo di Certificazione e CPF3 devono essere nominati come assicurati aggiuntivi.

\subsubsection{Processo di Certificazione}

\textbf{Fase 1: Domanda (Settimane 1-4)}

La società presenta domanda completa includendo:
\begin{itemize}
\item Informazioni organizzative
\item Lista del personale certificato con numeri di certificato
\item Documentazione del sistema di gestione della qualità
\item Framework di privacy e report di audit
\item Policy etiche e registri di formazione
\item Certificati assicurativi
\item Deliverable di esempio (minimo 3 report redatti)
\item Referenze clienti (minimo 5)
\item Organigramma
\item Tariffa di domanda
\end{itemize}

L'Organismo di Certificazione conduce:
\begin{itemize}
\item Revisione desktop della domanda
\item Verifica dello stato del personale certificato
\item Controlli delle referenze (minimo 3 contattate)
\item Valutazione preliminare dell'adeguatezza del QMS
\end{itemize}

\textbf{Fase 2: Audit di Certificazione (Settimane 6-8)}

Durata dell'audit: 2-4 giorni in base alla dimensione della società

\textit{Attività di Audit:}
\begin{itemize}
\item Riunione di apertura
\item Revisione e walkthrough del QMS
\item Interviste al personale (staff certificato, management, quality manager)
\item Revisione dei file di progetto (3-5 progetti completati)
\item Valutazione dell'infrastruttura (sicurezza dati, strumenti, sistemi)
\item Valutazione del framework di privacy
\item Valutazione del programma etico
\item Verifica assicurativa
\item Analisi della soddisfazione cliente
\item Riunione di chiusura con risultati
\end{itemize}

Risultati classificati come:
\begin{itemize}
\item Conformità: Requisiti soddisfatti
\item Non Conformità Minore: Mancanza isolata, correggibile entro 90 giorni
\item Non Conformità Maggiore: Fallimento sistemico, deve essere corretto prima della certificazione
\item Osservazione: Opportunità di miglioramento
\end{itemize}

Report di audit consegnato entro 15 giorni lavorativi.

\textbf{Fase 3: Azioni Correttive (Se Necessarie)}

Se identificate non conformità:
\begin{itemize}
\item La società presenta piano di azioni correttive entro 30 giorni
\item Le NC Maggiori devono essere corrette prima del rilascio della certificazione
\item Le NC Minori possono essere corrette entro 90 giorni dopo la certificazione
\item L'Organismo di Certificazione verifica l'efficacia
\end{itemize}

\textbf{Fase 4: Decisione di Certificazione}

Entro 15 giorni lavorativi dal completamento dell'audit o dalla verifica delle azioni correttive:

\textit{Se Concessa:}
\begin{itemize}
\item Certificato emesso (elettronico entro 3 giorni, fisico entro 10 giorni)
\item Badge digitale fornito
\item Aggiunta alla directory Fornitore di Servizi Autorizzato con profilo in evidenza
\item Pacchetto di benvenuto con toolkit di marketing
\item Accesso a risorse esclusive
\item Programma di sorveglianza stabilito
\end{itemize}

\textit{Se Negata:}
\begin{itemize}
\item Spiegazione scritta delle carenze
\item Guida per la remediation
\item Diritto di appello (30 giorni)
\item Opzione di ripresentare domanda dopo aver affrontato i problemi
\end{itemize}

\subsubsection{Benefici e Privilegi}

\textbf{Marketing e Branding:}
\begin{itemize}
\item Uso del marchio "Fornitore di Servizi Autorizzato CPF"
\item Logo ufficiale ASP in formati multipli
\item Badge digitale per sito web ed email
\item Autorizzazione a esporre il marchio su materiali di marketing, sito web, proposte
\item Template di messaggistica approvati
\end{itemize}

\textbf{Inserimento in Directory in Evidenza:}
\begin{itemize}
\item Posizionamento prioritario nei risultati di ricerca
\item Profilo migliorato con logo, descrizione (500 parole), servizi, copertura
\item Indicatori di dimensione team e personale certificato
\item Informazioni di contatto e CTA
\item Analytics su visualizzazioni profilo e richieste
\end{itemize}

\textbf{Sviluppo Business:}
\begin{itemize}
\item Referral da CPF3 per organizzazioni in cerca di servizi
\item Status di fornitore preferito per grandi incarichi
\item Accesso a RFP che richiedono status ASP
\item Riduzione dell'onere di due diligence del cliente
\item Tassi di successo più alti e giustificazione di prezzi premium
\end{itemize}

\textbf{Risorse e Supporto:}
\begin{itemize}
\item Strumenti e template di valutazione esclusivi
\item Template professionali per report
\item Accesso anticipato agli aggiornamenti del framework (90 giorni in anticipo)
\item Account manager CPF3 dedicato
\item Coda prioritaria per supporto tecnico
\item Sconti volume sulle certificazioni:
\begin{itemize}
\item 10\% per 1-5 certificazioni/anno
\item 15\% per 6-10 certificazioni/anno
\item 20\% per 11+ certificazioni/anno
\end{itemize}
\item Accesso gratuito a webinar CPF e e-learning
\item Pass gratuiti per conferenze (limitati)
\end{itemize}

\textbf{Comunità e Networking:}
\begin{itemize}
\item Summit Annuale dei Fornitori CPF
\item Tavole rotonde virtuali trimestrali
\item Forum online privato per fornitori
\item Partecipazione ai capitoli regionali
\item Opportunità di collaborazione
\end{itemize}

\subsubsection{Obblighi Continuativi}

\textbf{Aggiornamento Annuale della Certificazione:}

Da presentare 30 giorni prima dell'anniversario, include:
\begin{itemize}
\item Roster attuale del personale certificato
\item Riepilogo delle metriche di qualità
\item Numero di progetti completati
\item Dati sulla soddisfazione cliente
\item Certificati assicurativi aggiornati
\item Risultati dell'audit sulla privacy
\item Registri di formazione etica
\item Notifica di cambiamenti materiali
\end{itemize}

\textbf{Audit di Sorveglianza Semestrali:}

Ogni 18 mesi (1-3 giorni):
\begin{itemize}
\item Revisione focalizzata (non così completa come l'iniziale)
\item Approccio basato su campioni (2-3 progetti, sottoinsieme del personale)
\item Verifica della conformità continua
\item Revisione delle azioni correttive
\item Valutazione dei cambiamenti dall'ultimo audit
\end{itemize}

\textbf{Monitoraggio delle Performance:}

L'Organismo di Certificazione monitora:
\begin{itemize}
\item Mantenimento del personale certificato minimo
\item Trend di soddisfazione cliente (deve mantenere >4.0/5.0)
\item Tassi di reclami (deve rimanere <5\%)
\item Status della copertura assicurativa
\item Risultati dell'audit sulla privacy
\end{itemize}

\textbf{Notifiche di Cambiamento:}

\textit{Immediate (5 giorni lavorativi):}
\begin{itemize}
\item Scendere sotto il personale minimo
\item Lasso assicurativo
\item Violazione privacy/dati
\item Reclami etici
\item Azioni legali
\item Perdita di personale chiave
\end{itemize}

\textit{Notifica a 30 giorni:}
\begin{itemize}
\item Cambiamenti di proprietà
\item Riorganizzazioni
\item Cambiamenti di nome
\item Aperture/chiusure di uffici
\end{itemize}

\subsubsection{Sospensione e Revoca}

\textbf{Motivi di Sospensione (max 180 giorni):}
\begin{itemize}
\item Scendere sotto il personale certificato minimo
\item Lasso assicurativo
\item Mancato completamento della sorveglianza
\item Mancato pagamento delle tariffe
\item Reclami clienti sotto indagine
\item Violazione privacy che richiede indagine
\item NC Maggiore non corretta entro 90 giorni
\item Soddisfazione cliente <3.5/5.0 per due trimestri
\end{itemize}

\textit{Durante la sospensione:}
\begin{itemize}
\item Restrizione sul nuovo uso del Marchio ASP
\item Status directory: "Sospeso - In Revisione"
\item Referral sospesi
\item Deve aggiungere "Status In Revisione" ai materiali esistenti
\item Periodo di remediation di 180 giorni o revoca
\end{itemize}

\textbf{Motivi di Revoca:}
\begin{itemize}
\item Mancata remediation della sospensione
\item Violazioni gravi della privacy (profilazione, vendita dati, violazione maggiore)
\item Frode o falsa dichiarazione
\item Fallimenti sistematici di qualità
\item Violazioni etiche materiali
\item Uso improprio persistente del Marchio
\item Perdita assicurazione >60 giorni
\item Dissoluzione/fallimento della società
\end{itemize}

\textit{Processo di revoca:}
\begin{itemize}
\item Notifica scritta con motivazioni
\item 30 giorni per rispondere
\item Revisione indipendente da comitato etico
\item Decisione finale entro 45 giorni
\item Diritto di appello
\end{itemize}

\textit{Effetto della revoca:}
\begin{itemize}
\item Cessazione immediata di TUTTO l'uso del Marchio ASP
\item Rimozione dalla directory
\item Avviso pubblico (visibile 12 mesi)
\item Nessun rimborso
\item Divieto di ripresentare domanda: 2-5 anni o permanente
\item Deve notificare i clienti
\end{itemize}

\textbf{Ritiro Volontario:}
\begin{itemize}
\item Preavviso di 60 giorni
\item Nessun record negativo
\item Può ripresentare domanda dopo 12 mesi
\item Processo semplificato se entro 24 mesi
\end{itemize}

\subsubsection{Tariffe}

\textbf{Tariffa di Domanda:}

\begin{tabular}{|l|c|}
\hline
\textbf{Dimensione Società} & \textbf{Tariffa} \\
\hline
Micro & \$1.000 \\
Piccola & \$2.000 \\
Media & \$3.500 \\
Grande & \$5.000 \\
\hline
\end{tabular}

\textbf{Tariffa Audit di Certificazione:}

\begin{tabular}{|l|c|}
\hline
\textbf{Dimensione Società} & \textbf{Tariffa} \\
\hline
Micro & \$3.000 \\
Piccola & \$6.000 \\
Media & \$10.000 \\
Grande & \$15.000 \\
\hline
\end{tabular}

Giorni aggiuntivi: \$1.500/giorno

\textbf{Tariffa di Certificazione:}

\begin{tabular}{|l|c|}
\hline
\textbf{Dimensione Società} & \textbf{Tariffa} \\
\hline
Micro & \$1.500 \\
Piccola & \$2.500 \\
Media & \$4.000 \\
Grande & \$6.000 \\
\hline
\end{tabular}

\textbf{Sorveglianza Annuale:} Stessa della tariffa di certificazione

\textbf{Ricertificazione (ogni 3 anni):}
\begin{itemize}
\item Audit: 75\% dell'iniziale
\item Certificazione: Stessa dell'iniziale
\end{itemize}

Tutte le tariffe non rimborsabili. Termini di pagamento: 30 giorni. Pagamento in ritardo: interesse mensile 1,5\%.

\textit{Requisiti di Sorveglianza:}
\begin{itemize}
\item Valutazione completa annuale da CPF Auditor esterno certificato
\item Auto-monitoraggio continuo con validazione mensile
\item Monitoraggio degli stati convergenti in tempo reale con risposta automatizzata
\item Riesame della direzione mensile
\item Audit esterno trimestrale delle pratiche di privacy ed etiche
\item Peer review semestrale da altre organizzazioni di Livello 4
\end{itemize}

\section{Requisiti di Certificazione}

\subsection{Requisiti di Istruzione}

I requisiti di istruzione validano la conoscenza fondamentale necessaria per la competenza CPF. La natura interdisciplinare di CPF richiede la comprensione sia dei principi psicologici che di cybersecurity.

\textbf{Lauree Accettabili (Triennale o Superiore):}

\textit{Per Certificazione Assessor:}
\begin{itemize}
\item Psicologia
\item Scienze Comportamentali
\item Psicologia Organizzativa
\item Psicologia Industriale/Organizzativa
\item Scienze Cognitive
\item Cybersecurity (con formazione psicologica supplementare richiesta)
\item Sicurezza Informatica (con formazione psicologica supplementare richiesta)
\item Informatica (con formazione psicologica supplementare richiesta)
\end{itemize}

\textit{Per Certificazione Practitioner:}
\begin{itemize}
\item Tutte le lauree accettabili per certificazione Assessor
\item Amministrazione Aziendale con focus sulla sicurezza
\item Risorse Umane con focus su sicurezza o psicologia organizzativa
\item Gestione del Rischio
\end{itemize}

\textbf{Equivalenza di Laurea:}

I candidati senza lauree formali possono qualificarsi attraverso combinazione di:
\begin{itemize}
\item Certificazioni professionali rilevanti (CISSP, CISM, CEH per sicurezza; Psicologo Abilitato, SHRM-SCP per psicologia)
\item Esperienza professionale documentata (minimo 5 anni)
\item Completamento di tutta la formazione CPF richiesta
\item Superamento di esame migliorato che dimostra conoscenza equivalente ai requisiti di laurea
\end{itemize}

\textbf{Riconoscimento Lauree Internazionali:}

Lauree da istituzioni non-US valutate usando:
\begin{itemize}
\item Riconoscimento nazionale nel paese che rilascia la laurea
\item Valutazione di equivalenza da servizio di valutazione credenziali
\item Dimostrazione di competenza linguistica in inglese per gli esami
\end{itemize}

\subsection{Requisiti di Esperienza}

I requisiti di esperienza validano la capacità pratica oltre la conoscenza teorica. L'esperienza deve dimostrare lavoro rilevante per la sicurezza ed esposizione ai fattori umani organizzativi.

\textbf{Metodi di Verifica:}
\begin{itemize}
\item Lettere di verifica del datore di lavoro su carta intestata ufficiale
\item Portfolio professionale dettagliato che documenta progetti e responsabilità
\item Referenze professionali da supervisori o clienti
\item Deliverable di progetto documentati (con informazioni confidenziali redatte)
\end{itemize}

\textbf{Categorie di Esperienza Qualificante:}

\textit{Esperienza di Sicurezza:}
\begin{itemize}
\item Operazioni di sicurezza e monitoraggio
\item Risposta e investigazione incidenti
\item Valutazione e testing di sicurezza
\item Gestione del programma di sicurezza
\item Valutazione e gestione del rischio
\item Sviluppo del programma di security awareness
\item Sviluppo e implementazione di policy di sicurezza
\end{itemize}

\textit{Esperienza Psicologica:}
\begin{itemize}
\item Consulenza in psicologia organizzativa
\item Valutazione e intervento comportamentale
\item Analisi dei fattori umani
\item Sviluppo organizzativo
\item Gestione del cambiamento
\item Progettazione di programmi di formazione e sviluppo
\end{itemize}

\textit{Esperienza Integrata (Vale Doppio):}
\begin{itemize}
\item Psicologia del programma di security awareness
\item Fattori umani nella progettazione della sicurezza
\item Testing e analisi del social engineering
\item Psicologia del programma insider threat
\item Sviluppo della cultura di sicurezza
\end{itemize}

\subsection{Requisiti di Formazione}

La formazione obbligatoria assicura comprensione standardizzata della metodologia CPF, tecniche di valutazione e pratiche etiche. La formazione deve essere completata attraverso fornitori di formazione approvati CPF che soddisfano standard di qualità e curriculum.

\textbf{CPF-101: Fondamenti del Framework (40 ore)}

\textit{Obiettivi del Corso:}
\begin{itemize}
\item Comprendere i fondamenti teorici: teoria psicoanalitica, psicologia cognitiva, dinamiche di gruppo
\item Padroneggiare l'architettura CPF: 10 domini, 100 indicatori, punteggio ternario
\item Applicare principi di valutazione che preservano la privacy
\item Integrare CPF con framework di sicurezza esistenti (ISO 27001, NIST CSF)
\end{itemize}

\textit{Struttura del Corso:}
\begin{enumerate}
\item Introduzione alla Psicologia della Cybersecurity (4 ore)
\begin{itemize}
\item Fallimento degli interventi di sicurezza a livello conscio
\item Elaborazione pre-cognitiva e decisioni di sicurezza
\item Panoramica del framework CPF
\end{itemize}

\item Fondamenti Teorici (8 ore)
\begin{itemize}
\item Contributi psicoanalitici: Bion, Klein, Jung, Winnicott
\item Psicologia cognitiva: Kahneman, Cialdini, Miller
\item Dinamiche di gruppo e inconscio organizzativo
\item Psicologia dell'AI e interazione umano-AI
\end{itemize}

\item Approfondimento dei Domini CPF (20 ore - 2 ore per dominio)
\begin{itemize}
\item Vulnerabilità Basate sull'Autorità [1.x]
\item Vulnerabilità Temporali [2.x]
\item Vulnerabilità di Influenza Sociale [3.x]
\item Vulnerabilità Affettive [4.x]
\item Vulnerabilità da Sovraccarico Cognitivo [5.x]
\item Vulnerabilità delle Dinamiche di Gruppo [6.x]
\item Vulnerabilità della Risposta allo Stress [7.x]
\item Vulnerabilità dei Processi Inconsci [8.x]
\item Vulnerabilità dei Bias Specifici dell'AI [9.x]
\item Stati Convergenti Critici [10.x]
\end{itemize}

\item Privacy ed Etica (4 ore)
\begin{itemize}
\item Metodologia di valutazione che preserva la privacy
\item Privacy differenziale e requisiti di aggregazione
\item Considerazioni etiche nella valutazione psicologica
\item Divieto di profilazione individuale
\end{itemize}

\item Integrazione e Applicazione (4 ore)
\begin{itemize}
\item Integrazione con ISO 27001 e NIST CSF
\item Strategie di implementazione organizzativa
\item Case study e applicazioni pratiche
\end{itemize}
\end{enumerate}

\textbf{CPF-201: Metodologia di Valutazione (40 ore)}

\textit{Obiettivi del Corso:}
\begin{itemize}
\item Padroneggiare il processo di valutazione sistematico per tutti i 100 indicatori
\item Sviluppare competenze di raccolta e analisi dati
\item Applicare la metodologia di punteggio ternario in modo coerente
\item Creare report di valutazione attuabili
\end{itemize}

\textit{Struttura del Corso:}
\begin{enumerate}
\item Pianificazione della Valutazione (6 ore)
\item Metodi di Raccolta Dati (8 ore)
\item Punteggio e Analisi (12 ore)
\item Tecniche di Preservazione della Privacy (6 ore)
\item Scrittura Report e Comunicazione (8 ore)
\end{enumerate}

\textbf{CPF-301: Implementazione Avanzata (40 ore)}

\textit{Formazione Avanzata Opzionale per Practitioner}

\textit{Obiettivi del Corso:}
\begin{itemize}
\item Progettare interventi efficaci per vulnerabilità identificate
\item Implementare sistemi di monitoraggio continuo
\item Integrare controlli psicologici e tecnici
\item Misurare l'efficacia degli interventi
\end{itemize}

\textbf{CPF-401: Tecniche di Audit (40 ore)}

\textit{Richiesto per Certificazione Auditor}

\textit{Obiettivi del Corso:}
\begin{itemize}
\item Applicare i principi di audit ISO 19011 al contesto CPF
\item Condurre audit di conformità CPF-27001
\item Valutare i sistemi di gestione delle vulnerabilità psicologiche
\item Documentare risultati e raccomandazioni dell'audit
\end{itemize}

\subsection{Requisiti d'Esame}

Gli esami validano l'acquisizione di conoscenze e la capacità pratica. Tutti gli esami sono sviluppati utilizzando principi psicometrici che assicurano validità, affidabilità ed equità.

\textbf{Struttura dell'Esame Scritto:}

\textit{Sviluppo delle Domande:}
\begin{itemize}
\item Distribuzione della difficoltà degli item: 30\% facili, 50\% moderati, 20\% difficili
\item Copertura della tassonomia di Bloom: 20\% conoscenza, 40\% comprensione/applicazione, 40\% analisi/sintesi
\item Test pilota e validazione prima dell'uso operativo
\item Analisi statistica regolare e miglioramento degli item
\end{itemize}

\textit{Copertura dei Domini (Tutte le Certificazioni):}
\begin{itemize}
\item Vulnerabilità Basate sull'Autorità: 10\%
\item Vulnerabilità Temporali: 10\%
\item Vulnerabilità di Influenza Sociale: 10\%
\item Vulnerabilità Affettive: 10\%
\item Vulnerabilità da Sovraccarico Cognitivo: 10\%
\item Vulnerabilità delle Dinamiche di Gruppo: 10\%
\item Vulnerabilità della Risposta allo Stress: 10\%
\item Vulnerabilità dei Processi Inconsci: 10\%
\item Vulnerabilità dei Bias Specifici dell'AI: 10\%
\item Stati Convergenti Critici: 10\%
\end{itemize}

\textit{Amministrazione dell'Esame:}
\begin{itemize}
\item Testing basato su computer presso centri di testing autorizzati
\item Proctoring remoto disponibile con sicurezza migliorata
\item Esame a libro chiuso senza materiali di riferimento
\item Risultati preliminari immediati (in attesa di revisione qualità)
\item Risultati ufficiali entro 5 giorni lavorativi
\end{itemize}

\textit{Standard di Superamento:}
\begin{itemize}
\item Esame Scritto Assessor/Practitioner: minimo 70\%
\item Esame Scritto Auditor: minimo 75\% (standard più alto riflettendo ruolo avanzato)
\item Nessun punteggio minimo per dominio, ma richiesta copertura completa
\item I candidati che non superano possono ripetere dopo periodo di attesa di 30 giorni
\item Massimo 3 tentativi entro periodo di 12 mesi
\end{itemize}

\textbf{Struttura dell'Esame Pratico:}

\textit{Pratico Assessor:}
\begin{itemize}
\item Case study: Scenario organizzativo realistico con indicatori di vulnerabilità psicologica
\item Compito: Condurre valutazione, applicare punteggio ternario, giustificare valutazioni, raccomandare interventi
\item Durata: 4 ore
\item Criteri di valutazione: Accuratezza, qualità della giustificazione, protezione della privacy, chiarezza comunicativa
\end{itemize}

\textit{Pratico Auditor:}
\begin{itemize}
\item Mock audit: Simulazione di audit di una giornata intera includendo pianificazione, interviste, revisione documenti, documentazione dei risultati, generazione report
\item Durata: 8 ore (giornata lavorativa completa)
\item Criteri di valutazione: Metodologia di audit, raccolta evidenze, qualità dei risultati, condotta professionale, chiarezza del report
\end{itemize}

\section{Processo di Certificazione}

\subsection{Domanda}

Il processo di domanda assicura che i candidati soddisfino i requisiti di eleggibilità prima della registrazione all'esame.

\textbf{Passaggi della Domanda:}

\begin{enumerate}
\item \textbf{Auto-Valutazione dell'Eleggibilità}
\begin{itemize}
\item Revisione dei requisiti di certificazione
\item Verifica delle qualifiche educative
\item Conferma dei requisiti di esperienza
\item Assicurarsi del completamento della formazione
\end{itemize}

\item \textbf{Preparazione della Documentazione}
\begin{itemize}
\item Trascrizioni ufficiali o certificati di laurea
\item Lettere di verifica dell'esperienza o portfolio
\item Certificati di completamento della formazione
\item Referenze professionali (minimo 2)
\item CV/curriculum attuale
\end{itemize}

\item \textbf{Presentazione della Domanda}
\begin{itemize}
\item Completare il modulo di domanda online
\item Caricare la documentazione richiesta
\item Pagare la tariffa di revisione della domanda (non rimborsabile)
\item Firma elettronica sul Codice Etico
\end{itemize}

\item \textbf{Revisione della Domanda}
\begin{itemize}
\item Verifica dell'eleggibilità da parte dell'organismo di certificazione
\item Controllo della completezza della documentazione
\item Contatto con le referenze (se necessario)
\item Approvazione o richiesta di informazioni aggiuntive
\item Tempistica: 10 giorni lavorativi dalla domanda completa
\end{itemize}
\end{enumerate}

\textbf{Tariffe di Domanda:}
\begin{itemize}
\item CPF Assessor: \$300 (revisione domanda)
\item CPF Practitioner: \$200 (revisione domanda)
\item CPF Auditor: \$400 (revisione domanda)
\item Certificazione Organizzativa: \$500-\$2000 in base a dimensione e ambito dell'organizzazione
\end{itemize}

\subsection{Verifica}

La verifica assicura l'autenticità e l'accuratezza della documentazione presentata.

\textbf{Verifica dell'Istruzione:}
\begin{itemize}
\item Contatto diretto con l'istituzione che rilascia la laurea
\item Verifica del tipo di laurea, specializzazione e data di conferimento
\item Valutazione di equivalenza per lauree internazionali
\item Tempistica: 5-10 giorni lavorativi
\end{itemize}

\textbf{Verifica dell'Esperienza:}
\begin{itemize}
\item Contatto con i datori di lavoro o clienti elencati
\item Conferma delle date di impiego e responsabilità
\item Revisione del portfolio per candidati autonomi
\item Tempistica: 10-15 giorni lavorativi
\end{itemize}

\textbf{Controlli delle Referenze:}
\begin{itemize}
\item Contatto con minimo 2 referenze professionali
\item Verifica della competenza del candidato e condotta professionale
\item Valutazione dell'idoneità alla certificazione
\item Feedback confidenziale all'organismo di certificazione
\end{itemize}

\textbf{Verifica della Formazione:}
\begin{itemize}
\item Verifica diretta con fornitori di formazione approvati
\item Conferma del completamento del corso e date
\item Verifica della presenza e risultati delle valutazioni
\item Tempistica: 3-5 giorni lavorativi
\end{itemize}

\subsection{Esame}

L'esame valida conoscenze e competenza attraverso valutazione standardizzata.

\textbf{Programmazione:}
\begin{itemize}
\item Notifica di eleggibilità all'esame entro 3 giorni lavorativi dal completamento della verifica
\item Il candidato seleziona data e luogo dell'esame dalle opzioni disponibili
\item Richiesta programmazione anticipata minima di 14 giorni
\item Riprogrammazione permessa fino a 48 ore prima dell'esame (potrebbe applicarsi una tariffa)
\end{itemize}

\textbf{Opzioni di Erogazione dell'Esame:}

\textit{Testing In Presenza:}
\begin{itemize}
\item Centri di testing autorizzati Pearson VUE o Prometric
\item Ambiente di testing sicuro con proctoring
\item Richiesta verifica dell'identità
\item Nessun oggetto personale permesso nella stanza di testing
\end{itemize}

\textit{Testing Online Proctored:}
\begin{itemize}
\item Esame remoto via piattaforma sicura
\item Proctoring live tramite webcam
\item Richiesta scansione ambientale
\item Requisiti di sistema: Computer, webcam, microfono, internet stabile
\item Verifica dell'identità via documento d'identità con foto emesso dal governo
\end{itemize}

\textbf{Tariffe d'Esame:}
\begin{itemize}
\item CPF Assessor Scritto: \$400
\item CPF Assessor Pratico: \$600
\item CPF Practitioner Scritto: \$300
\item CPF Practitioner Revisione Portfolio: \$400
\item CPF Auditor Scritto: \$450
\item CPF Auditor Pratico: \$800
\item Tariffa Ripetizione: 50\% della tariffa d'esame originale
\end{itemize}

\textbf{Policy di Ripetizione:}
\begin{itemize}
\item I candidati che non superano possono ripetere dopo periodo di attesa di 30 giorni
\item Massimo 3 tentativi entro 12 mesi
\item Dopo 3 fallimenti, il candidato deve completare formazione aggiuntiva e attendere 6 mesi
\item Ogni ripetizione richiede nuova tariffa d'esame
\item L'esame pratico può essere ripetuto indipendentemente dall'esame scritto
\end{itemize}

\textbf{Accommodamenti:}
\begin{itemize}
\item Accomodamenti ragionevoli forniti per disabilità documentate
\item La richiesta deve essere presentata con la domanda
\item Richiesta documentazione da professionista qualificato
\item Esempi: Tempo esteso, stanza separata, screen reader, pause
\end{itemize}

\subsection{Decisione di Certificazione}

Le decisioni di certificazione sono prese da personale qualificato dell'organismo di certificazione basate su criteri standardizzati.

\textbf{Criteri di Decisione:}

\textit{Certificazione Individuale:}
\begin{itemize}
\item Verifica di tutti i requisiti di eleggibilità
\item Punteggio di superamento nell'esame scritto
\item Valutazione di superamento nell'esame pratico/portfolio
\item Controlli delle referenze soddisfacenti
\item Accordo al Codice Etico
\item Pagamento di tutte le tariffe applicabili
\end{itemize}

\textit{Certificazione Organizzativa:}
\begin{itemize}
\item Valutazione CPF valida da Assessor/Auditor certificato
\item Punteggio CPF nel range del livello di conformità target
\item Documentazione delle policy e procedure richieste
\item Evidenza di implementazione sistematica
\item Impegno della direzione dimostrato
\item Requisiti di sorveglianza concordati
\end{itemize}

\textbf{Tempistica della Decisione:}
\begin{itemize}
\item Certificazione Individuale: 10 giorni lavorativi dal completamento dell'esame
\item Certificazione Organizzativa: 15 giorni lavorativi dal completamento dell'audit
\item Revisione accelerata disponibile per tariffa aggiuntiva
\end{itemize}

\textbf{Esiti della Decisione:}

\textit{Certificazione Concessa:}
\begin{itemize}
\item Certificato emesso elettronicamente e in copia fisica
\item Inserimento nel registro pubblico delle certificazioni
\item Accesso ai benefici del titolare di certificazione
\item Autorizzazione a usare i marchi di certificazione
\end{itemize}

\textit{Certificazione Negata:}
\begin{itemize}
\item Spiegazione scritta delle carenze
\item Guida sui passaggi di remediation
\item Diritto di appello della decisione
\item Opzione di ripresentare domanda dopo aver affrontato le carenze
\end{itemize}

\textbf{Processo di Appello:}
\begin{itemize}
\item Gli appelli devono essere presentati per iscritto entro 30 giorni
\item Revisione indipendente da panel di appello (non coinvolto nella decisione originale)
\item Revisione di tutte le evidenze e razionale della decisione
\item Decisione entro 30 giorni dalla presentazione dell'appello
\item Tariffa di appello: \$200 (rimborsata se appello ha successo)
\item Decisione finale vincolante, ma il candidato può ripresentare domanda dopo aver affrontato i problemi
\end{itemize}

\textbf{Emissione del Certificato:}
\begin{itemize}
\item Certificato elettronico (PDF) emesso entro 3 giorni lavorativi
\item Certificato fisico spedito entro 10 giorni lavorativi
\item Badge digitale per profili professionali online
\item Inserimento nel registro pubblico delle certificazioni entro 5 giorni lavorativi
\item Tessera di certificazione per identificazione fisica
\end{itemize}

\section{Mantenimento della Certificazione}

\subsection{Formazione Continua}

L'Educazione Professionale Continua (CPE) assicura che gli individui certificati mantengano conoscenze attuali man mano che la metodologia CPF e il panorama della cybersecurity evolvono.

\textbf{Requisiti CPE:}

\textit{CPF Assessor:}
\begin{itemize}
\item 40 crediti CPE per anno (120 nel ciclo di 3 anni)
\item Minimo 20 crediti in argomenti specifici CPF
\item Massimo 10 crediti da singola attività/fonte per anno
\item Minimo 5 crediti in etica e privacy annualmente
\end{itemize}

\textit{CPF Practitioner:}
\begin{itemize}
\item 30 crediti CPE per anno (90 nel ciclo di 3 anni)
\item Minimo 15 crediti in argomenti specifici CPF
\item Massimo 10 crediti da singola attività/fonte per anno
\item Minimo 3 crediti in etica annualmente
\end{itemize}

\textit{CPF Auditor:}
\begin{itemize}
\item 50 crediti CPE per anno (150 nel ciclo di 3 anni)
\item Minimo 30 crediti in argomenti specifici di audit
\item Minimo 10 crediti in aggiornamenti della metodologia CPF
\item Massimo 10 crediti da singola attività/fonte per anno
\item Minimo 5 crediti in etica e condotta professionale annualmente
\end{itemize}

\textbf{Attività CPE Accettate:}

\textit{Categoria A: Educazione Formale (1 ora = 1 credito)}
\begin{itemize}
\item Corsi di formazione CPF approvati
\item Corsi accademici in psicologia o cybersecurity
\item Formazione per certificazioni professionali (CISSP, CISM, ecc.)
\item Webinar e formazione virtuale
\end{itemize}

\textit{Categoria B: Sviluppo Professionale (1 ora = 1 credito)}
\begin{itemize}
\item Partecipazione a conferenze (cybersecurity o psicologia)
\item Riunioni di associazioni professionali
\item Partecipazione alla comunità di pratica CPF
\item Mentoring di candidati certificati (massimo 5 crediti/anno)
\end{itemize}

\textit{Categoria C: Studio Autonomo (2 ore = 1 credito)}
\begin{itemize}
\item Lettura di riviste e pubblicazioni professionali
\item Revisione degli aggiornamenti della metodologia CPF
\item Ricerca indipendente in argomenti rilevanti
\item Corsi online senza valutazione
\end{itemize}

\textit{Categoria D: Contributi (Credito Speciale)}
\begin{itemize}
\item Pubblicazione di ricerca o case study CPF: 10 crediti
\item Sviluppo di materiali di formazione CPF: 15 crediti
\item Presentazioni a conferenze su argomenti CPF: 5 crediti per presentazione
\item Servizio in comitati consultivi CPF: 10 crediti/anno
\item Contributo allo sviluppo della metodologia CPF: 20 crediti
\end{itemize}

\textbf{Requisiti di Documentazione:}
\begin{itemize}
\item Certificato di completamento per formazione formale
\item Registri di partecipazione per conferenze
\item Diari di lettura con riassunti per studio autonomo
\item Citazioni delle pubblicazioni per opere scritte
\item Verifica dalle organizzazioni beneficiarie per lavoro volontario
\item Tutta la documentazione mantenuta per 5 anni
\end{itemize}

\textbf{Tracciamento CPE:}
\begin{itemize}
\item Portale CPE online per registrazione attività
\item Credito automatico per attività approvate
\item Capacità di upload per documentazione di supporto
\item Dashboard di progresso che mostra accumulo crediti
\item Promemoria automatizzati per scadenze in avvicinamento
\end{itemize}

\textbf{Audit CPE:}
\begin{itemize}
\item Audit random del 10\% dei titolari di certificazione annualmente
\item Richiesta di documentazione delle attività CPE dichiarate
\item Verifica del completamento attività e calcolo crediti
\item Periodo di risposta di 30 giorni per presentazione documentazione
\item La non conformità può risultare in sospensione della certificazione
\end{itemize}

\subsection{Ricertificazione}

La ricertificazione avviene ogni 3 anni e valida la competenza continua e la pratica etica.

\textbf{Requisiti di Ricertificazione:}

\textit{Tutte le Certificazioni Individuali:}
\begin{itemize}
\item Completamento di tutti i requisiti CPE per il ciclo di 3 anni
\item Esperienza professionale continua in ruolo rilevante
\item Nessuna violazione etica comprovata
\item Pagamento della tariffa di ricertificazione
\item Accordo aggiornato al Codice Etico
\item Dimostrazione di competenza attuale
\end{itemize}

\textit{Requisiti Aggiuntivi Assessor:}
\begin{itemize}
\item Minimo 5 valutazioni CPF documentate nel periodo di 3 anni
\item Peer review di almeno un report di valutazione
\item Partecipazione ad attività di calibrazione degli assessor
\end{itemize}

\textit{Requisiti Aggiuntivi Auditor:}
\begin{itemize}
\item Minimo 45 giorni di audit nel periodo di 3 anni (media 15/anno)
\item Ruolo di lead auditor in minimo 5 audit
\item Revisione della qualità dei report di audit
\item Partecipazione alla valutazione della competenza degli auditor
\end{itemize}

\textbf{Processo di Ricertificazione:}

\begin{enumerate}
\item \textbf{Notifica} (180 giorni prima della scadenza)
\begin{itemize}
\item L'organismo di certificazione invia avviso di ricertificazione
\item Il candidato rivede i requisiti e lo stato attuale
\item Identificazione del deficit CPE e piano di remediation se necessario
\end{itemize}

\item \textbf{Presentazione della Documentazione} (90 giorni prima della scadenza)
\begin{itemize}
\item Registri CPE presentati via portale online
\item Documentazione dell'esperienza caricata
\item Referenze professionali fornite (se richieste)
\item Attestazione etica completata
\end{itemize}

\item \textbf{Revisione e Verifica} (60 giorni prima della scadenza)
\begin{itemize}
\item L'organismo di certificazione rivede la presentazione
\item Audit CPE (se selezionato)
\item Verifica dell'esperienza
\item Controllo del record etico
\end{itemize}

\item \textbf{Decisione di Ricertificazione} (30 giorni prima della scadenza)
\begin{itemize}
\item Approvazione o richiesta di informazioni aggiuntive
\item Nuovo certificato emesso all'approvazione
\item Periodo di certificazione aggiornato nel registro
\end{itemize}
\end{enumerate}

\textbf{Tariffe di Ricertificazione:}
\begin{itemize}
\item CPF Assessor: \$400
\item CPF Practitioner: \$300
\item CPF Auditor: \$500
\item Ricertificazione tardiva (entro 90 giorni dopo la scadenza): Aggiuntivi \$100
\item Reinstatement (oltre 90 giorni dopo la scadenza): Processo di certificazione completo richiesto
\end{itemize}

\textbf{Periodo di Grazia:}
\begin{itemize}
\item Periodo di grazia di 90 giorni dopo la scadenza
\item Lo status della certificazione cambia in "Ricertificazione in Attesa"
\item Uso dei marchi di certificazione ristretto durante il periodo di grazia
\item Tariffa tardiva si applica per ricertificazione durante il periodo di grazia
\item Dopo il periodo di grazia, richiesto processo di ricertificazione completo
\end{itemize}

\textbf{Ricertificazione Organizzativa:}
\begin{itemize}
\item Audit di sorveglianza annuali richiesti
\item Rivalutazione completa ogni 3 anni
\item Monitoraggio continuo del Punteggio CPF
\item Il livello di conformità può essere aggiornato o declassato in base alle performance
\item Cambiamenti organizzativi significativi innescano rivalutazione
\end{itemize}

\subsection{Etica e Condotta Professionale}

I requisiti etici assicurano che i titolari di certificazione mantengano standard professionali e proteggano gli interessi degli stakeholder.

\textbf{Codice Etico - Principi Fondamentali:}

\begin{enumerate}
\item \textbf{Integrità}
\begin{itemize}
\item Rappresentazione onesta delle qualifiche e capacità
\item Reporting accurato dei risultati delle valutazioni
\item Comunicazione trasparente con gli stakeholder
\item Nessuna falsificazione di documentazione o dati
\end{itemize}

\item \textbf{Obiettività}
\begin{itemize}
\item Valutazione e analisi imparziali
\item Nessun conflitto di interesse
\item Indipendenza dalle pressioni commerciali
\item Decision making basato su evidenze
\end{itemize}

\item \textbf{Riservatezza}
\begin{itemize}
\item Protezione dei dati di valutazione
\item Gestione sicura delle informazioni sulle vulnerabilità psicologiche
\item Nessuna divulgazione non autorizzata
\item Protezione della privacy migliorata per dati sensibili
\end{itemize}

\item \textbf{Competenza}
\begin{itemize}
\item Praticare entro aree di competenza dimostrata
\item Sviluppo professionale continuo
\item Riconoscimento delle limitazioni di competenza
\item Referral quando l'expertise è insufficiente
\end{itemize}

\item \textbf{Responsabilità Professionale}
\begin{itemize}
\item Aderenza agli standard della metodologia CPF
\item Conformità alle leggi e regolamenti applicabili
\item Segnalazione di comportamento non etico
\item Contributo alla comunità professionale
\end{itemize}
\end{enumerate}

\textbf{Requisiti Etici Specifici:}

\textit{Protezione della Privacy:}
\begin{itemize}
\item Mai usare dati di valutazione per profilazione individuale
\item Mantenere unità di aggregazione minime (10 individui)
\item Implementare protezioni di privacy differenziale
\item Storage e trasmissione sicuri di tutti i dati
\item Divieto di uso secondario senza consenso esplicito
\item Reporting con ritardo temporale (minimo 72 ore)
\item Analisi basata sui ruoli piuttosto che individuale
\end{itemize}

\textit{Conflitto di Interesse:}
\begin{itemize}
\item Divulgazione di tutti i potenziali conflitti prima dell'incarico
\item Nessun interesse finanziario nei risultati della valutazione
\item Indipendenza dai fornitori di formazione durante la valutazione
\item Nessuna consulenza e audit per la stessa organizzazione simultaneamente
\item Divieto di accettare regali o incentivi
\end{itemize}

\textit{Ambito di Pratica:}
\begin{itemize}
\item La valutazione CPF è organizzativa, non valutazione psicologica clinica
\item Nessuna diagnosi individuale o interventi terapeutici
\item Riconoscimento che le vulnerabilità psicologiche sono caratteristiche umane normali
\item Nessuna stigmatizzazione o colpa degli individui per le vulnerabilità
\item Confini chiari tra CPF e psicologia clinica
\end{itemize}

\textit{Gestione dei Dati:}
\begin{itemize}
\item Crittografia di tutti i dati di valutazione a riposo e in transito
\item Controlli di accesso che limitano i dati al personale autorizzato
\item Trail di audit per tutti gli accessi ai dati
\item Limiti di retention (massimo 5 anni salvo requisiti legali)
\item Distruzione sicura dei dati dopo il periodo di retention
\item Nessun trasferimento transfrontaliero di dati senza salvaguardie appropriate
\end{itemize}

\textbf{Processo Disciplinare:}

\textit{Presentazione del Reclamo:}
\begin{itemize}
\item Chiunque può presentare reclamo contro individuo o organizzazione certificata
\item Reclamo presentato per iscritto con accuse specifiche
\item Evidenze di supporto fornite
\item Identità del reclamante protetta (opzione per reclami anonimi)
\end{itemize}

\textit{Indagine:}
\begin{itemize}
\item Revisione iniziale entro 10 giorni lavorativi
\item Indagine da comitato etico (indipendente dalle decisioni di certificazione)
\item Opportunità per la parte accusata di rispondere
\item Raccolta evidenze e interviste ai testimoni
\item Indagine completata entro 60 giorni (estendibile se complessa)
\end{itemize}

\textit{Risultati e Sanzioni:}

Risultato: Nessuna Violazione
\begin{itemize}
\item Reclamo respinto
\item Nessun record nel file di certificazione
\item Parti notificate dell'esito
\end{itemize}

Risultato: Violazione Minore
\begin{itemize}
\item Avvertimento scritto emesso
\item Piano di azioni correttive richiesto
\item Requisiti CPE migliorati
\item Monitoraggio del progresso
\end{itemize}

Risultato: Violazione Significativa
\begin{itemize}
\item Sospensione della certificazione (6-12 mesi)
\item Requisiti di remediation prima del reinstatement
\item Periodo di prova dopo il reinstatement
\item Divulgazione pubblica nel registro delle certificazioni
\end{itemize}

Risultato: Violazione Grave
\begin{itemize}
\item Revoca della certificazione
\item Divieto di ripresentare domanda (2-5 anni o permanente)
\item Divulgazione pubblica
\item Notifica alle autorità rilevanti se coinvolte violazioni legali
\end{itemize}

\textit{Appello dell'Azione Disciplinare:}
\begin{itemize}
\item L'appello deve essere presentato entro 30 giorni dalla decisione
\item Panel di appello indipendente rivede il caso
\item Nessuna nuova evidenza permessa (revisione del processo e proporzionalità)
\item Decisione entro 45 giorni
\item Tariffa di appello: \$500 (rimborsata se appello ha successo)
\end{itemize}

\section{Organismi di Certificazione}

\subsection{Requisiti di Accreditamento}

Gli organismi di certificazione che operano questo schema devono mantenere accreditamento appropriato e dimostrare competenze specifiche.

\textbf{Accreditamento ISO/IEC 17065:}
\begin{itemize}
\item Accreditamento da organismo di accreditamento riconosciuto a livello nazionale
\item Ambito di accreditamento include certificazione di processi e servizi
\item Audit di sorveglianza annuali da parte dell'organismo di accreditamento
\item Rivalutazione completa ogni 4 anni
\item Conformità a tutti i requisiti ISO/IEC 17065
\end{itemize}

\textbf{Competenze Specifiche CPF:}

\textit{Requisiti del Personale:}
\begin{itemize}
\item Manager dello schema di certificazione con expertise sia in cybersecurity che psicologia
\item Minimo 2 esperti tecnici con certificazione CPF Auditor
\item Accesso a esperti di materia in teoria psicoanalitica e psicologia cognitiva
\item Personale di sviluppo esami con expertise psicometrica
\item Specialisti in privacy ed etica
\end{itemize}

\textit{Competenza Tecnica:}
\begin{itemize}
\item Comprensione di tutti i dieci domini CPF e 100 indicatori
\item Conoscenza delle metodologie di valutazione che preservano la privacy
\item Familiarità con privacy differenziale e requisiti di aggregazione
\item Conoscenza di integrazione per ISO 27001 e NIST CSF
\item Competenza in etica psicologica e condotta professionale
\end{itemize}

\textit{Requisiti dell'Infrastruttura:}
\begin{itemize}
\item Sistemi sicuri di sviluppo e storage degli esami
\item Database candidati crittografato con controlli di accesso
\item Piattaforme online per domanda e tracciamento CPE
\item Canali di comunicazione sicuri per informazioni sensibili
\item Capacità di backup e disaster recovery
\end{itemize}

\textbf{Sistema di Gestione della Qualità:}

\textit{Documentazione:}
\begin{itemize}
\item Procedure dello schema di certificazione completamente documentate
\item Criteri di decision-making chiaramente definiti
\item Procedure di appelli e reclami stabilite
\item Procedure di retention e gestione dei record
\item Procedure di riservatezza e imparzialità
\end{itemize}

\textit{Controlli di Processo:}
\begin{itemize}
\item Processo standardizzato di revisione delle domande
\item Amministrazione degli esami coerente
\item Decision-making calibrato tra il personale
\item Valutazione regolare delle competenze dello staff
\item Audit interni dei processi di certificazione
\end{itemize}

\textit{Miglioramento Continuo:}
\begin{itemize}
\item Revisione regolare delle statistiche degli esami
\item Analisi di appelli e reclami per problemi sistemici
\item Meccanismi di feedback degli stakeholder
\item Benchmarking contro altri organismi di certificazione
\item Implementazione di azioni correttive e preventive
\end{itemize}

\subsection{Assicurazione della Qualità}

L'assicurazione della qualità assicura coerenza, affidabilità e credibilità delle decisioni di certificazione.

\textbf{Audit degli Organismi di Certificazione:}

\textit{Audit dell'Organismo di Accreditamento:}
\begin{itemize}
\item Sorveglianza annuale da parte dell'organismo di accreditamento ISO/IEC 17065
\item Revisione dei file di certificazione e decisioni
\item Witness assessment ed esami
\item Valutazione della gestione delle competenze
\item Rivalutazione completa ogni 4 anni
\end{itemize}

\textit{Audit del Proprietario dello Schema CPF:}
\begin{itemize}
\item Audit annuale della conformità ai requisiti specifici CPF
\item Revisione della qualità e validità degli esami
\item Valutazione della competenza tecnica nei domini CPF
\item Valutazione delle pratiche di protezione della privacy
\item Verifica delle procedure di etica e disciplinari
\end{itemize}

\textit{Peer Review:}
\begin{itemize}
\item Cross-audit tra organismi di certificazione
\item Condivisione delle best practice
\item Calibrazione del decision-making
\item Identificazione di opportunità di miglioramento
\end{itemize}

\textbf{Gestione dei Reclami:}

\textit{Tipi di Reclami:}
\begin{itemize}
\item Reclami sul processo di certificazione (ritardi, comunicazione, equità)
\item Reclami sull'esame (qualità, amministrazione, equità)
\item Reclami sul personale certificato (etica, competenza, condotta)
\item Reclami sulla certificazione organizzativa (qualità dell'audit, decisioni)
\end{itemize}

\textit{Processo di Reclamo:}
\begin{enumerate}
\item Presentazione del reclamo (scritto, con dettagli)
\item Acknowledgment entro 3 giorni lavorativi
\item Indagine entro 30 giorni
\item Risoluzione e risposta al reclamante
\item Azione correttiva se giustificata
\item Analisi dei trend per problemi sistemici
\end{enumerate}

\textit{Registri dei Reclami:}
\begin{itemize}
\item Tutti i reclami registrati nel registro dei reclami
\item Documentazione dell'indagine mantenuta
\item Risoluzione e azioni correttive registrate
\item Revisione regolare da parte della direzione
\item Reporting di sintesi annuale all'organismo di accreditamento
\end{itemize}

\textbf{Miglioramento Continuo:}

\textit{Metriche di Performance:}
\begin{itemize}
\item Tempo di elaborazione delle domande
\item Tassi di superamento esami e statistiche
\item Tassi di appelli e reclami
\item Tassi di retention dei titolari di certificazione
\item Punteggi di soddisfazione degli stakeholder
\end{itemize}

\textit{Meccanismi di Miglioramento:}
\begin{itemize}
\item Revisione regolare della direzione delle metriche di qualità
\item Analisi delle statistiche degli esami per validità
\item Revisione degli appelli per coerenza delle decisioni
\item Survey e feedback degli stakeholder
\item Implementazione dei miglioramenti identificati
\item Condivisione delle lezioni apprese tra organismi di certificazione
\end{itemize}

\subsection{Appelli e Reclami}

Procedure robuste di appelli e reclami assicurano equità e forniscono ricorso per gli stakeholder.

\textbf{Processo di Appello:}

\textit{Decisioni Appellabili:}
\begin{itemize}
\item Negazione della certificazione
\item Fallimento dell'esame (solo questioni procedurali, non punteggio)
\item Azioni disciplinari
\item Negazione della ricertificazione
\item Sospensione o revoca della certificazione
\end{itemize}

\textit{Procedura di Appello:}
\begin{enumerate}
\item Appello presentato per iscritto entro 30 giorni dalla decisione
\item Pagamento della tariffa di appello (\$200-\$500 in base al tipo di decisione)
\item Specificazione dei motivi dell'appello
\item Documentazione di supporto fornita

\item Panel di appello indipendente assegnato (nessun coinvolgimento nella decisione originale)
\item Revisione di tutte le evidenze e razionale della decisione
\item Opportunità per l'appellante di fornire informazioni aggiuntive
\item Deliberazione e decisione del panel

\item Decisione comunicata entro 30 giorni dalla presentazione dell'appello
\item Opzioni: Confermare decisione originale, Modificare decisione, Rovesciare decisione, Rinviare per riconsiderazione
\item Tariffa rimborsata se appello ha successo
\item La decisione è finale (nessun ulteriore appello)
\end{enumerate}

\textbf{Indagine del Reclamo:}

\textit{Processo di Indagine:}
\begin{enumerate}
\item Presentazione del reclamo con accuse specifiche
\item Revisione iniziale per completezza e giurisdizione
\item Assegnazione all'investigatore (indipendente dal soggetto)
\item Notifica al soggetto del reclamo con opportunità di rispondere
\item Raccolta evidenze e interviste ai testimoni
\item Report di indagine con risultati
\item Decisione sulla validità del reclamo e azioni correttive
\item Comunicazione al reclamante e al soggetto
\item Implementazione delle azioni correttive
\item Follow-up per verificare l'efficacia
\end{enumerate}

\textbf{Procedure di Risoluzione:}

\textit{Risoluzione Informale:}
\begin{itemize}
\item Mediazione tra le parti
\item Chiarimento dei malintesi
\item Azione correttiva da parte dell'organismo di certificazione
\item Ritiro del reclamo se risolto
\end{itemize}

\textit{Risoluzione Formale:}
\begin{itemize}
\item Risultati ufficiali dell'indagine
\item Azioni correttive o disciplinari
\item Cambiamenti alle procedure dell'organismo di certificazione
\item Compensazione o remediation se giustificata
\item Misure di prevenzione per evitare ricorrenze
\end{itemize}

\subsection{Licensing dello Schema}

CPF3, come Proprietario dello Schema, autorizza gli organismi di certificazione qualificati a operare lo Schema di Certificazione CPF attraverso un processo formale di licensing. Questo assicura coerenza, qualità e integrità delle certificazioni tra più organismi di certificazione a livello globale.

\subsubsection{Obiettivi del Licensing}

Il modello di licensing serve molteplici obiettivi strategici:

\begin{itemize}
\item \textit{Scalabilità:} Abilitare l'espansione geografica e di mercato oltre la capacità diretta di CPF3
\item \textit{Accessibilità:} Aumentare l'accesso ai servizi di certificazione globalmente
\item \textit{Localizzazione:} Permettere adattamento regionale entro framework coerente
\item \textit{Competizione:} Promuovere competizione sana che migliora la qualità del servizio
\item \textit{Distribuzione del Rischio:} Diversificare il rischio operativo tra entità indipendenti
\item \textit{Generazione di Ricavi:} Creare ricavi sostenibili per lo sviluppo dello schema
\item \textit{Controllo Qualità:} Mantenere supervisione centralizzata con operazioni distribuite
\item \textit{Protezione del Marchio:} Assicurare standard coerenti proteggendo la reputazione CPF
\end{itemize}

\textbf{Filosofia del Licensing:}

\begin{itemize}
\item \textit{Autorizzazione Selettiva:} Standard rigorosi assicurano che solo organismi capaci siano autorizzati
\item \textit{Supervisione Attiva:} CPF3 mantiene monitoraggio continuo e diritti di intervento
\item \textit{Relazione Collaborativa:} Gli organismi licenziati sono partner nel successo condiviso
\item \textit{Miglioramento Continuo:} Il feedback informa l'evoluzione dello schema
\item \textit{Coerenza Prima:} L'adattamento è accettabile solo dove la coerenza è mantenuta
\item \textit{Partnership a Lungo Termine:} Strutturato per relazioni sostenute
\end{itemize}

\subsubsection{Requisiti di Licensing}

\textbf{Prerequisiti di Accreditamento Fondamentali:}

\textit{Accreditamento ISO/IEC 17065:2012:}
\begin{itemize}
\item Accreditamento attuale e valido da organismo di accreditamento firmatario IAF MLA
\item Esempi: ANAB (USA), UKAS (UK), DAkkS (Germania), JAS-ANZ (Australia/NZ)
\item L'ambito deve includere:
\begin{itemize}
\item Certificazione di prodotti, processi e servizi
\item Certificazione di sistemi di gestione
\item Certificazione del personale
\end{itemize}
\item Buono stato: Nessuna non conformità maggiore nell'audit recente
\item Minimo 2 anni di storia operativa sotto l'accreditamento attuale
\end{itemize}

\textbf{Competenza Tecnica:}

\textit{Requisiti del Personale:}

\begin{itemize}
\item \textit{Manager dello Schema:}
\begin{itemize}
\item Laurea magistrale in Psicologia, Cybersecurity o correlate (o equivalente)
\item 5+ anni operazioni di certificazione O consulenza psicologia/cybersecurity
\item Comprensione sia dei contesti psicologici che di sicurezza
\item Minimo 0,5 FTE dedicato a CPF
\end{itemize}

\item \textit{Esperto Tecnico - Psicologia:}
\begin{itemize}
\item Laurea magistrale/Dottorato in Psicologia (I/O Psychology preferito)
\item 3+ anni esperienza in psicologia organizzativa
\item Ruolo: Revisione tecnica, sviluppo esami, formazione
\item Minimo 1 FTE totale
\end{itemize}

\item \textit{Esperto Tecnico - Cybersecurity:}
\begin{itemize}
\item Certificazioni professionali (CISSP, CISM, o equivalente)
\item 5+ anni esperienza cybersecurity/gestione del rischio
\item Ruolo: Revisione contesto sicurezza, guida all'integrazione
\item Minimo 1 FTE totale
\end{itemize}

\item \textit{CPF Auditor Certificati:}
\begin{itemize}
\item Minimo 2 CPF Auditor certificati nello staff o contratto esclusivo
\item Certificazione CPF Auditor attuale e valida
\item Disponibili per audit di certificazione organizzativa
\end{itemize}

\item \textit{Specialista Sviluppo Esami:}
\begin{itemize}
\item Competenza in psicometria e sviluppo test
\item Laurea in psicometria, misurazione educativa, o equivalente
\item 2+ anni esperienza sviluppo esami
\item Può essere specialista a contratto
\end{itemize}

\item \textit{Specialista Privacy e Protezione Dati:}
\begin{itemize}
\item Expertise in GDPR, CCPA, e leggi sulla protezione dati
\item CIPP (Certified Information Privacy Professional) o equivalente
\item Comprensione delle tecniche di privacy differenziale
\item Può essere specialista a contratto
\end{itemize}
\end{itemize}

\textit{Competenza Collettiva:}

Oltre ai ruoli individuali, deve dimostrare:
\begin{itemize}
\item Copertura di tutti i 10 domini CPF
\item Comprensione dei fondamenti psicoanalitici (Bion, Klein, Jung, Winnicott)
\item Comprensione della psicologia cognitiva (Kahneman, Cialdini, Miller)
\item Conoscenza di ISO 27001, NIST CSF
\item Capacità culturali e linguistiche per i territori di servizio
\end{itemize}

\textbf{Requisiti dell'Infrastruttura:}

\textit{Sicurezza degli Esami:}
\begin{itemize}
\item Ambiente di sviluppo sicuro (rete isolata, crittografia AES-256)
\item Autenticazione multi-fattore e logging di audit
\item Sicurezza fisica per materiali stampati (cassaforte chiusa, accesso ristretto)
\item Piattaforma di erogazione test sicura (Pearson VUE, Prometric, o equivalente)
\item Capacità di proctoring remoto
\item Procedure di verifica dell'identità
\item Protocolli di reporting incidenti
\end{itemize}

\textit{Gestione Candidati:}
\begin{itemize}
\item Database candidati crittografato (conforme GDPR/CCPA)
\item Controlli di accesso basati sui ruoli con trail di audit
\item Portale domande online
\item Piattaforma di tracciamento CPE
\item Integrazione con registro centrale CPF3
\item Generazione certificati con funzionalità anti-contraffazione
\item Distribuzione badge digitali
\end{itemize}

\textit{Infrastruttura di Comunicazione:}
\begin{itemize}
\item Email crittografata (S/MIME o PGP)
\item Condivisione file sicura (SFTP o portale sicuro)
\item Videoconferenza sicura
\item API o scambio dati con registro CPF3
\end{itemize}

\textit{Continuità Operativa:}
\begin{itemize}
\item Backup crittografati giornalieri con storage offsite
\item Piano di disaster recovery (RTO: 48 ore, RPO: 24 ore)
\item Procedure di recovery testate trimestralmente
\item Sito operazioni alternativo identificato
\item Conformità ISO 22301 raccomandata
\end{itemize}

\textbf{Requisiti Finanziari:}

\textit{Stabilità Dimostrata:}
\begin{itemize}
\item Minimo 3 anni di bilanci certificati
\item Posizione patrimoniale positiva
\item Riserve di cassa adeguate (6 mesi spese operative)
\item Nessun procedimento di fallimento o insolvenza
\end{itemize}

\textit{Tariffe di Licensing:}
\begin{itemize}
\item Tariffa di domanda: \$10.000 (non rimborsabile)
\item Tariffa di licensing iniziale: \$25.000 - \$100.000 (in base al territorio)
\item Tariffa di licensing annuale: \$15.000 - \$50.000
\item Tariffe royalty: 10-15\% dei ricavi di certificazione
\item Royalty annuale minima: \$10.000
\item Licensing esami: \$5.000 annualmente
\item Costi di setup stimati: \$50.000 - \$150.000
\end{itemize}

\textit{Requisiti Assicurativi:}
\begin{itemize}
\item Responsabilità professionale: \$5M per evento, \$10M aggregato
\item Responsabilità cyber: \$2M copertura
\item CPF3 nominato come assicurato aggiuntivo
\item Certificati forniti annualmente
\end{itemize}

\subsubsection{Processo di Licensing}

Tempistica: 6-12 mesi dalla domanda allo stato operativo

\textbf{Fase 1: Domanda (Mese 1-2)}

L'organismo prospettico presenta pacchetto completo:

\begin{itemize}
\item Sintesi esecutiva e motivazione
\item Informazioni organizzative e documentazione legale
\item Certificato di accreditamento ISO/IEC 17065 e report di audit recente
\item Documentazione delle competenze del personale (curriculum, certificazioni)
\item Documentazione dell'infrastruttura (sistemi IT, misure di sicurezza)
\item Manuale del sistema di gestione della qualità e procedure
\item Bilanci (3 anni) e business plan
\item Territorio proposto e analisi di mercato
\item Certificati assicurativi
\item Referenze (organismo di accreditamento, proprietari di schemi, clienti)
\item Pagamento tariffa di domanda (\$10.000)
\end{itemize}

\textit{Revisione Iniziale CPF3 (30 giorni):}

\begin{itemize}
\item Controllo di completezza
\item Valutazione dell'eleggibilità
\item Verifica dell'accreditamento
\item Revisione della stabilità finanziaria
\item Valutazione delle competenze del personale
\item Valutazione dell'adeguatezza dell'infrastruttura
\item Conferma della disponibilità del territorio
\item Esito: Approvato per assessment / Condizionale / Negato
\end{itemize}

\textbf{Fase 2: Audit di Assessment (Mese 3-5)}

\textit{Assessment On-Site (3-5 giorni):}

Il team di assessment CPF3 conduce audit completo:

\begin{itemize}
\item Giorno 1: Riunione di apertura, revisione QMS, walkthrough delle procedure
\item Giorno 2: Interviste al personale e valutazione delle competenze
\item Giorno 3: Test di infrastruttura e sicurezza, dimostrazione sistemi
\item Giorno 4: Operazioni business, facilities, gestione record
\item Giorno 5: Sviluppo risultati, riunione di chiusura
\end{itemize}

L'assessment copre:
\begin{itemize}
\item Efficacia del sistema di gestione della qualità
\item Procedure di certificazione e decision-making
\item Competenza e disponibilità del personale
\item Sicurezza e capacità dell'infrastruttura
\item Misure di sicurezza degli esami
\item Sistemi finanziari e business planning
\item Conformità a tutti i requisiti di licensing
\end{itemize}

\textit{Report di Assessment (2 settimane):}

\begin{itemize}
\item Sintesi esecutiva e determinazione della prontezza
\item Risultati dettagliati con evidenze
\item Classificazione: Conformità / Gap / Preoccupazioni
\item Punti di forza e capacità
\item Azioni correttive richieste (se presenti)
\item Raccomandazioni per miglioramento
\end{itemize}

\textit{Remediation dei Gap (Se Necessario):}

\begin{itemize}
\item L'organismo di certificazione presenta piano di remediation (30 giorni)
\item Implementa correzioni
\item CPF3 verifica remediation (desktop o visita di follow-up)
\item Deve completare prima di procedere al licensing
\end{itemize}

\textbf{Fase 3: Accordo di Licensing (Mese 6-7)}

\textit{Negoziazione Commerciale:}

Le parti negoziano i termini:
\begin{itemize}
\item Definizione del territorio (esclusivo o non esclusivo)
\item Tariffe di licensing iniziali e annuali
\item Tassi di royalty e garanzie minime
\item Metriche di performance e target
\item Servizi di supporto da CPF3
\item Termini di marketing e co-branding
\item Durata del termine (tipicamente 5 anni) e rinnovo
\end{itemize}

\textit{Esecuzione Legale:}

\begin{itemize}
\item Bozza dell'Accordo di Licensing dello Schema preparata
\item Revisione legale da entrambe le parti
\item Negoziazione di termini specifici
\item Esecuzione da firmatari autorizzati
\item Pagamento della tariffa di licensing iniziale
\item Data di efficacia stabilita
\end{itemize}

\textbf{Fase 4: Implementazione (Mese 8-12)}

\textit{Formazione del Personale (Mese 8-9):}

\begin{itemize}
\item Manager dello Schema: Formazione completa di 5 giorni
\item Esperti tecnici: CPF-101 e CPF-201 (80 ore)
\item Auditor: CPF-401 più procedure specifiche CB
\item Staff di supporto: Sistemi, procedure, customer service
\item Certificazione fast-track per personale chiave
\item Validazione delle competenze attraverso esercizi pratici
\end{itemize}

\textit{Integrazione dei Sistemi (Mese 9-10):}

\begin{itemize}
\item Setup API con registro centrale CPF3
\item Test scambio dati e sincronizzazione
\item Configurazione piattaforma esami
\item Setup proctoring remoto
\item Sviluppo materiali di marketing
\item Integrazione sito web
\item Tutti i sistemi testati end-to-end
\end{itemize}

\textit{Operazioni Shadow (Mese 11):}

Fase pilota con supervisione CPF3:
\begin{itemize}
\item Shadow audit: Auditor CB accompagnano CPF3 in 3 audit
\item Certificazioni pilota: Processare 5-10 sotto supervisione
\item Esami mock per testare procedure
\item Revisione qualità di tutti gli output pilota
\item Identificazione e risoluzione problemi
\item Aggiustamenti finali delle procedure
\end{itemize}

\textit{Approvazione al Lancio (Mese 12):}

\begin{itemize}
\item Assessment finale di prontezza
\item Approvazione a iniziare operazioni indipendenti
\item Annuncio pubblico del nuovo CB licenziato
\item Aggiunta alla directory sito web CPF3
\item Lancio marketing coordinato
\item Transizione al supporto standard continuativo
\end{itemize}

\subsubsection{Supervisione Continua}

\textbf{Audit Annuali da CPF3:}

Audit annuale completo (2-4 giorni):

\textit{Ambito:}
\begin{itemize}
\item Conformità del processo di certificazione (campione 10-15 decisioni)
\item Amministrazione e sicurezza degli esami
\item Mantenimento delle competenze del personale
\item Efficacia del QMS
\item Controlli di privacy e riservatezza
\item Gestione reclami e appelli
\item Accuratezza del registro
\item Reporting finanziario e calcoli royalty
\item Utilizzo marketing e marchio
\item Soddisfazione cliente
\end{itemize}

\textit{Processo:}
\begin{itemize}
\item Preavviso di 30 giorni
\item Risultati documentati e classificati
\item Report entro 15 giorni lavorativi
\item Azione correttiva per non conformità (30 giorni)
\item Verifica delle azioni
\item CPF3 sostiene costi di routine; CB paga per follow-up se NC maggiori
\end{itemize}

\textbf{Reporting Trimestrale:}

Gli organismi di certificazione presentano report dettagliati includendo:

\begin{itemize}
\item Statistiche di certificazione per tipo
\item Domande, approvazioni, negazioni
\item Tassi di superamento/fallimento esami
\item Riepilogo appelli e reclami
\item Dati finanziari e calcolo royalty
\item Metriche di qualità
\item Incidenti o cambiamenti significativi
\end{itemize}

Scadenza: Entro 30 giorni dalla fine del trimestre

\textbf{Calibrazione Inter-CB:}

Workshop semestrali (partecipazione obbligatoria):

\begin{itemize}
\item Esercizi di coerenza nel punteggio
\item Discussione di casi complessi
\item Condivisione best practice
\item Aggiornamenti e formazione sullo schema
\item Benchmarking delle performance
\item Apprendimento tra pari
\end{itemize}

Costi condivisi tra CPF3 e CB partecipanti.

\textbf{Monitoraggio delle Performance:}

CPF3 monitora indicatori chiave:

\begin{itemize}
\item Tassi di superamento esami (entro 15\% dalla media dello schema)
\item Tassi di appello (<5\% delle decisioni)
\item Comprovazione reclami (<10\%)
\item Tempo ciclo di certificazione (<90 giorni media)
\item Soddisfazione cliente (>4.0/5.0)
\end{itemize}

Intervento innescato se gli standard non sono soddisfatti per due trimestri consecutivi.

\subsubsection{Sospensione e Terminazione della Licenza}

\textbf{Motivi di Sospensione (max 180 giorni):}

\begin{itemize}
\item NC maggiore nell'audit annuale non corretta entro tempistica
\item Perdita accreditamento ISO/IEC 17065 o finding maggiore
\item Aumento significativo di appelli/reclami
\item Tassi di superamento/fallimento fuori dal range accettabile
\item Instabilità finanziaria o mancato pagamento tariffe (60+ giorni)
\item Violazione privacy o riservatezza
\item Carenze di competenza del personale
\item Lasso assicurativo
\end{itemize}

\textit{Durante la sospensione:}
\begin{itemize}
\item Nessuna nuova domanda accettata
\item Le certificazioni esistenti rimangono valide
\item Deve correggere entro 180 giorni o affrontare terminazione
\end{itemize}

\textbf{Motivi di Terminazione Immediata:}

\begin{itemize}
\item Mancata remediation della sospensione (180 giorni)
\item Perdita accreditamento ISO/IEC 17065
\item Frode, falsa dichiarazione, o violazioni etiche gravi
\item Violazione materiale dell'accordo di licensing
\item Fallimento o insolvenza
\item Uso non autorizzato della PI CPF
\item Fallimenti di qualità sistematici che danneggiano la reputazione dello schema
\end{itemize}

\textbf{Successione Post-Terminazione:}

Disposizioni critiche assicurano continuità:

\begin{itemize}
\item Tutte le certificazioni emesse rimangono valide fino alla scadenza normale
\item Le parti certificate possono trasferirsi a un altro CB licenziato
\item Il CB terminato trasferisce tutti i record a CPF3 (30 giorni)
\item CPF3 o successore designato fornisce supporto continuo
\item Il CB terminato cessa immediatamente tutto l'uso del marchio CPF
\item Nessun rimborso delle tariffe di licensing
\item Obbligazioni finanziarie risolte (90 giorni)
\end{itemize}

\subsubsection{Riepilogo dei Termini Finanziari}

\textbf{Tariffe Pagate a CPF3:}

\begin{tabular}{|l|c|p{6cm}|}
\hline
\textbf{Tipo di Tariffa} & \textbf{Importo} & \textbf{Quando Dovuta} \\
\hline
Domanda & \$10.000 & Con la domanda \\
Licenza Iniziale & \$25K-\$100K & All'esecuzione dell'accordo \\
Licenza Annuale & \$15K-\$50K & Annualmente all'anniversario \\
Materiali Esame & \$5.000/anno & Annualmente \\
Royalty & 10-15\% & Trimestralmente \\
Royalty Annuale Min. & \$10.000 & Se attuale <minimo \\
\hline
\end{tabular}

\textbf{Struttura Royalty:}

\begin{itemize}
\item Certificazioni individuali (Assessor, Practitioner, Auditor): 15\%
\item Certificazioni organizzative (Livelli 1-4): 10\%
\item Certificazioni Fornitore di Servizi Autorizzato: 12\%
\item Tariffe di ricertificazione: Stessa percentuale
\item Ripetizioni esame: 15\%
\end{itemize}

Termini di pagamento: Reporting trimestrale e rimessa entro 30 giorni dalla fine del trimestre.

\subsubsection{Benefici dello Status di Licenziato}

\textbf{Per gli Organismi di Certificazione:}

\begin{itemize}
\item Autorizzazione a operare schema di certificazione riconosciuto
\item Accesso a marchio e metodologia consolidati
\item Supporto marketing e opportunità di co-branding
\item Supporto tecnico da CPF3
\item Materiali d'esame e banche di item
\item Risorse e materiali di formazione
\item Integrazione con registro centrale
\item Partecipazione alla comunità di certificazione globale
\item Generazione ricavi da mercato in crescita
\end{itemize}

\textbf{Per l'Ecosistema:}

\begin{itemize}
\item Espansione geografica e accessibilità
\item Presenza nel mercato locale e adattamento culturale
\item Competizione che guida miglioramenti della qualità del servizio
\item Prospettive diverse che migliorano l'evoluzione dello schema
\item Distribuzione del rischio tra entità multiple
\item Scalabilità che abilita la crescita del mercato
\item Riconoscimento reciproco tra territori
\end{itemize}

Questo framework di licensing completo abilita l'espansione controllata dell'ecosistema di certificazione CPF mantenendo qualità, coerenza e integrità essenziali per credibilità a lungo termine e accettazione del mercato.


\section*{Appendici}

\appendix

\section{Domande d'Esame di Esempio}

\subsection{Domande di Esempio CPF Assessor}

\textbf{Domande a Scelta Multipla:}

\textbf{Domanda 1:} Quale delle seguenti descrive meglio lo scopo primario della valutazione CPF?

a) Identificare i dipendenti che pongono rischi di sicurezza\\
b) Misurare i livelli consci di security awareness\\
c) Identificare vulnerabilità psicologiche pre-cognitive a livello organizzativo\\
d) Valutare l'idoneità psicologica individuale per ruoli di sicurezza

\textit{Risposta Corretta: c}

\textbf{Domanda 2:} Secondo la metodologia CPF, qual è l'unità di aggregazione minima per il reporting dei dati di valutazione?

a) 5 individui\\
b) 10 individui\\
c) 25 individui\\
d) 50 individui

\textit{Risposta Corretta: b}

\textbf{Domanda 3:} Nel sistema di punteggio ternario, un indicatore Giallo (1) rappresenta:

a) Vulnerabilità minima senza azione richiesta\\
b) Vulnerabilità moderata che richiede monitoraggio\\
c) Vulnerabilità critica che richiede intervento immediato\\
d) Vulnerabilità eliminata

\textit{Risposta Corretta: b}

\textbf{Domande Basate su Scenari:}

\textbf{Domanda 4:} Il dipartimento finanziario di un'organizzazione processa costantemente richieste urgenti di bonifici da chiunque abbia "CEO" nella firma email senza verifica aggiuntiva. Questo comportamento indica principalmente vulnerabilità in quale dominio CPF?

a) [2.x] Vulnerabilità Temporali\\
b) [1.x] Vulnerabilità Basate sull'Autorità\\
c) [3.x] Vulnerabilità di Influenza Sociale\\
d) [5.x] Vulnerabilità da Sovraccarico Cognitivo

\textit{Risposta Corretta: b - Vulnerabilità Basate sull'Autorità, specificamente indicatore 1.1 (Conformità acritica all'autorità apparente)}

\textbf{Domanda 5:} Durante i periodi di fine trimestre, i tassi di incidenti di sicurezza aumentano del 35\%, coinvolgendo principalmente dipendenti che bypassano i processi di approvazione per rispettare le scadenze. Quale stato di vulnerabilità convergente rappresenta questo?

a) Vulnerabilità Temporale [2.x] pura\\
b) [10.4] Allineamento swiss cheese di vulnerabilità temporali e di autorità\\
c) [6.1] Punti ciechi di sicurezza da groupthink\\
d) [5.2] Errori da affaticamento decisionale

\textit{Risposta Corretta: b - Questo rappresenta la convergenza di pressione temporale ([2.3] Accettazione del rischio guidata dalle scadenze) con pattern organizzativi che creano condizioni di tempesta perfetta}

\subsection{Domande di Esempio CPF Practitioner}

\textbf{Domande di Applicazione:}

\textbf{Domanda 6:} Una valutazione rivela punteggi Rossi elevati nel Dominio 5 (Sovraccarico Cognitivo) con alert fatigue che colpisce il 70\% del team di sicurezza. Quale intervento sarebbe PIÙ efficace secondo i principi CPF?

a) Formazione aggiuntiva di security awareness\\
b) Azione disciplinare per alert ignorati\\
c) Riduzione di alert falsi positivi e consolidamento degli alert\\
d) Assunzione di personale di sicurezza aggiuntivo

\textit{Risposta Corretta: c - Affronta la causa radice del sovraccarico cognitivo piuttosto che i sintomi}

\textbf{Domanda 7:} Quando si integra CPF con un ISMS ISO 27001 esistente, i risultati della valutazione delle vulnerabilità psicologiche dovrebbero essere principalmente incorporati in quale componente ISMS?

a) Inventario degli asset\\
b) Processo di valutazione del rischio\\
c) Procedure di controllo accessi\\
d) Piano di risposta agli incidenti

\textit{Risposta Corretta: b - Le vulnerabilità psicologiche sono fattori di rischio che richiedono valutazione e trattamento sistematico del rischio}

\subsection{Domande di Esempio CPF Auditor}

\textbf{Domande sulla Metodologia di Audit:}

\textbf{Domanda 8:} Durante un audit di conformità CPF-27001, scopri che i dati di valutazione includono identificatori individuali che potrebbero abilitare la profilazione. Questo rappresenta una non conformità con quale requisito CPF-27001?

a) Sezione 7.3 (Consapevolezza)\\
b) Sezione 8.2.3 (Misure di Preservazione della Privacy)\\
c) Sezione 9.1 (Monitoraggio, Misurazione, Analisi e Valutazione)\\
d) Sezione 10.1 (Non Conformità e Azione Correttiva)

\textit{Risposta Corretta: b - Le Misure di Preservazione della Privacy proibiscono esplicitamente la profilazione individuale}

\textbf{Domanda 9:} Un'organizzazione dichiara certificazione CPF Livello 3 ma il Punteggio CPF è 75. Qual è la conclusione di audit appropriata?

a) La certificazione dovrebbe essere declassata a Livello 2 (range 70-99)\\
b) La certificazione dovrebbe essere mantenuta con sorveglianza\\
c) La certificazione dovrebbe essere sospesa in attesa di azione correttiva\\
d) La certificazione è appropriata poiché il punteggio è nel range del Livello 3

\textit{Risposta Corretta: a - Il Punteggio CPF di 75 ricade nel range del Livello 2 (70-99), non Livello 3 (40-69)}

\textbf{Domande sulla Condotta Professionale:}

\textbf{Domanda 10:} Ti viene offerto un incarico di consulenza per aiutare un'organizzazione a migliorare il loro Punteggio CPF prima del tuo audit programmato. Qual è la risposta appropriata?

a) Accettare se la consulenza avviene 6+ mesi prima dell'audit\\
b) Accettare ma ricusarsi dall'audit\\
c) Rifiutare per conflitto di interesse\\
d) Accettare ma divulgare all'organismo di certificazione

\textit{Risposta Corretta: c - Gli auditor devono mantenere indipendenza e non possono fornire consulenza alle organizzazioni che verificano}

\section{Curriculum di Formazione CPF}

\subsection{CPF-101: Fondamenti del Framework (40 ore)}

\textbf{Modulo 1: Introduzione alla Psicologia della Cybersecurity (4 ore)}
\begin{itemize}
\item 1.1 Il Gap del Fattore Umano nella Cybersecurity
\item 1.2 Fallimento degli Interventi a Livello Conscio
\item 1.3 Elaborazione Pre-Cognitiva e Decisioni di Sicurezza
\item 1.4 Panoramica del Framework CPF
\item 1.5 Integrazione CPF con Framework di Sicurezza
\end{itemize}

\textbf{Modulo 2: Fondamenti Psicoanalitici (4 ore)}
\begin{itemize}
\item 2.1 Teoria delle Assunzioni di Base di Bion
\item 2.2 Teoria delle Relazioni Oggettuali di Klein
\item 2.3 Psicologia Analitica di Jung
\item 2.4 Spazio Transizionale di Winnicott
\item 2.5 Applicazione alla Sicurezza Organizzativa
\end{itemize}

\textbf{Modulo 3: Fondamenti di Psicologia Cognitiva (4 ore)}
\begin{itemize}
\item 3.1 Teoria del Processo Duale di Kahneman
\item 3.2 Principi di Influenza di Cialdini
\item 3.3 Teoria del Carico Cognitivo di Miller
\item 3.4 Euristiche e Bias nella Sicurezza
\item 3.5 Decision-Making in Condizioni di Incertezza
\end{itemize}

\textbf{Moduli 4-13: Approfondimenti dei Domini CPF (20 ore, 2 ore ciascuno)}
\begin{itemize}
\item Modulo 4: Vulnerabilità Basate sull'Autorità [1.x]
\item Modulo 5: Vulnerabilità Temporali [2.x]
\item Modulo 6: Vulnerabilità di Influenza Sociale [3.x]
\item Modulo 7: Vulnerabilità Affettive [4.x]
\item Modulo 8: Vulnerabilità da Sovraccarico Cognitivo [5.x]
\item Modulo 9: Vulnerabilità delle Dinamiche di Gruppo [6.x]
\item Modulo 10: Vulnerabilità della Risposta allo Stress [7.x]
\item Modulo 11: Vulnerabilità dei Processi Inconsci [8.x]
\item Modulo 12: Vulnerabilità dei Bias Specifici dell'AI [9.x]
\item Modulo 13: Stati Convergenti Critici [10.x]
\end{itemize}

\textbf{Modulo 14: Privacy ed Etica (4 ore)}
\begin{itemize}
\item 14.1 Principi di Valutazione che Preservano la Privacy
\item 14.2 Privacy Differenziale e Requisiti di Aggregazione
\item 14.3 Considerazioni Etiche nella Valutazione Psicologica
\item 14.4 Divieto di Profilazione Individuale
\item 14.5 Condotta Professionale e Confini
\item 14.6 Gestione Dati e Riservatezza
\end{itemize}

\textbf{Modulo 15: Integrazione e Applicazione (4 ore)}
\begin{itemize}
\item 15.1 Integrazione CPF e ISO/IEC 27001:2022
\item 15.2 Integrazione CPF e NIST CSF 2.0
\item 15.3 Strategie di Implementazione Organizzativa
\item 15.4 Case Study e Applicazioni Pratiche
\item 15.5 Sfide di Implementazione Comuni
\item 15.6 Revisione del Corso e Valutazione
\end{itemize}

\subsection{CPF-201: Metodologia di Valutazione (40 ore)}

\textbf{Modulo 1: Pianificazione della Valutazione (6 ore)}
\begin{itemize}
\item 1.1 Definizione dell'Ambito e Confini
\item 1.2 Coinvolgimento degli Stakeholder
\item 1.3 Pianificazione delle Risorse
\item 1.4 Valutazione dell'Impatto sulla Privacy
\item 1.5 Programma e Timeline della Valutazione
\item 1.6 Valutazione del Rischio per il Processo di Valutazione
\end{itemize}

\textbf{Modulo 2: Metodi di Raccolta Dati (8 ore)}
\begin{itemize}
\item 2.1 Tecniche di Osservazione Comportamentale
\item 2.2 Metodologie di Intervista
\item 2.3 Revisione e Analisi Documentale
\item 2.4 Progettazione e Amministrazione di Survey
\item 2.5 Analisi dei Log Tecnici
\item 2.6 Combinazione di Fonti Dati Multiple
\item 2.7 Evitare l'Effetto Hawthorne
\item 2.8 Esercizio Pratico: Pianificazione della Raccolta Dati
\end{itemize}

\textbf{Modulo 3: Punteggio e Analisi (12 ore)}
\begin{itemize}
\item 3.1 Metodologia di Punteggio Ternario
\item 3.2 Decisioni di Rating Basate su Evidenze
\item 3.3 Valutazione Indicatore per Indicatore
\item 3.4 Calcolo del Punteggio di Categoria
\item 3.5 Calcolo del Punteggio CPF
\item 3.6 Analisi dell'Indice di Convergenza
\item 3.7 Tecniche di Analisi Statistica
\item 3.8 Analisi dei Trend e Confronto Storico
\item 3.9 Affidabilità Inter-Valutatore
\item 3.10 Esercizio Pratico: Scoring di Case Study
\end{itemize}

\textbf{Modulo 4: Tecniche di Preservazione della Privacy (6 ore)}
\begin{itemize}
\item 4.1 Implementazione delle Unità di Aggregazione Minime
\item 4.2 Matematica della Privacy Differenziale
\item 4.3 Meccanismi di Ritardo Temporale
\item 4.4 Analisi Basata sui Ruoli
\item 4.5 Tecniche di Anonimizzazione dei Dati
\item 4.6 Storage e Trasmissione Sicuri dei Dati
\item 4.7 Esercizio Pratico: Implementazione della Privacy
\end{itemize}

\textbf{Modulo 5: Scrittura Report e Comunicazione (8 ore)}
\begin{itemize}
\item 5.1 Sviluppo della Sintesi Esecutiva
\item 5.2 Documentazione dei Risultati Tecnici
\item 5.3 Visualizzazione dei Risultati della Valutazione
\item 5.4 Raccomandazioni per il Trattamento del Rischio
\item 5.5 Comunicazione Specifica per Stakeholder
\item 5.6 Competenze di Presentazione
\item 5.7 Gestione di Risultati Sensibili
\item 5.8 Valutazione Finale: Sviluppo di Report Completo
\end{itemize}

\subsection{CPF-301: Implementazione Avanzata (40 ore)}

\textbf{Modulo 1: Progettazione degli Interventi (10 ore)}
\begin{itemize}
\item 1.1 Dalla Valutazione all'Azione
\item 1.2 Selezione di Interventi Basati su Evidenze
\item 1.3 Principi di Intervento Psicologico
\item 1.4 Integrazione dei Controlli Tecnici
\item 1.5 Gestione del Cambiamento Organizzativo
\item 1.6 Test Pilota degli Interventi
\end{itemize}

\textbf{Modulo 2: Monitoraggio Continuo (10 ore)}
\begin{itemize}
\item 2.1 Monitoraggio degli Indicatori in Tempo Reale
\item 2.2 Strategie di Integrazione SIEM
\item 2.3 Sistemi di Alerting Automatizzato
\item 2.4 Progettazione e Implementazione Dashboard
\item 2.5 Rilevamento degli Stati Convergenti
\item 2.6 Protezioni della Privacy nel Monitoraggio Continuo
\end{itemize}

\textbf{Modulo 3: Strategie di Integrazione (10 ore)}
\begin{itemize}
\item 3.1 Integrazione con le Security Operations
\item 3.2 Miglioramento della Risposta agli Incidenti
\item 3.3 Potenziamento della Threat Intelligence
\item 3.4 Considerazioni sull'Architettura di Sicurezza
\item 3.5 Integrazione di Governance e Conformità
\item 3.6 Allineamento con la Gestione del Rischio Aziendale
\end{itemize}

\textbf{Modulo 4: Misurazione dell'Efficacia (10 ore)}
\begin{itemize}
\item 4.1 Metriche e KPI
\item 4.2 Metodologie di Calcolo del ROI
\item 4.3 Analisi della Riduzione degli Incidenti
\item 4.4 Studi di Confronto Prima-Dopo
\item 4.5 Processi di Miglioramento Continuo
\item 4.6 Progetto Capstone: Piano di Implementazione
\end{itemize}

\subsection{CPF-401: Tecniche di Audit (40 ore)}

\textbf{Modulo 1: Fondamenti dell'Audit (8 ore)}
\begin{itemize}
\item 1.1 Principi ISO 19011:2018
\item 1.2 Panoramica dei Requisiti CPF-27001:2025
\item 1.3 Panoramica del Processo di Audit
\item 1.4 Competenze ed Etica dell'Auditor
\item 1.5 Indipendenza e Obiettività
\item 1.6 Standard di Condotta Professionale
\end{itemize}

\textbf{Modulo 2: Pianificazione dell'Audit (8 ore)}
\begin{itemize}
\item 2.1 Ambito e Obiettivi dell'Audit
\item 2.2 Pianificazione dell'Audit Basata sul Rischio
\item 2.3 Selezione del Team di Audit
\item 2.4 Allocazione delle Risorse
\item 2.5 Sviluppo del Piano di Audit
\item 2.6 Comunicazione con l'Auditee
\end{itemize}

\textbf{Modulo 3: Esecuzione dell'Audit (12 ore)}
\begin{itemize}
\item 3.1 Conduzione della Riunione di Apertura
\item 3.2 Tecniche di Revisione Documentale
\item 3.3 Metodologie di Intervista
\item 3.4 Strategie di Campionamento
\item 3.5 Raccolta e Documentazione delle Evidenze
\item 3.6 Tecniche di Osservazione
\item 3.7 Sviluppo dei Risultati
\item 3.8 Conduzione della Riunione di Chiusura
\item 3.9 Esercizio Pratico: Mock Audit
\end{itemize}

\textbf{Modulo 4: Reporting dell'Audit (6 ore)}
\begin{itemize}
\item 4.1 Classificazione delle Non Conformità
\item 4.2 Documentazione di Osservazioni e Opportunità
\item 4.3 Struttura del Report di Audit
\item 4.4 Scrittura Chiara e Obiettiva
\item 4.5 Raccomandazioni per Azioni Correttive
\item 4.6 Revisione del Report e Assicurazione Qualità
\end{itemize}

\textbf{Modulo 5: Follow-Up e Chiusura (6 ore)}
\begin{itemize}
\item 5.1 Revisione del Piano di Azioni Correttive
\item 5.2 Verifica delle Correzioni
\item 5.3 Valutazione dell'Efficacia
\item 5.4 Criteri di Chiusura dell'Audit
\item 5.5 Miglioramento Continuo dai Risultati dell'Audit
\item 5.6 Preparazione all'Esame Pratico Finale
\end{itemize}

\section{Moduli di Domanda}

\subsection{Template Domanda di Certificazione Individuale}

\textbf{Domanda di Certificazione CPF}

\textit{Tipo di Certificazione (Selezionare Uno):}
\begin{itemize}
\item[$\square$] CPF Assessor
\item[$\square$] CPF Practitioner
\item[$\square$] CPF Auditor
\end{itemize}

\textbf{Informazioni Personali:}
\begin{itemize}
\item Nome Legale Completo: \underline{\hspace{10cm}}
\item Nome Preferito: \underline{\hspace{10cm}}
\item Data di Nascita: \underline{\hspace{4cm}}
\item Indirizzo Email: \underline{\hspace{10cm}}
\item Numero di Telefono: \underline{\hspace{6cm}}
\item Indirizzo Postale: \underline{\hspace{10cm}}
\item \underline{\hspace{12cm}}
\end{itemize}

\textbf{Istruzione (Laurea o Superiore):}
\begin{itemize}
\item Istituzione: \underline{\hspace{10cm}}
\item Tipo di Laurea: \underline{\hspace{6cm}} Specializzazione: \underline{\hspace{5cm}}
\item Data di Conferimento: \underline{\hspace{4cm}}
\item Trascrizione Ufficiale Allegata: $\square$ Sì $\square$ No
\end{itemize}

\textbf{Esperienza Professionale:}

\textit{Posizione Attuale/Più Recente:}
\begin{itemize}
\item Datore di Lavoro: \underline{\hspace{10cm}}
\item Titolo della Posizione: \underline{\hspace{10cm}}
\item Date: \underline{\hspace{3cm}} a \underline{\hspace{3cm}}
\item Responsabilità Rilevanti: \underline{\hspace{10cm}}
\item \underline{\hspace{12cm}}
\item \underline{\hspace{12cm}}
\end{itemize}

\textit{Esperienza Rilevante Aggiuntiva (allegare pagine aggiuntive se necessario):}

\textbf{Completamento della Formazione:}
\begin{itemize}
\item[$\square$] CPF-101: Fondamenti del Framework
\begin{itemize}
\item Data di Completamento: \underline{\hspace{4cm}}
\item Fornitore di Formazione: \underline{\hspace{6cm}}
\item Numero di Certificato: \underline{\hspace{6cm}}
\end{itemize}
\item[$\square$] CPF-201: Metodologia di Valutazione (solo Assessor/Auditor)
\begin{itemize}
\item Data di Completamento: \underline{\hspace{4cm}}
\item Fornitore di Formazione: \underline{\hspace{6cm}}
\item Numero di Certificato: \underline{\hspace{6cm}}
\end{itemize}
\item[$\square$] CPF-401: Tecniche di Audit (solo Auditor)
\begin{itemize}
\item Data di Completamento: \underline{\hspace{4cm}}
\item Fornitore di Formazione: \underline{\hspace{6cm}}
\item Numero di Certificato: \underline{\hspace{6cm}}
\end{itemize}
\item[$\square$] Formazione Auditor ISO 19011 (solo Auditor)
\begin{itemize}
\item Data di Completamento: \underline{\hspace{4cm}}
\item Fornitore di Formazione: \underline{\hspace{6cm}}
\end{itemize}
\end{itemize}

\textbf{Referenze Professionali (Minimo 2):}

\textit{Referenza 1:}
\begin{itemize}
\item Nome: \underline{\hspace{8cm}} Titolo: \underline{\hspace{5cm}}
\item Organizzazione: \underline{\hspace{10cm}}
\item Email: \underline{\hspace{8cm}} Telefono: \underline{\hspace{5cm}}
\item Relazione: \underline{\hspace{10cm}}
\end{itemize}

\textit{Referenza 2:}
\begin{itemize}
\item Nome: \underline{\hspace{8cm}} Titolo: \underline{\hspace{5cm}}
\item Organizzazione: \underline{\hspace{10cm}}
\item Email: \underline{\hspace{8cm}} Telefono: \underline{\hspace{5cm}}
\item Relazione: \underline{\hspace{10cm}}
\end{itemize}

\textbf{Riconoscimento del Codice Etico:}

Ho letto e accetto di rispettare il Codice Etico CPF, incluso:
\begin{itemize}
\item[$\square$] Mantenere integrità e obiettività
\item[$\square$] Proteggere la riservatezza dei dati di valutazione
\item[$\square$] Praticare entro i confini delle competenze
\item[$\square$] Implementare metodologie che preservano la privacy
\item[$\square$] Non usare mai dati di valutazione per profilazione individuale
\item[$\square$] Aderire a tutti i requisiti di condotta professionale
\end{itemize}

\textbf{Dichiarazione:}

Dichiaro che le informazioni fornite in questa domanda sono vere, complete e accurate secondo la mia migliore conoscenza. Comprendo che informazioni false o fuorvianti possono risultare nella negazione della certificazione o nella revoca della certificazione se già concessa.

Firma: \underline{\hspace{8cm}} Data: \underline{\hspace{4cm}}

\textbf{Allegati Richiesti:}
\begin{itemize}
\item[$\square$] Trascrizione(i) ufficiale(i) o certificato(i) di laurea
\item[$\square$] Lettere di verifica dell'esperienza o portfolio professionale
\item[$\square$] Certificati di completamento della formazione
\item[$\square$] CV/curriculum attuale
\item[$\square$] Conferma pagamento tariffa di domanda
\item[$\square$] Documento d'identità con foto rilasciato dal governo (copia)
\end{itemize}

\textbf{Tariffa di Domanda:}
\begin{itemize}
\item CPF Assessor: \$300
\item CPF Practitioner: \$200
\item CPF Auditor: \$400
\end{itemize}

Metodo di Pagamento: $\square$ Carta di Credito $\square$ Bonifico Bancario $\square$ Assegno

\textit{Presentare domanda completata con tutti gli allegati richiesti a:}

CPF Certification Body\\
Dipartimento Domande di Certificazione\\
Email: certification@cpf-cert.org\\
Portale Web: https://apply.cpf-cert.org

\subsection{Template Domanda di Certificazione Organizzativa}

\textbf{Domanda di Certificazione Organizzativa CPF}

\textit{Livello di Certificazione Target (Selezionare Uno):}
\begin{itemize}
\item[$\square$] Livello 1: Fondazione (Punteggio CPF 100-149)
\item[$\square$] Livello 2: Intermedio (Punteggio CPF 70-99)
\item[$\square$] Livello 3: Avanzato (Punteggio CPF 40-69)
\item[$\square$] Livello 4: Esemplare (Punteggio CPF 0-39)
\end{itemize}

\textbf{Informazioni sull'Organizzazione:}
\begin{itemize}
\item Nome Legale dell'Organizzazione: \underline{\hspace{10cm}}
\item Nome Operativo (se diverso): \underline{\hspace{10cm}}
\item Settore Industriale: \underline{\hspace{10cm}}
\item Dimensione dell'Organizzazione: $\square$ 1-50 $\square$ 51-250 $\square$ 251-1000 $\square$ 1000+
\item Sede Centrale: \underline{\hspace{10cm}}
\item Sito Web: \underline{\hspace{10cm}}
\end{itemize}

\textbf{Contatto Primario:}
\begin{itemize}
\item Nome: \underline{\hspace{8cm}} Titolo: \underline{\hspace{5cm}}
\item Email: \underline{\hspace{8cm}} Telefono: \underline{\hspace{5cm}}
\end{itemize}

\textbf{Coordinatore CPF:}
\begin{itemize}
\item Nome: \underline{\hspace{8cm}} Titolo: \underline{\hspace{5cm}}
\item Email: \underline{\hspace{8cm}} Telefono: \underline{\hspace{5cm}}
\item Certificazione CPF (se applicabile): \underline{\hspace{6cm}}
\end{itemize}

\textbf{Ambito della Certificazione:}
\begin{itemize}
\item Sedi Coperte: \underline{\hspace{10cm}}
\item \underline{\hspace{12cm}}
\item Unità di Business Coperte: \underline{\hspace{10cm}}
\item \underline{\hspace{12cm}}
\item Personale Totale nell'Ambito: \underline{\hspace{4cm}}
\item Esclusioni (se presenti): \underline{\hspace{10cm}}
\item \underline{\hspace{12cm}}
\end{itemize}

\textbf{Informazioni sulla Valutazione CPF:}
\begin{itemize}
\item Data di Valutazione: \underline{\hspace{4cm}}
\item Nome Assessor/Auditor Certificato: \underline{\hspace{8cm}}
\item Numero di Certificazione: \underline{\hspace{6cm}}
\item Punteggio CPF: \underline{\hspace{3cm}} (Range: 0-200)
\item Report di Valutazione Allegato: $\square$ Sì $\square$ No
\end{itemize}

\textbf{Certificazioni Esistenti:}
\begin{itemize}
\item[$\square$] ISO/IEC 27001 - Numero Certificato: \underline{\hspace{6cm}}
\item[$\square$] ISO 9001 - Numero Certificato: \underline{\hspace{6cm}}
\item[$\square$] SOC 2 - Data Report: \underline{\hspace{6cm}}
\item[$\square$] Altro: \underline{\hspace{10cm}}
\end{itemize}

\textbf{Stato di Implementazione:}

\textit{Elementi del Programma CPF (Selezionare tutti quelli implementati):}
\begin{itemize}
\item[$\square$] Policy CPF Documentata
\item[$\square$] Coordinatore CPF Designato
\item[$\square$] Procedure di Protezione della Privacy
\item[$\square$] Piani di Trattamento del Rischio
\item[$\square$] Monitoraggio Continuo (se applicabile)
\item[$\square$] Integrazione con ISMS
\item[$\square$] Processo di Riesame della Direzione
\item[$\square$] Programma CPE per lo Staff
\end{itemize}

\textbf{Impegno della Direzione:}

Io, come rappresentante autorizzato dell'organizzazione, mi impegno a:
\begin{itemize}
\item[$\square$] Mantenere la gestione sistematica delle vulnerabilità psicologiche
\item[$\square$] Fornire le risorse necessarie per l'implementazione CPF
\item[$\square$] Conformarmi ai requisiti di sorveglianza
\item[$\square$] Implementare azioni correttive come necessario
\item[$\square$] Proteggere la privacy in tutte le attività CPF
\item[$\square$] Partecipare ai riesami della direzione richiesti
\end{itemize}

\textbf{Autorizzazione:}

Nome: \underline{\hspace{8cm}} Titolo: \underline{\hspace{5cm}}

Firma: \underline{\hspace{8cm}} Data: \underline{\hspace{4cm}}

\textbf{Allegati Richiesti:}
\begin{itemize}
\item[$\square$] Report di Valutazione CPF (completo)
\item[$\square$] Documento di Policy CPF
\item[$\square$] Organigramma che mostra i ruoli CPF
\item[$\square$] Procedure di Protezione della Privacy
\item[$\square$] Piani di Trattamento del Rischio per Indicatori Rossi
\item[$\square$] Evidenza di integrazione ISMS (se applicabile)
\item[$\square$] Conferma pagamento tariffa di domanda
\end{itemize}

\textbf{Tariffa di Domanda (In Base alla Dimensione dell'Organizzazione):}
\begin{itemize}
\item 1-50 dipendenti: \$500
\item 51-250 dipendenti: \$1.000
\item 251-1000 dipendenti: \$1.500
\item 1000+ dipendenti: \$2.000
\end{itemize}

Metodo di Pagamento: $\square$ Carta di Credito $\square$ Bonifico Bancario $\square$ Assegno

\textit{Presentare domanda completata con tutti gli allegati richiesti a:}

CPF Certification Body\\
Dipartimento Certificazione Organizzativa\\
Email: org-certification@cpf-cert.org\\
Portale Web: https://apply.cpf-cert.org

\subsection{Template Domanda di Ricertificazione}

\textbf{Domanda di Ricertificazione CPF}

\textit{Certificazione Attuale:}
\begin{itemize}
\item Tipo di Certificazione: \underline{\hspace{8cm}}
\item Numero di Certificato: \underline{\hspace{8cm}}
\item Data di Certificazione Originale: \underline{\hspace{4cm}}
\item Data di Scadenza Attuale: \underline{\hspace{4cm}}
\end{itemize}

\textbf{Informazioni Personali:}
\begin{itemize}
\item Nome Completo: \underline{\hspace{10cm}}
\item Email: \underline{\hspace{8cm}} Telefono: \underline{\hspace{5cm}}
\item Le tue informazioni di contatto sono cambiate? $\square$ Sì $\square$ No
\item Se sì, fornire informazioni aggiornate: \underline{\hspace{8cm}}
\end{itemize}

\textbf{Educazione Professionale Continua (CPE):}

\textit{Riepilogo CPE (Ciclo di 3 Anni):}
\begin{itemize}
\item Crediti Anno 1: \underline{\hspace{3cm}} (Richiesti: \underline{\hspace{2cm}})
\item Crediti Anno 2: \underline{\hspace{3cm}} (Richiesti: \underline{\hspace{2cm}})
\item Crediti Anno 3: \underline{\hspace{3cm}} (Richiesti: \underline{\hspace{2cm}})
\item Crediti CPE Totali: \underline{\hspace{3cm}} (Richiesti: \underline{\hspace{2cm}})
\end{itemize}

\textit{Documentazione CPE:}
\begin{itemize}
\item[$\square$] Log CPE completo allegato (con date, attività, crediti)
\item[$\square$] Certificati/documentazione di supporto allegati
\item[$\square$] Tutte le attività conformi alla policy CPE
\end{itemize}

\textbf{Esperienza Professionale (Ultimi 3 Anni):}

\textit{Per Assessor:}
\begin{itemize}
\item Numero di Valutazioni CPF Condotte: \underline{\hspace{3cm}}
\item Riepilogo Valutazioni Allegato: $\square$ Sì $\square$ No
\end{itemize}

\textit{Per Practitioner:}
\begin{itemize}
\item Portfolio Aggiornato Allegato: $\square$ Sì $\square$ No
\item Numero di Progetti di Implementazione: \underline{\hspace{3cm}}
\end{itemize}

\textit{Per Auditor:}
\begin{itemize}
\item Giorni di Audit Totali: \underline{\hspace{3cm}} (Richiesti: 45)
\item Numero di Ruoli Lead Auditor: \underline{\hspace{3cm}} (Richiesti: 5)
\item Riepilogo Audit Allegato: $\square$ Sì $\square$ No
\end{itemize}

\textbf{Attestazione Etica:}

Attesto che durante il periodo di certificazione passato:
\begin{itemize}
\item[$\square$] Ho rispettato il Codice Etico CPF
\item[$\square$] Ho mantenuto i requisiti di riservatezza
\item[$\square$] Ho praticato entro i confini delle mie competenze
\item[$\square$] Ho mantenuto pratiche che preservano la privacy
\item[$\square$] Non ho reclami etici irrisolti
\item[$\square$] Non sono soggetto a nessuna azione disciplinare professionale
\end{itemize}

Accetto di continuare ad aderire al Codice Etico CPF per il prossimo periodo di certificazione.

Firma: \underline{\hspace{8cm}} Data: \underline{\hspace{4cm}}

\textbf{Tariffa di Ricertificazione:}
\begin{itemize}
\item CPF Assessor: \$400
\item CPF Practitioner: \$300
\item CPF Auditor: \$500
\item Ricertificazione Tardiva (entro 90 giorni): Aggiungere \$100
\end{itemize}

Metodo di Pagamento: $\square$ Carta di Credito $\square$ Bonifico Bancario $\square$ Assegno

\textbf{Allegati Richiesti:}
\begin{itemize}
\item[$\square$] Log CPE completo con documentazione di supporto
\item[$\square$] Documentazione dell'esperienza (valutazioni, portfolio, o audit)
\item[$\square$] Referenze professionali (se richieste)
\item[$\square$] Conferma pagamento tariffa di ricertificazione
\end{itemize}

\textit{Presentare domanda completata con tutti gli allegati richiesti a:}

CPF Certification Body\\
Dipartimento Ricertificazione\\
Email: recertification@cpf-cert.org\\
Portale Web: https://recertify.cpf-cert.org

\textit{Nota: Le domande dovrebbero essere presentate 90-180 giorni prima della scadenza per assicurare elaborazione tempestiva.}

\section*{Controllo del Documento}

\textbf{Cronologia delle Versioni:}

\begin{tabular}{llp{8cm}}
\toprule
Versione & Data & Modifiche \\
\midrule
1.0 & Gennaio 2025 & Rilascio iniziale \\
\bottomrule
\end{tabular}

\vspace{1em}

\textbf{Programma di Revisione:}
\begin{itemize}
\item Revisione annuale: Gennaio di ogni anno
\item Revisione maggiore: Come necessario in base a cambiamenti del settore, avanzamenti della ricerca, o feedback degli stakeholder
\item Prossima revisione programmata: Gennaio 2026
\end{itemize}

\textbf{Approvazione:}

Proprietario del Documento: Comitato dello Schema di Certificazione CPF

Approvato da: \underline{\hspace{8cm}} Data: \underline{\hspace{4cm}}

\textbf{Distribuzione:}
\begin{itemize}
\item Tutti gli organismi di certificazione CPF approvati
\item Fornitori di formazione CPF
\item Versione pubblica disponibile su: https://cpf3.org/certification-scheme
\end{itemize}

\textbf{Informazioni di Contatto:}

CPF Certification Body\\
Sito Web: https://cpf3.org\\
Email: info@cpf3.org\\
Domande sulla Certificazione: certification@cpf-cert.org\\
Supporto Tecnico: support@cpf-cert.org

\vspace{2em}

\begin{center}
\textit{Fine del Documento}
\end{center}

\end{document}
