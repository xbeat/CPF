\documentclass[11pt,a4paper]{article}

% Packages
\usepackage[utf8]{inputenc}
\usepackage[english]{babel}
\usepackage[margin=2.5cm]{geometry}
\usepackage{hyperref}
\usepackage{fancyhdr}
\usepackage{enumitem}
\usepackage{tabularx}
\usepackage{amssymb}

% Page style
\pagestyle{fancy}
\fancyhf{}
\renewcommand{\headrulewidth}{0.4pt}
\fancyhead[L]{Accordo di Servizi Professionali CPF}
\fancyhead[R]{Progetto \#: \_\_\_\_\_\_\_\_}
\fancyfoot[C]{\thepage}

% Spacing
\setlength{\parindent}{0pt}
\setlength{\parskip}{0.8em}

% Title
\title{\textbf{ACCORDO DI SERVIZI PROFESSIONALI\\CPF}}
\author{}
\date{}

\begin{document}

\maketitle

\section*{PARTI}

Questo Accordo di Servizi Professionali ("Accordo") è stipulato in data \_\_\_ del mese di \_\_\_\_\_\_\_\_\_\_\_, 20\_\_\_ ("Data di Efficacia"), tra:

\textbf{[NOME FORNITORE DI SERVIZI]} ("Fornitore di Servizi" o "Consulente")\\
\texttt{[Se Individuo: Professionista certificato CPF individuale]}\\
\texttt{[Se Società: Società di consulenza / Stato di Fornitore di Servizi Autorizzato se applicabile]}\\
Indirizzo: [Indirizzo]\\
Email: [Email]\\
Telefono: [Telefono]\\
Certificazione: [Valutatore CPF / Practitioner / Auditor / ASP]

E

\textbf{[NOME ORGANIZZAZIONE CLIENTE]} ("Cliente" o "Organizzazione")\\
Una [giurisdizione] [tipo di entità]\\
Numero di Registrazione: [Numero]\\
Sede Principale: [Indirizzo]\\
Contatto Primario: [Nome, Titolo]\\
Email: [Email]\\
Telefono: [Telefono]

Collettivamente denominati le "Parti" e individualmente una "Parte."

\section*{PREMESSA}

CONSIDERATO CHE, il Fornitore di Servizi è un professionista certificato CPF [o Fornitore di Servizi Autorizzato CPF] con competenza nella valutazione e gestione della vulnerabilità psicologica nei contesti di cybersecurity;

CONSIDERATO CHE, il Cliente desidera impegnare il Fornitore di Servizi per fornire servizi professionali correlati al CPF;

CONSIDERATO CHE, il Fornitore di Servizi accetta di fornire tali servizi in conformità con la metodologia e gli standard etici del CPF;

ORA, PERTANTO, in considerazione dei reciproci patenti e accordi contenuti nel presente, le Parti concordano come segue:

\section{DEFINIZIONI}

\textbf{1.1 "Servizi"} significa i servizi professionali CPF che saranno forniti dal Fornitore di Servizi come descritto nell'Allegato A (Dichiarazione di Lavoro).

\textbf{1.2 "Metodologia CPF"} significa la metodologia, gli standard e le migliori pratiche del Cybersecurity Psychology Framework come mantenuti da CPF3.

\textbf{1.3 "Dati di Valutazione"} significa tutti i dati raccolti, generati o elaborati durante le valutazioni CPF, inclusi punteggi di vulnerabilità, indicatori e informazioni correlate.

\textbf{1.4 "Prodotti Forniti"} significa tutti i rapporti, documentazione, strumenti e materiali che saranno forniti dal Fornitore di Servizi come specificato nell'Allegato A.

\textbf{1.5 "Informazioni Confidenziali"} significa tutte le informazioni non pubbliche divulgate da una Parte all'altra, incluse ma non limitate ai Dati di Valutazione, informazioni commerciali, dati tecnici e processi proprietari.

\textbf{1.6 "Codice Etico CPF"} significa gli standard di condotta professionale per professionisti certificati CPF.

\section{AMBITO DEI SERVIZI}

\textbf{2.1 Servizi da Fornire.} Il Fornitore di Servizi fornirà i seguenti servizi CPF (selezionare tutti quelli applicabili):

\begin{itemize}
\item[$\square$] \textbf{Servizi di Valutazione CPF:}
\begin{itemize}
\item Valutazione sistematica delle vulnerabilità psicologiche su 100 indicatori CPF
\item Raccolta dati attraverso interviste, osservazioni, revisione documentale e sondaggi
\item Applicazione della metodologia di punteggio ternario (Verde/Giallo/Rosso)
\item Calcolo del Punteggio CPF e dei Punteggi per Categoria
\item Identificazione degli stati di vulnerabilità convergenti
\item Gestione e reporting dei dati preservando la privacy
\end{itemize}

\item[$\square$] \textbf{Servizi di Implementazione CPF:}
\begin{itemize}
\item Progettazione e implementazione di interventi basati su CPF
\item Integrazione di CPF con programmi di sicurezza esistenti (ISO 27001, NIST CSF)
\item Sviluppo di sistemi di monitoraggio continuo che preservano la privacy
\item Pianificazione ed esecuzione del trattamento del rischio
\item Sviluppo di politiche e procedure CPF
\item Supporto all'implementazione del sistema di gestione
\end{itemize}

\item[$\square$] \textbf{Servizi di Formazione CPF:}
\begin{itemize}
\item Formazione di sensibilizzazione al CPF per il personale
\item Formazione specializzata per team di sicurezza
\item Formazione e supporto per Coordinatori CPF
\item Briefing esecutivi sulle vulnerabilità psicologiche
\end{itemize}

\item[$\square$] \textbf{Preparazione Audit CPF:}
\begin{itemize}
\item Valutazione di preparazione per la certificazione CPF
\item Analisi dei gap rispetto ai requisiti CPF-27001
\item Revisione e miglioramento della documentazione
\item Audit simulati e supporto al rimedio
\end{itemize}

\item[$\square$] \textbf{Altri Servizi CPF:}
\begin{itemize}
\item[] \underline{\hspace{10cm}}
\item[] \underline{\hspace{10cm}}
\end{itemize}
\end{itemize}

\textbf{2.2 Dichiarazione di Lavoro Dettagliata.} L'ambito dettagliato, i prodotti forniti, la tempistica e i criteri di successo sono specificati nell'Allegato A (Dichiarazione di Lavoro), che è incorporato per riferimento e forma parte di questo Accordo.

\textbf{2.3 Responsabilità del Fornitore di Servizi.} Il Fornitore di Servizi dovrà:

\begin{enumerate}[label=\alph*)]
\item Eseguire i Servizi in modo professionale conforme agli standard del settore
\item Applicare la metodologia CPF in modo accurato e coerente
\item Aderire al Codice Etico CPF in ogni momento
\item Implementare metodologie che preservano la privacy (unità minime di aggregazione, privacy differenziale ove applicabile)
\item Fornire personale qualificato con appropriate certificazioni CPF
\item Mantenere le certificazioni CPF attuali durante tutto l'impegno
\item Fornire i Servizi secondo la tempistica dell'Allegato A
\item Comunicare progressi e problemi tempestivamente
\item Mantenere un'assicurazione di responsabilità professionale
\end{enumerate}

\textbf{2.4 Responsabilità del Cliente.} Il Cliente dovrà:

\begin{enumerate}[label=\alph*)]
\item Fornire tempestivo accesso a strutture, sistemi, personale e documentazione necessari per i Servizi
\item Designare una persona di contatto primaria con autorità decisionale
\item Fornire informazioni accurate e complete richieste dal Fornitore di Servizi
\item Revisionare e fornire feedback sui Prodotti Forniti entro i tempi specificati
\item Rendere disponibile il personale per interviste, sondaggi e raccolta dati come necessario
\item Assicurare la cooperazione di tutti i reparti e personale pertinenti
\item Fornire spazio di lavoro sicuro e attrezzature necessarie (se servizi in loco)
\item Pagare i compensi secondo il calendario concordato
\end{enumerate}

\textbf{2.5 Fuori Ambito.} Quanto segue è esplicitamente fuori ambito a meno che non sia specificamente aggiunto tramite modifica scritta:

\begin{itemize}
\item Servizi non correlati alla metodologia CPF
\item Valutazioni psicologiche cliniche o interventi terapeutici
\item Profili individuali o valutazioni delle prestazioni
\item Servizi che richiedono certificazioni non possedute dal Fornitore di Servizi
\item {[Esclusioni aggiuntive come appropriato]}
\end{itemize}

\section{DURATA E RISOLUZIONE}

\textbf{3.1 Durata.} Questo Accordo ha inizio alla Data di Efficacia e continua fino a:

\begin{itemize}
\item[$\square$] Completamento dei Servizi come specificato nell'Allegato A (stimato: \underline{\hspace{4cm}})
\item[$\square$] Termine fisso che termina il: \underline{\hspace{6cm}}
\item[$\square$] Risolto da una delle Parti secondo la Sezione 3.3
\end{itemize}

\textbf{3.2 Estensioni.} La durata può essere estesa tramite accordo scritto reciproco, inclusa Dichiarazione di Lavoro e compensi aggiornati.

\textbf{3.3 Diritti di Risoluzione.}

\textit{Risoluzione per Comodità:}
\begin{itemize}
\item Entrambe le Parti possono risolvere senza causa con preavviso scritto di trenta (30) giorni
\item Il Cliente pagherà per i Servizi eseguiti fino alla data di risoluzione più i costi ragionevoli di chiusura
\item Il Fornitore di Servizi consegnerà tutti i Prodotti Forniti completati e il lavoro in corso
\end{itemize}

\textit{Risoluzione per Giusta Causa:}
Entrambe le Parti possono risolvere immediatamente con preavviso scritto se l'altra Parte:
\begin{itemize}
\item Viola materialmente questo Accordo e non rimedia entro quindici (15) giorni dal preavviso scritto
\item Diventa insolvente o soggetta a procedure di fallimento
\item Si impegna in frode, negligenza grave o condotta intenzionale
\end{itemize}

\textit{Risoluzione per Perdita di Certificazione:}
\begin{itemize}
\item Questo Accordo si risolve automaticamente se la certificazione CPF del Fornitore di Servizi viene sospesa o revocata
\item Il Fornitore di Servizi deve notificare immediatamente il Cliente di eventuali cambiamenti di stato della certificazione
\item Il Cliente ha diritto al rimborso dei compensi prepagati per Servizi non ancora forniti
\end{itemize}

\textbf{3.4 Effetti della Risoluzione.} In caso di risoluzione:

\begin{enumerate}[label=\alph*)]
\item Il Fornitore di Servizi cesserà immediatamente i Servizi (eccetto quanto necessario per una chiusura ordinata)
\item Il Fornitore di Servizi consegnerà tutti i Prodotti Forniti completati e il lavoro in corso entro cinque (5) giorni lavorativi
\item Il Cliente pagherà tutti gli importi non contestati dovuti per i Servizi eseguiti fino alla data di risoluzione
\item Il Fornitore di Servizi restituirà o distruggerà tutte le Informazioni Confidenziali del Cliente secondo le istruzioni scritte del Cliente
\item Le Sezioni 4 (Compensi), 6 (Confidenzialità), 7 (Protezione Dati), 8 (Proprietà Intellettuale), 10 (Limitazione di Responsabilità), 11 (Indennizzo) e 12 (Disposizioni Generali) sopravvivono alla risoluzione
\end{enumerate}

\section{COMPENSI E PAGAMENTO}

\textbf{4.1 Struttura dei Compensi.} Il Cliente pagherà il Fornitore di Servizi come segue (selezionare quelli applicabili):

\begin{itemize}
\item[$\square$] \textbf{Compenso Fisso:} \$\underline{\hspace{4cm}} per tutti i Servizi descritti nell'Allegato A
\begin{itemize}
\item Calendario di pagamento: \underline{\hspace{8cm}}
\end{itemize}

\item[$\square$] \textbf{Tempo e Materiali:}
\begin{itemize}
\item Tariffa/e oraria/e: \$\underline{\hspace{3cm}}/ora per [ruolo/persona]
\item Ore totali stimate: \underline{\hspace{4cm}}
\item importo non superabile: \$\underline{\hspace{4cm}} (se applicabile)
\item Frequenza di fatturazione: $\square$ Settimanale $\square$ Bisettimanale $\square$ Mensile
\end{itemize}

\item[$\square$] \textbf{Basato su Tappe:}
\begin{itemize}
\item Pagamenti legati al completamento delle tappe come specificato nell'Allegato A
\item Pagamento dovuto entro \underline{\hspace{3cm}} giorni dal completamento e accettazione della tappa
\end{itemize}

\item[$\square$] \textbf{Retainer:}
\begin{itemize}
\item Retainer mensile: \$\underline{\hspace{4cm}}
\item Include fino a \underline{\hspace{3cm}} ore al mese
\item Ore aggiuntive a \$\underline{\hspace{3cm}}/ora
\end{itemize}
\end{itemize}

\textbf{4.2 Spese.} Il Cliente rimborserà il Fornitore di Servizi per spese ragionevoli pre-approvate incluse:

\begin{itemize}
\item Viaggio (biglietto aereo, trasporto a terra, parcheggio)
\item Alloggio e pasti per diaria (se applicabile)
\item Materiali e forniture direttamente correlati ai Servizi
\item Servizi o strumenti di terze parti richiesti per il progetto (se pre-approvati)
\end{itemize}

Il rimborso delle spese richiede:
\begin{itemize}
\item Approvazione scritta preventiva per spese che superano \$\underline{\hspace{3cm}} individualmente o \$\underline{\hspace{3cm}} in totale
\item Presentazione di ricevute e report spese
\item Pagamento entro trenta (30) giorni dalla presentazione
\end{itemize}

\textbf{4.3 Fatturazione e Termini di Pagamento.}

\begin{enumerate}[label=\alph*)]
\item Il Fornitore di Servizi presenterà le fatture secondo il calendario concordato
\item Le fatture devono includere:
\begin{itemize}
\item Descrizione dettagliata dei Servizi eseguiti
\item Ore lavorate (se tempo e materiali)
\item Spese con ricevute di supporto
\item Data di scadenza pagamento
\item Istruzioni di pagamento
\end{itemize}
\item Pagamento dovuto entro trenta (30) giorni dalla data della fattura a meno che non sia specificato diversamente
\item I pagamenti in ritardo sono soggetti a interessi all'1.5\% al mese o al massimo tasso legale, whichever è inferiore
\item Il Cliente può trattenere il pagamento per importi contestati fornendo spiegazione scritta; gli importi non contestati rimangono dovuti
\item I Servizi possono essere sospesi per fatture non pagate sessanta (60) giorni dopo la scadenza
\end{enumerate}

\textbf{4.4 Adeguamenti dei Compensi.} I compensi possono essere adeguati tramite accordo scritto reciproco per:
\begin{itemize}
\item Cambiamenti o aggiunte all'ambito
\item Tempistiche estese a causa di ritardi del Cliente
\item Lavoro aggiuntivo richiesto dal Cliente
\item Complessità impreviste che richiedono sforzo aggiuntivo
\end{itemize}

Tutti gli adeguamenti dei compensi richiedono un ordine di modifica scritto firmato da entrambe le Parti prima che il lavoro proceda.

\section{PRODOTTI FORNITI E ACCETTAZIONE}

\textbf{5.1 Prodotti Forniti.} Il Fornitore di Servizi fornirà i seguenti Prodotti Forniti come dettagliato nell'Allegato A:

\begin{itemize}
\item Rapporti di valutazione e risultati
\item Punteggi CPF e analisi delle vulnerabilità
\item Raccomandazioni per il trattamento del rischio
\item Piani di implementazione e documentazione
\item Materiali di formazione
\item Altri prodotti forniti specificati
\end{itemize}

\textbf{5.2 Formato e Metodo di Consegna.}
\begin{itemize}
\item Formato: $\square$ Elettronico (PDF) $\square$ Stampato $\square$ Entrambi $\square$ Altro: \underline{\hspace{4cm}}
\item Metodo di consegna: $\square$ Email $\square$ Portale sicuro $\square$ Consegna fisica $\square$ Altro: \underline{\hspace{3cm}}
\item Numero di copie (se stampato): \underline{\hspace{3cm}}
\end{itemize}

\textbf{5.3 Processo di Accettazione.}

\begin{enumerate}[label=\alph*)]
\item Il Fornitore di Servizi consegna il Prodotto Fornito al Cliente con avviso di consegna
\item Il Cliente ha quindici (15) giorni lavorativi per revisionare e:
\begin{itemize}
\item Accettare il Prodotto Fornito per iscritto, o
\item Rifiutare il Prodotto Fornito con specifica spiegazione scritta delle carenze
\end{itemize}
\item Se rifiutato, il Fornitore di Servizi ha dieci (10) giorni lavorativi per correggere le carenze e riconsegnare
\item Se nessuna risposta entro quindici (15) giorni, il Prodotto Fornito si considera accettato
\item L'accettazione non può essere negata irragionevolmente
\item L'accettazione attiva qualsiasi pagamento di tappa dovuto per quel Prodotto Fornito
\end{enumerate}

\textbf{5.4 Standard dei Prodotti Forniti.} Tutti i Prodotti Forniti devono:
\begin{itemize}
\item Conformarsi alla metodologia e agli standard CPF
\item Essere di qualità professionale adatti allo scopo previsto
\item Includere adeguate protezioni della privacy (nessun identificativo individuale, unità minime di aggregazione mantenute)
\item Essere completi e accurati basandosi sulle informazioni fornite dal Cliente
\item Soddisfare le specifiche nell'Allegato A
\end{itemize}

\section{CONFIDENZIALITÀ}

\textbf{6.1 Informazioni Confidenziali.} Ogni Parte riconosce di poter ricevere Informazioni Confidenziali dall'altra Parte.

\textit{Le Informazioni Confidenziali del Fornitore di Servizi includono:}
\begin{itemize}
\item Metodologie di valutazione e strumenti proprietari
\item Prezzi e informazioni commerciali
\item Processi tecnici e know-how
\end{itemize}

\textit{Le Informazioni Confidenziali del Cliente includono:}
\begin{itemize}
\item Dati di Valutazione e Punteggi CPF
\item Politiche interne, procedure e documentazione
\item Operazioni commerciali e strategie
\item Vulnerabilità e incidenti di sicurezza
\item Informazioni del personale
\item Informazioni finanziarie
\end{itemize}

\textbf{6.2 Obblighi.} Ogni Parte dovrà:

\begin{enumerate}[label=\alph*)]
\item Mantenere le Informazioni Confidenziali in stretta confidenzialità
\item Utilizzare le Informazioni Confidenziali solo per scopi autorizzati in questo Accordo
\item Proteggerle utilizzando almeno lo stesso grado di cura riservato alle proprie informazioni confidenziali (minimo: cura ragionevole)
\item Limitare l'accesso al personale con necessità di conoscere
\item Non divulgare a terzi senza previo consenso scritto
\item Assicurarsi che il personale sia vincolato da obblighi di confidenzialità
\end{enumerate}

\textbf{6.3 Eccezioni.} Gli obblighi non si applicano a informazioni che:
\begin{itemize}
\item Erano pubblicamente disponibili al momento della divulgazione o diventano pubbliche senza violazione
\item Erano legittimamente possedute prima della divulgazione
\item Sono sviluppate indipendentemente senza l'uso di Informazioni Confidenziali
\item Devono essere divulgate per legge (con tempestiva notifica per consentire misure protettive)
\end{itemize}

\textbf{6.4 Protezione Speciale per i Dati di Valutazione.} Il Fornitore di Servizi dovrà:
\begin{itemize}
\item Non utilizzare mai i Dati di Valutazione per profili individuali o valutazioni delle prestazioni
\item Mantenere unità minime di aggregazione (10 individui) in tutti i report
\item Implementare report con ritardo temporale (minimo 72 ore) per il monitoraggio in tempo reale
\item Memorizzare i Dati di Valutazione crittografati a riposo e in transito
\item Limitare l'accesso ai Dati di Valutazione al personale direttamente coinvolto nei Servizi
\item Restituire o distruggere i Dati di Valutazione secondo le istruzioni del Cliente al completamento del progetto
\item Non utilizzare mai i Dati di Valutazione del Cliente per benchmarking, ricerca o marketing senza esplicito consenso scritto
\end{itemize}

\textbf{6.5 Durata.} Gli obblighi di confidenzialità sopravvivono per cinque (5) anni dopo la risoluzione, eccetto per i segreti commerciali che sono protetti indefinitamente.

\section{PROTEZIONE DEI DATI E PRIVACY}

\textbf{7.1 Leggi Applicabili.} Entrambe le Parti dovranno conformarsi a tutte le leggi applicabili di protezione dati e privacy, incluse ma non limitate a:
\begin{itemize}
\item Regolamento Generale sulla Protezione dei Dati (GDPR) - UE
\item California Consumer Privacy Act (CCPA) - California, USA
\item {[Leggi applicabili aggiuntive basate sulla giurisdizione]}
\end{itemize}

\textbf{7.2 Ruoli di Elaborazione Dati.}
\begin{itemize}
\item Il Cliente è il Titolare del Dati per i dati del personale raccolti durante le valutazioni
\item Il Fornitore di Servizi è il Responsabile del Trattamento che agisce per conto e secondo le istruzioni del Cliente
\item Il Fornitore di Servizi elabora i dati personali solo come necessario per eseguire i Servizi
\item Il Fornitore di Servizi non elaborerà dati personali per propri scopi
\end{itemize}

\textbf{7.3 Obblighi di Protezione Dati del Fornitore di Servizi.} Il Fornitore di Servizi dovrà:

\begin{enumerate}[label=\alph*)]
\item Elaborare i dati personali solo secondo le istruzioni documentate del Cliente
\item Implementare appropriate misure tecniche e organizzative per proteggere i dati personali:
\begin{itemize}
\item Crittografia dei dati a riposo e in transito
\item Controlli di accesso e autenticazione
\item Logging di audit
\item Valutazioni di sicurezza regolari
\item Formazione del personale sulla protezione dei dati
\end{itemize}
\item Assicurarsi che tutto il personale sia vincolato dalla confidenzialità
\item Assistere il Cliente nel rispondere alle richieste degli interessati (accesso, correzione, cancellazione, ecc.)
\item Notificare il Cliente entro ventiquattro (24) ore di qualsiasi violazione dei dati
\item Assistere il Cliente nella notifica della violazione e mitigazione
\item Cancellare o restituire tutti i dati personali al completamento del progetto (secondo le istruzioni del Cliente)
\item Rendere disponibili le informazioni necessarie per dimostrare la conformità
\item Consentire e contribuire agli audit da parte del Cliente o di un revisore nominato
\end{enumerate}

\textbf{7.4 Sub-responsabili del Trattamento.} Il Fornitore di Servizi non può impegnare sub-responsabili del trattamento senza:
\begin{itemize}
\item Previo consenso scritto del Cliente per specifico sub-responsabile del trattamento
\item Assicurarsi che il sub-responsabile del trattamento sia vincolato dagli stessi obblighi di protezione dati
\item Rimanere pienamente responsabile per atti e omissioni del sub-responsabile del trattamento
\end{itemize}

\textbf{7.5 Trasferimenti di Dati.} Se i dati personali verranno trasferiti internazionalmente:
\begin{itemize}
\item Il Fornitore di Servizi dovrà implementare appropriate salvaguardie (Clausole Contrattuali Standard, decisioni di adeguatezza, ecc.)
\item Il Fornitore di Servizi dovrà notificare il Cliente dei meccanismi di trasferimento utilizzati
\item Il Fornitore di Servizi dovrà conformarsi ai requisiti di localizzazione dei dati se applicabili
\end{itemize}

\textbf{7.6 Diritti degli Interessati.} Il Fornitore di Servizi dovrà assistere il Cliente nel soddisfare i diritti degli interessati:
\begin{itemize}
\item Diritto di accesso
\item Diritto di rettifica
\item Diritto alla cancellazione ("diritto all'oblio")
\item Diritto di limitazione del trattamento
\item Diritto alla portabilità dei dati
\item Diritto di opposizione
\end{itemize}

Tempo di risposta: Entro dieci (10) giorni lavorativi dalla richiesta del Cliente.

\section{PROPRIETÀ INTELLETTUALE}

\textbf{8.1 IP Pre-esistente.} Ogni Parte mantiene tutti i diritti alla sua proprietà intellettuale pre-esistente, inclusa:

\textit{Il Fornitore di Servizi mantiene:}
\begin{itemize}
\item Metodologia CPF e strumenti di valutazione (in licenza da CPF3)
\item Processi e metodologie proprietarie
\item Modelli e formati standard dei prodotti forniti
\item Conoscenza generale e competenza
\end{itemize}

\textit{Il Cliente mantiene:}
\begin{itemize}
\item Politiche, procedure e documentazione esistenti
\item Processi e sistemi aziendali
\item Dati e informazioni organizzative
\end{itemize}

\textbf{8.2 Proprietà del Prodotto del Lavoro.} Tutti i Prodotti Forniti e il prodotto del lavoro creati specificamente per il Cliente in base a questo Accordo saranno di proprietà del Cliente, inclusi:
\begin{itemize}
\item Rapporti di valutazione e risultati specifici del Cliente
\item Punteggi CPF e analisi delle vulnerabilità per il Cliente
\item Piani di implementazione e documentazione specifici del Cliente
\item Materiali di formazione personalizzati sviluppati per il Cliente
\item Raccomandazioni e roadmap per il Cliente
\end{itemize}

\textbf{8.3 Licenza al Fornitore di Servizi.} Il Cliente concede al Fornitore di Servizi una licenza limitata per utilizzare le Informazioni Confidenziali del Cliente e il prodotto del lavoro solo per eseguire i Servizi in base a questo Accordo.

\textbf{8.4 Licenza al Cliente.} Il Fornitore di Servizi concede al Cliente:
\begin{itemize}
\item Licenza non esclusiva, perpetua, gratuita di utilizzare i Prodotti Forniti per scopi commerciali interni
\item Diritto di modificare i Prodotti Forniti per uso interno
\item Nessun diritto di sublicenziare, distribuire esternamente o creare opere derivate per scopi commerciali
\end{itemize}

\textbf{8.5 Utilizzo di Dati Anonimizzati.} Il Fornitore di Servizi può:
\begin{itemize}
\item Utilizzare dati aggregati e anonimizzati da più clienti per benchmarking e ricerca
\item Solo se i dati sono completamente de-identificati e non possono essere ricondotti al Cliente
\item Solo con il previo consenso scritto del Cliente
\item Non divulgare mai l'identità del Cliente in connessione con tali dati
\end{itemize}

\textbf{8.6 IP di Terze Parti.} Se i Servizi richiedono l'utilizzo di IP di terze parti (software, strumenti, ecc.):
\begin{itemize}
\item Il Fornitore di Servizi dovrà ottenere le licenze necessarie
\item Il Cliente è responsabile dei costi delle licenze necessarie per l'uso continuativo
\item Il Fornitore di Servizi garantisce di avere la corretta autorizzazione per utilizzare l'IP di terze parti nei Servizi
\end{itemize}

\section{DICHIARAZIONI E GARANZIE}

\textbf{9.1 Dichiarazioni Reciproche.} Ogni Parte dichiara e garantisce che:
\begin{itemize}
\item È debitamente organizzata e validamente esistente
\item Ha piena autorità per stipulare ed eseguire questo Accordo
\item L'esecuzione non viola alcun altro accordo o obbligo
\item Si conformerà a tutte le leggi e regolamenti applicabili
\end{itemize}

\textbf{9.2 Dichiarazioni e Garanzie del Fornitore di Servizi.} Il Fornitore di Servizi dichiara e garantisce che:

\begin{enumerate}[label=\alph*)]
\item Detiene certificazioni CPF attuali e valide in buona standing:
\begin{itemize}
\item Tipo/i di certificazione: \underline{\hspace{8cm}}
\item Numero/i di certificato: \underline{\hspace{8cm}}
\item Data/e di scadenza: \underline{\hspace{8cm}}
\end{itemize}
\item Manterrà la certificazione durante tutto l'impegno
\item I Servizi saranno eseguiti in modo professionale conforme agli standard del settore
\item I Servizi si conformeranno alla metodologia e al Codice Etico CPF
\item Ha le competenze, conoscenze ed esperienza necessarie per eseguire i Servizi
\item Utilizzerà personale qualificato con certificazioni appropriate
\item Mantiene un'assicurazione di responsabilità professionale con copertura minima di \$\underline{\hspace{4cm}}
\item Non ha conflitti di interesse che potrebbero compromettere l'oggettività
\item Implementerà metodologie che preservano la privacy secondo gli standard CPF
\item I Prodotti Forniti saranno lavoro originale o correttamente autorizzato
\item Non farà dichiarazioni false o fuorvianti
\end{enumerate}

\textbf{9.3 Dichiarazioni e Garanzie del Cliente.} Il Cliente dichiara e garantisce che:
\begin{itemize}
\item Fornirà informazioni accurate e complete
\item Ha l'autorità per divulgare informazioni al Fornitore di Servizi
\item Coopererà in buona fede per consentire i Servizi
\item Le informazioni fornite non violano diritti di terze parti
\item Pagherà i compensi come concordato
\end{itemize}

\textbf{9.4 Disclaimer.} ECCETTO QUANTO ESPRESSAMENTE STATO IN QUESTA SEZIONE 9, IL FORNITORE DI SERVIZI NON FORNISCE ALCUNA GARANZIA, ESPRESSA O IMPLICITA, INCLUSE GARANZIE DI COMMERCIABILITÀ O IDONEITÀ PER UNO SCOPO PARTICOLARE. IL FORNITORE DI SERVIZI NON GARANTISCE:
\begin{itemize}
\item Che i Servizi elimineranno tutte le vulnerabilità di sicurezza o preveniranno tutti gli incidenti
\item Risultati di sicurezza specifici o percentuali di riduzione degli incidenti
\item Conformità con specifici requisiti normativi (oltre la metodologia CPF)
\item Che le raccomandazioni saranno implementate con successo
\item Risultati di certificazione se il Cliente persegue la certificazione organizzativa CPF
\end{itemize}

\section{LIMITAZIONE DI RESPONSABILITÀ}

\textbf{10.1 ESCLUSIONE DEI DANNI.} IN NESSUN CASO UNA PARTE SARÀ RESPONSABILE VERSO L'ALTRA PER DANNI INDIRETTI, INCIDENTALI, CONSEGUENZIALI, SPECIALI, ESEMPLARI O PUNITIVI, INCLUSI:
\begin{itemize}
\item Profitti o ricavi persi
\item Opportunità commerciali perse
\item Perdita di dati
\item Costo di servizi sostitutivi
\item Danno alla reputazione
\item Interruzione dell'attività
\end{itemize}

ANCHE SE AVVISATA DELLA POSSIBILITÀ DI TALI DANNI E INDIPENDENTEMENTE DALLA TEORIA DI RESPONSABILITÀ (CONTRATTO, ILLECITO CIVILE, NEGLIGENZA, RESPONSABILITÀ OGGETTIVA O ALTRE).

\textbf{10.2 LIMITE SULLA RESPONSABILITÀ.} LA RESPONSABILITÀ TOTALE CUMULATIVA DI CIASCUNA PARTE AI SENSI DI QUESTO ACCORDO NON SUPERERÀ IL TOTALE DEI COMPENSI PAGATI O PAGABILI DAL CLIENTE AL FORNITORE DI SERVIZI AI SENSI DI QUESTO ACCORDO NEI DODICI (12) MESI PRECEDENTI LA RICHIESTA.

\textbf{10.3 Eccezioni.} Le limitazioni nelle Sezioni 10.1 e 10.2 non si applicano a:
\begin{itemize}
\item Violazioni degli obblighi di confidenzialità (Sezione 6)
\item Violazioni della protezione dei dati (Sezione 7)
\item Frode, condotta intenzionale o negligenza grave
\item Obblighi di indennizzo (Sezione 11)
\item Lesioni personali o danni alla proprietà causati da negligenza
\item Violazioni non consentite di essere limitate secondo la legge applicabile
\end{itemize}

\textbf{10.4 Base della Negoziazione.} Le Parti riconoscono che le limitazioni in questa Sezione 10 sono elementi fondamentali della negoziazione e che il Fornitore di Servizi non fornirebbe i Servizi senza tali limitazioni.

\section{INDENNIZZO}

\textbf{11.1 Indennizzo da parte del Fornitore di Servizi.} Il Fornitore di Servizi indennizzerà, difenderà e manterrà indenne il Cliente, i suoi funzionari, direttori, dipendenti e agenti da e contro qualsiasi rivendicazione di terze parti, danni, responsabilità, costi e spese (incluse ragionevoli spese legali) derivanti da:

\begin{enumerate}[label=\alph*)]
\item Negligenza o condotta intenzionale del Fornitore di Servizi
\item Violazione di questo Accordo da parte del Fornitore di Servizi
\item Violazione delle leggi applicabili da parte del Fornitore di Servizi
\item Violazione dei diritti di proprietà intellettuale di terze parti da parte dei Prodotti Forniti (eccetto nella misura basata su specifiche o materiali del Cliente)
\item Violazione del Codice Etico CPF da parte del Fornitore di Servizi
\item Violazioni della privacy o protezione dei dati da parte del Fornitore di Servizi
\item Utilizzo o divulgazione non autorizzata delle Informazioni Confidenziali del Cliente
\end{enumerate}

\textbf{11.2 Indennizzo da parte del Cliente.} Il Cliente indennizzerà, difenderà e manterrà indenne il Fornitore di Servizi, i suoi funzionari, direttori, dipendenti e agenti da e contro qualsiasi rivendicazione di terze parti, danni, responsabilità, costi e spese (incluse ragionevoli spese legali) derivanti da:

\begin{enumerate}[label=\alph*)]
\item Violazione di questo Accordo da parte del Cliente
\item Violazione delle leggi applicabili da parte del Cliente
\item Informazioni inaccurate o fuorvianti fornite dal Cliente
\item Utilizzo da parte del Cliente dei Prodotti Forniti oltre l'ambito autorizzato in questo Accordo
\item Implementazione da parte del Cliente delle raccomandazioni (il Fornitore di Servizi non è responsabile per le decisioni o risultati dell'implementazione)
\item Rivendicazioni di terze parti che il Cliente non aveva l'autorità per fornire informazioni o materiali al Fornitore di Servizi
\end{enumerate}

\textbf{11.3 Processo di Indennizzo.} L'Indennizzato dovrà:

\begin{enumerate}[label=\alph*)]
\item Notificare tempestivamente per iscritto l'Indennitore della rivendicazione
\item Fornire ragionevole cooperazione nella difesa
\item Consentire all'Indennitore di controllare la difesa e il regolamento (con il consenso dell'Indennizzato, non negato irragionevolmente)
\item Non ammettere responsabilità o regolare senza il previo consenso scritto dell'Indennitore
\end{enumerate}

Gli obblighi dell'Indennitore sono condizionati sulla conformità dell'Indennizzato con questo processo.

\textbf{11.4 Verifica Assicurativa.} Su richiesta, il Fornitore di Servizi fornirà al Cliente un certificato di assicurazione che attesti la copertura di responsabilità professionale.

\section{DISPOSIZIONI GENERALI}

\textbf{12.1 Legge Applicabile.} Questo Accordo sarà regolato e interpretato in conformità con le leggi di [Giurisdizione], senza riguardo ai suoi principi di conflitto di leggi.

\textbf{12.2 Risoluzione delle Controversie.}

\begin{enumerate}[label=\alph*)]
\item \textit{Negoziazione in Buona Fede:} Le controversie saranno prima affrontate attraverso negoziazioni in buona fede tra rappresentanti senior di entrambe le Parti per trenta (30) giorni.

\item \textit{Mediazione:} Se la negoziazione fallisce, le Parti tenteranno la mediazione amministrata da [Servizio di Mediazione] in conformità con le sue regole. Ogni Parte sostiene i propri costi più una quota pari delle commissioni del mediatore.

\item \textit{Arbitrato:} Se la mediazione fallisce, la controversia sarà risolta attraverso arbitrato vincolante:
\begin{itemize}
\item Amministrato da [Servizio di Arbitrato] secondo le sue regoles
\item Un arbitro concordato reciprocamente o nominato secondo le regole del servizio
\item Condotto in [Città, Giurisdizione]
\item Lingua inglese
\item La decisione dell'arbitro è finale e vincolante
\item La sentenza può essere iscritta in qualsiasi corte di giurisdizione competente
\item Ogni parte sostiene i propri costi a meno che l'arbitro non decida diversamente
\end{itemize}

\item \textit{Eccezioni:} Entrambe le Parti possono cercare un provvedimento ingiuntivo o un rimedio equitativo in tribunale senza prima perseguire mediazione o arbitrato per:
\begin{itemize}
\item Violazioni della confidenzialità
\item Violazione della proprietà intellettuale
\item Violazioni della protezione dei dati
\item Questioni urgenti che richiedono un rimedio immediato
\end{itemize}
\end{enumerate}

\textbf{12.3 Contrattista Indipendente.} Il Fornitore di Servizi è un contrattista indipendente, non dipendente, partner o agente del Cliente. Il Fornitore di Servizi:
\begin{itemize}
\item Controlla il modo e i mezzi di esecuzione dei Servizi
\item È responsabile delle proprie tasse, assicurazioni e benefit
\item Non ha diritto ai benefit dei dipendenti
\item Può servire altri clienti (soggetto a nessun conflitto di interessi)
\item Non è autorizzato a vincolare il Cliente o a fare impegni per conto del Cliente
\end{itemize}

\textbf{12.4 Cessione.}
\begin{itemize}
\item Nessuna Parte può cedere questo Accordo senza il previo consenso scritto dell'altra Parte
\item Il consenso non sarà negato irragionevolmente
\item Eccezione: Entrambe le Parti possono cedere in connessione con fusione, acquisizione o vendita di sostanzialmente tutti i beni
\item Il tentativo di cessione senza consenso è nullo
\end{itemize}

\textbf{12.5 Subappalto.}
\begin{itemize}
\item Il Fornitore di Servizi non può subappaltare i Servizi senza il previo consenso scritto del Cliente
\item Il Fornitore di Servizi rimane pienamente responsabile per la prestazione del subappaltatore
\item Tutti i subappaltatori devono essere vincolati dagli stessi obblighi di confidenzialità e protezione dei dati
\item Le valutazioni CPF devono essere condotte da professionisti certificati CPF
\end{itemize}

\textbf{12.6 Notifiche.} Tutte le notifiche ai sensi di questo Accordo dovranno essere per iscritto e consegnate tramite:
\begin{itemize}
\item Consegna personale
\item Corriere internazionale riconosciuto (DHL, FedEx, UPS)
\item Email con conferma di ricezione (accettabile per comunicazioni di routine)
\end{itemize}

Notifiche inviate agli indirizzi sopra indicati o come aggiornate per iscritto. Le notifiche sono efficaci alla ricezione.

\textbf{12.7 Forza Maggiore.} Nessuna Parte è responsabile per il fallimento o il ritardo nell'esecuzione a causa di cause al di là del suo ragionevole controllo, incluse:
\begin{itemize}
\item Atti di Dio (disastri naturali, maltempo grave)
\item Guerra, terrorismo, disordini civili
\item Epidemia, pandemia
\item Azioni o restrizioni governative
\item Scioperi, dispute di lavoro (non coinvolgenti i dipendenti della Parte)
\item Fallimento delle infrastrutture di telecomunicazioni o internet
\item Cyberattacchi che colpiscono i sistemi della Parte
\end{itemize}

La Parte interessata dovrà:
\begin{itemize}
\item Notificare tempestivamente l'altra Parte
\item Usare ragionevoli sforzi per mitigare l'impatto
\item Riprendere l'esecuzione quando le circostanze lo consentono
\end{itemize}

Se la forza maggiore continua per più di sessanta (60) giorni, entrambe le Parti possono risolvere con preavviso scritto.

\textbf{12.8 Accordo Integrale.} Questo Accordo, inclusi tutti gli Allegati, costituisce l'intero accordo tra le Parti riguardo all'oggetto e sostituisce tutti gli accordi precedenti, intese, negoziazioni e discussioni, siano esse orali o scritte.

\textbf{12.9 Modifica.} Questo Accordo può essere modificato solo tramite strumento scritto firmato da entrambe le Parti. Gli scambi di email possono costituire modifica scritta se documentano chiaramente l'accordo reciproco su cambiamenti specifici.

\textbf{12.10 Rinuncia.} La rinuncia a qualsiasi disposizione deve essere per iscritto e firmata dalla Parte che rinuncia. Il mancato rispetto di qualsiasi disposizione non costituisce rinuncia al diritto di farla rispettare in seguito o qualsiasi altra disposizione.

\textbf{12.11 Separabilità.} Se qualsiasi disposizione è trovata invalida o ineseguibile, le disposizioni rimanenti continuano in pieno vigore. La disposizione invalida sarà riformata nella massima misura possibile per raggiungere l'intento delle Parti.

\textbf{12.12 Controparti e Firme Elettroniche.} Questo Accordo può essere eseguito in controparti, ciascuna considerata originale e tutte insieme costituenti un unico strumento. Le firme elettroniche (incluse DocuSign, Adobe Sign, ecc.) hanno lo stesso effetto legale delle firme originali.

\textbf{12.13 Intestazioni.} Le intestazioni delle sezioni sono solo per convenienza e non influenzano l'interpretazione.

\textbf{12.14 Sopravvivenza.} Le seguenti sezioni sopravvivono alla risoluzione o scadenza: 4 (obblighi di pagamento), 5.3 (accettazione), 6 (Confidenzialità), 7 (Protezione Dati), 8 (Proprietà Intellettuale), 10 (Limitazione di Responsabilità), 11 (Indennizzo) e 12 (Disposizioni Generali).

\textbf{12.15 Conformità con i Requisiti CPF.} Il Fornitore di Servizi riconosce che:
\begin{itemize}
\item CPF3 è proprietario del framework CPF e della proprietà intellettuale
\item Il Fornitore di Servizi opera sotto certificazione CPF concessa dall'Organismo di Certificazione
\item Questo Accordo deve conformarsi al Codice Etico CPF
\item La violazione degli standard CPF può risultare nella sospensione o revoca della certificazione
\item Il Cliente ha il diritto di segnalare violazioni etiche all'Organismo di Certificazione del Fornitore di Servizi
\end{itemize}

\textbf{12.16 Pubblicità e Marketing.}

\textit{Utilizzo del Nome del Cliente:}
\begin{itemize}
\item Il Fornitore di Servizi non può utilizzare il nome, il logo del Cliente o identificare il Cliente come cliente senza previo consenso scritto
\item Eccezione: Il Fornitore di Servizi può elencare il Cliente (senza dettagli) nell'elenco clienti se ottenuto il consenso
\end{itemize}

\textit{Casi di Studio e Testimonianze:}
\begin{itemize}
\item Il Fornitore di Servizi può richiedere il permesso di sviluppare un caso di studio o una testimonianza
\item Il Cliente ha pieni diritti di approvazione sul contenuto prima della pubblicazione
\item Il Cliente può richiedere l'anonimizzazione
\item Dati di Valutazione e Informazioni Confidenziali esclusi a meno che non siano esplicitamente approvati
\end{itemize}

\textit{Riferimento Generale:}
\begin{itemize}
\item Il Fornitore di Servizi può fare un riferimento generale al tipo di lavoro eseguito (es. "valutazioni CPF per organizzazioni di servizi finanziari") senza identificare clienti specifici
\end{itemize}

\section*{ALLEGATI}

I seguenti Allegati sono allegati e formano parte di questo Accordo:

\begin{itemize}
\item \textbf{Allegato A:} Dichiarazione di Lavoro (Ambito, Prodotti Forniti, Tempistica, Criteri di Successo)
\item \textbf{Allegato B:} Calendario dei Compensi e Termini di Pagamento (se richiesto un dettaglio)
\item \textbf{Allegato C:} Personale Chiave e Qualifiche
\item \textbf{Allegato D:} Matrice delle Responsabilità del Cliente (se impegno complesso)
\end{itemize}

\vspace{2em}

\section*{FIRME}

IN TESTE DI QUANTO SOPRA, le Parti hanno eseguito questo Accordo alla Data di Efficacia.

\vspace{2em}

\textbf{FORNITORE DI SERVIZI: [NOME]}

\vspace{1.5em}

Firma: \underline{\hspace{6cm}} Data: \underline{\hspace{3cm}}

Nome Stampato: \underline{\hspace{6cm}}

Titolo (se società): \underline{\hspace{6cm}}

Certificazione/i CPF: \underline{\hspace{6cm}}

\vspace{2em}

\textbf{CLIENTE: [NOME ORGANIZZAZIONE]}

\vspace{1.5em}

Da: \underline{\hspace{6cm}} Data: \underline{\hspace{3cm}}

Nome: \underline{\hspace{6cm}}

Titolo: \underline{\hspace{6cm}}

\newpage

\section*{ALLEGATO A: DICHIARAZIONE DI LAVORO}

\subsection*{1. Panoramica del Progetto}

\textbf{Nome del Progetto:} \underline{\hspace{10cm}}

\textbf{Obiettivi del Progetto:}
\begin{itemize}
\item[] \underline{\hspace{12cm}}
\item[] \underline{\hspace{12cm}}
\item[] \underline{\hspace{12cm}}
\end{itemize}

\textbf{Riepilogo dell'Ambito:} \underline{\hspace{10cm}}

\underline{\hspace{12cm}}

\underline{\hspace{12cm}}

\subsection*{2. Ambito Dettagliato dei Servizi}

\textbf{Fase 1: [Nome Fase]}

\textit{Durata:} \underline{\hspace{4cm}}

\textit{Attività:}
\begin{itemize}
\item[] \underline{\hspace{12cm}}
\item[] \underline{\hspace{12cm}}
\item[] \underline{\hspace{12cm}}
\end{itemize}

\textit{Prodotti Forniti:}
\begin{itemize}
\item[] \underline{\hspace{12cm}}
\item[] \underline{\hspace{12cm}}
\end{itemize}

\textbf{Fase 2: [Nome Fase]}

\textit{Durata:} \underline{\hspace{4cm}}

\textit{Attività:}
\begin{itemize}
\item[] \underline{\hspace{12cm}}
\item[] \underline{\hspace{12cm}}
\item[] \underline{\hspace{12cm}}
\end{itemize}

\textit{Prodotti Forniti:}
\begin{itemize}
\item[] \underline{\hspace{12cm}}
\item[] \underline{\hspace{12cm}}
\end{itemize}

[Fasi aggiuntive come necessario]

\subsection*{3. Riepilogo dei Prodotti Forniti}

\begin{tabular}{|p{5cm}|p{3cm}|p{3cm}|}
\hline
\textbf{Prodotto Fornito} & \textbf{Data di Scadenza} & \textbf{Formato} \\
\hline
 &  &  \\
\hline
 &  &  \\
\hline
 &  &  \\
\hline
 &  &  \\
\hline
\end{tabular}

\subsection*{4. Tempistica del Progetto}

\textbf{Data di Inizio Progetto:} \underline{\hspace{4cm}}

\textbf{Data di Fine Progetto:} \underline{\hspace{4cm}}

\textbf{Tappe Chiave:}

\begin{tabular}{|p{6cm}|p{3cm}|}
\hline
\textbf{Tappa} & \textbf{Data Target} \\
\hline
 &  \\
\hline
 &  \\
\hline
 &  \\
\hline
\end{tabular}

\subsection*{5. Criteri di Successo}

\textit{Il progetto sarà considerato di successo quando:}
\begin{itemize}
\item[] \underline{\hspace{12cm}}
\item[] \underline{\hspace{12cm}}
\item[] \underline{\hspace{12cm}}
\end{itemize}

\subsection*{6. Presupposti e Dipendenze}

\textbf{Presupposti:}
\begin{itemize}
\item[] \underline{\hspace{12cm}}
\item[] \underline{\hspace{12cm}}
\end{itemize}

\textbf{Dipendenze:}
\begin{itemize}
\item[] \underline{\hspace{12cm}}
\item[] \underline{\hspace{12cm}}
\end{itemize}

\subsection*{7. Fuori Ambito}

Quanto segue è esplicitamente escluso da questo impegno:
\begin{itemize}
\item[] \underline{\hspace{12cm}}
\item[] \underline{\hspace{12cm}}
\item[] \underline{\hspace{12cm}}
\end{itemize}

\subsection*{8. Gestione dei Cambiamenti}

I cambiamenti a questa Dichiarazione di Lavoro richiedono:
\begin{itemize}
\item Richiesta di modifica scritta che descrive la modifica proposta
\item Analisi di impatto (ambito, tempistica, budget)
\item Approvazione scritta da entrambe le Parti
\item Ordine di modifica eseguito prima che il lavoro proceda
\end{itemize}

\vspace{2em}

\textbf{Approvato da:}

\vspace{1em}

Fornitore di Servizi: \underline{\hspace{5cm}} Data: \underline{\hspace{3cm}}

Cliente: \underline{\hspace{5cm}} Data: \underline{\hspace{3cm}}

\newpage

\section*{ALLEGATO C: PERSONALE CHIAVE E QUALIFICHE}

\textbf{Personale del Fornitore di Servizi:}

\subsection*{[Nome], [Titolo]}

\textbf{Ruolo nel Progetto:} \underline{\hspace{8cm}}

\textbf{Certificazione CPF:} \underline{\hspace{6cm}} (Certificato \#: \underline{\hspace{4cm}})

\textbf{Esperienza Rilevante:}
\begin{itemize}
\item[] \underline{\hspace{12cm}}
\item[] \underline{\hspace{12cm}}
\end{itemize}

\textbf{Formazione:} \underline{\hspace{10cm}}

\textbf{Anni di Esperienza:} \underline{\hspace{4cm}}

\vspace{1em}

[Personale aggiuntivo come necessario]

\vspace{2em}

\textbf{Cambiamenti del Personale:}

\begin{itemize}
\item Il Fornitore di Servizi non può sostituire il personale chiave senza l'approvazione del Cliente
\item Se il personale chiave non è disponibile a causa di circostanze impreviste, il Fornitore di Servizi proporrà un sostituto qualificato entro 5 giorni lavorativi
\item Il Cliente ha il diritto di approvare o rifiutare il sostituto proposto
\item Il sostituto deve avere qualifiche equivalenti o superiori
\end{itemize}

\vspace{2em}

\textbf{Contatti del Cliente:}

\textbf{Contatto Primario:} \underline{\hspace{8cm}}

Titolo: \underline{\hspace{6cm}} Email: \underline{\hspace{6cm}}

Telefono: \underline{\hspace{4cm}}

\vspace{1em}

\textbf{Coordinatore CPF:} \underline{\hspace{8cm}}

Titolo: \underline{\hspace{6cm}} Email: \underline{\hspace{6cm}}

Telefono: \underline{\hspace{4cm}}

\vspace{1em}

\textbf{Contatti Aggiuntivi:} [Come necessario]

\vspace{2em}

\begin{center}
\textit{Fine dell'Accordo di Servizi Professionali}
\end{center}

\end{document}