\documentclass[11pt,a4paper]{article}

% Packages
\usepackage[utf8]{inputenc}
\usepackage[italian]{babel}
\usepackage[margin=2.5cm]{geometry}
\usepackage{hyperref}
\usepackage{fancyhdr}
\usepackage{enumitem}
\usepackage{tabularx}
\usepackage{amssymb}

% Page style
\pagestyle{fancy}
\fancyhf{}
\renewcommand{\headrulewidth}{0.4pt}
\fancyhead[L]{Accordo di Certificazione Individuale CPF}
\fancyhead[R]{Candidato \#: \_\_\_\_\_\_\_\_}
\fancyfoot[C]{\thepage}

% Spacing
\setlength{\parindent}{0pt}
\setlength{\parskip}{0.8em}

% Title
\title{\textbf{ACCORDO DI CERTIFICAZIONE\\PROFESSIONALE INDIVIDUALE CPF}}
\author{}
\date{}

\begin{document}

\maketitle

\section*{PARTI}

Il presente Accordo di Certificazione Professionale Individuale ("Accordo") è stipulato alla data di sottoscrizione da parte del Candidato ("Data di Efficacia"), tra:

\textbf{[NOME ORGANISMO DI CERTIFICAZIONE]} ("Organismo di Certificazione" o "OdC")\\
Una [tipo di entità] di [giurisdizione]\\
Organismo di Certificazione CPF Autorizzato\\
Sede Principale: [Indirizzo]\\
Email: [Email]

E

\textbf{[NOME CANDIDATO]} ("Candidato" o "Professionista Certificato" al momento della certificazione)\\
Indirizzo: [Indirizzo]\\
Email: [Email]\\
Telefono: [Telefono]

Collettivamente denominati le "Parti" e individualmente una "Parte."

\section*{PREMESSE}

CONSIDERATO CHE, l'Organismo di Certificazione è autorizzato da CPF3 a gestire lo Schema di Certificazione CPF e certificare individui come CPF Assessor, CPF Practitioner o CPF Auditor;

CONSIDERATO CHE, il Candidato desidera ottenere la certificazione professionale secondo lo Schema di Certificazione CPF;

CONSIDERATO CHE, l'Organismo di Certificazione è disposto a valutare le qualifiche del Candidato e, se appropriato, concedere la certificazione soggetta ai termini e condizioni qui stabiliti;

PERTANTO, in considerazione dei reciproci impegni e accordi qui contenuti, le Parti convengono quanto segue:

\section{DEFINIZIONI}

\textbf{1.1 "Certificazione"} indica l'attestazione formale da parte dell'Organismo di Certificazione che il Candidato ha soddisfatto i requisiti per uno dei seguenti:
\begin{itemize}
\item CPF Certified Assessor
\item CPF Certified Practitioner
\item CPF Certified Auditor
\end{itemize}

\textbf{1.2 "Marchio di Certificazione"} indica il marchio, logo e designazione associati alla specifica certificazione concessa al Candidato.

\textbf{1.3 "Codice Etico CPF"} indica gli standard di condotta professionale stabiliti per i professionisti CPF certificati, come possono essere modificati di volta in volta.

\textbf{1.4 "CPE"} indica i crediti di Formazione Professionale Continua richiesti per la ricertificazione.

\textbf{1.5 "Periodo di Certificazione"} indica il periodo di tre (3) anni dalla data di certificazione iniziale o ricertificazione.

\textbf{1.6 "Organismo di Certificazione"} include il suo personale autorizzato, comitati e rappresentanti.

\section{TIPO DI CERTIFICAZIONE}

Il Candidato si candida per la seguente certificazione (selezionare una):

\begin{itemize}
\item[$\square$] \textbf{CPF Certified Assessor}\\
Requisiti: Laurea (Psicologia o Cybersecurity con formazione supplementare), 2 anni di esperienza, 80 ore di formazione (CPF-101 + CPF-201), esami scritti e pratici.

\item[$\square$] \textbf{CPF Certified Practitioner}\\
Requisiti: Laurea in campo pertinente, 1 anno di esperienza nell'implementazione CPF, 40 ore di formazione (CPF-101), esame scritto e revisione del portfolio.

\item[$\square$] \textbf{CPF Certified Auditor}\\
Requisiti: Certificazione CPF Assessor corrente, 1 anno come Assessor, 10 assessment completati, 64 ore di formazione aggiuntiva (CPF-401 + ISO 19011), esami scritti e pratici.
\end{itemize}

\section{PROCESSO DI DOMANDA E CERTIFICAZIONE}

\textbf{3.1 Presentazione della Domanda.} Il Candidato deve:

\begin{enumerate}[label=\alph*)]
\item Compilare il modulo di domanda in modo accurato e veritiero;
\item Presentare tutta la documentazione richiesta inclusi:
\begin{itemize}
\item Trascrizioni ufficiali o certificati di laurea;
\item Lettere di verifica dell'esperienza o portfolio professionale;
\item Certificati di completamento della formazione;
\item Referenze professionali (minimo 2);
\item Curriculum vitae aggiornato;
\item Documento d'identità con foto rilasciato dal governo;
\end{itemize}
\item Pagare la tariffa di revisione domanda non rimborsabile;
\item Fornire firma elettronica sul Codice Etico CPF.
\end{enumerate}

\textbf{3.2 Revisione della Domanda.} L'Organismo di Certificazione deve:

\begin{enumerate}[label=\alph*)]
\item Revisionare la domanda per completezza e idoneità;
\item Verificare le credenziali educative con le istituzioni che conferiscono il titolo;
\item Verificare l'impiego e l'esperienza con i datori di lavoro elencati o attraverso la revisione del portfolio;
\item Contattare le referenze professionali;
\item Verificare il completamento della formazione con i fornitori di formazione approvati;
\item Completare la revisione entro quindici (15) giorni lavorativi dalla ricezione della domanda completa;
\item Notificare al Candidato la determinazione dell'idoneità o richiedere informazioni aggiuntive.
\end{enumerate}

\textbf{3.3 Esame.} Dopo l'approvazione della domanda:

\begin{enumerate}[label=\alph*)]
\item Il Candidato deve programmare e completare gli esami richiesti entro dodici (12) mesi;
\item Esame scritto amministrato tramite piattaforma di test sicura;
\item Esame pratico o revisione del portfolio come applicabile;
\item Risultati forniti entro cinque (5) giorni lavorativi;
\item Punteggi di superamento richiesti: 70\% per Assessor/Practitioner, 75\% per Auditor.
\end{enumerate}

\textbf{3.4 Politica di Ripetizione.} Se il Candidato non supera l'esame:

\begin{enumerate}[label=\alph*)]
\item Può ripetere dopo un periodo di attesa di trenta (30) giorni;
\item Massimo tre (3) tentativi entro dodici (12) mesi;
\item Ogni ripetizione richiede il pagamento della tariffa di ripetizione (50\% della tariffa d'esame);
\item Dopo tre fallimenti, deve completare formazione aggiuntiva e attendere sei (6) mesi prima di riapplicare;
\item Esami scritti e pratici possono essere ripetuti indipendentemente.
\end{enumerate}

\textbf{3.5 Decisione di Certificazione.} L'Organismo di Certificazione deve:

\begin{enumerate}[label=\alph*)]
\item Prendere la decisione di certificazione entro dieci (10) giorni lavorativi dal completamento dell'esame;
\item Concedere la certificazione se tutti i requisiti sono soddisfatti;
\item Negare la certificazione con spiegazione scritta se i requisiti non sono soddisfatti;
\item Fornire diritti di appello se la certificazione è negata;
\item Emettere certificato e badge digitale all'approvazione;
\item Aggiungere il Professionista Certificato al registro pubblico di certificazione entro cinque (5) giorni lavorativi.
\end{enumerate}

\section{CONCESSIONE DELLA CERTIFICAZIONE E DIRITTI}

\textbf{4.1 Concessione della Certificazione.} Al completamento con successo di tutti i requisiti e pagamento della tariffa di certificazione, l'Organismo di Certificazione concede al Professionista Certificato:

\begin{enumerate}[label=\alph*)]
\item Certificazione professionale nella categoria richiesta;
\item Diritto di utilizzare il Marchio di Certificazione applicabile;
\item Inserimento nel registro pubblico di certificazione;
\item Accesso ai benefici dei titolari di certificazione;
\item Certificato valido per tre (3) anni dalla data di emissione.
\end{enumerate}

\textbf{4.2 Uso del Marchio di Certificazione.} Il Professionista Certificato può:

\begin{enumerate}[label=\alph*)]
\item Utilizzare il Marchio di Certificazione dopo il proprio nome (es. "CPF Certified Assessor");
\item Visualizzare il logo di certificazione su biglietti da visita, carta intestata, firme email e profili professionali;
\item Fare riferimento alla certificazione nei materiali marketing e proposte;
\item Utilizzare il badge digitale sui siti di networking professionale (LinkedIn, etc.);
\item Indicare lo status di certificazione nelle informazioni biografiche e presentazioni.
\end{enumerate}

\textbf{4.3 Restrizioni sull'Uso del Marchio di Certificazione.} Il Professionista Certificato NON deve:

\begin{enumerate}[label=\alph*)]
\item Modificare, alterare o creare versioni derivate del Marchio di Certificazione;
\item Utilizzare il Marchio di Certificazione in modo da suggerire la certificazione di prodotti, servizi o organizzazioni (a meno che non siano certificati separatamente);
\item Utilizzare il Marchio di Certificazione dopo che la certificazione scade, è sospesa o revocata;
\item Trasferire o sublicenziare il diritto di utilizzare il Marchio di Certificazione;
\item Utilizzare il Marchio di Certificazione in modo che porti discredito a CPF o all'Organismo di Certificazione;
\item Rappresentare che la certificazione si estende oltre la specifica categoria concessa.
\end{enumerate}

\textbf{4.4 Benefici della Certificazione.} Il Professionista Certificato riceve:

\begin{enumerate}[label=\alph*)]
\item Certificato elettronico e fisico;
\item Badge digitale per uso online;
\item Inserimento nel registro pubblico di certificazione con pagina profilo;
\item Accesso al portale online di tracciamento CPE;
\item Inviti a eventi e webinar della comunità CPF;
\item Accesso a risorse e strumenti esclusivi (come applicabile);
\item Newsletter e aggiornamenti sugli sviluppi CPF;
\item Opportunità di networking attraverso la comunità dei professionisti certificati.
\end{enumerate}

\section{OBBLIGHI DEL PROFESSIONISTA CERTIFICATO}

\textbf{5.1 Conformità al Codice Etico.} Il Professionista Certificato deve:

\begin{enumerate}[label=\alph*)]
\item Aderire al Codice Etico CPF in ogni momento;
\item Mantenere integrità, obiettività e condotta professionale;
\item Esercitare solo nelle aree di competenza dimostrata;
\item Proteggere la riservatezza dei dati di assessment e delle informazioni del cliente;
\item Mai utilizzare i dati di assessment per la profilazione individuale;
\item Implementare metodologie che preservano la privacy in tutto il lavoro CPF;
\item Segnalare sospette violazioni etiche da parte di altri professionisti certificati.
\end{enumerate}

\textbf{5.2 Formazione Professionale Continua (CPE).} Il Professionista Certificato deve:

\begin{enumerate}[label=\alph*)]
\item Completare i crediti CPE richiesti annualmente:
\begin{itemize}
\item CPF Assessor: 40 crediti all'anno (120 in 3 anni)
\item CPF Practitioner: 30 crediti all'anno (90 in 3 anni)
\item CPF Auditor: 50 crediti all'anno (150 in 3 anni)
\end{itemize}
\item Documentare tutte le attività CPE nel portale CPE online;
\item Conservare la documentazione di supporto per cinque (5) anni;
\item Sottoporsi ad audit CPE se selezionato (10\% casuale annualmente);
\item Assicurare il completamento annuale dei crediti CPE minimi in etica.
\end{enumerate}

\textbf{5.3 Requisiti di Pratica Professionale.}

\textit{Per CPF Assessor:}
\begin{itemize}
\item Condurre minimo cinque (5) assessment CPF durante il periodo di certificazione di 3 anni;
\item Partecipare alle attività di calibrazione degli assessor;
\item Presentare almeno un report di assessment per peer review;
\item Mantenere la conoscenza aggiornata degli aggiornamenti della metodologia CPF.
\end{itemize}

\textit{Per CPF Practitioner:}
\begin{itemize}
\item Mantenere un portfolio aggiornato che dimostri l'applicazione pratica continua;
\item Documentare minimo tre (3) progetti di implementazione durante il periodo di certificazione;
\item Partecipare alla comunità di pratica dei practitioner.
\end{itemize}

\textit{Per CPF Auditor:}
\begin{itemize}
\item Condurre minimo quindici (15) giorni di audit all'anno (45 in 3 anni);
\item Servire come lead auditor in minimo cinque (5) audit durante il periodo di certificazione;
\item Presentare i report di audit per la revisione qualità;
\item Partecipare alle attività di valutazione della competenza degli auditor;
\item Mantenere l'indipendenza dalle attività di consulenza secondo ISO 19011.
\end{itemize}

\textbf{5.4 Obblighi di Notifica.} Il Professionista Certificato deve notificare immediatamente l'Organismo di Certificazione di:

\begin{enumerate}[label=\alph*)]
\item Modifiche alle informazioni di contatto;
\item Condanne penali o azioni disciplinari professionali;
\item Perdita delle qualifiche sottostanti (revoca del titolo, sospensione della licenza);
\item Coinvolgimento in reclami o indagini etiche significative;
\item Fallimento o circostanze finanziarie che influenzano la reputazione professionale;
\item Qualsiasi circostanza che possa influenzare lo status o l'idoneità alla certificazione.
\end{enumerate}

\textbf{5.5 Cooperazione con le Indagini.} Il Professionista Certificato deve:

\begin{enumerate}[label=\alph*)]
\item Cooperare pienamente con le indagini sui reclami etici;
\item Rispondere alle richieste dell'Organismo di Certificazione entro i tempi specificati;
\item Fornire la documentazione e le informazioni richieste;
\item Partecipare ai colloqui se richiesto;
\item Non ritorcersi contro i denuncianti o i testimoni.
\end{enumerate}

\textbf{5.6 Rappresentazione Accurata.} Il Professionista Certificato deve:

\begin{enumerate}[label=\alph*)]
\item Rappresentare accuratamente lo status e l'ambito della certificazione;
\item Non travisare le qualifiche o l'esperienza;
\item Distinguere chiaramente i servizi CPF dagli altri servizi offerti;
\item Fornire informazioni veritiere nel marketing e nelle proposte;
\item Correggere prontamente qualsiasi dichiarazione errata quando scoperta.
\end{enumerate}

\section{RICERTIFICAZIONE}

\textbf{6.1 Requisito di Ricertificazione.} La certificazione scade tre (3) anni dalla data di emissione. Per mantenere la certificazione, il Professionista Certificato deve richiedere la ricertificazione.

\textbf{6.2 Processo di Ricertificazione.}

\begin{enumerate}[label=\alph*)]
\item L'Organismo di Certificazione invia l'avviso di ricertificazione 180 giorni prima della scadenza;
\item Il Professionista Certificato presenta la domanda di ricertificazione 90 giorni prima della scadenza;
\item La domanda di ricertificazione include:
\begin{itemize}
\item Registri CPE completi per il periodo di 3 anni;
\item Documentazione dei requisiti di pratica professionale;
\item Referenze professionali aggiornate (se richiesto);
\item Attestazione etica;
\item Pagamento della tariffa di ricertificazione;
\end{itemize}
\item L'Organismo di Certificazione revisiona la presentazione entro 60 giorni;
\item Se approvato, nuovo certificato emesso con data di scadenza aggiornata;
\item Se negato, il Professionista Certificato riceve spiegazione scritta e diritti di appello.
\end{enumerate}

\textbf{6.3 Periodo di Grazia.} Se la ricertificazione non è completata entro la scadenza:

\begin{enumerate}[label=\alph*)]
\item Si applica un periodo di grazia di novanta (90) giorni;
\item Lo status di certificazione cambia in "In Attesa di Ricertificazione";
\item L'uso del Marchio di Certificazione è limitato durante il periodo di grazia;
\item Si applica la tariffa di ricertificazione tardiva (aggiuntivi \$100);
\item Dopo il periodo di grazia, è richiesto il processo completo di ricertificazione inclusi gli esami.
\end{enumerate}

\textbf{6.4 Rimedio al Deficit CPE.} Se i requisiti CPE non sono soddisfatti:

\begin{enumerate}[label=\alph*)]
\item Il Professionista Certificato può richiedere fino a 90 giorni di proroga per completare i CPE rimanenti;
\item La proroga è concessa a discrezione dell'Organismo di Certificazione;
\item Lo status di certificazione cambia in "Condizionato" durante la proroga;
\item Se i CPE non sono completati entro la proroga, la certificazione decade;
\item Può essere applicata una tariffa di proroga.
\end{enumerate}

\section{TARIFFE}

\textbf{7.1 Tariffa di Domanda.}
\begin{itemize}
\item CPF Assessor: \$300 (non rimborsabile)
\item CPF Practitioner: \$200 (non rimborsabile)
\item CPF Auditor: \$400 (non rimborsabile)
\end{itemize}

\textbf{7.2 Tariffe d'Esame.}
\begin{itemize}
\item CPF Assessor Scritto: \$400
\item CPF Assessor Pratico: \$600
\item CPF Practitioner Scritto: \$300
\item CPF Practitioner Revisione Portfolio: \$400
\item CPF Auditor Scritto: \$450
\item CPF Auditor Pratico: \$800
\item Tariffa di Ripetizione: 50\% della tariffa d'esame originale
\end{itemize}

\textbf{7.3 Tariffa di Certificazione.} Al completamento con successo di tutti i requisiti:
\begin{itemize}
\item CPF Assessor: \$200
\item CPF Practitioner: \$150
\item CPF Auditor: \$250
\end{itemize}

\textbf{7.4 Tariffe di Ricertificazione.}
\begin{itemize}
\item CPF Assessor: \$400
\item CPF Practitioner: \$300
\item CPF Auditor: \$500
\item Ricertificazione Tardiva (entro il periodo di grazia di 90 giorni): Aggiungere \$100
\end{itemize}

\textbf{7.5 Altre Tariffe.}
\begin{itemize}
\item Richiesta Proroga CPE: \$50
\item Certificato Duplicato: \$25
\item Lettera di Verifica Certificazione: \$15
\item Tariffa di Appello: \$200 (rimborsata se l'appello ha successo)
\end{itemize}

\textbf{7.6 Termini di Pagamento.}
\begin{enumerate}[label=\alph*)]
\item Tutte le tariffe pagabili in USD;
\item Pagamento tramite carta di credito, bonifico bancario o assegno;
\item Tariffe non rimborsabili eccetto dove specificamente indicato;
\item Servizi non forniti fino alla ricezione del pagamento;
\item Tariffe in ritardo possono risultare nella sospensione della certificazione.
\end{enumerate}

\section{SOSPENSIONE E REVOCA}

\textbf{8.1 Motivi di Sospensione.} L'Organismo di Certificazione può sospendere la certificazione per:

\begin{enumerate}[label=\alph*)]
\item Mancato completamento dei CPE richiesti entro la scadenza;
\item Mancato pagamento delle tariffe richieste;
\item Reclamo etico sotto indagine;
\item Mancato soddisfacimento dei requisiti di pratica professionale;
\item Mancata risposta alle richieste dell'Organismo di Certificazione;
\item Perdita delle qualifiche sottostanti in attesa di indagine.
\end{enumerate}

\textbf{8.2 Processo di Sospensione.}

\begin{enumerate}[label=\alph*)]
\item Notifica scritta di sospensione con motivazioni specifiche;
\item Cessazione immediata dell'uso del Marchio di Certificazione;
\item Status del registro cambiato in "Sospeso";
\item Periodo di sospensione: Massimo 90 giorni;
\item Piano di risoluzione richiesto entro 30 giorni;
\item Reintegro alla risoluzione con successo;
\item Se non risolto entro 90 giorni: Procedimenti di revoca avviati.
\end{enumerate}

\textbf{8.3 Motivi di Revoca.} L'Organismo di Certificazione può revocare la certificazione per:

\begin{enumerate}[label=\alph*)]
\item Gravi violazioni etiche incluse:
\begin{itemize}
\item Frode, dichiarazioni false o disonestà;
\item Violazione della riservatezza o uso improprio dei dati;
\item Profilazione individuale utilizzando i dati di assessment;
\item Condanna penale relativa alla condotta professionale;
\end{itemize}
\item Mancata risoluzione della sospensione entro 90 giorni;
\item Violazioni ripetute o sistematiche del Codice Etico CPF;
\item Perdita delle qualifiche sottostanti (revoca del titolo);
\item Fornitura di informazioni false nella domanda o ricertificazione;
\item Sublicenza o trasferimento non autorizzato della certificazione;
\item Violazione sostanziale di questo Accordo.
\end{enumerate}

\textbf{8.4 Processo di Revoca.}

\begin{enumerate}[label=\alph*)]
\item Notifica scritta dell'intenzione di revocare con motivazioni specifiche;
\item Opportunità di rispondere entro 30 giorni;
\item Revisione indipendente da parte del comitato etico dell'Organismo di Certificazione;
\item Decisione finale comunicata entro 45 giorni;
\item Se revocato:
\begin{itemize}
\item Cessazione immediata di tutto l'uso del Marchio di Certificazione;
\item Rimozione dal registro di certificazione;
\item Avviso pubblico di revoca;
\item Restituzione del certificato all'Organismo di Certificazione;
\item Divieto di riapplicazione per minimo 2 anni (o permanente);
\end{itemize}
\item Diritto di appellarsi alla decisione di revoca.
\end{enumerate}

\textbf{8.5 Rinuncia Volontaria.} Il Professionista Certificato può rinunciare volontariamente alla certificazione tramite:

\begin{enumerate}[label=\alph*)]
\item Notifica scritta all'Organismo di Certificazione;
\item Cessazione immediata dell'uso del Marchio di Certificazione;
\item Restituzione del certificato;
\item Nessun rimborso delle tariffe;
\item Può riapplicare per la certificazione in qualsiasi momento completando il processo completo di certificazione.
\end{enumerate}

\section{APPELLI}

\textbf{9.1 Diritto di Appello.} Il Professionista Certificato può appellarsi per:

\begin{enumerate}[label=\alph*)]
\item Diniego di certificazione;
\item Fallimento dell'esame dovuto a irregolarità procedurali (non il punteggio);
\item Diniego di ricertificazione;
\item Decisione di sospensione;
\item Decisione di revoca;
\item Azioni disciplinari.
\end{enumerate}

\textbf{9.2 Processo di Appello.}

\begin{enumerate}[label=\alph*)]
\item Appello presentato per iscritto entro 30 giorni dalla decisione;
\item Pagamento della tariffa di appello (\$200);
\item Specificazione delle motivazioni dell'appello e documentazione di supporto;
\item Pannello di appello indipendente assegnato (nessun coinvolgimento nella decisione originale);
\item Il pannello revisiona tutte le prove e la motivazione della decisione;
\item L'appellante può fornire informazioni scritte aggiuntive;
\item Il pannello rende la decisione entro 30 giorni;
\item Opzioni di decisione: Conferma, Modifica, Ribalta o Rimanda per riconsiderazione;
\item Tariffa rimborsata se l'appello ha successo;
\item La decisione del pannello di appello è finale e vincolante.
\end{enumerate}

\textbf{9.3 Composizione del Pannello di Appello.}

\begin{itemize}
\item Tre membri: un professionista CPF certificato, un esperto della materia, un rappresentante dell'Organismo di Certificazione non coinvolto nella decisione originale;
\item I membri del pannello non hanno conflitti di interesse;
\item Le decisioni sono prese a maggioranza;
\item Le deliberazioni del pannello sono riservate.
\end{itemize}

\section{RISERVATEZZA E PROTEZIONE DEI DATI}

\textbf{10.1 Riservatezza.} L'Organismo di Certificazione deve:

\begin{enumerate}[label=\alph*)]
\item Mantenere la riservatezza delle informazioni del Candidato/Professionista Certificato;
\item Limitare l'accesso alle informazioni al personale con necessità di conoscere;
\item Proteggere le risposte agli esami e i dati di assessment;
\item Non divulgare informazioni riservate senza consenso, eccetto:
\begin{itemize}
\item Informazioni del registro pubblico (nome, tipo di certificazione, status, data di scadenza);
\item Come richiesto dalla legge o ordine del tribunale;
\item A CPF3 per scopi di supervisione della qualità;
\item A organismi di accreditamento durante gli audit;
\item Indagine sui reclami etici come necessario;
\end{itemize}
\item Implementare misure di sicurezza tecniche e organizzative appropriate;
\item Conformarsi alle leggi applicabili sulla protezione dei dati (GDPR, CCPA, etc.).
\end{enumerate}

\textbf{10.2 Diritti di Protezione dei Dati.} Il Professionista Certificato ha diritto a:

\begin{enumerate}[label=\alph*)]
\item Accedere ai dati personali detenuti dall'Organismo di Certificazione;
\item Richiedere la correzione di informazioni inesatte;
\item Richiedere la cancellazione dei dati (soggetto ai requisiti di conservazione dei registri);
\item Opporsi al trattamento per determinati scopi;
\item Ricevere i dati in formato portabile;
\item Presentare reclamo all'autorità di protezione dei dati.
\end{enumerate}

\textbf{10.3 Conservazione dei Dati.} L'Organismo di Certificazione deve:

\begin{enumerate}[label=\alph*)]
\item Conservare i registri di certificazione per sette (7) anni dopo la scadenza o revoca della certificazione;
\item Conservare i registri delle indagini etiche per dieci (10) anni;
\item Distruggere in modo sicuro i dati dopo il periodo di conservazione a meno che non si applichi un blocco legale;
\item Mantenere audit trail per l'accesso ai dati e le modifiche.
\end{enumerate}

\textbf{10.4 Notifica di Violazione dei Dati.} In caso di violazione dei dati:

\begin{enumerate}[label=\alph*)]
\item L'Organismo di Certificazione deve notificare i Professionisti Certificati interessati entro 72 ore;
\item La notifica include la natura della violazione, i dati interessati e le misure di mitigazione;
\item L'Organismo di Certificazione deve notificare le autorità di protezione dei dati applicabili come richiesto dalla legge;
\item L'Organismo di Certificazione deve adottare misure per prevenire ulteriori violazioni.
\end{enumerate}

\section{LIMITAZIONE DI RESPONSABILITÀ}

\textbf{11.1 Esclusione di Garanzie.} L'ORGANISMO DI CERTIFICAZIONE NON FORNISCE GARANZIE, ESPLICITE O IMPLICITE, RIGUARDO AI RISULTATI DELLA CERTIFICAZIONE, AI BENEFICI DI CARRIERA O AL POTENZIALE DI REDDITO. LA CERTIFICAZIONE È FORNITA "COSÌ COM'È" SENZA GARANZIA DI COMMERCIABILITÀ O IDONEITÀ PER UNO SCOPO PARTICOLARE.

\textbf{11.2 Limitazione dei Danni.} IN NESSUN CASO L'ORGANISMO DI CERTIFICAZIONE SARÀ RESPONSABILE PER DANNI INDIRETTI, INCIDENTALI, CONSEQUENZIALI, SPECIALI, ESEMPLARI O PUNITIVI, INCLUSI REDDITO PERSO, OPPORTUNITÀ COMMERCIALI PERSE O DANNI REPUTAZIONALI, DERIVANTI DALLA CERTIFICAZIONE O DAL SUO DINIEGO.

\textbf{11.3 Tetto di Responsabilità.} LA RESPONSABILITÀ TOTALE DELL'ORGANISMO DI CERTIFICAZIONE AI SENSI DI QUESTO ACCORDO NON ECCEDERÀ LE TARIFFE TOTALI PAGATE DAL PROFESSIONISTA CERTIFICATO NEI DODICI (12) MESI PRECEDENTI IL RECLAMO.

\textbf{11.4 Eccezioni.} Le limitazioni non si applicano a:

\begin{enumerate}[label=\alph*)]
\item Negligenza grave o condotta dolosa dell'Organismo di Certificazione;
\item Violazioni degli obblighi di riservatezza;
\item Violazioni della protezione dei dati;
\item Reclami non limitabili secondo la legge applicabile.
\end{enumerate}

\section{INDENNIZZO}

\textbf{12.1 Indennizzo da parte del Professionista Certificato.} Il Professionista Certificato deve indennizzare, difendere e tenere indenne l'Organismo di Certificazione da reclami derivanti da:

\begin{enumerate}[label=\alph*)]
\item Servizi professionali del Professionista Certificato a terzi;
\item Negligenza o cattiva condotta del Professionista Certificato;
\item Violazione di questo Accordo o del Codice Etico CPF da parte del Professionista Certificato;
\item Uso non autorizzato dei Marchi di Certificazione da parte del Professionista Certificato;
\item Informazioni false o fuorvianti fornite nella domanda o ricertificazione.
\end{enumerate}

\textbf{12.2 Assicurazione di Responsabilità Professionale.} Il Professionista Certificato che fornisce servizi CPF professionalmente deve mantenere:

\begin{itemize}
\item Assicurazione di responsabilità professionale (errori e omissioni) con copertura minima appropriata all'ambito della pratica;
\item Assicurazione di responsabilità generale come applicabile;
\item Prova dell'assicurazione fornita ai clienti su richiesta;
\item L'Organismo di Certificazione non è responsabile della verifica della copertura assicurativa.
\end{itemize}

\section{DISPOSIZIONI GENERALI}

\textbf{13.1 Legge Applicabile.} Questo Accordo sarà regolato dalle leggi di [Giurisdizione], senza riguardo ai principi di conflitto di leggi.

\textbf{13.2 Risoluzione delle Controversie.}

\begin{enumerate}[label=\alph*)]
\item Negoziazione in buona fede richiesta prima della risoluzione formale delle controversie;
\item Le controversie non risolte attraverso la negoziazione o il processo di appello saranno risolte attraverso arbitrato vincolante;
\item Arbitrato condotto secondo le regole di [Servizio di Arbitrato];
\item Arbitrato in [Città, Giurisdizione], lingua inglese;
\item La decisione dell'arbitro è finale e vincolante;
\item Ciascuna parte sostiene i propri costi a meno che l'arbitro non determini diversamente.
\end{enumerate}

\textbf{13.3 Intero Accordo.} Questo Accordo, incluso il Codice Etico CPF incorporato e i requisiti dello Schema di Certificazione, costituisce l'intero accordo e sostituisce tutte le intese precedenti.

\textbf{13.4 Modifica.} L'Organismo di Certificazione può modificare questo Accordo o il Codice Etico CPF fornendo 90 giorni di preavviso scritto. La certificazione continua dopo la data di efficacia costituisce accettazione. Se il Professionista Certificato non accetta le modifiche, può rinunciare volontariamente alla certificazione.

\textbf{13.5 Cessione.} Il Professionista Certificato non può cedere o trasferire la certificazione. L'Organismo di Certificazione può cedere questo Accordo in connessione con trasferimento aziendale o fusione.

\textbf{13.6 Notifiche.} Tutte le notifiche devono essere inviate agli indirizzi sopra indicati o come aggiornati per iscritto. L'email con conferma di ricezione è accettabile per le comunicazioni di routine.

\textbf{13.7 Separabilità.} Se qualsiasi disposizione è ritenuta non valida, le disposizioni rimanenti continuano in pieno effetto.

\textbf{13.8 Rinuncia.} Il mancato esercizio di qualsiasi disposizione non costituisce rinuncia al diritto di esercitarla in seguito.

\textbf{13.9 Contraente Indipendente.} Il Professionista Certificato è contraente indipendente, non dipendente o agente dell'Organismo di Certificazione.

\textbf{13.10 Sopravvivenza.} Le Sezioni 5.1 (Etica), 8 (effetti di Sospensione/Revoca), 10 (Riservatezza), 11 (Limitazione di Responsabilità), 12 (Indennizzo) e 13 (Disposizioni Generali) sopravvivono alla cessazione della certificazione.

\section{RICONOSCIMENTI}

Firmando di seguito, il Candidato riconosce e accetta che:

\begin{enumerate}[label=\alph*)]
\item Ha letto e compreso questo Accordo nella sua interezza;
\item Ha letto e accetta di conformarsi al Codice Etico CPF;
\item Ha fornito informazioni accurate e veritiere nella domanda;
\item Comprende i requisiti di certificazione e gli obblighi continuativi;
\item Comprende che le tariffe non sono rimborsabili;
\item Comprende che la certificazione può essere sospesa o revocata per violazioni;
\item Comprende che deve mantenere i requisiti CPE e di pratica;
\item Autorizza l'Organismo di Certificazione a verificare le informazioni fornite;
\item Autorizza la pubblicazione del nome e dello status di certificazione nel registro pubblico;
\item Acconsente al trattamento dei dati personali come descritto;
\item Comprende che la certificazione non garantisce impiego o reddito;
\item Accetta di risolvere le controversie attraverso l'arbitrato;
\item Cesserà immediatamente l'uso del Marchio di Certificazione se la certificazione termina.
\end{enumerate}

\vspace{2em}

\section*{FIRME}

\textbf{ORGANISMO DI CERTIFICAZIONE: [NOME]}

\vspace{1.5em}

Per: \underline{\hspace{6cm}} Data: \underline{\hspace{3cm}}

Nome: \underline{\hspace{6cm}}

Titolo: \underline{\hspace{6cm}}

\vspace{2em}

\textbf{CANDIDATO/PROFESSIONISTA CERTIFICATO}

\vspace{1.5em}

Firma: \underline{\hspace{6cm}} Data: \underline{\hspace{3cm}}

Nome Stampato: \underline{\hspace{6cm}}

\vspace{2em}

\section*{REGISTRO DI CERTIFICAZIONE (Solo per Uso OdC)}

\begin{tabular}{|l|p{10cm}|}
\hline
\textbf{Tipo di Certificazione} & \hspace{8cm} \\
\hline
\textbf{Numero Certificato} & \\
\hline
\textbf{Data di Emissione} & \\
\hline
\textbf{Data di Scadenza} & \\
\hline
\textbf{Emesso Da} & \\
\hline
\end{tabular}

\vspace{2em}

\begin{center}
\textit{Fine dell'Accordo di Certificazione Professionale Individuale}
\end{center}

\end{document}
