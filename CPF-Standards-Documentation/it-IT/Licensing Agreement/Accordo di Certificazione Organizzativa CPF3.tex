\documentclass[11pt,a4paper]{article}

% Packages
\usepackage[utf8]{inputenc}
\usepackage[italian]{babel}
\usepackage[margin=2.5cm]{geometry}
\usepackage{hyperref}
\usepackage{fancyhdr}
\usepackage{enumitem}
\usepackage{tabularx}
\usepackage{amssymb}

% Page style
\pagestyle{fancy}
\fancyhf{}
\renewcommand{\headrulewidth}{0.4pt}
\fancyhead[L]{Accordo di Certificazione Organizzativa CPF}
\fancyhead[R]{Cert \#: \_\_\_\_\_\_\_\_}
\fancyfoot[C]{\thepage}

% Spacing
\setlength{\parindent}{0pt}
\setlength{\parskip}{0.8em}

% Title
\title{\textbf{ACCORDO DI CERTIFICAZIONE\\ORGANIZZATIVA CPF}}
\author{}
\date{}

\begin{document}

\maketitle

\section*{PARTI}

Il presente Accordo di Certificazione Organizzativa ("Accordo") è stipulato in data \_\_\_ del mese di \_\_\_\_\_\_\_\_\_\_\_, 20\_\_\_ ("Data di Efficacia"), tra:

\textbf{[NOME ORGANISMO DI CERTIFICAZIONE]} ("Organismo di Certificazione" o "OdC")\\
Una [tipo di entità] di [giurisdizione]\\
Organismo di Certificazione CPF Autorizzato\\
Sede Principale: [Indirizzo]\\
Email: [Email]

E

\textbf{[NOME ORGANIZZAZIONE]} ("Organizzazione" o "Organizzazione Certificata" al momento della certificazione)\\
Una [tipo di entità] di [giurisdizione]\\
Numero di Registrazione: [Numero]\\
Sede Principale: [Indirizzo]\\
Email: [Email]

Collettivamente denominati le "Parti" e individualmente una "Parte."

\section*{PREMESSE}

CONSIDERATO CHE, l'Organismo di Certificazione è autorizzato da CPF3 a gestire lo Schema di Certificazione CPF e certificare organizzazioni per la maturità nella gestione delle vulnerabilità psicologiche;

CONSIDERATO CHE, l'Organizzazione desidera ottenere la certificazione organizzativa secondo lo Schema di Certificazione CPF a uno dei quattro livelli di conformità (Livello 1-4);

CONSIDERATO CHE, l'Organizzazione ha implementato o sta implementando la metodologia CPF e i requisiti CPF-27001:2025;

CONSIDERATO CHE, l'Organismo di Certificazione è disposto a valutare l'implementazione dell'Organizzazione e concedere la certificazione se i requisiti sono soddisfatti;

PERTANTO, in considerazione dei reciproci impegni e accordi qui contenuti, le Parti convengono quanto segue:

\section{DEFINIZIONI}

\textbf{1.1 "Certificazione"} indica l'attestazione formale da parte dell'Organismo di Certificazione che l'Organizzazione ha raggiunto uno dei seguenti Livelli di Conformità CPF:
\begin{itemize}
\item Livello 1: Foundation (Punteggio CPF 100-149)
\item Livello 2: Intermediate (Punteggio CPF 70-99)
\item Livello 3: Advanced (Punteggio CPF 40-69)
\item Livello 4: Exemplary (Punteggio CPF 0-39)
\end{itemize}

\textbf{1.2 "Punteggio CPF"} indica il punteggio aggregato di vulnerabilità (range 0-200) dove punteggi inferiori indicano una migliore postura di sicurezza.

\textbf{1.3 "Ambito di Certificazione"} indica le unità organizzative, le sedi e il personale coperti dalla certificazione, come dettagliato nell'Allegato A.

\textbf{1.4 "CPF-27001:2025"} indica lo standard dei requisiti del sistema di gestione CPF.

\textbf{1.5 "Audit di Sorveglianza"} indica l'audit periodico per verificare la conformità continuativa.

\textbf{1.6 "Non Conformità"} indica il mancato soddisfacimento di un requisito.

\section{AMBITO E LIVELLO DI CERTIFICAZIONE}

\textbf{2.1 Ambito di Certificazione.} La certificazione copre:

\begin{itemize}
\item[] Entità Legale: \underline{\hspace{10cm}}
\item[] Unità Operative: \underline{\hspace{10cm}}
\item[] Sedi: \underline{\hspace{10cm}}
\item[] Personale Totale nell'Ambito: \underline{\hspace{4cm}}
\item[] Esclusioni: \underline{\hspace{10cm}}
\end{itemize}

Ambito dettagliato nell'Allegato A.

\textbf{2.2 Livello di Certificazione Target:}

\begin{itemize}
\item[$\square$] \textbf{Livello 1: Foundation} (Punteggio CPF 100-149)
\item[$\square$] \textbf{Livello 2: Intermediate} (Punteggio CPF 70-99)
\item[$\square$] \textbf{Livello 3: Advanced} (Punteggio CPF 40-69)
\item[$\square$] \textbf{Livello 4: Exemplary} (Punteggio CPF 0-39)
\end{itemize}

\section{PROCESSO DI CERTIFICAZIONE}

\textbf{3.1 Fase di Domanda.} L'Organizzazione deve presentare:
\begin{itemize}
\item Modulo di domanda compilato
\item Report di assessment CPF valido da Assessor/Auditor certificato
\item Policy CPF approvata dall'alta direzione
\item Organigramma che mostra i ruoli CPF
\item Procedure di protezione della privacy
\item Piani di trattamento dei rischi per indicatori Red
\item Evidenza dell'integrazione ISMS
\item Lettera di impegno della direzione
\item Pagamento della tariffa di domanda
\end{itemize}

\textbf{3.2 Revisione della Domanda.} Entro 15 giorni lavorativi, l'Organismo di Certificazione deve:
\begin{itemize}
\item Verificare la completezza
\item Verificare il Punteggio CPF e la validità dell'assessment
\item Rivedere la documentazione per la conformità al livello target
\item Approvare per l'audit o richiedere informazioni aggiuntive
\item Assegnare un Auditor CPF qualificato
\end{itemize}

\textbf{3.3 Audit di Certificazione.}

\textit{Fase 1 (Revisione Documentale, 1-3 giorni):}
\begin{itemize}
\item Revisione delle policy e procedure CPF
\item Valutazione della prontezza per la Fase 2
\item Identificazione delle lacune che richiedono correzione
\end{itemize}

\textit{Fase 2 (Revisione dell'Implementazione, 3-10 giorni):}
\begin{itemize}
\item Verifica del Punteggio CPF attraverso campionamento
\item Revisione della metodologia e protezioni della privacy
\item Verifica del trattamento dei rischi
\item Interviste con direzione e personale
\item Revisione delle evidenze per i requisiti del livello target
\item Valutazione dell'integrazione ISMS
\item Valutazione dell'efficacia
\end{itemize}

\textit{Reportistica dell'Audit (15 giorni lavorativi):}
\begin{itemize}
\item Riunioni di apertura e chiusura
\item Report di audit scritto
\item Risultati: NC Maggiore, NC Minore, Osservazione, Opportunità
\end{itemize}

\textbf{3.4 Azioni Correttive.} Se non conformità:
\begin{itemize}
\item L'Organizzazione presenta il piano entro 30 giorni
\item NC maggiori corrette prima della certificazione
\item NC minori correggibili entro 90 giorni dopo
\item Verifica dell'efficacia
\end{itemize}

\textbf{3.5 Decisione di Certificazione.} Entro 15 giorni lavorativi:
\begin{itemize}
\item Concedere al livello appropriato
\item Emettere il certificato e autorizzare l'uso del Marchio
\item Aggiungere al registro pubblico
\item Stabilire il programma di sorveglianza
\item Oppure negare con spiegazione e diritti di appello
\end{itemize}

\section{CONCESSIONE DELLA CERTIFICAZIONE E DIRITTI}

\textbf{4.1 Concessione della Certificazione:}
\begin{itemize}
\item Certificazione del Livello di Conformità CPF
\item Diritto di utilizzare il Marchio di Certificazione
\item Inserimento nel registro pubblico
\item Certificato valido 3 anni
\item Accesso alle risorse
\end{itemize}

\textbf{4.2 Uso del Marchio di Certificazione:}
\begin{itemize}
\item Sito web e materiali marketing
\item Proposte e presentazioni
\item Sedi degli uffici
\item Firme email
\item Social media
\item Indicare: "Organizzazione Certificata CPF - Livello [X]"
\end{itemize}

\textbf{4.3 Restrizioni:}
\begin{itemize}
\item Nessuna modifica al Marchio
\item Non su prodotti/servizi (si applica all'organizzazione)
\item Non per livello superiore a quello certificato
\item Non al di fuori dell'ambito di certificazione
\item Non dopo scadenza/sospensione/revoca
\item Nessun trasferimento o sublicenza
\item Nessuna dichiarazione fuorviante
\end{itemize}

\section{OBBLIGHI}

\textbf{5.1 Mantenimento:}
\begin{itemize}
\item Mantenere la gestione sistematica delle vulnerabilità
\item Continuare l'implementazione CPF-27001:2025
\item Mantenere/migliorare il Punteggio CPF entro il livello
\item Aggiornare i trattamenti dei rischi
\item Mantenere le pratiche di preservazione della privacy
\item Fornire risorse adeguate
\end{itemize}

\textbf{5.2 Personale:}
\begin{itemize}
\item Mantenere il Coordinatore CPF
\item Livello 2+: Minimo 1 Assessor certificato
\item Livello 3+: Minimo 2 Assessor certificati
\item Livello 4: Team dedicato con Auditor
\item Assicurare il mantenimento CPE
\item Fornire formazione di consapevolezza
\end{itemize}

\textbf{5.3 Assessment e Monitoraggio:}
\begin{itemize}
\item Livello 1: Assessment annuale
\item Livello 2: Cicli trimestrali
\item Livello 3+: Monitoraggio continuo
\item Utilizzare professionisti certificati
\item Mantenere la documentazione
\item Tracciare le tendenze
\item Riportare gli indicatori Red secondo i requisiti del livello
\end{itemize}

\textbf{5.4 Riesame della Direzione:}
\begin{itemize}
\item Livello 1: Annuale
\item Livello 2: Semestrale
\item Livello 3+: Trimestrale
\item Documentare i riesami con metriche, decisioni, azioni
\end{itemize}

\textbf{5.5 Riduzione degli Incidenti:}
\begin{itemize}
\item Tracciare gli incidenti legati al fattore umano
\item Stabilire la baseline
\item Livello 2: 20\% di riduzione
\item Livello 3: 40\% di riduzione
\item Livello 4: 60\% di riduzione
\item Documentare le evidenze
\end{itemize}

\textbf{5.6 Privacy ed Etica:}
\begin{itemize}
\item Framework di protezione della privacy
\item Mai utilizzare per profilazione individuale
\item Aggregazione minima (10 individui)
\item Livello 3+: Differential privacy ($\varepsilon \leq 0.1$)
\item Reportistica con ritardo temporale (72 ore)
\item Archiviazione e trasmissione sicure
\item Livello 3-4: Audit privacy esterno annuale
\end{itemize}

\textbf{5.7 Modifiche all'Ambito.} Notificare entro 30 giorni:
\begin{itemize}
\item Cambiamenti organizzativi
\item Espansioni/riduzioni dell'ambito
\item Cambiamenti del personale (>20\%)
\item Cambiamenti del Coordinatore CPF
\item Qualsiasi cosa impatti la certificazione
\end{itemize}

\textbf{5.8 Cooperazione:}
\begin{itemize}
\item Concedere accesso per la sorveglianza
\item Rispondere tempestivamente alle richieste
\item Notificare immediatamente: violazioni, aumenti del punteggio, reclami, azioni legali, perdita di personale
\item Implementare le azioni correttive
\end{itemize}

\section{SORVEGLIANZA}

\textbf{6.1 Requisiti per Livello:}

\textit{Livello 1:}
\begin{itemize}
\item Sorveglianza annuale da Assessor (1-2 giorni)
\item Revisione del programma e dei risultati
\end{itemize}

\textit{Livello 2:}
\begin{itemize}
\item Biennale da Auditor (2-3 giorni)
\item Revisione documentale trimestrale
\item Verificare la riduzione degli incidenti
\end{itemize}

\textit{Livello 3:}
\begin{itemize}
\item Annuale da Auditor (3-5 giorni)
\item Revisione documentale trimestrale del monitoraggio
\item Verifica annuale dell'audit privacy
\end{itemize}

\textit{Livello 4:}
\begin{itemize}
\item Annuale da Auditor esterno (5-7 giorni)
\item Revisione documentale mensile
\item Audit privacy esterno trimestrale
\item Peer review biennale
\end{itemize}

\textbf{6.2 Processo:}
\begin{itemize}
\item Preavviso di 30 giorni
\item Focus: Tendenze del punteggio, metodologia, privacy, riesame della direzione, incidenti, modifiche
\item Risultati documentati
\item Azioni correttive per le NC
\end{itemize}

\textbf{6.3 Risultati:}
\begin{itemize}
\item Nessuna NC: Continuare
\item NC minori: Piano entro 30 giorni, implementare entro 90
\item NC maggiori: Azione immediata, sospensione se non corrette in 90 giorni
\end{itemize}

\textbf{6.4 Monitoraggio del Punteggio CPF:}
\begin{itemize}
\item Miglioramento: Può richiedere upgrade
\item Degradazione fuori range: 90 giorni per ripristinare o downgrade
\item Punteggio >149: Sospensione in attesa di azione correttiva
\end{itemize}

\section{RICERTIFICAZIONE}

\textbf{7.1 Requisito.} Ogni 3 anni.

\textbf{7.2 Processo:}
\begin{itemize}
\item Notifica 180 giorni prima della scadenza
\item Domanda 120 giorni prima
\item Audit di ricertificazione completo
\item Assessment completo del Punteggio CPF
\item Revisione delle tendenze triennali
\item Valutazione del miglioramento continuo
\item Audit minimo 60 giorni prima della scadenza
\item Decisione entro 30 giorni
\item Nuovo certificato con date aggiornate
\item Il livello può cambiare in base al punteggio attuale
\end{itemize}

\textbf{7.3 Tempistiche:}
\begin{itemize}
\item Anticipata: Fino a 6 mesi prima (nuovo periodo dalla data effettiva)
\item Ritardata: Ricertificazione completa come nuovo richiedente
\item Nessun periodo di grazia
\end{itemize}

\section{TARIFFE}

\textbf{8.1 Tariffa di Domanda:}

\begin{tabular}{|l|c|}
\hline
1-50 dipendenti & \$500 \\
51-250 & \$1.000 \\
251-1000 & \$1.500 \\
1000+ & \$2.000 \\
\hline
\end{tabular}

Non rimborsabile.

\textbf{8.2 Tariffe di Audit:}

\begin{tabular}{|l|c|c|}
\hline
Dimensione & Fase 1 & Fase 2 \\
\hline
1-50 & \$2.000 & \$4.000 \\
51-250 & \$3.000 & \$7.000 \\
251-1000 & \$5.000 & \$12.000 \\
1000+ & \$8.000 & \$20.000 \\
\hline
\end{tabular}

Complesso/multi-sito: \$1.500/giorno aggiuntivo

\textbf{8.3 Tariffa di Certificazione:}

\begin{tabular}{|l|c|}
\hline
1-50 & \$1.000 \\
51-250 & \$2.000 \\
251-1000 & \$3.500 \\
1000+ & \$5.000 \\
\hline
\end{tabular}

\textbf{8.4 Sorveglianza Annuale:}

\begin{tabular}{|l|c|}
\hline
Livello 1 & 30\% dell'audit iniziale \\
Livello 2 & 40\% (biennale) \\
Livello 3 & 50\% \\
Livello 4 & 60\% \\
\hline
\end{tabular}

\textbf{8.5 Ricertificazione:}
\begin{itemize}
\item Audit: 75\% dell'iniziale
\item Tariffa: Stessa dell'iniziale
\end{itemize}

\textbf{8.6 Altro:}
\begin{itemize}
\item Espansione ambito: \$1.000-\$5.000
\item Follow-up per NC maggiori: \$1.500/giorno
\item Upgrade di livello: \$2.000-\$8.000
\item Urgente: 25\% sovrapprezzo
\item Viaggio: Costi effettivi
\end{itemize}

\textbf{8.7 Pagamento:}
\begin{itemize}
\item Domanda: Con presentazione
\item Fase 1: Prima dell'audit
\item Fase 2: Prima dell'audit
\item Certificazione: Alla decisione
\item Sorveglianza: 30 giorni prima
\item Tutte le tariffe in USD
\item Ritardo: 1,5\% interesse mensile
\item Servizi sospesi se >60 giorni di ritardo
\end{itemize}

\section{SOSPENSIONE E REVOCA}

\textbf{9.1 Motivi di Sospensione:}
\begin{itemize}
\item Punteggio fuori range
\item NC maggiore non corretta (90 giorni)
\item Mancato completamento della sorveglianza
\item Mancato pagamento delle tariffe
\item Violazione della privacy
\item Perdita di personale chiave
\item Cambiamenti organizzativi maggiori
\item Uso improprio del Marchio
\end{itemize}

\textbf{9.2 Processo di Sospensione:}
\begin{itemize}
\item Notifica scritta con motivazioni
\item Restrizione immediata sul nuovo uso del Marchio
\item Registro: "Sospeso"
\item Usi esistenti: Aggiungere "Certificazione Sospesa"
\item Max 180 giorni
\item Piano di risoluzione (30 giorni)
\item Potrebbe essere richiesto audit di verifica
\item Reintegro alla risoluzione
\item Revoca se non risolto
\end{itemize}

\textbf{9.3 Motivi di Revoca:}
\begin{itemize}
\item Mancata risoluzione (180 giorni)
\item Violazioni gravi della privacy
\item Frode/dichiarazioni false/falsificazione
\item Violazioni sistematiche di CPF-27001
\item Profilazione individuale
\item Violazione sostanziale
\item Uso improprio persistente del Marchio
\item Rifiuto di cooperare
\item Insolvenza/fallimento
\end{itemize}

\textbf{9.4 Processo di Revoca:}
\begin{itemize}
\item Notifica scritta con motivazioni
\item 30 giorni per rispondere
\item Revisione da comitato indipendente
\item Decisione entro 45 giorni
\item Se revocato: Cessazione immediata, rimozione dal registro, avviso pubblico, restituzione certificato, nessun rimborso, divieto di riapplicazione per 2 anni
\item Diritto di appello
\end{itemize}

\textbf{9.5 Recesso Volontario:}
\begin{itemize}
\item Preavviso 30 giorni
\item Cessazione immediata
\item Restituzione certificato
\item Nessun rimborso
\item Può riapplicare in qualsiasi momento
\end{itemize}

\section{APPELLI}

\textbf{10.1 Diritto di Appello:}
\begin{itemize}
\item Diniego di certificazione
\item Determinazione del livello
\item Sospensione
\item Revoca
\item Downgrade
\item Dispute su NC maggiori
\end{itemize}

\textbf{10.2 Processo:}
\begin{itemize}
\item Scritto entro 30 giorni
\item Tariffa: \$500
\item Motivazioni e prove
\item Pannello indipendente
\item Decisione entro 45 giorni
\item Opzioni: Conferma/Modifica/Ribalta/Rimanda
\item Tariffa rimborsata se con successo
\item Finale e vincolante
\end{itemize}

\section{RISERVATEZZA}

\textbf{11.1 Riservatezza dell'OdC:}
\begin{itemize}
\item Mantenere la riservatezza di: Dati di assessment, punteggi, documenti interni, informazioni aziendali, metodi di privacy, risultati
\item Limitare l'accesso al team di audit
\item Non divulgare eccetto: Info del registro pubblico, a CPF3, a organismi di accreditamento, come richiesto dalla legge
\item Il personale firma accordi di riservatezza
\end{itemize}

\textbf{11.2 Protezione dei Dati:}
\begin{itemize}
\item Conformità GDPR/CCPA
\item Implementare misure di sicurezza
\item Elaborare solo per la certificazione
\item Notificare le violazioni (24 ore)
\item Cooperare nella risposta alle violazioni
\end{itemize}

\textbf{11.3 Conservazione:}
\begin{itemize}
\item Registri: 7 anni dopo scadenza/revoca
\item Report di audit: 7 anni
\item Appelli/reclami: 10 anni
\item Distruzione sicura
\end{itemize}

\section{LIMITAZIONE DI RESPONSABILITÀ}

\textbf{12.1 Esclusione.} NESSUNA GARANZIA RIGUARDO A RISULTATI AZIENDALI, PREVENZIONE INCIDENTI, CONFORMITÀ NORMATIVA O MIGLIORAMENTI ASSICURATIVI.

\textbf{12.2 Limitazione.} NESSUNA RESPONSABILITÀ PER DANNI INDIRETTI, CONSEQUENZIALI, SPECIALI O PUNITIVI.

\textbf{12.3 Tetto.} RESPONSABILITÀ TOTALE NON OLTRE LE TARIFFE PAGATE NEI 12 MESI PRECEDENTI IL RECLAMO.

\textbf{12.4 Eccezioni:} Negligenza grave, violazioni di riservatezza, violazioni della protezione dati, reclami non limitabili per legge.

\section{INDENNIZZO}

\textbf{13.1 Dall'Organizzazione:} Da reclami derivanti da uso improprio del Marchio, dichiarazioni false, violazioni della privacy, informazioni false, reclami di terze parti.

\textbf{13.2 Dall'OdC:} Da violazione di riservatezza, negligenza nell'audit, violazioni dei dati.

\section{DISPOSIZIONI GENERALI}

\textbf{14.1 Legge Applicabile.} [Giurisdizione]

\textbf{14.2 Controversie.} Negoziazione, mediazione, poi arbitrato.

\textbf{14.3 Intero Accordo.} Questo Accordo e Allegati.

\textbf{14.4 Modifica.} L'OdC può modificare CPF-27001 (180 giorni di preavviso).

\textbf{14.5 Cessione.} L'Organizzazione non può cedere; l'OdC può per trasferimento aziendale.

\textbf{14.6 Forza Maggiore.} Nessuno responsabile per eventi oltre il controllo.

\textbf{14.7 Notifiche.} Scritte agli indirizzi indicati.

\textbf{14.8 Separabilità.} Disposizioni non valide riformate.

\textbf{14.9 Sopravvivenza.} Sezioni 10, 12, 13, 14 sopravvivono.

\section*{FIRME}

\textbf{ORGANISMO DI CERTIFICAZIONE:}

Per: \underline{\hspace{6cm}} Data: \underline{\hspace{3cm}}

Nome: \underline{\hspace{6cm}} Titolo: \underline{\hspace{6cm}}

\vspace{2em}

\textbf{ORGANIZZAZIONE:}

Per: \underline{\hspace{6cm}} Data: \underline{\hspace{3cm}}

Nome: \underline{\hspace{6cm}} Titolo: \underline{\hspace{6cm}}

\newpage

\section*{ALLEGATO A: AMBITO DI CERTIFICAZIONE}

Entità Legale: \underline{\hspace{12cm}}

Unità Operative: \underline{\hspace{12cm}}

Sedi: \underline{\hspace{12cm}}

Personale Totale: \underline{\hspace{4cm}}

Esclusioni: \underline{\hspace{12cm}}

Giustificazione: \underline{\hspace{12cm}}

Approvato da:

OdC: \underline{\hspace{5cm}} Data: \underline{\hspace{3cm}}

Org: \underline{\hspace{5cm}} Data: \underline{\hspace{3cm}}

\end{document}
