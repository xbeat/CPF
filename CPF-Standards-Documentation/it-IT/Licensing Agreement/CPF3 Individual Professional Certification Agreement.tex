\documentclass[11pt,a4paper]{article}

% Packages
\usepackage[utf8]{inputenc}
\usepackage[english]{babel}
\usepackage[margin=2.5cm]{geometry}
\usepackage{hyperref}
\usepackage{fancyhdr}
\usepackage{enumitem}
\usepackage{tabularx}
\usepackage{amssymb}

% Page style
\pagestyle{fancy}
\fancyhf{}
\renewcommand{\headrulewidth}{0.4pt}
\fancyhead[L]{CPF Individual Certification Agreement}
\fancyhead[R]{Candidate #: ________}
\fancyfoot[C]{\thepage}

% Spacing
\setlength{\parindent}{0pt}
\setlength{\parskip}{0.8em}

% Title
\title{\textbf{ACCORDO DI CERTIFICAZIONE PROFESSIONALE INDIVIDUALE CPF}}
\author{}
\date{}

\begin{document}

\maketitle

\section*{PARTI}

Il presente Accordo di Certificazione Professionale Individuale ("Accordo") è stipulato a partire dalla data di esecuzione da parte del Candidato ("Data di Efficacia"), da e tra:

\textbf{[NOME DELL'ORGANISMO DI CERTIFICAZIONE]} ("Organismo di Certificazione" o "CB")\
Una [giurisdizione] [tipo di entità]\
Organismo di Certificazione CPF Autorizzato\
Ufficio Principale: [Indirizzo]\
Email: [Email]

E

\textbf{[NOME DEL CANDIDATO]} ("Candidato" o "Professionista Certificato" previa certificazione)\
Indirizzo: [Indirizzo]\
Email: [Email]\
Telefono: [Telefono]

Collettivamente denominate le "Parti" e individualmente una "Parte".

\section*{PREMESSE}

CONSIDERATO CHE, l'Organismo di Certificazione è autorizzato da CPF3 a gestire il CPF Certification Scheme e a certificare individui come CPF Assessor, CPF Practitioner o CPF Auditor;

CONSIDERATO CHE, il Candidato desidera ottenere la certificazione professionale sotto il CPF Certification Scheme;

CONSIDERATO CHE, l'Organismo di Certificazione è disposto a valutare le qualifiche del Candidato e, se appropriato, concedere la certificazione soggetta ai termini e condizioni stabiliti qui di seguito;

ORA, PERTANTO, in considerazione delle reciproche promesse e accordi contenuti qui di seguito, le Parti concordano quanto segue:

\section{DEFINIZIONI}

\textbf{1.1 "Certificazione"} significa l'attestazione formale da parte dell'Organismo di Certificazione che il Candidato ha soddisfatto i requisiti per una delle seguenti:
\begin{itemize}
\item CPF Certified Assessor
\item CPF Certified Practitioner
\item CPF Certified Auditor
\end{itemize}

\textbf{1.2 "Certification Mark"} significa il marchio, il logo e la designazione associati alla specifica certificazione concessa al Candidato.

\textbf{1.3 "CPF Code of Ethics"} significa gli standard di condotta professionale stabiliti per i professionisti CPF certificati, come possono essere modificati di volta in volta.

\textbf{1.4 "CPE"} significa i crediti di Formazione Professionale Continua richiesti per la ricertificazione.

\textbf{1.5 "Certification Period"} significa il periodo di tre (3) anni dalla data della certificazione iniziale o della ricertificazione.

\textbf{1.6 "Certification Body"} include il suo personale autorizzato, comitati e rappresentanti.

\section{TIPO DI CERTIFICAZIONE}

Il Candidato sta richiedendo la seguente certificazione (spuntare una):

\begin{itemize}
\item[$\square$] \textbf{CPF Certified Assessor}\
Requisiti: Laurea (Psicologia o Cybersecurity con formazione integrativa), 2 anni di esperienza, 80 ore di formazione (CPF-101 + CPF-201), esami scritti e pratici.

\item[$\square$] \textbf{CPF Certified Practitioner}\
Requisiti: Laurea in campo pertinente, 1 anno di esperienza di implementazione CPF, 40 ore di formazione (CPF-101), esame scritto e revisione del portfolio.

\item[$\square$] \textbf{CPF Certified Auditor}\
Requisiti: Certificazione corrente di CPF Assessor, 1 anno come Assessor, 10 valutazioni completate, 64 ore di formazione aggiuntiva (CPF-401 + ISO 19011), esami scritti e pratici.
\end{itemize}

\section{APPLICAZIONE E PROCESSO DI CERTIFICAZIONE}

\textbf{3.1 Invio dell'Applicazione.} Il Candidato deve:

\begin{enumerate}[label=\alph*)]
\item Completare il modulo di applicazione in modo accurato e veritiero;
\item Inviare tutta la documentazione richiesta inclusa:
\begin{itemize}
\item Trascrizioni ufficiali o certificati di laurea;
\item Lettere di verifica dell'esperienza o portfolio professionale;
\item Certificati di completamento della formazione;
\item Referenze professionali (minimo 2);
\item Curriculum Vitae/Resume attuale;
\item Identificazione fotografica rilasciata dal governo;
\end{itemize}
\item Pagare la tariffa non rimborsabile di revisione dell'applicazione;
\item Fornire la firma elettronica sul CPF Code of Ethics.
\end{enumerate}

\textbf{3.2 Revisione dell'Applicazione.} L'Organismo di Certificazione deve:

\begin{enumerate}[label=\alph*)]
\item Revisionare l'applicazione per completezza e idoneità;
\item Verificare le credenziali educative con le istituzioni che rilasciano i titoli di studio;
\item Verificare l'impiego e l'esperienza con i datori di lavoro elencati o attraverso la revisione del portfolio;
\item Contattare i referenti professionali;
\item Verificare il completamento della formazione con i fornitori di formazione approvati;
\item Completare la revisione entro quindici (15) giorni lavorativi dal ricevimento dell'applicazione completa;
\item Notificare al Candidato la determinazione di idoneità o richiedere informazioni aggiuntive.
\end{enumerate}

\textbf{3.3 Esame.} All'approvazione dell'applicazione:

\begin{enumerate}[label=\alph*)]
\item Il Candidato deve programmare e completare gli esami richiesti entro dodici (12) mesi;
\item Esame scritto somministrato tramite piattaforma di testing sicura;
\item Esame pratico o revisione del portfolio come applicabile;
\item Risultati forniti entro cinque (5) giorni lavorativi;
\item Punteggi di sufficienza richiesti: 70% per Assessor/Practitioner, 75% per Auditor.
\end{enumerate}

\textbf{3.4 Politica di Ripetizione.} Se il Candidato fallisce l'esame:

\begin{enumerate}[label=\alph*)]
\item Può ripeterlo dopo un periodo di attesa di trenta (30) giorni;
\item Massimo tre (3) tentativi entro dodici (12) mesi;
\item Ogni ripetizione richiede il pagamento di una tariffa di ripetizione (50% della tariffa d'esame);
\item Dopo tre fallimenti, deve completare una formazione aggiuntiva e attendere sei (6) mesi prima di riapplicare;
\item Gli esami scritti e pratici possono essere ripetuti indipendentemente.
\end{enumerate}

\textbf{3.5 Decisione di Certificazione.} L'Organismo di Certificazione deve:

\begin{enumerate}[label=\alph*)]
\item Prendere una decisione di certificazione entro dieci (10) giorni lavorativi dal completamento dell'esame;
\item Concedere la certificazione se tutti i requisiti sono soddisfatti;
\item Negare la certificazione con spiegazione scritta se i requisiti non sono soddisfatti;
\item Fornire il diritto di appello se la certificazione è negata;
\item Rilasciare il certificato e il badge digitale all'approvazione;
\item Aggiungere il Professionista Certificato al registro pubblico delle certificazioni entro cinque (5) giorni lavorativi.
\end{enumerate}

\section{CONCESSIONE DELLA CERTIFICAZIONE E DIRITTI}

\textbf{4.1 Concessione della Certificazione.} Al completamento con successo di tutti i requisiti e al pagamento della tariffa di certificazione, l'Organismo di Certificazione concede al Professionista Certificato:

\begin{enumerate}[label=\alph*)]
\item La certificazione professionale nella categoria richiesta;
\item Il diritto di utilizzare l'applicabile Certification Mark;
\item L'iscrizione nel registro pubblico delle certificazioni;
\item L'accesso ai benefici del titolare della certificazione;
\item Certificato valido per tre (3) anni dalla data di rilascio.
\end{enumerate}

\textbf{4.2 Utilizzo del Certification Mark.} Il Professionista Certificato può:

\begin{enumerate}[label=\alph*)]
\item Utilizzare il Certification Mark dopo il proprio nome (es., "CPF Certified Assessor");
\item Visualizzare il logo della certificazione su biglietti da visita, carta intestata, firme email e profili professionali;
\item Fare riferimento alla certificazione nei materiali di marketing e nelle proposte;
\item Utilizzare il badge digitale sui siti di networking professionale (LinkedIn, ecc.);
\item Dichiarare lo stato di certificazione nelle informazioni biografiche e nelle presentazioni.
\end{enumerate}

\textbf{4.3 Restrizioni sull'Utilizzo del Certification Mark.} Il Professionista Certificato NON deve:

\begin{enumerate}[label=\alph*)]
\item Modificare, alterare o creare versioni derivate del Certification Mark;
\item Utilizzare il Certification Mark in un modo che suggerisca la certificazione di prodotti, servizi o organizzazioni (a meno che non siano separatamente certificati);
\item Utilizzare il Certification Mark dopo la scadenza, sospensione o revoca della certificazione;
\item Trasferire o concedere in sublicenza il diritto di utilizzare il Certification Mark;
\item Utilizzare il Certification Mark in un modo che rechi discredito a CPF o all'Organismo di Certificazione;
\item Rappresentare che la certificazione si estende oltre la specifica categoria concessa.
\end{enumerate}

\textbf{4.4 Benefici della Certificazione.} Il Professionista Certificato riceve:

\begin{enumerate}[label=\alph*)]
\item Certificato elettronico e fisico;
\item Badge digitale per l'uso online;
\item Iscrizione nel registro pubblico delle certificazioni con pagina del profilo;
\item Accesso al portale online di tracciamento CPE;
\item Inviti a eventi della community CPF e webinar;
\item Accesso a risorse e strumenti esclusivi (come applicabile);
\item Newsletter e aggiornamenti sugli sviluppi CPF;
\item Opportunità di networking attraverso la community dei professionisti certificati.
\end{enumerate}

\section{OBBLIGHI DEL PROFESSIONISTA CERTIFICATO}

\textbf{5.1 Conformità al Code of Ethics.} Il Professionista Certificato deve:

\begin{enumerate}[label=\alph*)]
\item Aderire al CPF Code of Ethics in ogni momento;
\item Mantenere integrità, obiettività e condotta professionale;
\item Praticare solo all'interno di aree di competenza dimostrata;
\item Proteggere la riservatezza dei dati di valutazione e delle informazioni del cliente;
\item Non utilizzare mai i dati di valutazione per la profilazione individuale;
\item Implementare metodologie che preservano la privacy in tutto il lavoro CPF;
\item Segnalare sospette violazioni etiche da parte di altri professionisti certificati.
\end{enumerate}

\textbf{5.2 Continuing Professional Education (CPE).} Il Professionista Certificato deve:

\begin{enumerate}[label=\alph*)]
\item Completare i crediti CPE richiesti annualmente:
\begin{itemize}
\item CPF Assessor: 40 crediti all'anno (120 in 3 anni)
\item CPF Practitioner: 30 crediti all'anno (90 in 3 anni)
\item CPF Auditor: 50 crediti all'anno (150 in 3 anni)
\end{itemize}
\item Documentare tutte le attività CPE nel portale CPE online;
\item Conservare la documentazione di supporto per cinque (5) anni;
\item Sottoporsi a audit CPE se selezionato (10% casuale annualmente);
\item Assicurare il completamento dei crediti CPE etici minimi annualmente.
\end{enumerate}

\textbf{5.3 Requisiti di Pratica Professionale.}

\textit{Per CPF Assessors:}
\begin{itemize}
\item Condurre un minimo di cinque (5) valutazioni CPF durante il periodo di certificazione di 3 anni;
\item Partecipare alle attività di calibrazione degli assessor;
\item Sottoporre almeno un rapporto di valutazione per la revisione tra pari;
\item Mantenere una conoscenza attuale degli aggiornamenti della metodologia CPF.
\end{itemize}

\textit{Per CPF Practitioners:}
\begin{itemize}
\item Mantenere un portfolio aggiornato che dimostri la continua applicazione pratica;
\item Documentare un minimo di tre (3) progetti di implementazione durante il periodo di certificazione;
\item Partecipare alla community of practice dei practitioner.
\end{itemize}

\textit{Per CPF Auditors:}
\begin{itemize}
\item Condurre un minimo di quindici (15) giorni di audit all'anno (45 in 3 anni);
\item Servire come lead auditor in un minimo di cinque (5) audit durante il periodo di certificazione;
\item Sottoporre i rapporti di audit per la revisione della qualità;
\item Partecipare alle attività di valutazione della competenza degli auditor;
\item Mantenere l'indipendenza dalle attività di consulenza per ISO 19011.
\end{itemize}

\textbf{5.4 Obblighi di Notifica.} Il Professionista Certificato deve notificare immediatamente all'Organismo di Certificazione:

\begin{enumerate}[label=\alph*)]
\item Cambiamenti alle informazioni di contatto;
\item Condanne penali o azioni disciplinari professionali;
\item Perdita delle qualifiche sottostanti (revoca della laurea, sospensione della licenza);
\item Coinvolgimento in reclami etici significativi o indagini;
\item Fallimento o circostanze finanziarie che influenzano la reputazione professionale;
\item Qualsiasi circostanza che possa impattare lo stato di certificazione o l'idoneità.
\end{enumerate}

\textbf{5.5 Cooperazione con le Indagini.} Il Professionista Certificato deve:

\begin{enumerate}[label=\alph*)]
\item Cooperare pienamente con le indagini sui reclami etici;
\item Rispondere alle richieste dell'Organismo di Certificazione entro i termini specificati;
\item Fornire la documentazione e le informazioni richieste;
\item Partecipare a interviste se richiesto;
\item Non retaliare contro i querelanti o i testimoni.
\end{enumerate}

\textbf{5.6 Rappresentazione Accurata.} Il Professionista Certificato deve:

\begin{enumerate}[label=\alph*)]
\item Rappresentare accuratamente lo stato e l'ambito della certificazione;
\item Non travisare qualifiche o esperienza;
\item Distinguere chiaramente i servizi CPF dagli altri servizi offerti;
\item Fornire informazioni veritiere nel marketing e nelle proposte;
\item Correggere eventuali false rappresentazioni prontamente quando scoperte.
\end{enumerate}

\section{RICERTIFICAZIONE}

\textbf{6.1 Requisito di Ricertificazione.} La certificazione scade tre (3) anni dalla data di rilascio. Per mantenere la certificazione, il Professionista Certificato deve richiedere la ricertificazione.

\textbf{6.2 Processo di Ricertificazione.}

\begin{enumerate}[label=\alph*)]
\item L'Organismo di Certificazione invia un avviso di ricertificazione 180 giorni prima della scadenza;
\item Il Professionista Certificato invia l'applicazione di ricertificazione 90 giorni prima della scadenza;
\item L'applicazione di ricertificazione include:
\begin{itemize}
\item Record CPE completi per il periodo di 3 anni;
\item Documentazione dei requisiti di pratica professionale;
\item Referenze professionali aggiornate (se richieste);
\item Attestazione etica;
\item Pagamento della tariffa di ricertificazione;
\end{itemize}
\item L'Organismo di Certificazione revisiona la sottomissione entro 60 giorni;
\item Se approvata, viene rilasciato un nuovo certificato con data di scadenza aggiornata;
\item Se negata, il Professionista Certificato riceve una spiegazione scritta e i diritti di appello.
\end{enumerate}

\textbf{6.3 Periodo di Grazia.} Se la ricertificazione non è completata entro la scadenza:

\begin{enumerate}[label=\alph*)]
\item Si applica un periodo di grazia di novanta (90) giorni;
\item Lo stato di certificazione cambia in "Pending Recertification";
\item L'uso del Certification Mark è limitato durante il periodo di grazia;
\item Si applica una tariffa di ricertificazione tardiva (ulteriori $100);
\item Dopo il periodo di grazia, è richiesto il processo di ricertificazione completo inclusi gli esami.
\end{enumerate}

\textbf{6.4 Rimedio al Deficit di CPE.} Se i requisiti CPE non sono soddisfatti:

\begin{enumerate}[label=\alph*)]
\item Il Professionista Certificato può richiedere una proroga fino a 90 giorni per completare i CPE rimanenti;
\item La proroga è concessa a discrezione dell'Organismo di Certificazione;
\item Lo stato di certificazione cambia in "Conditional" durante la proroga;
\item Se i CPE non sono completati entro la proroga, la certificazione decade;
\item Può applicarsi una tariffa di proroga.
\end{enumerate}

\section{TARIFFE}

\textbf{7.1 Tariffa di Applicazione.}
\begin{itemize}
\item CPF Assessor: $300 (non rimborsabile)
\item CPF Practitioner: $200 (non rimborsabile)
\item CPF Auditor: $400 (non rimborsabile)
\end{itemize}

\textbf{7.2 Tariffe d'Esame.}
\begin{itemize}
\item CPF Assessor Scritto: $400
\item CPF Assessor Pratico: $600
\item CPF Practitioner Scritto: $300
\item CPF Practitioner Revisione Portfolio: $400
\item CPF Auditor Scritto: $450
\item CPF Auditor Pratico: $800
\item Tariffa di Ripetizione: 50% della tariffa d'esame originale
\end{itemize}

\textbf{7.3 Tariffa di Certificazione.} Al completamento con successo di tutti i requisiti:
\begin{itemize}
\item CPF Assessor: $200
\item CPF Practitioner: $150
\item CPF Auditor: $250
\end{itemize}

\textbf{7.4 Tariffe di Ricertificazione.}
\begin{itemize}
\item CPF Assessor: $400
\item CPF Practitioner: $300
\item CPF Auditor: $500
\item Ricertificazione Tardiva (entro il periodo di grazia di 90 giorni): Aggiungi $100
\end{itemize}

\textbf{7.5 Altre Tariffe.}
\begin{itemize}
\item Richiesta di Proroga CPE: $50
\item Duplicato Certificato: $25
\item Lettera di Verifica della Certificazione: $15
\item Tariffa di Appello: $200 (rimborsata se l'appello ha successo)
\end{itemize}

\textbf{7.6 Termini di Pagamento.}
\begin{enumerate}[label=\alph*)]
\item Tutte le tariffe sono pagabili in USD;
\item Pagamento con carta di credito, bonifico bancario o assegno;
\item Tariffe non rimborsabili se non specificamente dichiarato;
\item Servizi non forniti fino al ricevimento del pagamento;
\item Tariffe scadute possono comportare la sospensione della certificazione.
\end{enumerate}

\section{SOSPENSIONE E REVOCA}

\textbf{8.1 Motivi per la Sospensione.} L'Organismo di Certificazione può sospendere la certificazione per:

\begin{enumerate}[label=\alph*)]
\item Mancato completamento dei CPE richiesti entro la scadenza;
\item Mancato pagamento delle tariffe richieste;
\item Reclamo etico sotto indagine;
\item Mancato soddisfacimento dei requisiti di pratica professionale;
\item Mancata risposta alle richieste dell'Organismo di Certificazione;
\item Perdita delle qualifiche sottostanti in attesa di indagine.
\end{enumerate}

\textbf{8.2 Processo di Sospensione.}

\begin{enumerate}[label=\alph*)]
\item Notifica scritta di sospensione con motivi specifici;
\item Cessazione immediata dell'uso del Certification Mark;
\item Stato nel registro cambiato in "Suspended";
\item Periodo di sospensione: Massimo 90 giorni;
\item Piano di rimedio richiesto entro 30 giorni;
\item Reintegro al successo del rimedio;
\item Se non rimediato entro 90 giorni: Avvio delle procedure di revoca.
\end{enumerate}

\textbf{8.3 Motivi per la Revoca.} L'Organismo di Certificazione può revocare la certificazione per:

\begin{enumerate}[label=\alph*)]
\item Gravi violazioni etiche inclusi:
\begin{itemize}
\item Frode, falsa rappresentazione o disonestà;
\item Violazione della riservatezza o uso improprio dei dati;
\item Profilazione individuale utilizzando i dati di valutazione;
\item Condanna penale relativa alla condotta professionale;
\end{itemize}
\item Mancato rimedio della sospensione entro 90 giorni;
\item Violazioni ripetute o sistematiche del CPF Code of Ethics;
\item Perdita delle qualifiche sottostanti (revoca della laurea);
\item Fornitura di informazioni false nell'applicazione o nella ricertificazione;
\item Sublicenza non autorizzata o trasferimento della certificazione;
\item Violazione materiale del presente Accordo.
\end{enumerate}

\textbf{8.4 Processo di Revoca.}

\begin{enumerate}[label=\alph*)]
\item Notifica scritta dell'intenzione di revocare con motivi specifici;
\item Opportunità di risposta entro 30 giorni;
\item Revisione indipendente da parte del comitato etico dell'Organismo di Certificazione;
\item Decisione finale comunicata entro 45 giorni;
\item Se revocata:
\begin{itemize}
\item Cessazione immediata di tutto l'uso del Certification Mark;
\item Rimozione dal registro delle certificazioni;
\item Avviso pubblico della revoca;
\item Restituzione del certificato all'Organismo di Certificazione;
\item Proibizione di riapplicare per un minimo di 2 anni (o permanente);
\end{itemize}
\item Diritto di appellare la decisione di revoca.
\end{enumerate}

\textbf{8.5 Rinuncia Volontaria.} Il Professionista Certificato può rinunciare volontariamente alla certificazione mediante:

\begin{enumerate}[label=\alph*)]
\item Notifica scritta all'Organismo di Certificazione;
\item Cessazione immediata dell'uso del Certification Mark;
\item Restituzione del certificato;
\item Nessun rimborso delle tariffe;
\item Può riapplicare per la certificazione in qualsiasi momento completando il processo di certificazione completo.
\end{enumerate}

\section{APPELI}

\textbf{9.1 Diritto di Appello.} Il Professionista Certificato può appellare:

\begin{enumerate}[label=\alph*)]
\item La negazione della certificazione;
\item Il fallimento dell'esame dovuto a irregolarità procedurali (non il punteggio);
\item La negazione della ricertificazione;
\item La decisione di sospensione;
\item La decisione di revoca;
\item Le azioni disciplinari.
\end{enumerate}

\textbf{9.2 Processo di Appello.}

\begin{enumerate}[label=\alph*)]
\item Appello sottomesso per iscritto entro 30 giorni dalla decisione;
\item Pagamento della tariffa di appello ($200);
\item Specificazione dei motivi dell'appello e documentazione di supporto;
\item Assegnazione di un panel di appello indipendente (nessun coinvolgimento nella decisione originale);
\item Il panel revisiona tutte le prove e la logica della decisione;
\item L'appellante può fornire informazioni scritte aggiuntive;
\item Il panel emette una decisione entro 30 giorni;
\item Opzioni di decisione: Confermare, Modificare, Rovesciare o Rimettere per riconsiderazione;
\item Tariffa rimborsata se l'appello ha successo;
\item La decisione del panel di appello è finale e vincolante.
\end{enumerate}

\textbf{9.3 Composizione del Panel di Appello.}

\begin{itemize}
\item Tre membri: un professionista CPF certificato, un esperto della materia, un rappresentante dell'Organismo di Certificazione non coinvolto nella decisione originale;
\item I membri del panel non hanno conflitti di interesse;
\item Decisioni prese a maggioranza di voti;
\item Deliberazioni del panel riservate.
\end{itemize}

\section{RISERVATEZZA E PROTEZIONE DEI DATI}

\textbf{10.1 Riservatezza.} L'Organismo di Certificazione deve:

\begin{enumerate}[label=\alph*)]
\item Mantenere la riservatezza delle informazioni del Candidato/Professionista Certificato;
\item Limitare l'accesso alle informazioni al personale che ha bisogno di sapere;
\item Proteggere le risposte agli esami e i dati di valutazione;
\item Non divulgare informazioni riservate senza consenso, eccetto:
\begin{itemize}
\item Informazioni del registro pubblico (nome, tipo di certificazione, stato, data di scadenza);
\item Come richiesto dalla legge o da ordine del tribunale;
\item A CPF3 per scopi di supervisione della qualità;
\item Agli organismi di accreditamento durante gli audit;
\item Indagine sui reclami etici come necessario;
\end{itemize}
\item Implementare misure di sicurezza tecniche e organizzative appropriate;
\item Conformarsi alle leggi applicabili sulla protezione dei dati (GDPR, CCPA, ecc.).
\end{enumerate}

\textbf{10.2 Diritti di Protezione dei Dati.} Il Professionista Certificato ha il diritto di:

\begin{enumerate}[label=\alph*)]
\item Accedere ai dati personali detenuti dall'Organismo di Certificazione;
\item Richiedere la correzione di informazioni inaccurata;
\item Richiedere la cancellazione dei dati (soggetto ai requisiti di conservazione dei record);
\item Opporsi al trattamento per determinati scopi;
\item Ricevere i dati in formato portabile;
\item Presentare un reclamo all'autorità per la protezione dei dati.
\end{enumerate}

\textbf{10.3 Conservazione dei Dati.} L'Organismo di Certificazione deve:

\begin{enumerate}[label=\alph*)]
\item Conservare i record di certificazione per sette (7) anni dopo la scadenza o revoca della certificazione;
\item Conservare i record delle indagini etiche per dieci (10) anni;
\item Distruggere in modo sicuro i dati dopo il periodo di conservazione a meno che non si applichi un blocco legale;
\item Mantenere tracce di audit per l'accesso e le modifiche ai dati.
\end{enumerate}

\textbf{10.4 Notifica della Violazione dei Dati.} In caso di violazione dei dati:

\begin{enumerate}[label=\alph*)]
\item L'Organismo di Certificazione deve notificare i Professionisti Certificati interessati entro 72 ore;
\item La notifica include la natura della violazione, i dati interessati e le azioni di mitigazione;
\item L'Organismo di Certificazione deve notificare le autorità applicabili per la protezione dei dati come richiesto dalla legge;
\item L'Organismo di Certificazione deve intraprendere azioni per prevenire ulteriori violazioni.
\end{enumerate}

\section{LIMITAZIONE DI RESPONSABILITÀ}

\textbf{11.1 Esclusione di Garanzie.} L'ORGANISMO DI CERTIFICAZIONE NON FORNISCE ALCUNA GARANZIA, ESPRESSA O IMPLICITA, RIGUARDO AGLI ESITI DELLA CERTIFICAZIONE, AI BENEFICI DI CARRIERA O AL POTENZIALE DI REDDITO. LA CERTIFICAZIONE È FORNITA "COSÌ COM'È" SENZA GARANZIA DI COMMERCIABILITÀ O IDONEITÀ PER UN PARTICOLARE SCOPO.

\textbf{11.2 Limitazione dei Danni.} IN NESSUN CASO L'ORGANISMO DI CERTIFICAZIONE SARÀ RESPONSABILE PER DANNI INDIRETTI, INCIDENTALI, CONSEGUENZIALI, SPECIALI, ESEMPLARI O PUNITIVI, INCLUSI REDDITO PERDUTO, OPPORTUNITÀ DI BUSINESS PERDUTE O DANNI ALLA REPUTAZIONE, DERIVANTI DALLA CERTIFICAZIONE O DALLA SUA NEGAZIONE.

\textbf{11.3 Tetto alla Responsabilità.} LA RESPONSABILITÀ TOTALE DELL'ORGANISMO DI CERTIFICAZIONE SOTTO QUESTO ACCORDO NON SUPERERÀ LE TARIFFE TOTALI PAGATE DAL PROFESSIONISTA CERTIFICATO NEI DODICI (12) MESI PRECEDENTI LA RICHIESTA.

\textbf{11.4 Eccezioni.} Le limitazioni non si applicano a:

\begin{enumerate}[label=\alph*)]
\item Negligenza grave o cattiva condotta intenzionale dell'Organismo di Certificazione;
\item Violazioni degli obblighi di riservatezza;
\item Violazioni della protezione dei dati;
\item Richieste che non possono essere limitate dalla legge applicabile.
\end{enumerate}

\section{INDENNIZZO}

\textbf{12.1 Indennizzo da parte del Professionista Certificato.} Il Professionista Certificato deve indennizzare, difendere e tenere indenne l'Organismo di Certificazione da richieste derivanti da:

\begin{enumerate}[label=\alph*)]
\item Servizi professionali del Professionista Certificato a terze parti;
\item Negligenza o cattiva condotta del Professionista Certificato;
\item Violazione del presente Accordo o del CPF Code of Ethics da parte del Professionista Certificato;
\item Utilizzo non autorizzato dei Certification Marks da parte del Professionista Certificato;
\item Informazioni false o fuorvianti fornite nell'applicazione o nella ricertificazione.
\end{enumerate}

\textbf{12.2 Assicurazione di Responsabilità Professionale.} Il Professionista Certificato che fornisce servizi CPF professionalmente deve mantenere:

\begin{itemize}
\item Assicurazione di responsabilità professionale (errori e omissioni) con copertura minima appropriata all'ambito di pratica;
\item Assicurazione di responsabilità generale come applicabile;
\item Prova dell'assicurazione fornita ai clienti su richiesta;
\item L'Organismo di Certificazione non è responsabile per la verifica della copertura assicurativa.
\end{itemize}

\section{DISPOSIZIONI GENERALI}

\textbf{13.1 Legge Applicabile.} Il presente Accordo sarà governato dalle leggi di [Giurisdizione], senza considerare i principi di conflitto di leggi.

\textbf{13.2 Risoluzione delle Controversie.}

\begin{enumerate}[label=\alph*)]
\item È richiesta una negoziazione di buona fede prima della risoluzione formale delle controversie;
\item Le controversie non risolte attraverso negoziazione o processo di appello saranno risolte attraverso arbitrato vincolante;
\item L'arbitrato è condotto secondo le regole di [Servizio di Arbitrato];
\item Arbitrato in [Città, Giurisdizione], lingua inglese;
\item La decisione dell'arbitro è finale e vincolante;
\item Ciascuna parte sostiene i propri costi a meno che l'arbitro non determini altrimenti.
\end{enumerate}

\textbf{13.3 Accordo Completo.} Il presente Accordo, incluso il CPF Code of Ethics incorporato e i requisiti del Certification Scheme, costituisce l'accordo completo e sostituisce tutti i precedenti accordi.

\textbf{13.4 Emendamento.} L'Organismo di Certificazione può emendare il presente Accordo o il CPF Code of Ethics fornendo un preavviso scritto di 90 giorni. La prosecuzione della certificazione dopo la data di efficacia costituisce accettazione. Se il Professionista Certificato non accetta gli emendamenti, può rinunciare volontariamente alla certificazione.

\textbf{13.5 Assegnazione.} Il Professionista Certificato non può assegnare o trasferire la certificazione. L'Organismo di Certificazione può assegnare il presente Accordo in connessione con un trasferimento di attività o una fusione.

\textbf{13.6 Notifiche.} Tutte le notifiche devono essere inviate agli indirizzi indicati sopra o come aggiornati per iscritto. L'email con conferma di ricevuta è accettabile per le comunicazioni di routine.

\textbf{13.7 Severabilità.} Se qualsiasi disposizione è ritenuta invalida, le disposizioni rimanenti continuano in pieno effetto.

\textbf{13.8 Rinuncia.} Il mancato esercizio di qualsiasi diritto non costituisce rinuncia al diritto di esercitarlo in seguito.

\textbf{13.9 Contraente Indipendente.} Il Professionista Certificato è un contraente indipendente, non un dipendente o agente dell'Organismo di Certificazione.

\textbf{13.10 Sopravvivenza.} Le Sezioni 5.1 (Etica), 8 (Effetti di Sospensione/Revoca), 10 (Riservatezza), 11 (Limitazione di Responsabilità), 12 (Indennizzo) e 13 (Disposizioni Generali) sopravvivono alla terminazione della certificazione.

\section{DICHIARAZIONI DI CONSAPEVOLEZZA}

Sottoscrivendo di seguito, il Candidato riconosce e accetta di:

\begin{enumerate}[label=\alph*)]
\item Aver letto e compreso il presente Accordo nella sua interezza;
\item Aver letto e accettato di conformarsi al CPF Code of Ethics;
\item Aver fornito informazioni accurate e veritiere nell'applicazione;
\item Comprendere i requisiti di certificazione e gli obblighi continui;
\item Comprendere che le tariffe sono non rimborsabili;
\item Comprendere che la certificazione può essere sospesa o revocata per violazioni;
\item Comprendere di dover mantenere i requisiti CPE e di pratica;
\item Autorizzare l'Organismo di Certificazione a verificare le informazioni fornite;
\item Autorizzare la pubblicazione del nome e dello stato di certificazione nel registro pubblico;
\item Acconsentire al trattamento dei dati personali come descritto;
\item Comprendere che la certificazione non garantisce l'impiego o il reddito;
\item Accettare di risolvere le controversie attraverso l'arbitrato;
\item Cessare immediatamente l'uso del Certification Mark se la certificazione termina.
\end{enumerate}

\vspace{2em}

\section*{FIRME}

\textbf{ORGANISMO DI CERTIFICAZIONE: [NOME]}

\vspace{1.5em}

Per: \underline{\hspace{6cm}} Data: \underline{\hspace{3cm}}

Nome: \underline{\hspace{6cm}}

Titolo: \underline{\hspace{6cm}}

\vspace{2em}

\textbf{CANDIDATO/PROFESSIONISTA CERTIFICATO}

\vspace{1.5em}

Firma: \underline{\hspace{6cm}} Data: \underline{\hspace{3cm}}

Nome Stampato: \underline{\hspace{6cm}}

\vspace{2em}

\section*{REGISTRO DI CERTIFICAZIONE (Solo per uso CB)}

\begin{tabular}{|l|p{10cm}|}
\hline
\textbf{Tipo di Certificazione} & \hspace{8cm} \
\hline
\textbf{Numero di Certificato} & \
\hline
\textbf{Data di Rilascio} & \
\hline
\textbf{Data di Scadenza} & \
\hline
\textbf{Rilasciato Da} & \
\hline
\end{tabular}

\vspace{2em}

\begin{center}
\textit{Fine dell'Accordo di Certificazione Professionale Individuale}
\end{center}

\end{document}