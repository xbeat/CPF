\documentclass[11pt,a4paper]{article}

% Pacchetti
\usepackage[utf8]{inputenc}
\usepackage[english]{babel}
\usepackage[margin=2.5cm]{geometry}
\usepackage{hyperref}
\usepackage{fancyhdr}
\usepackage{enumitem}
\usepackage{tabularx}
\usepackage{graphicx}

% Stile della pagina
\pagestyle{fancy}
\fancyhf{}
\renewcommand{\headrulewidth}{0.4pt}
\fancyhead[L]{Accordo di Licenza dello Schema CPF}
\fancyhead[R]{Contratto \#: \_\_\_\_\_\_\_\_}
\fancyfoot[C]{\thepage}

% Spaziatura
\setlength{\parindent}{0pt}
\setlength{\parskip}{0.8em}

% Titolo
\title{\textbf{ACCORDO DI LICENZA\\SCHEMA DI CERTIFICAZIONE CPF}}
\author{}
\date{}

\begin{document}

\maketitle

\section*{PARTI}

Questo Accordo di Licenza dello Schema di Certificazione ("Accordo") è stipulato in data \_\_\_ del mese di \_\_\_\_\_\_\_\_\_\_\_, 20\_\_\_ ("Data di Efficacia"), tra:

\textbf{CPF3} ("Licenziante" o "Proprietario dello Schema")\\
Un [giurisdizione] [tipo di entità]\\
Sede Principale: [Indirizzo]\\
Email: legal@cpf3.org

E

\textbf{[NOME DELL'ENTE DI CERTIFICAZIONE]} ("Licenziatario" o "Ente di Certificazione")\\
Un [giurisdizione] [tipo di entità]\\
Numero di Registrazione: [Numero]\\
Sede Principale: [Indirizzo]\\
Email: [Email]

Collettivamente indicati come le "Parti" e individualmente come una "Parte".

\section*{PREMESSA}

CONSIDERATO CHE, il Licenziante è il proprietario e sviluppatore del Cybersecurity Psychology Framework (CPF), un framework completo per la valutazione e la gestione delle vulnerabilità psicologiche nei contesti di cybersecurity;

CONSIDERATO CHE, il Licenziante ha sviluppato uno schema di certificazione ("Schema di Certificazione CPF") per certificare individui e organizzazioni nella metodologia CPF, come dettagliato nel documento dello Schema di Certificazione CPF versione 1.0 datato gennaio 2025;

CONSIDERATO CHE, il Licenziatario è un ente di certificazione accreditato secondo ISO/IEC 17065:2012 da [Nome dell'Ente di Accreditamento] (Certificato n. [Numero], scadenza [Data]);

CONSIDERATO CHE, il Licenziatario desidera ottenere una licenza per operare lo Schema di Certificazione CPF all'interno di un territorio definito;

CONSIDERATO CHE, il Licenziante è disposto a concedere tale licenza soggetta ai termini e condizioni stabiliti in questo Accordo;

ORA, PERTANTO, in considerazione dei reciproci patteggiamenti e accordi contenuti nel presente, e per altra buona e valida considerazione, la cui ricezione e adeguatezza sono qui riconosciute, le Parti concordano quanto segue:

\section{DEFINIZIONI}

\textbf{1.1 "Territorio Autorizzato"} significa le regioni geografiche specificate nell'Allegato A dove il Licenziatario è autorizzato a operare lo Schema di Certificazione CPF.

\textbf{1.2 "Certificazione"} significa il rilascio di un'attestazione formale da parte del Licenziatario che un individuo o un'organizzazione ha soddisfatto i requisiti dello Schema di Certificazione CPF.

\textbf{1.3 "Marchi di Certificazione"} significa i marchi, marchi di servizio, loghi e marchi di certificazione di proprietà del Licenziante e utilizzati in connessione con lo Schema di Certificazione CPF, come dettagliato nell'Allegato B.

\textbf{1.4 "Informazioni Confidenziali"} significa tutte le informazioni non pubbliche divulgate da una Parte all'altra, incluse ma non limitate a voci di esame, metodologie di valutazione, processi proprietari, informazioni commerciali e dati tecnici.

\textbf{1.5 "Materiali CPF"} significa tutta la proprietà intellettuale relativa al framework CPF, inclusi ma non limitati a: il documento dello Schema di Certificazione CPF, programmi di formazione, voci di esame, strumenti di valutazione, metodologie, modelli e documentazione correlata.

\textbf{1.6 "Personale del Licenziatario"} significa tutti i dipendenti, appaltatori e agenti del Licenziatario che partecipano alle attività di certificazione CPF.

\textbf{1.7 "Certificazione Organizzativa"} significa la certificazione di un'organizzazione ai Livelli di Conformità CPF 1-4, o come Fornitore di Servizi Autorizzato.

\textbf{1.8 "Certificazione Professionale"} significa la certificazione di un individuo come Valutatore CPF, Praticante CPF o Auditor CPF.

\section{CONCESSIONE DELLA LICENZA}

\textbf{2.1 Concessione della Licenza.} Soggetta ai termini e condizioni di questo Accordo, il Licenziante concede qui al Licenziatario una licenza non esclusiva, non trasferibile, non sublicenziabile per:

\begin{enumerate}[label=\alph*)]
\item Operare lo Schema di Certificazione CPF all'interno del Territorio Autorizzato;
\item Rilasciare Certificazioni Professionali a individui qualificati;
\item Rilasciare Certificazioni Organizzative a organizzazioni qualificate;
\item Utilizzare i Marchi di Certificazione in connessione con le attività di certificazione autorizzate;
\item Accedere e utilizzare i Materiali CPF esclusivamente allo scopo di operare lo schema di certificazione;
\item Rappresentarsi come un "Ente di Certificazione CPF Autorizzato" all'interno del Territorio Autorizzato.
\end{enumerate}

\textbf{2.2 Territorio Autorizzato.} Il Territorio Autorizzato è specificato nell'Allegato A e può essere modificato solo con accordo scritto di entrambe le Parti.

\textbf{2.3 Limitazioni della Licenza.} Il Licenziatario riconosce e accetta che:

\begin{enumerate}[label=\alph*)]
\item La licenza concessa è non esclusiva; il Licenziante può concedere licenze ad altri enti di certificazione;
\item Il Licenziatario non può sublicenziare, cedere o trasferire alcun diritto ai sensi di questo Accordo;
\item Il Licenziatario non può operare lo schema di certificazione al di fuori del Territorio Autorizzato;
\item Il Licenziatario non può modificare i Materiali CPF senza previa approvazione scritta del Licenziante;
\item Tutte le certificazioni rilasciate devono rigorosamente conformarsi ai requisiti dello Schema di Certificazione CPF.
\end{enumerate}

\textbf{2.4 Riserva dei Diritti.} Tutti i diritti non espressamente concessi al Licenziatario sono riservati dal Licenziante. Il Licenziante mantiene tutti i diritti di proprietà, titolo e interesse nel e verso il framework CPF, i Materiali CPF e i Marchi di Certificazione.

\section{OBBLIGHI DEL LICENZIATARIO}

\textbf{3.1 Mantenimento dell'Accreditamento.} Il Licenziatario deve:

\begin{enumerate}[label=\alph*)]
\item Mantenere l'accreditamento corrente ISO/IEC 17065:2012 durante tutta la durata di questo Accordo;
\item Notificare immediatamente al Licenziante qualsiasi sospensione, ritiro o riduzione dell'ambito dell'accreditamento;
\item Fornire al Licenziante certificati di accreditamento annuali e rapporti di audit di sorveglianza;
\item Rimediare a qualsiasi non conformità di accreditamento entro i tempi specificati dall'ente di accreditamento.
\end{enumerate}

\textbf{3.2 Requisiti di Competenza.} Il Licenziatario deve:

\begin{enumerate}[label=\alph*)]
\item Impiegare o contrattare un minimo di due (2) Auditor CPF certificati;
\item Mantenere personale con competenza dimostrata sia in cybersecurity che in psicologia;
\item Fornire valutazioni annuali della competenza per tutto il Personale del Licenziatario coinvolto nelle attività di certificazione;
\item Assicurare che tutto il personale completi la formazione specifica CPF richiesta (minimo 40 ore inizialmente);
\item Mantenere registri delle qualifiche del personale, formazione e valutazioni della competenza.
\end{enumerate}

\textbf{3.3 Requisiti Operativi.} Il Licenziatario deve:

\begin{enumerate}[label=\alph*)]
\item Operare lo Schema di Certificazione CPF in stretta conformità con il documento dello Schema di Certificazione CPF e tutti gli aggiornamenti emessi dal Licenziante;
\item Implementare procedure sicure di amministrazione degli esami garantendo integrità e riservatezza;
\item Mantenere database crittografati per le informazioni dei candidati con controlli di accesso appropriati;
\item Stabilire e mantenere misure di protezione della privacy conformi ai requisiti di privacy differenziale ($\varepsilon \leq 0.1$);
\item Implementare processi di gestione della qualità garantendo coerenza e affidabilità delle decisioni di certificazione;
\item Mantenere un'assicurazione di responsabilità professionale con copertura minima di USD \$5.000.000 per evento;
\item Mantenere un'assicurazione di responsabilità cyber con copertura minima di USD \$2.000.000.
\end{enumerate}

\textbf{3.4 Requisiti di Reportistica.} Il Licenziatario deve fornire al Licenziante:

\begin{enumerate}[label=\alph*)]
\item Report trimestrali delle attività di certificazione inclusi:
\begin{itemize}
\item Numero di domande ricevute per tipo di certificazione;
\item Numero di certificazioni rilasciate per tipo di certificazione;
\item Numero di certificazioni rinnovate;
\item Numero di certificazioni sospese o revocate;
\item Statistiche di superamento/fallimento degli esami;
\item Riepilogo di appelli e reclami;
\end{itemize}
\item Notifica immediata (entro 5 giorni lavorativi) di:
\begin{itemize}
\item Reclami etici che coinvolgono individui o organizzazioni certificate;
\item Violazioni della privacy o riservatezza;
\item Incidenti significativi di sicurezza degli esami;
\item Azioni legali che coinvolgono certificazioni CPF;
\end{itemize}
\item Sottomissione annuale di:
\begin{itemize}
\item Certificato di accreditamento aggiornato;
\item Certificati assicurativi;
\item Dichiarazioni finanziarie auditate;
\item Revisione gestionale delle prestazioni del programma di certificazione.
\end{itemize}
\end{enumerate}

\textbf{3.5 Assicurazione Qualità.} Il Licenziatario deve:

\begin{enumerate}[label=\alph*)]
\item Partecipare a workshop di calibrazione inter-CB semestrali organizzati dal Licenziante;
\item Sottomettersi ad audit annuali da parte del Licenziante delle attività di certificazione e conformità con questo Accordo;
\item Implementare azioni correttive per non conformità identificate durante gli audit del Licenziante entro i tempi specificati;
\item Mantenere registri di tutte le decisioni di certificazione, appelli e reclami per un minimo di sette (7) anni;
\item Fornire al Licenziante l'accesso ai file e registri di certificazione su richiesta ragionevole per scopi di monitoraggio della qualità.
\end{enumerate}

\textbf{3.6 Utilizzo dei Marchi.} Il Licenziatario deve:

\begin{enumerate}[label=\alph*)]
\item Utilizzare i Marchi di Certificazione solo in conformità con le linee guida di utilizzo dei marchi del Licenziante (Allegato B);
\item Non modificare, alterare o creare opere derivate dei Marchi di Certificazione;
\item Includere avvisi di marchio appropriati (® o TM) con tutti gli utilizzi dei Marchi di Certificazione;
\item Cessare immediatamente l'uso dei Marchi di Certificazione alla risoluzione o scadenza di questo Accordo;
\item Non registrare o tentare di registrare marchi confusivamente simili ai Marchi di Certificazione.
\end{enumerate}

\section{TERMINI FINANZIARI}

\textbf{4.1 Tariffa Iniziale di Licenza.} Il Licenziatario deve pagare al Licenziante una tariffa iniziale di licenza di USD \$\_\_\_\_\_\_\_\_\_\_ all'esecuzione di questo Accordo. Questa tariffa non è rimborsabile.

\textbf{4.2 Tariffa Annuale di Licenza.} Il Licenziatario deve pagare al Licenziante una tariffa annuale di licenza di USD \$\_\_\_\_\_\_\_\_\_\_ per ogni anno della durata dell'Accordo. Le tariffe annuali sono dovute nell'anniversario della Data di Efficacia e non sono rimborsabili.

\textbf{4.3 Tariffe di Royalty.} Il Licenziatario deve pagare al Licenziante tariffe di royalty calcolate come segue:

\begin{enumerate}[label=\alph*)]
\item Certificazioni Professionali (Valutatore, Praticante, Auditor): 15% della tariffa di certificazione addebitata al richiedente;
\item Certificazioni Organizzative (Livelli 1-4): 10% della tariffa di certificazione addebitata all'organizzazione;
\item Certificazioni di Fornitore di Servizi Autorizzato: 12% della tariffa di certificazione addebitata alla società;
\item Tariffe di ricertificazione: Stessa percentuale della certificazione iniziale;
\item Tariffe di ripetizione degli esami: 15% della tariffa d'esame.
\end{enumerate}

\textbf{4.4 Royalty Minima Annuale.} Il Licenziatario deve pagare una royalty minima annuale di USD \$10.000, indipendentemente dal volume effettivo di certificazioni. Se le royalty effettive per qualsiasi periodo annuale scendono al di sotto del minimo, il Licenziatario deve pagare la differenza al Licenziante.

\textbf{4.5 Licenza dei Materiali d'Esame.} Il Licenziatario deve pagare una tariffa annuale di licenza dei materiali d'esame di USD \$5.000 per l'accesso alle banche di voci d'esame proprietarie e strumenti di valutazione del Licenziante.

\textbf{4.6 Termini di Pagamento.}

\begin{enumerate}[label=\alph*)]
\item Le tariffe di royalty devono essere calcolate e riportate trimestralmente;
\item Il pagamento delle royalty è dovuto entro trenta (30) giorni successivi alla fine di ogni trimestre solare;
\item Ogni pagamento deve essere accompagnato da un report dettagliato che mostra il calcolo delle royalty;
\item Tutti i pagamenti devono essere effettuati in USD tramite bonifico bancario al conto specificato dal Licenziante;
\item I pagamenti in ritardo accumuleranno interessi al tasso dell'1,5% al mese o al tasso massimo consentito dalla legge, whichever is less;
\item Il Licenziante può sospendere le operazioni del Licenziatario ai sensi di questo Accordo per pagamenti in ritardo di oltre sessanta (60) giorni.
\end{enumerate}

\textbf{4.7 Diritti di Audit.} Il Licenziante avrà il diritto, con preavviso ragionevole, di auditare i libri e registri del Licenziatario relativi alle attività di certificazione allo scopo di verificare i calcoli delle royalty. Se un audit rivela una sottoreportistica di oltre il cinque percento (5%), il Licenziatario dovrà pagare il costo dell'audit oltre a qualsiasi royalty non pagata e interessi.

\textbf{4.8 Tasse.} Tutte le tariffe e royalty sono escluse di tasse. Il Licenziatario è responsabile di tutte le tasse, dazi e addebiti simili imposti da qualsiasi autorità governativa sugli importi pagabili ai sensi di questo Accordo, ad eccezione delle tasse basate sul reddito netto del Licenziante.

\section{PROPRIETÀ INTELLETTUALE}

\textbf{5.1 Proprietà.} Il Licenziatario riconosce e accetta che il Licenziante possiede tutti i diritti, titolo e interesse nel e verso:

\begin{enumerate}[label=\alph*)]
\item Il framework e metodologia CPF;
\item Tutti i Materiali CPF;
\item Tutti i Marchi di Certificazione;
\item Tutte le voci d'esame e strumenti di valutazione;
\item Tutti gli aggiornamenti, modifiche e opere derivate dei precedenti.
\end{enumerate}

\textbf{5.2 Nessun Trasferimento di Proprietà.} Nulla in questo Accordo trasferisce diritti di proprietà al Licenziatario. Il Licenziatario acquisisce solo una licenza limitata per utilizzare la proprietà intellettuale come espressamente permesso qui.

\textbf{5.3 Miglioramenti e Feedback.} Qualsiasi miglioramento, potenziamento o modifica ai Materiali CPF sviluppato dal Licenziatario deve essere:

\begin{enumerate}[label=\alph*)]
\item Tempestivamente divulgato al Licenziante per iscritto;
\item Di proprietà esclusiva del Licenziante salvo diverso accordo scritto;
\item Concesso in licenza retroattivamente al Licenziatario per l'uso all'interno del Territorio Autorizzato senza costi aggiuntivi;
\item Reso disponibile ad altri enti di certificazione licenziati a discrezione del Licenziante.
\end{enumerate}

\textbf{5.4 Protezione dei Marchi.} Il Licenziatario deve:

\begin{enumerate}[label=\alph*)]
\item Notificare immediatamente al Licenziante qualsiasi uso non autorizzato dei Marchi di Certificazione;
\item Cooperare con il Licenziante in qualsiasi azione di applicazione;
\item Non contestare la validità dei marchi o la proprietà del Licenziante;
\item Non utilizzare i Marchi di Certificazione in modo che ne diminuisca il valore o la reputazione.
\end{enumerate}

\textbf{5.5 Protezione Contro Violazioni.} Se il Licenziatario viene a conoscenza di qualsiasi violazione o appropriazione indebita della proprietà intellettuale CPF, il Licenziatario deve immediatamente notificare il Licenziante e fornire tutte le informazioni disponibili. Il Licenziante avrà il diritto esclusivo di intraprendere azioni di applicazione.

\section{RISERVATEZZA}

\textbf{6.1 Informazioni Confidenziali.} Ogni Parte riconosce che potrebbe ricevere Informazioni Confidenziali dall'altra Parte. Le Informazioni Confidenziali includono:

\begin{enumerate}[label=\alph*)]
\item Voci d'esame e metodologie di valutazione;
\item Informazioni commerciali e finanziarie;
\item Dati tecnici e processi proprietari;
\item Informazioni sui candidati e individui certificati;
\item Dati di valutazione organizzativa;
\item Piani strategici e analisi di mercato;
\item Qualsiasi informazione contrassegnata come "Confidenziale" o ragionevolmente intesa come tale.
\end{enumerate}

\textbf{6.2 Obblighi.} Ogni Parte deve:

\begin{enumerate}[label=\alph*)]
\item Mantenere le Informazioni Confidenziali in stretta riservatezza;
\item Utilizzare le Informazioni Confidenziali solo per scopi autorizzati ai sensi di questo Accordo;
\item Limitare l'accesso alle Informazioni Confidenziali al personale con necessità di conoscenza;
\item Proteggere le Informazioni Confidenziali utilizzando almeno lo stesso grado di cura utilizzato per le proprie informazioni confidenziali, ma non meno di una ragionevole cura;
\item Non divulgare le Informazioni Confidenziali a terzi senza previo consenso scritto;
\item Assicurare che tutto il personale con accesso alle Informazioni Confidenziali sia vincolato da obblighi di riservatezza almeno tanto stringenti quanto quelli in questo Accordo.
\end{enumerate}

\textbf{6.3 Eccezioni.} Gli obblighi di riservatezza non si applicano a informazioni che:

\begin{enumerate}[label=\alph*)]
\item Erano pubblicamente disponibili al momento della divulgazione o diventano pubblicamente disponibili senza violazione di questo Accordo;
\item Erano legittimamente in possesso della Parte ricevente prima della divulgazione;
\item Sono sviluppate indipendentemente dalla Parte ricevente senza l'uso di Informazioni Confidenziali;
\item Sono richieste per essere divulgate per legge, regolamento o ordine del tribunale, purché la Parte divulgante dia preavviso tempestivo per consentire all'altra Parte di cercare misure protettive.
\end{enumerate}

\textbf{6.4 Protezione Dati e Privacy.} Il Licenziatario deve:

\begin{enumerate}[label=\alph*)]
\item Conformarsi a tutte le leggi applicabili di protezione dati e privacy, incluse ma non limitate a GDPR (UE), CCPA (California) e leggi equivalenti nel Territorio Autorizzato;
\item Implementare misure tecniche e organizzative per proteggere i dati personali dei candidati e individui certificati;
\item Non elaborare dati personali per scopi diversi dalle attività di certificazione;
\item Rispondere tempestivamente alle richieste degli interessati (accesso, correzione, cancellazione);
\item Notificare al Licenziante entro ventiquattro (24) ore qualsiasi violazione di dati che interessi dati personali relativi a CPF;
\item Cooperare con il Licenziante nella gestione delle conseguenze e notifiche di violazione dati.
\end{enumerate}

\textbf{6.5 Sopravvivenza.} Gli obblighi di riservatezza sopravviveranno alla risoluzione di questo Accordo per un periodo di cinque (5) anni, ad eccezione delle Informazioni Confidenziali che costituiscono un segreto commerciale, che dovranno essere protette indefinitamente.

\section{ASSICURAZIONE QUALITÀ E AUDIT}

\textbf{7.1 Audit Annuali del Licenziante.} Il Licenziante condurrà un audit annuale sul posto o virtuale delle operazioni di certificazione del Licenziatario. L'ambito dell'audit include:

\begin{enumerate}[label=\alph*)]
\item Efficacia del sistema di gestione della qualità;
\item Conformità del processo di certificazione con lo Schema di Certificazione CPF;
\item Sicurezza e integrità dell'amministrazione degli esami;
\item Mantenimento della competenza del personale;
\item Controlli di privacy e riservatezza;
\item Gestione di reclami e appelli;
\item Accuratezza e tempestività del registro;
\item Accuratezza della reportistica finanziaria.
\end{enumerate}

\textbf{7.2 Processo di Audit.} Gli audit saranno condotti come segue:

\begin{enumerate}[label=\alph*)]
\item Il Licenziante fornirà almeno trenta (30) giorni di preavviso;
\item Il Licenziatario fornirà pieno accesso a strutture, registri e personale;
\item I risultati dell'audit saranno documentati per iscritto e forniti al Licenziatario entro quindici (15) giorni;
\item Il Licenziatario risponderà con piani di azione correttiva entro trenta (30) giorni;
\item Il Licenziante verificherà l'implementazione delle azioni correttive.
\end{enumerate}

\textbf{7.3 Costi di Audit.} Il Licenziante sosterrà i costi degli audit annuali di routine. Se un audit rivela non conformità principali o fallimenti sistemici, il Licenziante può richiedere un audit di follow-up a spese del Licenziatario.

\textbf{7.4 Calibrazione Inter-CB.} Il Licenziatario deve partecipare a workshop di calibrazione inter-ente di certificazione semestrali organizzati dal Licenziante, inclusi:

\begin{enumerate}[label=\alph*)]
\item Esercizi di coerenza di punteggio;
\item Discussioni decisionali per casi complessi;
\item Condivisione di best practice;
\item Aggiornamenti e formazione dello schema;
\item Benchmarking delle metriche di performance.
\end{enumerate}

\textbf{7.5 Standard di Performance.} Il Licenziatario deve mantenere:

\begin{enumerate}[label=\alph*)]
\item Tassi di superamento esami entro il 15% della media dello schema;
\item Tassi di appello inferiori al 5% delle decisioni di certificazione;
\item Tassi di conferma reclami inferiori al 10% del totale dei reclami;
\item Tempo medio del ciclo di certificazione non superiore a 90 giorni dalla domanda completa;
\item Punteggi di soddisfazione del cliente superiori a 4.0 su 5.0.
\end{enumerate}

Se il Licenziatario non riesce a raggiungere gli standard di performance per due trimestri consecutivi, il Licenziante può richiedere un piano di miglioramento della performance.

\section{RECLAMI E APPELLI}

\textbf{8.1 Gestione dei Reclami.} Il Licenziatario deve stabilire e mantenere procedure per la gestione dei reclami relativi a:

\begin{enumerate}[label=\alph*)]
\item Equità e coerenza del processo di certificazione;
\item Qualità e amministrazione degli esami;
\item Condotta di individui o organizzazioni certificate;
\item Operazioni e personale del Licenziatario.
\end{enumerate}

\textbf{8.2 Indagini Etiche.} Il Licenziatario deve:

\begin{enumerate}[label=\alph*)]
\item Indagare tutti i reclami etici contro individui o organizzazioni certificate;
\item Mantenere l'indipendenza degli investigatori dalle decisioni di certificazione;
\item Completare le indagini entro sessanta (60) giorni a meno che la complessità non richieda un'estensione;
\item Notificare al Licenziante entro cinque (5) giorni lavorativi la ricezione di reclami etici significativi;
\item Fornire al Licenziante copie dei rapporti di indagine e decisioni.
\end{enumerate}

\textbf{8.3 Processo di Appelli.} Il Licenziatario deve fornire un processo di appelli che consenta ai candidati e alle parti certificate di appellare:

\begin{enumerate}[label=\alph*)]
\item Dinieghi di certificazione;
\item Azioni disciplinari;
\item Dinieghi di ricertificazione;
\item Decisioni di sospensione o revoca.
\end{enumerate}

\textbf{8.4 Reportistica al Licenziante.} Il Licenziatario deve riportare al Licenziante:

\begin{enumerate}[label=\alph*)]
\item Riepilogo trimestrale di tutti i reclami e appelli;
\item Notifica immediata di problemi significativi che potrebbero impattare la reputazione del marchio CPF;
\item Analisi annuale delle tendenze dei reclami e azioni correttive.
\end{enumerate}

\section{DURATA E RISOLUZIONE}

\textbf{9.1 Durata Iniziale.} Questo Accordo inizierà alla Data di Efficacia e continuerà per una durata iniziale di cinque (5) anni a meno che non sia risolto prima in conformità con questa Sezione.

\textbf{9.2 Rinnovo.} Questo Accordo può essere rinnovato per periodi consecutivi di cinque (5) anni a condizione che:

\begin{enumerate}[label=\alph*)]
\item Ci sia un accordo scritto reciproco almeno cento ottanta (180) giorni prima della scadenza;
\item Ci sia una revisione della performance soddisfacente da parte del Licenziante;
\item Si rinegozino i termini finanziari se applicabile;
\item Si continui a conformarsi a tutti i termini dell'Accordo.
\end{enumerate}

\textbf{9.3 Risoluzione per Giusta Causa da parte del Licenziante.} Il Licenziante può risolvere questo Accordo immediatamente con notifica scritta se:

\begin{enumerate}[label=\alph*)]
\item Il Licenziatario perde l'accreditamento ISO/IEC 17065;
\item Il Licenziatario commette una violazione materiale e non rimedia entro sessanta (60) giorni dalla notifica scritta;
\item Il Licenziatario diventa insolvente, fallito o soggetto a ricevimento;
\item Il Licenziatario si impegna in frode, falsa rappresentazione o condotta non etica;
\item Il Licenziatario utilizza impropriamente la proprietà intellettuale CPF o i Marchi di Certificazione;
\item Il Licenziatario non paga gli importi dovuti e tale mancato pagamento continua per sessanta (60) giorni dopo la notifica;
\item Il Licenziatario subisce una violazione di dati importante e non risponde adeguatamente;
\item Il Licenziatario riceve risultanze importanti dall'ente di accreditamento che impattano la capacità di certificazione.
\end{enumerate}

\textbf{9.4 Risoluzione per Giusta Causa da parte del Licenziatario.} Il Licenziatario può risolvere questo Accordo con novanta (90) giorni di notifica scritta se:

\begin{enumerate}[label=\alph*)]
\item Il Licenziante viola materialmente questo Accordo e non rimedia entro sessanta (60) giorni;
\item Il Licenziante cessa di operare lo schema di certificazione CPF;
\item Il Licenziante diventa insolvente o fallito.
\end{enumerate}

\textbf{9.5 Risoluzione per Comodità.} Entrambe le Parti possono risolvere questo Accordo senza giusta causa con cento ottanta (180) giorni di notifica scritta. In tal caso, il Licenziatario pagherà una tariffa di risoluzione pari al 50% della tariffa annuale di licenza allora attuale.

\textbf{9.6 Sospensione.} Il Licenziante può sospendere le operazioni del Licenziatario ai sensi di questo Accordo fino a cento ottanta (180) giorni se:

\begin{enumerate}[label=\alph*)]
\item Viene identificata una non conformità importante nell'audit annuale;
\item Si verifica un aumento significativo di appelli o reclami;
\item Il Licenziatario scende al di sotto dei requisiti minimi di royalty;
\item La copertura assicurativa decade;
\item Vengono identificate carenze nella competenza del personale.
\end{enumerate}

Durante la sospensione, il Licenziatario non accetterà nuove domande ma completerà l'elaborazione delle domande già presentate.

\textbf{9.7 Effetto della Risoluzione.} Alla risoluzione o scadenza di questo Accordo:

\begin{enumerate}[label=\alph*)]
\item Il Licenziatario cesserà immediatamente tutte le attività di certificazione;
\item Il Licenziatario cesserà immediatamente l'uso di tutti i Marchi di Certificazione;
\item Tutte le certificazioni precedentemente rilasciate dal Licenziatario rimarranno valide fino alla loro normale scadenza;
\item Il Licenziatario trasferirà tutti i registri di certificazione al Licenziante o al successore designato del Licenziante entro trenta (30) giorni;
\item Gli individui e le organizzazioni certificate possono trasferirsi a un altro ente di certificazione licenziato;
\item Il Licenziatario restituirà o distruggerà tutti i Materiali CPF entro trenta (30) giorni;
\item Il Licenziatario pagherà tutte le tariffe e royalty arretrate entro trenta (30) giorni;
\item Le Sezioni 5 (Proprietà Intellettuale), 6 (Riservatezza), 10 (Limitazione di Responsabilità), 11 (Indennizzo) e 12 (Disposizioni Generali) sopravviveranno alla risoluzione.
\end{enumerate}

\textbf{9.8 Assistenza nella Transizione.} In caso di risoluzione, il Licenziatario deve:

\begin{enumerate}[label=\alph*)]
\item Cooperare con il Licenziante nella transizione delle parti certificate ad altri enti di certificazione;
\item Fornire agli individui e organizzazioni certificate la documentazione del loro stato di certificazione;
\item Rispondere alle richieste di verifica per certificazioni precedentemente rilasciate;
\item Non denigrare il programma CPF o interferire con le operazioni continue da parte di altri enti di certificazione.
\end{enumerate}

\section{LIMITAZIONE DI RESPONSABILITÀ}

\textbf{10.1 Esclusione delle Garanzie.} ECCETTO QUANTO ESPRESSAMENTE STATO IN QUESTO ACCORDO, IL LICENZIANTE NON FORNISCE ALCUNA GARANZIA, ESPRESSA O IMPLICITA, INCLUSE MA NON LIMITATE A GARANZIE DI COMMERCIABILITÀ, IDONEITÀ PER UNO SCOPO PARTICOLARE, O NON VIOLAZIONE. IL LICENZIANTE NON GARANTISCE CHE LO SCHEMA DI CERTIFICAZIONE CPF SIA SENZA ERRORI O ININTERROTTO.

\textbf{10.2 Limitazione dei Danni.} IN NESSUN CASO UNA PARTE SARÀ RESPONSABILE VERSO L'ALTRA PER DANNI INDIRETTI, INCIDENTALI, CONSEQUENZIALI, SPECIALI, ESEMPLARI O PUNITIVI, INCLUSI MA NON LIMITATI A PROFITTI PERSI, ENTRATE PERSE O OPPORTUNITÀ COMMERCIALI PERSE, ANCHE SE AVVISATA DELLA POSSIBILITÀ DI TALI DANNI.

\textbf{10.3 Limite alla Responsabilità.} ECCETTO PER VIOLAZIONI DI RISERVATEZZA, VIOLAZIONE DELLA PROPRIETÀ INTELLETTUALE, OBBLIGHI DI INDENNIZZO, LA RESPONSABILITÀ TOTALE CUMULATIVA DI CIASCUNA PARTE AI SENSI DI QUESTO ACCORDO NON SUPERERÀ IL TOTALE DELLE TARIFFE PAGATE DAL LICENZIATARIO AL LICENZIANTE NEI DODICI (12) MESI PRECEDENTI LA RICHIESTA.

\textbf{10.4 Eccezioni.} Le limitazioni in questa Sezione non si applicano a:

\begin{enumerate}[label=\alph*)]
\item Violazioni degli obblighi di riservatezza;
\item Violazione della proprietà intellettuale;
\item Frode o condotta intenzionale;
\item Negligenza grave;
\item Obblighi di indennizzo ai sensi della Sezione 11.
\end{enumerate}

\section{INDENNIZZO}

\textbf{11.1 Indennizzo da parte del Licenziatario.} Il Licenziatario indennizzerà, difenderà e manterrà indenne il Licenziante, i suoi funzionari, direttori, dipendenti e agenti da e contro qualsiasi richiesta, danno, responsabilità, costo e spesa (incluse ragionevoli spese legali) derivanti da:

\begin{enumerate}[label=\alph*)]
\item Negligenza o condotta intenzionale del Licenziatario;
\item Violazione di questo Accordo da parte del Licenziatario;
\item Violazione delle leggi o regolamenti applicabili da parte del Licenziatario;
\item Richieste da parte di individui o organizzazioni certificate relative alle decisioni o processi di certificazione del Licenziatario;
\item Violazioni della privacy o dati causate dal fallimento del Licenziatario nell'implementare adeguate misure di sicurezza;
\item Uso non autorizzato da parte del Licenziatario della proprietà intellettuale CPF o dei Marchi di Certificazione.
\end{enumerate}

\textbf{11.2 Indennizzo da parte del Licenziante.} Il Licenziante indennizzerà, difenderà e manterrà indenne il Licenziatario, i suoi funzionari, direttori, dipendenti e agenti da e contro qualsiasi richiesta, danno, responsabilità, costo e spesa (incluse ragionevoli spese legali) derivanti da:

\begin{enumerate}[label=\alph*)]
\item Richieste che i Materiali CPF violino i diritti di proprietà intellettuale di terzi;
\item Violazione di questo Accordo da parte del Licenziante;
\item Negligenza o condotta intenzionale del Licenziante.
\end{enumerate}

\textbf{11.3 Processo di Indennizzo.} Una Parte che cerca indennizzo (Indennizzato) deve:

\begin{enumerate}[label=\alph*)]
\item Notificare tempestivamente alla Parte indennizzante (Indennizzatore) qualsiasi richiesta;
\item Fornire ragionevole cooperazione nella difesa;
\item Consentire all'Indennizzatore di controllare la difesa e il regolamento;
\item Non regolare o compromettere la richiesta senza il consenso dell'Indennizzatore.
\end{enumerate}

\textbf{11.4 Assicurazione.} Ogni Parte manterrà una copertura assicurativa adeguata a supportare i suoi obblighi di indennizzo ai sensi di questo Accordo.

\section{DISPOSIZIONI GENERALI}

\textbf{12.1 Legge Applicabile.} Questo Accordo sarà regolato e interpretato in conformità con le leggi di [Giurisdizione], senza riguardo ai suoi principi di conflitto di leggi.

\textbf{12.2 Risoluzione delle Controversie.}

\begin{enumerate}[label=\alph*)]
\item \textit{Negoziazione:} Qualsiasi controversia sarà prima affrontata attraverso negoziazioni di buona fede tra dirigenti senior di entrambe le Parti per un periodo di trenta (30) giorni.
\item \textit{Mediazione:} Se la negoziazione fallisce, le Parti tenteranno di risolvere la controversia attraverso mediazione amministrata da [Servizio di Mediazione] in conformità con le sue regole.
\item \textit{Arbitrato:} Se la mediazione fallisce, la controversia sarà risolta attraverso arbitrato vincolante amministrato da [Servizio di Arbitrato] in conformità con le sue regole. L'arbitrato sarà condotto in [Città, Paese] in lingua inglese. La decisione dell'arbitro sarà finale e vincolante.
\item \textit{Eccezioni:} Entrambe le Parti possono cercare un provvedimento ingiuntivo in tribunale per violazioni dei diritti di proprietà intellettuale o obblighi di riservatezza senza prima perseguire mediazione o arbitrato.
\end{enumerate}

\textbf{12.3 Notifiche.} Tutte le notifiche ai sensi di questo Accordo devono essere per iscritto e consegnate tramite:

\begin{enumerate}[label=\alph*)]
\item Consegna personale;
\item Corriere internazionale riconosciuto (es. DHL, FedEx);
\item Email con conferma di ricezione.
\end{enumerate}

Le notifiche al Licenziante devono essere inviate a:
\begin{quote}
CPF3\\
Attenzione: Dipartimento Legale\\
\texttt{[Indirizzo]}\\ 
Email: legal@cpf3.org
\end{quote}

Le notifiche al Licenziatario devono essere inviate all'indirizzo sopra indicato o come aggiornato per iscritto.

\textbf{12.4 Appaltatori Indipendenti.} Le Parti sono appaltatori indipendenti. Nulla in questo Accordo crea una partnership, joint venture, agenzia o rapporto di lavoro.

\textbf{12.5 Cessione.} Nessuna delle Parti può cedere questo Accordo senza il previo consenso scritto dell'altra Parte, ad eccezione che il Licenziante può cedere a un'affiliata o in connessione con una fusione, acquisizione o vendita di sostanzialmente tutti i beni.

\textbf{12.6 Forza Maggiore.} Nessuna Parte sarà responsabile per il fallimento nell'esecuzione dovuto a cause oltre il suo ragionevole controllo, inclusi atti di Dio, guerra, terrorismo, epidemia, pandemia, restrizioni governative o disastri naturali. Se la forza maggiore continua per più di novanta (90) giorni, entrambe le Parti possono risolvere questo Accordo.

\textbf{12.7 Modifica.} Questo Accordo può essere modificato solo con accordo scritto firmato da entrambe le Parti.

\textbf{12.8 Rinuncia.} La rinuncia a qualsiasi disposizione deve essere per iscritto. Il fallimento nell'applicare qualsiasi disposizione non rinuncia al diritto di applicarla in seguito.

\textbf{12.9 Separabilità.} Se qualsiasi disposizione è trovata invalida o inapplicabile, le disposizioni rimanenti continueranno in pieno vigore, e la disposizione invalida sarà riformata nella massima misura possibile per raggiungere l'intento delle Parti.

\textbf{12.10 Accordo Integrale.} Questo Accordo, inclusi tutti gli Allegati, costituisce l'intero accordo tra le Parti e sostituisce tutti gli accordi precedenti, intese e comunicazioni riguardanti l'oggetto.

\textbf{12.11 Controparti.} Questo Accordo può essere eseguito in controparti, ciascuna delle quali sarà considerata un originale e tutte insieme costituiranno uno strumento unico. Le firme elettroniche avranno lo stesso effetto delle firme originali.

\textbf{12.12 Sopravvivenza.} Le disposizioni che per loro natura dovrebbero sopravvivere alla risoluzione sopravviveranno, incluse le Sezioni 5, 6, 10, 11 e 12.

\section{ALLEGATI}

I seguenti Allegati sono allegati e formano parte di questo Accordo:

\begin{itemize}
\item \textbf{Allegato A:} Territorio Autorizzato
\item \textbf{Allegato B:} Marchi di Certificazione e Linee Guida di Utilizzo dei Marchi
\item \textbf{Allegato C:} Schema Tariffario
\item \textbf{Allegato D:} Modelli di Reportistica
\item \textbf{Allegato E:} Metriche e Standard di Performance
\end{itemize}

\vspace{2em}

\section*{FIRME}

IN TESTIMONIANZA DI CUI SOPRA, le Parti hanno eseguito questo Accordo alla Data di Efficacia.

\vspace{2em}

\textbf{LICENZIANTE: CPF3}

\vspace{1.5em}

Da: \underline{\hspace{6cm}} Data: \underline{\hspace{3cm}}

Nome: \underline{\hspace{6cm}}

Titolo: \underline{\hspace{6cm}}

\vspace{2em}

\textbf{LICENZIATARIO: [NOME DELL'ENTE DI CERTIFICAZIONE]}

\vspace{1.5em}

Da: \underline{\hspace{6cm}} Data: \underline{\hspace{3cm}}

Nome: \underline{\hspace{6cm}}

Titolo: \underline{\hspace{6cm}}

\newpage

\section*{ALLEGATO A: TERRITORIO AUTORIZZATO}

\textbf{Ambito Geografico:}

Il Licenziatario è autorizzato a operare lo Schema di Certificazione CPF nel seguente territorio:

\begin{itemize}
\item[] Paese/Paesi: \underline{\hspace{10cm}}
\item[] \underline{\hspace{12cm}}
\item[] Regioni/Stati (se applicabile): \underline{\hspace{8cm}}
\item[] \underline{\hspace{12cm}}
\item[] Esclusioni (se presenti): \underline{\hspace{10cm}}
\item[] \underline{\hspace{12cm}}
\end{itemize}

\textbf{Modifiche del Territorio:}

Qualsiasi modifica al Territorio Autorizzato richiede un emendamento scritto firmato da entrambe le Parti.

\vspace{2em}

\textbf{Approvato da:}

\vspace{1em}

CPF3: \underline{\hspace{5cm}} Data: \underline{\hspace{3cm}}

Licenziatario: \underline{\hspace{5cm}} Data: \underline{\hspace{3cm}}

\newpage

\section*{ALLEGATO B: MARCHI DI CERTIFICAZIONE E UTILIZZO DEI MARCHI}

\textbf{Marchi di Certificazione Autorizzati:}

Il Licenziatario è autorizzato a utilizzare i seguenti marchi in connessione con le attività di certificazione CPF:

\begin{enumerate}
\item CPF® (marchio registrato)
\item Cybersecurity Psychology Framework™
\item Marchio di certificazione CPF Certified Assessor®
\item Marchio di certificazione CPF Certified Practitioner®
\item Marchio di certificazione CPF Certified Auditor®
\item Designazione CPF Authorized Certification Body™
\item [Marchi aggiuntivi come approvati]
\end{enumerate}

\textbf{Linee Guida di Utilizzo:}

\textit{Requisiti Generali:}
\begin{itemize}
\item Utilizzare sempre i simboli ® o ™ come indicato
\item Non modificare, abbreviare o alterare i marchi
\item Utilizzare i marchi solo in connessione con le attività di certificazione autorizzate
\item Mantenere l'integrità visiva come specificato nelle Linee Guida del Marchio CPF3
\item Includere l'attribuzione appropriata: "CPF è un marchio registrato di CPF3"
\end{itemize}

\textit{Usi Approvati:}
\begin{itemize}
\item Certificati rilasciati a individui e organizzazioni certificate
\item Materiali promozionali che promuovono servizi di certificazione
\item Contenuti del sito web che descrivono programmi di certificazione
\item Corrispondenza professionale relativa alle certificazioni
\item Presentazioni a conferenze sulla certificazione CPF
\end{itemize}

\textit{Usi Proibiti:}
\begin{itemize}
\item Utilizzo in connessione con servizi o prodotti non CPF
\item Modifiche o variazioni stilistiche
\item Utilizzo in modo che suggerisca l'approvazione di altri servizi del Licenziatario
\item Utilizzo dopo la risoluzione dell'Accordo
\item Sublicenza o trasferimento a terzi
\end{itemize}

\textbf{Controllo Qualità:}

Il Licenziatario sottometterà campioni di tutti i materiali che utilizzano i Marchi di Certificazione al Licenziante per l'approvazione prima del primo utilizzo. Il Licenziante approverà o richiederà modifiche entro quindici (15) giorni lavorativi.

\newpage

\section*{ALLEGATO C: SCHEMA TARIFFARIO}

\textbf{Tariffa Iniziale di Licenza:} USD \$\_\_\_\_\_\_\_\_\_\_

Pagamento dovuto: All'esecuzione dell'Accordo

\vspace{1em}

\textbf{Tariffa Annuale di Licenza:} USD \$\_\_\_\_\_\_\_\_\_\_ all'anno

Pagamento dovuto: Annualmente nell'anniversario della Data di Efficacia

\vspace{1em}

\textbf{Tariffa di Licenza dei Materiali d'Esame:} USD \$5.000 all'anno

Pagamento dovuto: Annualmente nell'anniversario della Data di Efficacia

\vspace{1em}

\textbf{Tassi di Royalty:}

\begin{tabular}{|l|c|}
\hline
\textbf{Tipo di Certificazione} & \textbf{Tasso di Royalty} \\
\hline
CPF Assessor (iniziale \& rinnovo) & 15\% \\
CPF Practitioner (iniziale \& rinnovo) & 15\% \\
CPF Auditor (iniziale \& rinnovo) & 15\% \\
Organizzativo Livello 1-4 (iniziale \& rinnovo) & 10\% \\
Fornitore di Servizi Autorizzato (iniziale \& rinnovo) & 12\% \\
Ripetizione esami & 15\% \\
\hline
\end{tabular}

\vspace{1em}

\textbf{Royalty Minima Annuale:} USD \$10.000

Se le royalty effettive scendono al di sotto del minimo, il Licenziatario paga la differenza.

\vspace{1em}

\textbf{Programma di Pagamento:}

\begin{itemize}
\item Royalty calcolate e riportate trimestralmente
\item Pagamento dovuto entro 30 giorni dalla fine del trimestre
\item Interesse per pagamento in ritardo: 1,5% al mese
\end{itemize}

\vspace{1em}

\textbf{Aggiustamenti delle Tariffe:}

Il Licenziante può adeguare le tariffe annualmente con novanta (90) giorni di notifica scritta. Gli aggiustamenti non supereranno il maggiore tra il 5% o l'aumento dell'Indice dei Prezzi al Consumo (CPI) USA per l'anno precedente.

\newpage

\section*{ALLEGATO D: MODELLI DI REPORTISTICA}

\textbf{Report Trimestrale delle Attività di Certificazione}

Scadenza: Entro 30 giorni dalla fine del trimestre

\vspace{1em}

\textit{Statistiche di Certificazione:}

\begin{tabular}{|l|c|c|c|c|}
\hline
\textbf{Tipo di Certificazione} & \textbf{Domande} & \textbf{Rilasciate} & \textbf{Negate} & \textbf{In Sospeso} \\
\hline
CPF Assessor & & & & \\
CPF Practitioner & & & & \\
CPF Auditor & & & & \\
Organizzativo Livello 1 & & & & \\
Organizzativo Livello 2 & & & & \\
Organizzativo Livello 3 & & & & \\
Organizzativo Livello 4 & & & & \\
Fornitore di Servizi Autorizzato & & & & \\
\hline
\end{tabular}

\vspace{1em}

\textit{Statistiche degli Esami:}

\begin{tabular}{|l|c|c|c|}
\hline
\textbf{Tipo di Esame} & \textbf{Somministrati} & \textbf{Superati} & \textbf{Falliti} \\
\hline
Assessor Scritto & & & \\
Assessor Pratico & & & \\
Practitioner Scritto & & & \\
Auditor Scritto & & & \\
Auditor Pratico & & & \\
\hline
\end{tabular}

\vspace{1em}

\textit{Riepilogo Finanziario:}

\begin{itemize}
\item Tariffe di Certificazione Riscosse Totali: \$\_\_\_\_\_\_\_\_\_\_
\item Royalty Dovute al Licenziante: \$\_\_\_\_\_\_\_\_\_\_
\item Pagamento Allegato: \$\_\_\_\_\_\_\_\_\_\_
\end{itemize}

\vspace{1em}

\textit{Metriche di Qualità:}

\begin{itemize}
\item Reclami Ricevuti: \_\_\_\_\_\_
\item Appelli Presentati: \_\_\_\_\_\_
\item Certificazioni Sospese: \_\_\_\_\_\_
\item Certificazioni Revocate: \_\_\_\_\_\_
\item Tempo Medio del Ciclo di Certificazione: \_\_\_\_\_ giorni
\end{itemize}

\vspace{1em}

Preparato da: \underline{\hspace{5cm}} Data: \underline{\hspace{3cm}}

\newpage

\section*{ALLEGATO E: METRICHE E STANDARD DI PERFORMANCE}

\textbf{Tassi di Superamento degli Esami:}

\begin{itemize}
\item Intervallo Target: Entro il 15% della media dello schema
\item Media dello Schema (aggiornata annualmente dal Licenziante)
\item Se al di fuori dell'intervallo per 2 trimestri consecutivi: Richiesta revisione della performance
\end{itemize}

\textbf{Appelli e Reclami:}

\begin{itemize}
\item Target Tasso di Appello: Inferiore al 5% delle decisioni di certificazione
\item Target Tasso di Conferma Reclami: Inferiore al 10% dei reclami
\item Se i target superati per 2 trimestri consecutivi: Richiesta analisi delle cause radici
\end{itemize}

\textbf{Tempo del Ciclo di Certificazione:}

\begin{itemize}
\item Target: Completare il processo di certificazione entro 90 giorni dalla domanda completa
\item Misurazione: Dalla domanda completa alla decisione finale
\item Se superato per più del 20% delle domande: Richiesto miglioramento del processo
\end{itemize}

\textbf{Soddisfazione del Cliente:}

\begin{itemize}
\item Target: Punteggio medio di soddisfazione superiore a 4.0 su 5.0
\item Metodo di Indagine: Indagine di soddisfazione post-certificazione
\item Frequenza: Tutte le certificazioni indagate
\item Se al di sotto del target per 2 trimestri consecutivi: Richiesta azione correttiva
\end{itemize}

\textbf{Accuratezza del Registro:}

\begin{itemize}
\item Target: Accuratezza del 100% nel registro di certificazione
\item Tempestività di Aggiornamento: Entro 5 giorni lavorativi dalla decisione di certificazione
\item Verifica mensile da parte del Licenziante
\end{itemize}

\textbf{Conformità di Sorveglianza:}

\begin{itemize}
\item Target: Completamento al 100% degli audit di sorveglianza programmati
\item Se gli audit di sorveglianza ritardati: Richiesta notifica immediata al Licenziante
\end{itemize}

\vspace{2em}

\begin{center}
\textit{Fine dell'Accordo di Licenza dello Schema}
\end{center}

\end{document}