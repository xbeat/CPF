\documentclass[11pt,a4paper]{article}

% Packages
\usepackage[utf8]{inputenc}
\usepackage[italian]{babel}
\usepackage[margin=2.5cm]{geometry}
\usepackage{amsmath}
\usepackage{booktabs}
\usepackage{hyperref}
\usepackage{fancyhdr}
\usepackage{enumitem}
\usepackage{longtable}

% Page style
\pagestyle{fancy}
\fancyhf{}
\renewcommand{\headrulewidth}{0.4pt}
\fancyhead[L]{CPF-101 Progetto di Formazione}
\fancyhead[R]{Version 1.0}
\fancyfoot[C]{\thepage}

% Spacing
\setlength{\parindent}{0pt}
\setlength{\parskip}{0.5em}

% Hyperref
\hypersetup{
    colorlinks=true,
    linkcolor=blue,
    citecolor=blue,
    urlcolor=blue,
    pdftitle={CPF-101 Progetto di Formazione},
    pdfauthor={Giuseppe Canale, CISSP}
}

\title{\textbf{CPF-101 Progetto di Formazione}\\
\large Framework Fundamentals Course Design\\
40 Hours | 80 Slides}
\author{CPF3 Training Development\\
Giuseppe Canale, CISSP\\
\small g.canale@cpf3.org}
\date{January 2025}

\begin{document}

\maketitle

\begin{abstract}
This training blueprint defines the instructional design for CPF-101: Framework Fundamentals, the foundational 40-hour course required for all CPF professional certifications. The document provides module-level outlines enabling systematic slide generation, exercise development, and assessment creation. Each module includes learning objectives, content structure, teaching methods, slide breakdowns, required materials, and assessment items. This blueprint serves as the reference document for creating instructor-led presentations, self-paced learning materials, and certification examination items.
\end{abstract}

\tableofcontents
\newpage

\section{Course Overview}

\subsection{Course Identification}
\textbf{Code:} CPF-101 | \textbf{Title:} Framework Fundamentals | \textbf{Duration:} 40 hours (5 days intensive or 10 half-days) | \textbf{Slides:} 80 total | \textbf{Format:} Instructor-led or self-paced

\subsection{Target Audience}
Cybersecurity professionals, information security practitioners, security auditors, and risk management professionals pursuing CPF Assessor, Practitioner, or Auditor certification. Prerequisites include bachelor's degree (or equivalent) and 2+ years experience in cybersecurity or psychology.

\subsection{Learning Objectives}
Upon completion, participants will: (1) Explain pre-cognitive psychological mechanisms underlying 82-85\% of security incidents, (2) Identify theoretical foundations from psychoanalysis and cognitive psychology, (3) Describe all 10 CPF domains and 100 indicators, (4) Articulate privacy protection requirements, (5) Apply ternary scoring methodology to scenarios, (6) Map CPF to ISO 27001 and NIST CSF 2.0.

\subsection{Course Structure}
\textbf{Part I - Foundations (12h, Modules 1-3):} Introduction to Cybersecurity Psychology (4h), Psychoanalytic Foundations (4h), Cognitive Psychology Foundations (4h).

\textbf{Part II - CPF Domains (20h, Modules 4-13):} Ten 2-hour modules covering Authority [1.x], Temporal [2.x], Social Influence [3.x], Affective [4.x], Cognitive Overload [5.x], Group Dynamics [6.x], Stress Response [7.x], Unconscious Process [8.x], AI-Specific Bias [9.x], Critical Convergent States [10.x].

\textbf{Part III - Application (8h, Modules 14-15):} Privacy and Ethics (4h), Integration and Application (4h).

\subsection{Assessment Method}
Formative: Module quizzes (3-5 questions each), 15 practical exercises, 5 case studies. Summative: 100-question written examination (60 multiple-choice, 30 scenario-based, 10 case analysis), 3 hours, 70\% passing score. Certification requires 90\% attendance and signed ethics agreement.

\subsection{Materials Provided}
CPF-101 Participant Workbook (80 pages), CPF Taxonomy Quick Reference Card, Field Kit Example (Indicator 1.1 complete), Case Study Packet (5 scenarios), Assessment Template, CPF framework papers (taxonomy, CPF-27001 requirements, certification scheme).

\newpage

\section{Module Structures}

\subsection{Module 1: Introduction to Cybersecurity Psychology}

\subsubsection{Overview}
\textbf{Duration:} 4 hours | \textbf{Slides:} 6

\textbf{Learning Objectives:} Articulate why traditional security awareness fails to prevent 82-85\% of incidents; explain neuroscience evidence for pre-cognitive decision-making; describe CPF architecture (10 domains, 100 indicators, ternary scoring); map CPF integration with ISO 27001 and NIST CSF; analyze major breach through CPF lens.

\textbf{Key Concepts:} Pre-cognitive processing, System 1 vs System 2, human factor gap, framework architecture, integration strategy.

\subsubsection{Content Outline}
\textbf{1. Human Factor Gap (45 min):} Global spending vs increasing breaches, Verizon DBIR statistics, failure of rational actor model, why awareness training provides false confidence, real-world examples (Target, Anthem, SolarWinds).

\textbf{2. Pre-Cognitive Processing (60 min):} Neuroscience evidence (Libet 1983, Soon 2008), amygdala activation 300-500ms before conscious awareness, fMRI decision studies, Damasio somatic markers, evolutionary hardwiring creates vulnerabilities, implications for security training, video demonstration.

\textbf{3. CPF Architecture (60 min):} 10x10 structure overview, brief introduction to all 10 domains, ternary scoring (Green/Yellow/Red), privacy-first design (aggregation, differential privacy, temporal delays), assessment methodology overview.

\textbf{4. Integration with Frameworks (45 min):} ISO 27001:2022 mapping (Clause 7.2 Competence, 7.3 Awareness, Annex A enhancement), NIST CSF 2.0 integration (Identify/Protect/Detect/Respond/Recover), CPF as psychological intelligence layer, complementary relationship.

\textbf{5. Case Study: Target Breach (30 min):} Technical narrative, CPF psychological analysis (authority vulnerability, alert fatigue, group dynamics, temporal pressure), how assessment could predict convergence, preventive interventions, discussion.

\subsubsection{Teaching Methods}
\textbf{Lecture:} Statistics/charts for human factor gap, neuroscience video with fMRI images, animated framework diagram, ISO/NIST comparison tables.

\textbf{Exercises:} (1) Awareness Failure Analysis - share failed training examples (15 min), (2) Pre-Cognitive Decision Experiment - live authority/urgency demonstration (10 min), (3) Framework Navigation - speed drill with Quick Reference Card (20 min).

\textbf{Discussion:} "Think of an incident - could awareness training have prevented it?", "What does 300-500ms pre-conscious decision mean for your program?", "Explain CPF to CISO in 60 seconds."

\textbf{Case Study:} Target breach presented chronologically, groups identify vulnerabilities, connect to CPF domains, synthesize convergent state prediction.

\subsubsection{Slide Breakdown}
\textbf{Slide 1.1:} "The Human Factor Crisis" - Spending vs breach trends chart, Verizon statistics, human silhouette vs fortress visual.

\textbf{Slide 1.2:} "Pre-Cognitive Decision-Making" - Timeline diagram (0ms→300ms→500ms→800ms), Libet/Soon results, brain diagram (amygdala vs prefrontal cortex).

\textbf{Slide 1.3:} "CPF Framework Architecture" - 10x10 grid with domain names/icons, 100 indicators, ternary scoring callout, privacy-first design.

\textbf{Slide 1.4:} "Ternary Scoring System" - Three-column comparison (Green/Yellow/Red), example indicator 1.1, category/CPF score formulas.

\textbf{Slide 1.5:} "Integration with ISO 27001 and NIST CSF" - Split diagram showing enhancement points, "CPF complements not replaces" message.

\textbf{Slide 1.6:} "Target Breach Through CPF Lens" - Timeline, technical story, CPF analysis overlay (4 domains identified), exercise prompt.

\subsubsection{Materials Needed}
Workbook Module 1 (pages 1-15), Taxonomy Quick Reference Card, neuroscience video (5 min), Target case study handout (2 pages), Exercise Worksheets 1.1 and 1.3, whiteboard/digital collaboration tool.

\subsubsection{Assessment Items}
\textbf{Quiz (5 questions):} Q1: Verizon DBIR human factor \% → 82-85\% correct. Q2: Amygdala activation timing → 300-500ms correct. Q3: Total CPF indicators → 100 correct. Q4: Yellow score meaning → moderate vulnerability/monitoring correct. Q5: ISO clause CPF enhances → 7.2 Competence correct.

\textbf{Exercise Rubric (Framework Navigation):} Speed 5 indicators <2 min (2 pts), accuracy domain/number (3 pts), comprehension explanation (3 pts), application scenario (2 pts). Total 10 pts (7+ pass).

\subsection{Module 2: Psychoanalytic Foundations}

\subsubsection{Overview}
\textbf{Duration:} 4 hours | \textbf{Slides:} 7

\textbf{Learning Objectives:} Explain Bion's basic assumptions (baD/baF/baP) in security contexts; apply Klein's object relations (splitting, projection) to organizational blind spots; identify Jung's shadow and collective unconscious in threat perception; describe Winnicott's transitional space relevance to digital security; analyze security postures through psychoanalytic lens.

\textbf{Key Concepts:} Basic assumptions, object relations, splitting, projection, shadow, collective unconscious, transitional space, social defense systems.

\subsubsection{Content Outline}
\textbf{1. Bion's Basic Assumptions (75 min):} Bion "Experiences in Groups" (1961), three defensive postures under anxiety. baD (Dependency): Omnipotent leader/technology seeking, security tool magical thinking, [6.6] indicator. baF (Fight-Flight): External enemy focus, fortress mentality, insider threat blindness, [6.7] indicator. baP (Pairing): Future salvation hope, continuous tool shopping, [6.8] indicator. Recognition in security teams, vulnerability creation.

\textbf{2. Kleinian Object Relations (60 min):} Klein contributions, Splitting mechanism (all good/all bad division, insiders idealized vs attackers demonized, insider threat invisibility, [6.9] organizational splitting), Projection mechanism (organizational vulnerabilities attributed externally, learning blocked, [8.1] shadow projection), Menzies Lyth social defense systems (ritualistic checkbox compliance, [6.10] collective defenses), pattern identification.

\textbf{3. Jungian Psychology (60 min):} Shadow concept (disowned aspects projected, black hat hackers embody organizational aggression, security team attacker identification, red team as shadow expression, [8.1][8.2] indicators), Archetypes (Hero CISO, Trickster hacker, Wise consultant, Shadow insider threat, [8.8] archetypal activation), Collective unconscious (shared industry denials, "too small to target" myth, [8.9] patterns), shadow work for vulnerability reduction.

\textbf{4. Winnicott Transitional Space (30 min):} Transitional objects/spaces concept, digital environments as transitional (neither real nor imaginary, reduced reality testing, omnipotent online fantasies, digital identity confusion, social media guard lowering, [8.10] dream logic, [8.7] symbolic equation), exploitation mechanisms, design implications.

\textbf{5. Exercise: Psychoanalytic Case Analysis (15 min):} Healthcare ransomware case, groups identify basic assumption, splitting evidence, projections, shadow elements, collective defenses. Presentations, facilitator maps to CPF indicators [6.x] and [8.x].

\subsubsection{Teaching Methods}
\textbf{Lecture:} Bion organizational diagrams, Klein splitting/projection visuals, Jung archetypal images, Winnicott physical vs digital space comparison.

\textbf{Exercises:} (1) Basic Assumption ID - 3 vignettes, identify baD/baF/baP (20 min), (2) Splitting in Security Culture - list trusted/threatening entities, discuss blind spots (15 min), (3) Shadow Recognition - anonymous reflection "what org refuses to acknowledge" (15 min).

\textbf{Discussion:} "Seen silver bullet thinking?", "How does org describe attackers?", "What security rituals provide false comfort?"

\textbf{Media:} Group dynamics video clip (5 min), Jung shadow animation, splitting diagram.

\subsubsection{Slide Breakdown}
\textbf{Slide 2.1:} "Bion's Basic Assumptions Overview" - Three postures with icons (crown, sword/shield, hope), Bion quote, unconscious not deliberate.

\textbf{Slide 2.2:} "Basic Assumptions in Security" - Three-column table (baD/baF/baP with behaviors, examples, vulnerabilities, CPF indicators).

\textbf{Slide 2.3:} "Kleinian Splitting and Projection" - Split diagram good/bad, splitting mechanism (insiders trusted, insider threats invisible), projection mechanism (blame external, no learning), Klein quote, indicators [6.9][8.1].

\textbf{Slide 2.4:} "Jung's Shadow in Security" - Light/dark silhouette, shadow concept, security manifestations (black hats, red teams, identification), collective shadow (industry denials), Jung quote, indicators [8.1][8.2][8.9].

\textbf{Slide 2.5:} "Archetypes in Security" - Four archetypal images (Hero CISO, Trickster hacker, Wise consultant, Shadow insider), descriptions with vulnerabilities, unconscious patterns note, indicator [8.8].

\textbf{Slide 2.6:} "Winnicott's Transitional Space" - Physical vs digital comparison table, security implications (omnipotence, identity confusion, dream logic, symbolic equations), Winnicott quote, indicators [8.10][8.7], social media example.

\textbf{Slide 2.7:} "Psychoanalytic Case Analysis Exercise" - Healthcare ransomware brief, analysis framework (basic assumption, splitting, projection, shadow, defenses), group task instructions, expected findings, debrief question.

\subsubsection{Materials Needed}
Workbook Module 2 (pages 16-30), Exercise 2.1 three vignettes worksheet, Exercise 2.2 splitting template, Exercise 2.3 anonymous shadow cards, healthcare case handout (2 pages), archetypal images, group dynamics video (5 min), whiteboard.

\subsubsection{Assessment Items}
\textbf{Quiz (5 questions):} Q1: baD characteristic → seeking omnipotent protector correct. Q2: Splitting definition → unconscious all good/bad division correct. Q3: Shadow projection indicator → [8.1] correct. Q4: Transitional space relevance → neither real nor imaginary, reduced reality testing correct. Q5: Continuous tool shopping basic assumption → baP correct.

\textbf{Exercise Rubric (Splitting):} List 3+ trusted and threatening entities (2 pts), recognize idealization/demonization (3 pts), identify 2+ blind spots (3 pts), suggest addressing splitting (2 pts). Total 10 pts (7+ pass).

\subsection{Module 3: Cognitive Psychology Foundations}

\subsubsection{Overview}
\textbf{Duration:} 4 hours | \textbf{Slides:} 7

\textbf{Learning Objectives:} Explain Kahneman's dual-process theory (System 1/2) implications for security; apply Cialdini's six influence principles to social engineering; analyze Miller's cognitive load impact on security tasks; identify heuristics/biases enabling exploitation; evaluate organizational structures through cognitive lens.

\textbf{Key Concepts:} System 1/2, heuristics, cognitive biases, influence principles (reciprocity, commitment, social proof, authority, liking, scarcity), cognitive load, working memory.

\subsubsection{Content Outline}
\textbf{1. Kahneman Dual-Process (75 min):} "Thinking Fast and Slow" (2011), System 1 (fast, automatic, unconscious, pattern recognition, emotional, heuristics, always on), System 2 (slow, deliberate, conscious, effortful, analytical, limited capacity, depletes). Security decision problem (most decisions <1 sec System 1, verification requires 10-30 sec System 2, time pressure = System 1 dominance, attackers exploit System 1). Key heuristics (availability, representativeness, anchoring, confirmation, optimism bias). CPF indicators leverage System 1 vulnerabilities.

\textbf{2. Cialdini Six Principles (75 min):} "Influence" (2007), each principle = attack vector. (1) Reciprocity: Obligation to return favors, quid pro quo attacks, [3.1] exploitation. (2) Commitment/Consistency: Align with prior commitments, gradual escalation foot-in-door, [3.2] traps. (3) Social Proof: Look to others, "everyone clicked" manipulation, [3.3] manipulation, [3.8] conformity. (4) Authority: Deference to authority, CEO fraud, fake IT, Domain [1.x] entire. (5) Liking: Comply with liked people, rapport building, [3.4] trust override. (6) Scarcity: Value limited things, urgency attacks, [3.5] scarcity decisions. Recognition of combined principles in attacks (BEC = Authority + Scarcity + Commitment).

\textbf{3. Miller Cognitive Load (45 min):} "Magical Number 7±2" (1956), working memory limits (7±2 items now 4±1, 15-30 sec duration, easily overwhelmed). Security task load (intrinsic complexity, extraneous unnecessary complexity, germane learning, total = sum of three, overload = errors/heuristics/System 1). Overload manifestations (alert fatigue, decision fatigue, multitasking degradation, complexity errors). Domain [5.x] all 10 indicators relate to capacity limits. Design principle: reduce extraneous load.

\textbf{4. Decision Under Uncertainty (30 min):} Prospect Theory (Kahneman/Tversky 1979), Loss aversion (losses larger than gains, ransomware effectiveness, [4.1] fear paralysis, [2.3] deadline risk), Framing effects (presentation changes decisions, protection vs restriction framing), Sunk cost fallacy (continue based on past investment, ineffective tool persistence, [4.4] legacy attachment).

\textbf{5. Exercise: Cognitive Exploitation Analysis (15 min):} Three social engineering emails, tasks (identify Cialdini principles, System 1 or 2 target, cognitive load manipulation, countermeasures based on cognitive psychology), group discussion how each fools System 1, facilitator connects to CPF indicators.

\subsubsection{Teaching Methods}
\textbf{Lecture:} Dual-process with optical illusions and rapid decisions, Cialdini with advertising examples then security, Cognitive load with digit span test, Decision-making with framing exercise.

\textbf{Exercises:} (1) System 1 vs 2 Speed Test - 20 emails 3 sec each then unlimited, compare accuracy (15 min), (2) Cialdini Mapping - 6 scenarios map principles, discuss combinations (20 min), (3) Cognitive Load Simulation - security task with distractions, experience overload (15 min).

\textbf{Discussion:} "Clicked something you shouldn't - System 1 or 2?", "Most dangerous Cialdini principle?", "How many daily security decisions? Cognitive cost?"

\textbf{Media:} Kahneman TED excerpt (5 min), Cialdini Candid Camera demonstrations (10 min), cognitive load animation (3 min).

\subsubsection{Slide Breakdown}
\textbf{Slide 3.1:} "Thinking Fast and Slow" - Split-screen System 1 vs 2 table (speed, mode, energy, method, security role, vulnerability/limitation), key insight most decisions System 1 timeframe, Kahneman quote, brain diagram.

\textbf{Slide 3.2:} "Heuristics and Biases" - Five heuristics with definitions, security examples, vulnerabilities (availability, representativeness, anchoring, confirmation, optimism).

\textbf{Slide 3.3:} "Cialdini's Six Principles Overview" - Six-box grid with icons, principle names, one-line descriptions, "Each = attack vector" message.

\textbf{Slide 3.4:} "Six Principles in Social Engineering" - Detailed table (principle, mechanism, security example, CPF indicator) for all six, note on combined principles, BEC example.

\textbf{Slide 3.5:} "Cognitive Load Theory" - Working memory capacity visual (7±2 now 4±1), three load types diagram (intrinsic, extraneous, germane), security overload manifestations (alert fatigue, decision fatigue, multitasking, complexity), Miller quote, Domain [5.x] note.

\textbf{Slide 3.6:} "Decision-Making Under Uncertainty" - Loss aversion explanation with ransomware example, framing effects demonstration, sunk cost fallacy with tool persistence, indicators [4.1][2.3][4.4].

\textbf{Slide 3.7:} "Cognitive Exploitation Exercise" - Three email examples displayed, analysis framework (Cialdini principles, System target, load manipulation, countermeasures), group task instructions, discussion prompt.

\subsubsection{Materials Needed}
Workbook Module 3 (pages 31-45), Exercise 3.1 20-email test set, Exercise 3.2 six scenario cards, Exercise 3.3 distraction simulation setup, three phishing emails for analysis, Kahneman video (5 min), Cialdini clips (10 min), cognitive load animation (3 min), whiteboard.

\subsubsection{Assessment Items}
\textbf{Quiz (5 questions):} Q1: System 1 characteristics → fast, automatic, unconscious correct. Q2: Cialdini reciprocity principle → obligation to return favors correct. Q3: Miller working memory capacity → 7±2 (or 4±1) correct. Q4: Which Domain addresses cognitive overload → [5.x] correct. Q5: Loss aversion explains → ransomware effectiveness correct.

\textbf{Exercise Rubric (Email Analysis):} Identify 2+ Cialdini principles (3 pts), determine System 1/2 target correctly (2 pts), explain cognitive load manipulation (3 pts), suggest effective countermeasure (2 pts). Total 10 pts (7+ pass).

\subsection{Module 4: Domain [1.x] Authority-Based Vulnerabilities}

\subsubsection{Overview}
\textbf{Duration:} 2 hours | \textbf{Slides:} 4

\textbf{Learning Objectives:} Understand Milgram obedience research and cybersecurity implications; identify 10 authority indicators; apply ternary scoring to authority compliance scenarios; recommend top 3 remediation strategies.

\textbf{Key Concepts:} Authority deference, diffusion of responsibility, CEO fraud, BEC, verification protocols.

\subsubsection{Content Outline}
\textbf{1. Psychological Foundation (20 min):} Milgram experiments 65\% obedience, neuroscience amygdala hijack, evolution authority deference survival mechanism, System 1 vs 2 in authority contexts.

\textbf{2. Ten Authority Indicators (30 min):} Overview table 1.1-1.10, Deep-dive 1.1 (unquestioning compliance, observables, assessment questions, G/Y/R scoring), Deep-dive 1.3 (impersonation susceptibility, email authentication gaps, verification failures, multi-channel), patterns across remaining indicators.

\textbf{3. Attack Vectors and Incidents (30 min):} BEC \$43B FBI losses, Target breach via vendor authority case, spear phishing with authority claims, IT support social engineering, technical failure points (MFA bypass, privilege escalation).

\textbf{4. Assessment and Solutions (30 min):} Observable indicators in organizations, ternary scoring decision tree, Top 3 solutions (dual-channel verification protocol, authority challenge training, simulation testing program), implementation priorities (high/medium/long-term).

\textbf{5. Exercise (10 min):} Case scenario CFO wire transfer email, students assess score/rationale/solutions, group discussion and feedback.

\subsubsection{Teaching Methods}
\textbf{Lecture:} Milgram with historical footage, neuroscience diagrams, attack flow visuals.

\textbf{Exercise:} CFO fraud scenario, individual assessment then group comparison.

\textbf{Discussion:} "Seen authority bypass in your org?", "How verify unusual executive requests?"

\subsubsection{Slide Breakdown}
\textbf{Slide 4.1:} "Why We Obey: Authority Vulnerability" - Milgram visual, neuroscience diagram, evolution context, 65\% stat, 300-500ms activation, transition to CEO fraud.

\textbf{Slide 4.2:} "10 Authority Indicators" - Table 1.1-1.10 with brief descriptions, highlight 1.1 and 1.3 for deep-dive, indicator icons.

\textbf{Slide 4.3:} "Attack Vectors in Action" - BEC \$43B stat, Target breach timeline, spear phishing examples, attack flow diagram, real email snippet (redacted).

\textbf{Slide 4.4:} "Assessment and Solutions" - Observable checklist, scoring decision tree G/Y/R, top 3 solutions with icons/timelines, exercise prompt CFO scenario.

\subsubsection{Materials Needed}
Field Kit 1.1 and 1.3 (reference), Taxonomy section [1.x], Target case study, CFO exercise handout.

\subsubsection{Assessment Items}
\textbf{Quiz:} Q1: Milgram obedience \% → 65\% correct. Q2: Indicator for security bypass for superiors → 1.4 correct. Q3: Primary authority mechanism → amygdala hijack before rational processing correct.

\textbf{Exercise Rubric:} Vulnerability ID (2 pts), ternary score with justification (3 pts), relevant solution (3 pts), implementation priority understanding (2 pts). Total 10 pts (7+ pass).

\subsection{Module 5: Domain [2.x] Temporal Vulnerabilities}

\subsubsection{Overview}
\textbf{Duration:} 2 hours | \textbf{Slides:} 4

\textbf{Learning Objectives:} Explain time pressure effects on security decisions; identify 10 temporal indicators; recognize deadline-driven attack patterns; apply temporal vulnerability scoring.

\textbf{Key Concepts:} Urgency exploitation, time pressure degradation, deadline attacks, present bias, hyperbolic discounting.

\subsubsection{Content Outline}
\textbf{1. Psychological Foundation (20 min):} Time pressure cognitive degradation (Kahneman/Tversky), present bias (immediate rewards overweighted), hyperbolic discounting (future threats discounted), System 2 shutdown under time pressure.

\textbf{2. Ten Temporal Indicators (30 min):} Overview table 2.1-2.10, Deep-dive 2.1 (urgency-induced bypass, "act now" attacks, scoring), Deep-dive 2.3 (deadline-driven risk acceptance, end-of-quarter pressure, scoring), temporal patterns.

\textbf{3. Attack Vectors and Incidents (30 min):} Deadline attacks (tax season, end-of-quarter), time-of-day exploitation (shift changes, late hours), weekend/holiday vulnerabilities, temporal social engineering case examples.

\textbf{4. Assessment and Solutions (30 min):} Temporal vulnerability observables, scoring criteria, Top 3 solutions (cooling-off periods for urgent requests, shift-change security protocols, temporal anomaly detection), implementation.

\textbf{5. Exercise (10 min):} Urgent invoice payment scenario with time pressure, assess temporal vulnerabilities, recommend controls.

\subsubsection{Teaching Methods}
\textbf{Lecture:} Time pressure experiments, present bias demonstrations, temporal attack timelines.

\textbf{Exercise:} Urgent payment scenario under simulated time pressure.

\textbf{Discussion:} "When are security decisions worst?", "End-of-quarter security compromises?"

\subsubsection{Slide Breakdown}
\textbf{Slide 5.1:} "Time Pressure and Security" - Cognitive degradation under time constraints, present bias explanation, hyperbolic discounting, System 2 shutdown.

\textbf{Slide 5.2:} "10 Temporal Indicators" - Table 2.1-2.10, highlight 2.1 and 2.3, temporal vulnerability patterns.

\textbf{Slide 5.3:} "Temporal Attack Vectors" - Deadline attacks (tax, quarter-end), time-of-day windows, weekend/holiday exploitation, case examples with timelines.

\textbf{Slide 5.4:} "Assessment and Solutions" - Temporal observables, scoring decision tree, top 3 solutions (cooling-off, shift protocols, anomaly detection), exercise prompt.

\subsubsection{Materials Needed}
Field Kit 2.1 and 2.3, Taxonomy [2.x], urgent payment exercise handout.

\subsubsection{Assessment Items}
\textbf{Quiz:} Q1: Present bias definition → immediate rewards overweighted correct. Q2: Indicator for urgency bypass → 2.1 correct. Q3: Temporal vulnerability amplifier → time pressure, deadlines correct.

\textbf{Exercise Rubric:} Temporal vulnerability ID (3 pts), urgency pressure analysis (2 pts), scoring with justification (3 pts), temporal controls recommendation (2 pts). Total 10 pts (7+ pass).

\subsection{Module 6: Domain [3.x] Social Influence Vulnerabilities}

\subsubsection{Overview}
\textbf{Duration:} 2 hours | \textbf{Slides:} 4

\textbf{Learning Objectives:} Apply Cialdini principles to security contexts; identify 10 social influence indicators; analyze social engineering attack patterns; design social influence countermeasures.

\textbf{Key Concepts:} Reciprocity, commitment/consistency, social proof, liking, scarcity, unity principle, peer pressure, conformity.

\subsubsection{Content Outline}
\textbf{1. Psychological Foundation (20 min):} Cialdini six principles review (from Module 3), social influence as evolutionary mechanism, compliance psychology, combination effects in attacks.

\textbf{2. Ten Social Indicators (30 min):} Overview table 3.1-3.10, Deep-dive 3.1 (reciprocity exploitation, quid pro quo, scoring), Deep-dive 3.3 (social proof manipulation, "everyone clicked" attacks, scoring), social influence patterns across indicators.

\textbf{3. Attack Vectors and Incidents (30 min):} Social engineering campaigns (pretext building, rapport establishment), peer pressure attacks (fake IT broadcasts), conformity exploitation (everyone updating passwords), unity principle (fake team requests), real-world social engineering cases.

\textbf{4. Assessment and Solutions (30 min):} Social influence observables, scoring criteria, Top 3 solutions (social proof verification protocols, peer validation systems, influence awareness training), implementation priorities.

\textbf{5. Exercise (10 min):} Social engineering email combining multiple Cialdini principles, identify which principles used, assess vulnerability score, recommend defenses.

\subsubsection{Teaching Methods}
\textbf{Lecture:} Cialdini principles demonstrations, social engineering video examples, influence combination analysis.

\textbf{Exercise:} Multi-principle phishing email analysis, group identification of techniques.

\textbf{Discussion:} "Most effective Cialdini principle in your experience?", "How resist social proof in security?"

\subsubsection{Slide Breakdown}
\textbf{Slide 6.1:} "Social Influence Mechanisms" - Six Cialdini principles visual review, evolutionary basis, compliance psychology, combination attack effects.

\textbf{Slide 6.2:} "10 Social Influence Indicators" - Table 3.1-3.10, highlight 3.1 and 3.3, social influence vulnerability patterns.

\textbf{Slide 6.3:} "Social Engineering Attack Patterns" - Campaign examples (pretext, rapport, pressure), conformity exploitation scenarios, unity principle fake requests, real-world cases.

\textbf{Slide 6.4:} "Assessment and Solutions" - Social observables checklist, scoring decision tree, top 3 solutions (verification protocols, peer validation, awareness training), exercise prompt multi-principle email.

\subsubsection{Materials Needed}
Field Kit 3.1 and 3.3, Taxonomy [3.x], social engineering video clips (5 min), multi-principle exercise handout.

\subsubsection{Assessment Items}
\textbf{Quiz:} Q1: Reciprocity principle definition → obligation to return favors correct. Q2: Social proof indicator → 3.3 correct. Q3: Most dangerous combination → multiple answers acceptable (Authority + Scarcity, Social Proof + Liking).

\textbf{Exercise Rubric:} Identify 3+ Cialdini principles (3 pts), explain influence mechanism (2 pts), vulnerability scoring with justification (3 pts), defense recommendations (2 pts). Total 10 pts (7+ pass).

\subsection{Module 7: Domain [4.x] Affective Vulnerabilities}

\subsubsection{Overview}
\textbf{Duration:} 2 hours | \textbf{Slides:} 4

\textbf{Learning Objectives:} Explain emotion-driven security decisions; identify 10 affective indicators; recognize emotional manipulation attacks; apply affective vulnerability assessment.

\textbf{Key Concepts:} Fear exploitation, anger-induced risk, trust transference, attachment, shame hiding, emotional contagion.

\subsubsection{Content Outline}
\textbf{1. Psychological Foundation (20 min):} Emotion primacy over cognition (LeDoux), affective heuristic (Slovic), emotional contagion (Hatfield), amygdala hijack in security contexts, emotional states alter risk perception.

\textbf{2. Ten Affective Indicators (30 min):} Overview table 4.1-4.10, Deep-dive 4.1 (fear-based paralysis, ransomware FUD, scoring), Deep-dive 4.5 (shame-based security hiding, incident non-reporting, scoring), affective patterns.

\textbf{3. Attack Vectors and Incidents (30 min):} Fear-based attacks (ransomware, scare tactics), anger manipulation (provocative content), trust exploitation (fake support), shame prevention of reporting, emotional contagion in breaches, case examples.

\textbf{4. Assessment and Solutions (30 min):} Affective vulnerability observables, scoring criteria, Top 3 solutions (psychological safety for reporting, emotional regulation training, FUD resistance protocols), implementation.

\textbf{5. Exercise (10 min):} Ransomware scenario with fear manipulation, assess emotional vulnerabilities, design psychological safety interventions.

\subsubsection{Teaching Methods}
\textbf{Lecture:} Emotion neuroscience, affective heuristic demonstrations, emotional attack examples.

\textbf{Exercise:} Ransomware fear scenario, individual then group analysis of emotional manipulation.

\textbf{Discussion:} "Fear-based security decisions made?", "How create psychological safety for incident reporting?"

\subsubsection{Slide Breakdown}
\textbf{Slide 7.1:} "Emotion and Security Decisions" - Emotion primacy (LeDoux), affective heuristic explanation, amygdala hijack diagram, emotional risk perception alteration.

\textbf{Slide 7.2:} "10 Affective Indicators" - Table 4.1-4.10, highlight 4.1 and 4.5, emotional vulnerability patterns.

\textbf{Slide 7.3:} "Emotional Manipulation Attacks" - Fear-based (ransomware FUD), anger manipulation, trust exploitation, shame reporting prevention, emotional contagion effects, case examples.

\textbf{Slide 7.4:} "Assessment and Solutions" - Affective observables, scoring decision tree, top 3 solutions (psychological safety, emotional regulation, FUD resistance), exercise prompt ransomware scenario.

\subsubsection{Materials Needed}
Field Kit 4.1 and 4.5, Taxonomy [4.x], ransomware exercise handout, emotional contagion video (3 min).

\subsubsection{Assessment Items}
\textbf{Quiz:} Q1: Affective heuristic definition → decisions based on emotional state correct. Q2: Fear paralysis indicator → 4.1 correct. Q3: Shame-based hiding indicator → 4.5 correct.

\textbf{Exercise Rubric:} Fear manipulation identification (3 pts), emotional vulnerability assessment (2 pts), scoring with justification (3 pts), psychological safety intervention design (2 pts). Total 10 pts (7+ pass).

\subsection{Module 8: Domain [5.x] Cognitive Overload Vulnerabilities}

\subsubsection{Overview}
\textbf{Duration:} 2 hours | \textbf{Slides:} 4

\textbf{Learning Objectives:} Apply Miller cognitive load theory to security; identify 10 overload indicators; recognize capacity exploitation attacks; design cognitive load reduction interventions.

\textbf{Key Concepts:} Working memory limits, alert fatigue, decision fatigue, multitasking degradation, cognitive tunneling.

\subsubsection{Content Outline}
\textbf{1. Psychological Foundation (20 min):} Miller 7±2 working memory (now 4±1), cognitive load types (intrinsic, extraneous, germane), capacity overload consequences, System 2 depletion, security task complexity.

\textbf{2. Ten Overload Indicators (30 min):} Overview table 5.1-5.10, Deep-dive 5.1 (alert fatigue desensitization, SOC overload, scoring), Deep-dive 5.2 (decision fatigue errors, repeated security choices, scoring), overload patterns.

\textbf{3. Attack Vectors and Incidents (30 min):} Alert fatigue exploitation (noise before attack), decision fatigue timing (end-of-day attacks), multitasking vulnerability windows, complexity-induced errors, Target SOC alert fatigue case.

\textbf{4. Assessment and Solutions (30 min):} Overload vulnerability observables, scoring criteria, Top 3 solutions (alert consolidation and tuning, decision simplification, cognitive load budgets), implementation.

\textbf{5. Exercise (10 min):} SOC analyst overload scenario with 50+ alerts, assess cognitive vulnerabilities, design alert reduction strategy.

\subsubsection{Teaching Methods}
\textbf{Lecture:} Working memory demonstrations (digit span), alert fatigue visualization, decision fatigue experiments.

\textbf{Exercise:} SOC overload simulation, experience cognitive capacity limits.

\textbf{Discussion:} "How many security decisions daily?", "Alert fatigue in your SOC?"

\subsubsection{Slide Breakdown}
\textbf{Slide 8.1:} "Cognitive Load and Capacity Limits" - Miller 7±2 (4±1) visual, three load types diagram, overload consequences, System 2 depletion, security complexity impact.

\textbf{Slide 8.2:} "10 Cognitive Overload Indicators" - Table 5.1-5.10, highlight 5.1 and 5.2, overload vulnerability patterns.

\textbf{Slide 8.3:} "Cognitive Exploitation Attacks" - Alert fatigue exploitation examples, decision fatigue timing attacks, multitasking windows, complexity errors, Target SOC case with 40+ ignored alerts.

\textbf{Slide 8.4:} "Assessment and Solutions" - Overload observables (alert counts, decision frequency), scoring decision tree, top 3 solutions (consolidation, simplification, load budgets), exercise prompt SOC scenario.

\subsubsection{Materials Needed}
Field Kit 5.1 and 5.2, Taxonomy [5.x], SOC overload exercise with 50-alert scenario, digit span test materials.

\subsubsection{Assessment Items}
\textbf{Quiz:} Q1: Miller working memory capacity → 7±2 or 4±1 correct. Q2: Alert fatigue indicator → 5.1 correct. Q3: Decision fatigue indicator → 5.2 correct.

\textbf{Exercise Rubric:} Cognitive overload identification (3 pts), capacity limit analysis (2 pts), vulnerability scoring with justification (3 pts), alert reduction strategy (2 pts). Total 10 pts (7+ pass).

\subsection{Module 9: Domain [6.x] Group Dynamic Vulnerabilities}

\subsubsection{Overview}
\textbf{Duration:} 2 hours | \textbf{Slides:} 4

\textbf{Learning Objectives:} Apply Bion basic assumptions to security teams; identify 10 group dynamic indicators; recognize collective vulnerability patterns; design group-level interventions.

\textbf{Key Concepts:} Groupthink, risky shift, diffusion of responsibility, social loafing, bystander effect, basic assumptions (baD/baF/baP).

\subsubsection{Content Outline}
\textbf{1. Psychological Foundation (20 min):} Bion basic assumptions review (from Module 2), groupthink (Janis), risky shift phenomenon, diffusion of responsibility (Latané), collective unconscious in security teams.

\textbf{2. Ten Group Indicators (30 min):} Overview table 6.1-6.10, Deep-dive 6.1 (groupthink security blind spots, consensus pressure, scoring), Deep-dive 6.3 (diffusion of responsibility, "someone else will check", scoring), group patterns.

\textbf{3. Attack Vectors and Incidents (30 min):} Organizational disruption attacks (exploiting group dynamics), collective decision failures, bystander effect in incident response, group-level social engineering, NASA Challenger as groupthink parallel.

\textbf{4. Assessment and Solutions (30 min):} Group vulnerability observables, scoring criteria, Top 3 solutions (red team dissent roles, responsibility assignment, group decision protocols), implementation.

\textbf{5. Exercise (10 min):} Security committee decision scenario with groupthink pressure, identify group vulnerabilities, design dissent mechanisms.

\subsubsection{Teaching Methods}
\textbf{Lecture:} Groupthink video examples, risky shift demonstrations, basic assumptions in teams.

\textbf{Exercise:} Committee decision with groupthink, experience consensus pressure.

\textbf{Discussion:} "Groupthink in your security team?", "How encourage dissent safely?"

\subsubsection{Slide Breakdown}
\textbf{Slide 9.1:} "Group Psychology in Security" - Bion basic assumptions brief review, groupthink concept (Janis), risky shift, diffusion of responsibility, collective vulnerabilities.

\textbf{Slide 9.2:} "10 Group Dynamic Indicators" - Table 6.1-6.10, highlight 6.1 and 6.3, group vulnerability patterns.

\textbf{Slide 9.3:} "Group-Level Attack Vectors" - Organizational disruption techniques, collective decision failures, bystander effect in IR, group social engineering, Challenger groupthink parallel.

\textbf{Slide 9.4:} "Assessment and Solutions" - Group observables (meeting dynamics, decision patterns), scoring decision tree, top 3 solutions (dissent roles, responsibility assignment, decision protocols), exercise prompt committee scenario.

\subsubsection{Materials Needed}
Field Kit 6.1 and 6.3, Taxonomy [6.x], groupthink video (5 min), committee exercise handout.

\subsubsection{Assessment Items}
\textbf{Quiz:} Q1: Groupthink definition → consensus pressure prevents critical evaluation correct. Q2: Groupthink indicator → 6.1 correct. Q3: Diffusion of responsibility indicator → 6.3 correct.

\textbf{Exercise Rubric:} Groupthink identification (3 pts), consensus pressure analysis (2 pts), vulnerability scoring with justification (3 pts), dissent mechanism design (2 pts). Total 10 pts (7+ pass).

\subsection{Module 10: Domain [7.x] Stress Response Vulnerabilities}

\subsubsection{Overview}
\textbf{Duration:} 2 hours | \textbf{Slides:} 4

\textbf{Learning Objectives:} Explain stress physiology impact on security; identify 10 stress response indicators; recognize stress exploitation attacks; design stress resilience interventions.

\textbf{Key Concepts:} Acute vs chronic stress, fight/flight/freeze/fawn responses, cortisol impairment, stress contagion, burnout.

\subsubsection{Content Outline}
\textbf{1. Psychological Foundation (20 min):} Selye stress physiology, HPA axis activation, cortisol effects on cognition/memory, acute vs chronic stress, four F responses (fight/flight/freeze/fawn), stress contagion mechanisms.

\textbf{2. Ten Stress Indicators (30 min):} Overview table 7.1-7.10, Deep-dive 7.1 (acute stress impairment, incident response degradation, scoring), Deep-dive 7.2 (chronic stress burnout, SOC analyst exhaustion, scoring), stress patterns.

\textbf{3. Attack Vectors and Incidents (30 min):} Stress induction attacks (create chaos then exploit), burnout exploitation windows, incident response under acute stress, stress contagion during breaches, healthcare ransomware stress cases.

\textbf{4. Assessment and Solutions (30 min):} Stress vulnerability observables, scoring criteria, Top 3 solutions (stress inoculation training, burnout prevention programs, incident stress management), implementation.

\textbf{5. Exercise (10 min):} Incident response scenario under simulated stress, assess stress vulnerabilities, design resilience protocols.

\subsubsection{Teaching Methods}
\textbf{Lecture:} Stress physiology diagrams, cortisol effects, four F response explanations.

\textbf{Exercise:} Incident response with time pressure and incomplete information, experience stress impairment.

\textbf{Discussion:} "Stress level during last incident?", "Burnout in your security team?"

\subsubsection{Slide Breakdown}
\textbf{Slide 10.1:} "Stress and Security Performance" - Selye stress physiology, HPA axis diagram, cortisol cognitive effects, acute vs chronic comparison, four F responses.

\textbf{Slide 10.2:} "10 Stress Response Indicators" - Table 7.1-7.10, highlight 7.1 and 7.2, stress vulnerability patterns.

\textbf{Slide 10.3:} "Stress Exploitation Attacks" - Stress induction techniques, burnout exploitation timing, IR under acute stress, stress contagion effects, healthcare ransomware cases.

\textbf{Slide 10.4:} "Assessment and Solutions" - Stress observables (incident performance, team exhaustion), scoring decision tree, top 3 solutions (inoculation training, burnout prevention, stress management), exercise prompt IR scenario.

\subsubsection{Materials Needed}
Field Kit 7.1 and 7.2, Taxonomy [7.x], stressful IR exercise scenario, stress physiology diagrams.

\subsubsection{Assessment Items}
\textbf{Quiz:} Q1: Four F responses → fight/flight/freeze/fawn correct. Q2: Acute stress indicator → 7.1 correct. Q3: Chronic stress/burnout indicator → 7.2 correct.

\textbf{Exercise Rubric:} Stress vulnerability identification (3 pts), stress impairment analysis (2 pts), vulnerability scoring with justification (3 pts), resilience protocol design (2 pts). Total 10 pts (7+ pass).

\subsection{Module 11: Domain [8.x] Unconscious Process Vulnerabilities}

\subsubsection{Overview}
\textbf{Duration:} 2 hours | \textbf{Slides:} 4

\textbf{Learning Objectives:} Apply psychoanalytic concepts to security; identify 10 unconscious indicators; recognize unconscious exploitation; design shadow-aware interventions.

\textbf{Key Concepts:} Shadow projection, unconscious identification, transference, countertransference, defense mechanisms, archetypes.

\subsubsection{Content Outline}
\textbf{1. Psychological Foundation (20 min):} Psychoanalytic unconscious review (from Module 2), Jung shadow, Klein projection, transference/countertransference, defense mechanisms in security, archetypal patterns.

\textbf{2. Ten Unconscious Indicators (30 min):} Overview table 8.1-8.10, Deep-dive 8.1 (shadow projection onto attackers, external attribution, scoring), Deep-dive 8.4 (transference to authority figures, security leader idealization, scoring), unconscious patterns.

\textbf{3. Attack Vectors and Incidents (30 min):} Symbolic attacks exploiting unconscious, transference manipulation, archetypal activation (hero/trickster), defense mechanism exploitation, unconscious identification with attackers.

\textbf{4. Assessment and Solutions (30 min):} Unconscious vulnerability observables, scoring criteria, Top 3 solutions (shadow work facilitation, transference awareness training, defense mechanism recognition), implementation.

\textbf{5. Exercise (10 min):} Organizational breach post-mortem with external blame, identify unconscious defenses, design shadow integration.

\subsubsection{Teaching Methods}
\textbf{Lecture:} Unconscious mechanisms review, shadow examples, transference in organizations.

\textbf{Exercise:} Breach post-mortem analysis for projection and shadow.

\textbf{Discussion:} "What does org refuse to acknowledge?", "Idealization of security leaders?"

\subsubsection{Slide Breakdown}
\textbf{Slide 11.1:} "The Unconscious in Security" - Psychoanalytic unconscious concept, shadow projection mechanism, transference/countertransference, defense mechanisms, archetypes.

\textbf{Slide 11.2:} "10 Unconscious Process Indicators" - Table 8.1-8.10, highlight 8.1 and 8.4, unconscious vulnerability patterns.

\textbf{Slide 11.3:} "Unconscious Exploitation" - Symbolic attack examples, transference manipulation, archetypal activation (hero/trickster exploitation), defense mechanism use, identification with attackers.

\textbf{Slide 11.4:} "Assessment and Solutions" - Unconscious observables (external blame patterns, idealization), scoring decision tree, top 3 solutions (shadow work, transference awareness, defense recognition), exercise prompt breach post-mortem.

\subsubsection{Materials Needed}
Field Kit 8.1 and 8.4, Taxonomy [8.x], breach post-mortem exercise with external blame narrative.

\subsubsection{Assessment Items}
\textbf{Quiz:} Q1: Shadow projection definition → disowned aspects projected onto others correct. Q2: Shadow projection indicator → 8.1 correct. Q3: Transference indicator → 8.4 correct.

\textbf{Exercise Rubric:} Unconscious defense identification (3 pts), projection/shadow analysis (2 pts), vulnerability scoring with justification (3 pts), shadow integration design (2 pts). Total 10 pts (7+ pass).

\subsection{Module 12: Domain [9.x] AI-Specific Bias Vulnerabilities}

\subsubsection{Overview}
\textbf{Duration:} 2 hours | \textbf{Slides:} 4

\textbf{Learning Objectives:} Explain AI-human interaction psychology; identify 10 AI bias indicators; recognize AI vulnerability exploitation; design AI-aware security controls.

\textbf{Key Concepts:} Anthropomorphization, automation bias, algorithm aversion, AI authority transfer, uncanny valley, hallucination acceptance.

\subsubsection{Content Outline}
\textbf{1. Psychological Foundation (20 min):} Human-AI interaction psychology, anthropomorphization tendency, automation bias (over-reliance), algorithm aversion (under-trust), AI authority transfer, uncanny valley effects, emerging AI psychology research.

\textbf{2. Ten AI Indicators (30 min):} Overview table 9.1-9.10, Deep-dive 9.1 (anthropomorphization of AI systems, emotional attachment, scoring), Deep-dive 9.2 (automation bias override, uncritical AI acceptance, scoring), AI vulnerability patterns.

\textbf{3. Attack Vectors and Incidents (30 min):} AI social engineering (chatbot manipulation), deepfake exploitation (voice/video), AI recommendation poisoning, automation bias attacks (malicious ML), hallucination exploitation, adversarial ML cases.

\textbf{4. Assessment and Solutions (30 min):} AI vulnerability observables, scoring criteria, Top 3 solutions (AI literacy training, human-in-loop requirements, AI output verification protocols), implementation.

\textbf{5. Exercise (10 min):} AI-generated phishing scenario with chatbot pretext, assess AI-specific vulnerabilities, design verification controls.

\subsubsection{Teaching Methods}
\textbf{Lecture:} AI psychology research, anthropomorphization demonstrations, automation bias examples.

\textbf{Exercise:} AI chatbot phishing scenario, experience anthropomorphization pressure.

\textbf{Discussion:} "Trust AI security tools how much?", "Experienced AI hallucinations?"

\subsubsection{Slide Breakdown}
\textbf{Slide 12.1:} "AI-Human Interaction Psychology" - Anthropomorphization tendency, automation bias vs algorithm aversion, AI authority transfer, uncanny valley, emerging research.

\textbf{Slide 12.2:} "10 AI-Specific Bias Indicators" - Table 9.1-9.10, highlight 9.1 and 9.2, AI vulnerability patterns.

\textbf{Slide 12.3:} "AI Exploitation Attacks" - AI social engineering (chatbot), deepfakes (voice/video examples), recommendation poisoning, automation bias attacks, hallucination exploitation, adversarial ML cases.

\textbf{Slide 12.4:} "Assessment and Solutions" - AI observables (AI tool trust levels, verification practices), scoring decision tree, top 3 solutions (AI literacy, human-in-loop, verification protocols), exercise prompt chatbot phishing.

\subsubsection{Materials Needed}
Field Kit 9.1 and 9.2, Taxonomy [9.x], AI chatbot phishing exercise, deepfake video examples (3 min).

\subsubsection{Assessment Items}
\textbf{Quiz:} Q1: Anthropomorphization definition → attributing human intentions to AI correct. Q2: Automation bias indicator → 9.2 correct. Q3: AI authority transfer indicator → 9.4 correct.

\textbf{Exercise Rubric:} AI vulnerability identification (3 pts), anthropomorphization analysis (2 pts), vulnerability scoring with justification (3 pts), verification control design (2 pts). Total 10 pts (7+ pass).

\subsection{Module 13: Domain [10.x] Critical Convergent States}

\subsubsection{Overview}
\textbf{Duration:} 2 hours | \textbf{Slides:} 4

\textbf{Learning Objectives:} Explain vulnerability convergence concept; identify 10 convergent state indicators; recognize perfect storm conditions; design convergence monitoring systems.

\textbf{Key Concepts:} Perfect storm, cascade failures, tipping points, Swiss cheese alignment, black swans, gray rhinos, complexity catastrophe.

\subsubsection{Content Outline}
\textbf{1. Psychological Foundation (20 min):} System theory emergence, Reason Swiss cheese model, Perrow normal accidents, convergence mathematics (multiplication not addition), tipping point dynamics, complexity theory in security.

\textbf{2. Ten Convergent Indicators (30 min):} Overview table 10.1-10.10, Deep-dive 10.1 (perfect storm conditions, multiple vulnerability alignment, scoring), Deep-dive 10.4 (Swiss cheese alignment, hole alignment critical, scoring), convergence patterns.

\textbf{3. Attack Vectors and Incidents (30 min):} APT perfect storms (multiple vulnerabilities exploited), cascade failure attacks, tipping point breaches, Swiss cheese exploitation, SolarWinds as convergence case, black swan vs gray rhino events.

\textbf{4. Assessment and Solutions (30 min):} Convergent state observables, scoring criteria (exponential not linear), Top 3 solutions (convergence monitoring dashboards, vulnerability correlation analysis, early warning systems), implementation.

\textbf{5. Exercise (10 min):} Multi-domain vulnerability scenario, calculate convergence risk, design monitoring for perfect storm detection.

\subsubsection{Teaching Methods}
\textbf{Lecture:} Swiss cheese model visual, convergence mathematics demonstration, SolarWinds timeline.

\textbf{Exercise:} Multi-vulnerability scenario, calculate exponential risk.

\textbf{Discussion:} "Experienced perfect storm breach?", "How monitor convergence in your org?"

\subsubsection{Slide Breakdown}
\textbf{Slide 13.1:} "Vulnerability Convergence" - System emergence concept, Reason Swiss cheese visual, convergence mathematics (multiplication), tipping point dynamics, complexity catastrophe.

\textbf{Slide 13.2:} "10 Critical Convergent Indicators" - Table 10.1-10.10, highlight 10.1 and 10.4, convergence vulnerability patterns.

\textbf{Slide 13.3:} "Convergent State Attacks" - APT perfect storms, cascade failures, tipping point breaches, Swiss cheese hole alignment, SolarWinds convergence timeline, black swan vs gray rhino.

\textbf{Slide 13.4:} "Assessment and Solutions" - Convergent observables (multiple simultaneous vulnerabilities), exponential scoring approach, top 3 solutions (monitoring dashboards, correlation analysis, early warning), exercise prompt multi-domain scenario.

\subsubsection{Materials Needed}
Field Kit 10.1 and 10.4, Taxonomy [10.x], Swiss cheese model visual, multi-vulnerability exercise, SolarWinds case study.

\subsubsection{Assessment Items}
\textbf{Quiz:} Q1: Convergence risk calculation → multiplication not addition correct. Q2: Perfect storm indicator → 10.1 correct. Q3: Swiss cheese alignment indicator → 10.4 correct.

\textbf{Exercise Rubric:} Multi-vulnerability identification (3 pts), convergence calculation (2 pts), exponential risk scoring (3 pts), monitoring system design (2 pts). Total 10 pts (7+ pass).

\subsection{Module 14: Privacy and Ethics}

\subsubsection{Overview}
\textbf{Duration:} 4 hours | \textbf{Slides:} 8

\textbf{Learning Objectives:} Articulate privacy-preserving assessment principles; apply differential privacy mathematics; implement minimum aggregation requirements; design temporal delay mechanisms; explain ethical boundaries in psychological assessment.

\textbf{Key Concepts:} Differential privacy, aggregation units, temporal delays, prohibition on individual profiling, ethical assessment, data handling.

\subsubsection{Content Outline}
\textbf{1. Privacy-Preserving Principles (60 min):} Why privacy matters in psychological assessment, aggregation-only principle (never individual), minimum aggregation unit (10 individuals), differential privacy concept (epsilon = 0.1), temporal delay requirement (72 hours minimum), role-based not individual analysis, prohibition on performance evaluation use.

\textbf{2. Differential Privacy Mathematics (45 min):} Epsilon privacy budget concept, noise injection mechanisms, privacy-utility tradeoff, CPF epsilon = 0.1 rationale, practical implementation examples, verification methods.

\textbf{3. Data Handling Requisiti (45 min):} Encryption at rest (AES-256) and in transit (TLS 1.3), access controls and audit trails, retention limits (5 years maximum), secure destruction procedures, cross-border data transfer considerations, GDPR/CCPA compliance mapping.

\textbf{4. Ethical Boundaries (45 min):} CPF is organizational not clinical assessment, no individual diagnosis or therapy, psychological vulnerabilities are normal human characteristics not failures, prohibition on stigmatization or blame, informed consent requirements, opt-out mechanisms while maintaining statistical validity, whistleblower protections.

\textbf{5. Case Studies: Privacy Violations (30 min):} Historical psychological profiling abuses, Cambridge Analytica lessons, employee surveillance concerns, three violation scenarios analysis, discussion of ethical boundaries.

\textbf{6. Exercise: Privacy Impact Assessment (15 min):} Design assessment for 50-person department, ensure aggregation units, calculate differential privacy parameters, implement temporal delays, verify no individual profiling possible.

\subsubsection{Teaching Methods}
\textbf{Lecture:} Privacy principles with violation examples, differential privacy mathematics with visualizations, ethical framework with case comparisons.

\textbf{Exercises:} (1) Aggregation unit calculation for various org sizes (15 min), (2) Differential privacy parameter selection (15 min), (3) Privacy impact assessment design (15 min).

\textbf{Discussion:} "Tension between assessment and privacy?", "How ensure no individual profiling?", "Ethical concerns with psychological assessment?"

\textbf{Case Studies:} Three privacy violation scenarios, group analysis of what went wrong, design safeguards.

\subsubsection{Slide Breakdown}
\textbf{Slide 14.1:} "Privacy-First Assessment Principles" - Why privacy matters, aggregation-only principle, minimum 10 individuals, differential privacy epsilon, 72-hour temporal delay, role-based analysis, no performance evaluation use.

\textbf{Slide 14.2:} "Differential Privacy Explained" - Epsilon privacy budget concept, noise injection visual, privacy-utility tradeoff graph, CPF epsilon = 0.1 rationale, implementation example.

\textbf{Slide 14.3:} "Minimum Aggregation Units" - 10-individual requirement, calculation for different org sizes, small organization challenges, aggregation unit examples (departments, roles, locations).

\textbf{Slide 14.4:} "Temporal Delay Mechanisms" - 72-hour minimum rationale, delayed reporting workflow diagram, prevents real-time surveillance, balances timeliness with privacy.

\textbf{Slide 14.5:} "Data Handling Requisiti" - Encryption (AES-256, TLS 1.3), access controls and audit, retention limits 5 years, secure destruction, cross-border considerations, GDPR/CCPA mapping.

\textbf{Slide 14.6:} "Ethical Boundaries" - Organizational not clinical assessment, no individual diagnosis, vulnerabilities are normal, prohibition on stigma/blame, informed consent, opt-out mechanisms, whistleblower protections.

\textbf{Slide 14.7:} "Privacy Violation Case Studies" - Cambridge Analytica lessons, employee surveillance concerns, three scenario examples (what went wrong, safeguards needed), discussion prompts.

\textbf{Slide 14.8:} "Privacy Impact Assessment Exercise" - 50-person department scenario, aggregation unit calculation, differential privacy parameters, temporal delay implementation, verification no individual profiling, group exercise instructions.

\subsubsection{Materials Needed}
Workbook Module 14 (pages 61-75), differential privacy calculator tool, aggregation unit worksheet, three privacy violation case studies (2 pages each), privacy impact assessment template, GDPR/CCPA compliance checklist.

\subsubsection{Assessment Items}
\textbf{Quiz (5 questions):} Q1: Minimum aggregation unit → 10 individuals correct. Q2: CPF differential privacy epsilon → 0.1 correct. Q3: Temporal delay minimum → 72 hours correct. Q4: Data retention maximum → 5 years correct. Q5: CPF assessment type → organizational not clinical correct.

\textbf{Exercise Rubric (Privacy Impact Assessment):} Correct aggregation unit calculation (2 pts), appropriate differential privacy parameters (2 pts), temporal delay implementation (2 pts), verification no profiling possible (2 pts), complete data handling plan (2 pts). Total 10 pts (7+ pass).

\subsection{Module 15: Integration and Application}

\subsubsection{Overview}
\textbf{Duration:} 4 hours | \textbf{Slides:} 7

\textbf{Learning Objectives:} Map CPF to ISO 27001:2022 clauses; integrate CPF with NIST CSF 2.0 functions; design organizational implementation strategies; overcome common challenges; apply framework to capstone assessment.

\textbf{Key Concepts:} ISO 27001 integration, NIST CSF mapping, implementation strategy, change management, common challenges, maturity progression.

\subsubsection{Content Outline}
\textbf{1. CPF and ISO 27001:2022 Integration (60 min):} CPF-27001:2025 standard overview, mapping to ISO clauses (4.1 context, 6.1 risk assessment, 7.2 competence, 7.3 awareness, 8.2 operational planning, 9.1 monitoring, 10.1 improvement), Annex A control enhancement, PVMS as parallel to ISMS, documentation requirements, audit considerations.

\textbf{2. CPF and NIST CSF 2.0 Integration (45 min):} NIST functions mapping (Identify: CPF assessment identifies human risks, Protect: psychological controls complement technical, Detect: behavioral indicators enable detection, Respond: psychological first aid during incidents, Recover: address psychological trauma post-breach), subcategories enhancement examples, CPF as human-factor intelligence layer.

\textbf{3. Implementation Strategies (60 min):} Phased approach (assessment, pilot, rollout, continuous), stakeholder engagement (executive buy-in, middle management, staff participation), change management considerations, resource planning (personnel, technology, budget), training requirements, communication planning, quick wins identification.

\textbf{4. Common Challenges and Solutions (30 min):} Challenge: "This feels invasive" - Solution: Emphasize privacy protections and aggregation, Challenge: "Too psychological for security team" - Solution: Focus on observables not therapy, Challenge: "We don't have psychologists" - Solution: CPF training creates competence, Challenge: "ROI unclear" - Solution: Incident reduction metrics, Challenge: "Integration complexity" - Solution: Start small pilot, Challenge: "Resistance to change" - Solution: Demonstrate value through pilot.

\textbf{5. Maturity Progression (30 min):} Level 1 Foundation (CPF Score 100-149), Level 2 Intermediate (70-99), Level 3 Advanced (40-69), Level 4 Exemplary (0-39), progression pathways, capability building over time.

\textbf{6. Capstone Exercise: Complete Mini-Assessment (45 min):} Realistic organizational scenario with multiple indicators across domains, students conduct abbreviated assessment using Quick Reference Card, apply ternary scoring, calculate category and CPF scores, identify convergent states, recommend top 5 interventions, present findings to group.

\subsubsection{Teaching Methods}
\textbf{Lecture:} ISO/NIST frameworks with CPF overlay diagrams, implementation roadmap visualization, maturity level progression charts.

\textbf{Exercises:} (1) ISO clause mapping exercise - assign CPF domains to ISO clauses (20 min), (2) NIST function enhancement - design CPF integration for one function (20 min), (3) Implementation planning - create 90-day pilot plan (20 min), (4) Capstone mini-assessment (45 min).

\textbf{Discussion:} "Biggest implementation challenge anticipated?", "How gain executive buy-in?", "Integration with existing security program?"

\textbf{Case Study:} Healthcare organization CPF implementation journey (pilot to Level 2 in 18 months), lessons learned, success factors.

\subsubsection{Slide Breakdown}
\textbf{Slide 15.1:} "CPF and ISO 27001:2022" - CPF-27001:2025 overview, clause mapping table (4.1, 6.1, 7.2, 7.3, 8.2, 9.1, 10.1), Annex A enhancement examples, PVMS parallel to ISMS.

\textbf{Slide 15.2:} "CPF and NIST CSF 2.0" - Five functions with CPF integration points (Identify human risks, Protect with psychological controls, Detect via behavioral indicators, Respond with psychological first aid, Recover addressing trauma), subcategory enhancement examples, human-factor intelligence layer visual.

\textbf{Slide 15.3:} "Implementation Strategy" - Phased approach diagram (assessment, pilot, rollout, continuous), stakeholder engagement pyramid, change management considerations, resource planning checklist, quick wins identification.

\textbf{Slide 15.4:} "Common Challenges and Solutions" - Six challenge-solution pairs table (feels invasive/privacy protections, too psychological/observables focus, no psychologists/training creates competence, ROI unclear/incident metrics, integration complex/start small, resistance/demonstrate value).

\textbf{Slide 15.5:} "Maturity Progression" - Four levels visual (Foundation 100-149, Intermediate 70-99, Advanced 40-69, Exemplary 0-39), characteristics of each level, progression pathways, capability building timeline.

\textbf{Slide 15.6:} "Case Study: Healthcare Implementation" - Organization background, implementation timeline (pilot to Level 2 in 18 months), challenges encountered and solutions, key success factors, lessons learned, measurable outcomes.

\textbf{Slide 15.7:} "Capstone Exercise: Mini-Assessment" - Organizational scenario description (multi-domain vulnerabilities), assessment task instructions (identify indicators, apply scoring, calculate scores, find convergence, recommend interventions), presentation format, evaluation criteria.

\subsubsection{Materials Needed}
Workbook Module 15 (pages 76-90), ISO 27001:2022 standard (reference), NIST CSF 2.0 document (reference), CPF-27001:2025 requirements document, Quick Reference Card for capstone, capstone scenario packet (5 pages), implementation planning template, healthcare case study handout (3 pages).

\subsubsection{Assessment Items}
\textbf{Quiz (5 questions):} Q1: ISO clause CPF primarily addresses → 7.2 Competence and 7.3 Awareness correct. Q2: NIST Identify function CPF contribution → human risk identification correct. Q3: CPF Level 2 score range → 70-99 correct. Q4: First implementation phase → assessment correct. Q5: CPF-27001 standard type → organizational PVMS requirements correct.

\textbf{Capstone Rubric:} Indicator identification across domains (3 pts), accurate ternary scoring with justification (3 pts), correct category and CPF score calculation (2 pts), convergent state recognition (1 pt), appropriate intervention recommendations (1 pt). Total 10 pts (7+ pass).

\textbf{Course Completion Rubric:} All 15 module quizzes passed (15 pts), active participation in exercises (10 pts), capstone mini-assessment passed (10 pts), ethics agreement signed (5 pts). Total 40 pts (28+ pass for course completion, separate written exam required for certification).

\newpage

\section{Appendices}

\subsection{Appendix A: Complete Slide Inventory}

\begin{longtable}{|p{2cm}|p{1cm}|p{7cm}|p{2cm}|p{1.5cm}|}
\hline
\textbf{Module} & \textbf{Slide} & \textbf{Title} & \textbf{Type} & \textbf{Duration} \\
\hline
\endhead

Module 1 & 1.1 & The Human Factor Crisis & Lecture & 10 min \\
Module 1 & 1.2 & Pre-Cognitive Decision-Making & Lecture & 15 min \\
Module 1 & 1.3 & CPF Framework Architecture & Lecture & 15 min \\
Module 1 & 1.4 & Ternary Scoring System & Lecture & 15 min \\
Module 1 & 1.5 & Integration with ISO 27001 and NIST CSF & Lecture & 15 min \\
Module 1 & 1.6 & Target Breach Through CPF Lens & Case Study & 30 min \\
\hline

Module 2 & 2.1 & Bion's Basic Assumptions Overview & Lecture & 15 min \\
Module 2 & 2.2 & Basic Assumptions in Security & Lecture & 20 min \\
Module 2 & 2.3 & Kleinian Splitting and Projection & Lecture & 20 min \\
Module 2 & 2.4 & Jung's Shadow in Security & Lecture & 15 min \\
Module 2 & 2.5 & Archetypes in Security & Lecture & 15 min \\
Module 2 & 2.6 & Winnicott's Transitional Space & Lecture & 10 min \\
Module 2 & 2.7 & Psychoanalytic Case Analysis Exercise & Exercise & 15 min \\
\hline

Module 3 & 3.1 & Thinking Fast and Slow & Lecture & 20 min \\
Module 3 & 3.2 & Heuristics and Biases & Lecture & 15 min \\
Module 3 & 3.3 & Cialdini's Six Principles Overview & Lecture & 15 min \\
Module 3 & 3.4 & Six Principles in Social Engineering & Lecture & 20 min \\
Module 3 & 3.5 & Cognitive Load Theory & Lecture & 15 min \\
Module 3 & 3.6 & Decision-Making Under Uncertainty & Lecture & 10 min \\
Module 3 & 3.7 & Cognitive Exploitation Exercise & Exercise & 15 min \\
\hline

Module 4 & 4.1 & Why We Obey: Authority Vulnerability & Lecture & 20 min \\
Module 4 & 4.2 & 10 Authority Indicators & Lecture & 30 min \\
Module 4 & 4.3 & Attack Vectors in Action & Lecture & 30 min \\
Module 4 & 4.4 & Assessment and Solutions & Lecture/Exercise & 40 min \\
\hline

Module 5 & 5.1 & Time Pressure and Security & Lecture & 20 min \\
Module 5 & 5.2 & 10 Temporal Indicators & Lecture & 30 min \\
Module 5 & 5.3 & Temporal Attack Vectors & Lecture & 30 min \\
Module 5 & 5.4 & Assessment and Solutions & Lecture/Exercise & 40 min \\
\hline

Module 6 & 6.1 & Social Influence Mechanisms & Lecture & 20 min \\
Module 6 & 6.2 & 10 Social Influence Indicators & Lecture & 30 min \\
Module 6 & 6.3 & Social Engineering Attack Patterns & Lecture & 30 min \\
Module 6 & 6.4 & Assessment and Solutions & Lecture/Exercise & 40 min \\
\hline

Module 7 & 7.1 & Emotion and Security Decisions & Lecture & 20 min \\
Module 7 & 7.2 & 10 Affective Indicators & Lecture & 30 min \\
Module 7 & 7.3 & Emotional Manipulation Attacks & Lecture & 30 min \\
Module 7 & 7.4 & Assessment and Solutions & Lecture/Exercise & 40 min \\
\hline

Module 8 & 8.1 & Cognitive Load and Capacity Limits & Lecture & 20 min \\
Module 8 & 8.2 & 10 Cognitive Overload Indicators & Lecture & 30 min \\
Module 8 & 8.3 & Cognitive Exploitation Attacks & Lecture & 30 min \\
Module 8 & 8.4 & Assessment and Solutions & Lecture/Exercise & 40 min \\
\hline

Module 9 & 9.1 & Group Psychology in Security & Lecture & 20 min \\
Module 9 & 9.2 & 10 Group Dynamic Indicators & Lecture & 30 min \\
Module 9 & 9.3 & Group-Level Attack Vectors & Lecture & 30 min \\
Module 9 & 9.4 & Assessment and Solutions & Lecture/Exercise & 40 min \\
\hline

Module 10 & 10.1 & Stress and Security Performance & Lecture & 20 min \\
Module 10 & 10.2 & 10 Stress Response Indicators & Lecture & 30 min \\
Module 10 & 10.3 & Stress Exploitation Attacks & Lecture & 30 min \\
Module 10 & 10.4 & Assessment and Solutions & Lecture/Exercise & 40 min \\
\hline

Module 11 & 11.1 & The Unconscious in Security & Lecture & 20 min \\
Module 11 & 11.2 & 10 Unconscious Process Indicators & Lecture & 30 min \\
Module 11 & 11.3 & Unconscious Exploitation & Lecture & 30 min \\
Module 11 & 11.4 & Assessment and Solutions & Lecture/Exercise & 40 min \\
\hline

Module 12 & 12.1 & AI-Human Interaction Psychology & Lecture & 20 min \\
Module 12 & 12.2 & 10 AI-Specific Bias Indicators & Lecture & 30 min \\
Module 12 & 12.3 & AI Exploitation Attacks & Lecture & 30 min \\
Module 12 & 12.4 & Assessment and Solutions & Lecture/Exercise & 40 min \\
\hline

Module 13 & 13.1 & Vulnerability Convergence & Lecture & 20 min \\
Module 13 & 13.2 & 10 Critical Convergent Indicators & Lecture & 30 min \\
Module 13 & 13.3 & Convergent State Attacks & Lecture & 30 min \\
Module 13 & 13.4 & Assessment and Solutions & Lecture/Exercise & 40 min \\
\hline

Module 14 & 14.1 & Privacy-First Assessment Principles & Lecture & 20 min \\
Module 14 & 14.2 & Differential Privacy Explained & Lecture & 15 min \\
Module 14 & 14.3 & Minimum Aggregation Units & Lecture & 15 min \\
Module 14 & 14.4 & Temporal Delay Mechanisms & Lecture & 10 min \\
Module 14 & 14.5 & Data Handling Requisiti & Lecture & 15 min \\
Module 14 & 14.6 & Ethical Boundaries & Lecture & 15 min \\
Module 14 & 14.7 & Privacy Violation Case Studies & Case Study & 20 min \\
Module 14 & 14.8 & Privacy Impact Assessment Exercise & Exercise & 30 min \\
\hline

Module 15 & 15.1 & CPF and ISO 27001:2022 & Lecture & 30 min \\
Module 15 & 15.2 & CPF and NIST CSF 2.0 & Lecture & 30 min \\
Module 15 & 15.3 & Implementation Strategy & Lecture & 30 min \\
Module 15 & 15.4 & Common Challenges and Solutions & Lecture & 15 min \\
Module 15 & 15.5 & Maturity Progression & Lecture & 15 min \\
Module 15 & 15.6 & Case Study: Healthcare Implementation & Case Study & 20 min \\
Module 15 & 15.7 & Capstone Exercise: Mini-Assessment & Exercise & 45 min \\
\hline

\multicolumn{5}{|c|}{\textbf{Total: 80 slides, 40 hours}} \\
\hline

\end{longtable}

\subsection{Appendix B: Exercise Bank Summary}

\textbf{Module 1 Exercises:}
\begin{itemize}
\item 1.1 Awareness Failure Analysis (15 min): Share failed training examples, identify why conscious interventions didn't work
\item 1.2 Pre-Cognitive Decision Experiment (10 min): Live demonstration authority/urgency scenarios, reflect on rapid decisions
\item 1.3 Framework Navigation (20 min): Speed drill with Quick Reference Card, locate indicators across domains
\end{itemize}

\textbf{Module 2 Exercises:}
\begin{itemize}
\item 2.1 Basic Assumption Identification (20 min): Three vignettes, identify baD/baF/baP and vulnerabilities
\item 2.2 Splitting in Security Culture (15 min): List trusted/threatening entities, discuss blind spots from idealization/demonization
\item 2.3 Shadow Recognition (15 min): Anonymous reflection "what org refuses to acknowledge"
\item 2.4 Psychoanalytic Case Analysis (15 min): Healthcare ransomware, identify basic assumption, splitting, projections
\end{itemize}

\textbf{Module 3 Exercises:}
\begin{itemize}
\item 3.1 System 1 vs 2 Speed Test (15 min): 20 emails 3 sec each then unlimited time, compare accuracy rates
\item 3.2 Cialdini Principle Mapping (20 min): Six scenarios, map influence principles, discuss combinations
\item 3.3 Cognitive Load Simulation (15 min): Security task with distractions, experience overload degradation
\item 3.4 Cognitive Exploitation Analysis (15 min): Three phishing emails, identify principles/System/load manipulation
\end{itemize}

\textbf{Modules 4-13 Domain Exercises (10 min each):}
Each domain includes scenario-based exercise applying ternary scoring to realistic case with discussion.

\textbf{Module 14 Exercises:}
\begin{itemize}
\item 14.1 Aggregation Unit Calculation (15 min): Calculate for various org sizes, ensure privacy requirements
\item 14.2 Differential Privacy Parameters (15 min): Select appropriate epsilon, understand privacy-utility tradeoff
\item 14.3 Privacy Impact Assessment (30 min): Design assessment for 50-person department, verify no profiling
\end{itemize}

\textbf{Module 15 Exercises:}
\begin{itemize}
\item 15.1 ISO Clause Mapping (20 min): Assign CPF domains to ISO 27001 clauses
\item 15.2 NIST Function Enhancement (20 min): Design CPF integration for one NIST function
\item 15.3 Implementation Planning (20 min): Create 90-day pilot plan with stakeholders/resources
\item 15.4 Capstone Mini-Assessment (45 min): Complete abbreviated assessment with scoring and recommendations
\end{itemize}

\textbf{Total: 25 exercises across 40 hours}

\subsection{Appendix C: Examination Blueprint}

\textbf{CPF-101 Written Examination Structure:}

\textbf{Format:} 100 questions, 3 hours, closed-book, computer-based

\textbf{Question Types:}
\begin{itemize}
\item 60 Multiple-Choice: Single correct answer from 4 options
\item 30 Scenario-Based: Short scenario with question requiring analysis
\item 10 Case Analysis: Extended case study with complex multi-step questions
\end{itemize}

\textbf{Content Distribution by Module:}

\begin{tabular}{|l|c|p{6cm}|}
\hline
\textbf{Module} & \textbf{Questions} & \textbf{Focus Areas} \\
\hline
Module 1 & 8 & Pre-cognitive processes, framework architecture, integration \\
Module 2 & 8 & Bion, Klein, Jung, Winnicott concepts in security \\
Module 3 & 8 & Kahneman, Cialdini, Miller, cognitive biases \\
Modules 4-13 & 50 & 5 questions per domain, ternary scoring, attack vectors, solutions \\
Module 14 & 12 & Privacy requirements, differential privacy, ethics \\
Module 15 & 14 & ISO/NIST integration, implementation, capstone scenarios \\
\hline
\textbf{Total} & \textbf{100} & \\
\hline
\end{tabular}

\textbf{Cognitive Level Distribution (Bloom's Taxonomy):}
\begin{itemize}
\item Knowledge/Recall: 20\% (20 questions) - Facts, definitions, terminology
\item Comprehension/Application: 40\% (40 questions) - Explain concepts, apply to scenarios
\item Analysis/Synthesis: 40\% (40 questions) - Analyze complex situations, integrate multiple concepts
\end{itemize}

\textbf{Passing Standard:} 70\% (70 correct responses)

\textbf{Question Development Process:}
\begin{itemize}
\item Psychometric validation with pilot groups
\item Item difficulty distribution: 30\% easy, 50\% moderate, 20\% difficult
\item Regular statistical analysis (discrimination index, difficulty index)
\item Continuous improvement based on performance data
\end{itemize}

\textbf{Retake Policy:}
\begin{itemize}
\item First retake: 30-day waiting period, 50\% fee
\item Second retake: 30-day waiting period, 50\% fee
\item After three failures: Additional training required, 6-month waiting period
\end{itemize}

\subsection{Appendix D: Reference Materials}

\textbf{CPF Framework Documents:}
\begin{itemize}
\item The Cybersecurity Psychology Framework: Complete taxonomy paper with all 100 indicators
\item CPF-27001:2025 Requisiti: Organizational PVMS standard
\item CPF Schema di Certificazione: Professional certification pathways and requirements
\item Field Kit Example: Indicator 1.1 complete (foundation, operational, field kit)
\end{itemize}

\textbf{Foundational Research Papers:}
\begin{itemize}
\item Milgram, S. (1974). Obedience to Authority
\item Bion, W. R. (1961). Experiences in Groups
\item Klein, M. (1946). Notes on some schizoid mechanisms
\item Jung, C. G. (1969). The Archetypes and the Collective Unconscious
\item Winnicott, D. W. (1971). Playing and Reality
\item Kahneman, D. (2011). Thinking, Fast and Slow
\item Kahneman, D. \& Tversky, A. (1979). Prospect Theory
\item Cialdini, R. B. (2007). Influence: The Psychology of Persuasion
\item Miller, G. A. (1956). The Magical Number Seven, Plus or Minus Two
\end{itemize}

\textbf{Security Framework Standards:}
\begin{itemize}
\item ISO/IEC 27001:2022 Information Security Management Systems
\item ISO/IEC 27002:2022 Code of Practice for Information Security Controls
\item NIST Cybersecurity Framework 2.0
\item ISO 19011:2018 Guidelines for Auditing Management Systems (for Auditor track)
\end{itemize}

\textbf{Privacy and Ethics References:}
\begin{itemize}
\item GDPR (General Data Protection Regulation)
\item CCPA (California Consumer Privacy Act)
\item Differential Privacy: A Survey of Results (Dwork, 2008)
\item APA Ethical Principles of Psychologists and Code of Conduct
\end{itemize}

\textbf{Field Kits Referenced (by module):}
\begin{itemize}
\item Module 4: Field Kit 1.1 (Unquestioning compliance), 1.3 (Impersonation susceptibility)
\item Module 5: Field Kit 2.1 (Urgency bypass), 2.3 (Deadline risk acceptance)
\item Module 6: Field Kit 3.1 (Reciprocity exploitation), 3.3 (Social proof manipulation)
\item Module 7: Field Kit 4.1 (Fear paralysis), 4.5 (Shame hiding)
\item Module 8: Field Kit 5.1 (Alert fatigue), 5.2 (Decision fatigue)
\item Module 9: Field Kit 6.1 (Groupthink), 6.3 (Diffusion of responsibility)
\item Module 10: Field Kit 7.1 (Acute stress), 7.2 (Chronic burnout)
\item Module 11: Field Kit 8.1 (Shadow projection), 8.4 (Transference)
\item Module 12: Field Kit 9.1 (Anthropomorphization), 9.2 (Automation bias)
\item Module 13: Field Kit 10.1 (Perfect storm), 10.4 (Swiss cheese alignment)
\end{itemize}

\textbf{Note:} All 100 Field Kits available as separate reference library for certified assessors.

\section*{Document Control}

\textbf{Version History:}

\begin{tabular}{llp{8cm}}
\toprule
Version & Date & Changes \\
\midrule
1.0 & January 2025 & Initial release \\
\bottomrule
\end{tabular}

\vspace{1em}

\textbf{Review Schedule:}
Annual review following each course delivery, major revision based on examination statistics, participant feedback, and framework updates.

\textbf{Approval:}

Document Owner: CPF3 Training Development

Approved by: Giuseppe Canale, CISSP

Date: January 2025

\textbf{Usage Instructions:}

This blueprint enables modular slide generation using the following workflow:

\begin{enumerate}
\item Select module from Section 2 (Module Structures)
\item Review module overview, content outline, teaching methods, and slide breakdown
\item Generate slide content using AI assistance with prompt structure: "Generate slide content for [Module X, Slide Y] based on CPF-101-Training-Blueprint.tex Section 2.X. Include [specified materials]. Output format: [title, bullets, notes, visual suggestions]."
\item Reference appropriate Field Kits and Taxonomy sections as specified in Materials Needed
\item Implement exercises from Appendix B with provided rubrics
\item Develop assessment items following Appendix C blueprint
\end{enumerate}

\textbf{Contact Information:}

CPF3 Training Development

Website: https://cpf3.org

Email: training@cpf3.org

\vspace{2em}

\begin{center}
\textit{End of CPF-101 Progetto di Formazione}
\end{center}

\end{document}