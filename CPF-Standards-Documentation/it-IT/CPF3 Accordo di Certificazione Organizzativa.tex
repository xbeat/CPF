\documentclass[11pt,a4paper]{article}

% Packages
\usepackage[utf8]{inputenc}
\usepackage[italian]{babel}
\usepackage[margin=2.5cm]{geometry}
\usepackage{hyperref}
\usepackage{fancyhdr}
\usepackage{enumitem}
\usepackage{tabularx}
\usepackage{amssymb}

% Page style
\pagestyle{fancy}
\fancyhf{}
\renewcommand{\headrulewidth}{0.4pt}
\fancyhead[L]{CPF Accordo di Certificazione Organizzativa}
\fancyhead[R]{Cert \#: \_\_\_\_\_\_\_\_}
\fancyfoot[C]{\thepage}

% Spacing
\setlength{\parindent}{0pt}
\setlength{\parskip}{0.8em}

% Title
\title{\textbf{CPF ACCORDO DI\\CERTIFICAZIONE ORGANIZZATIVA}}
\author{}
\date{}

\begin{document}

\maketitle

\section*{PARTI}

Questo Accordo di Certificazione Organizzativa ("Accordo") è stipulato alla data del \_\_\_ giorno di \_\_\_\_\_\_\_\_\_\_\_, 20\_\_\_ ("Data di Efficacia"), tra:

\textbf{[NOME ORGANISMO DI CERTIFICAZIONE]} ("Organismo di Certificazione" o "OC")\\
Una [giurisdizione] [tipo di entità]\\
Organismo di Certificazione CPF Autorizzato\\
Sede Principale: [Indirizzo]\\
Email: [Email]

E

\textbf{[NOME ORGANIZZAZIONE]} ("Organizzazione" o "Organizzazione Certificata" al momento della certificazione)\\
Una [giurisdizione] [tipo di entità]\\
Numero di Registrazione: [Numero]\\
Sede Principale: [Indirizzo]\\
Email: [Email]

Collettivamente denominati le "Parti" e individualmente come "Parte."

\section*{PREMESSE}

CONSIDERATO CHE, l'Organismo di Certificazione è autorizzato da CPF3 a gestire lo Schema di Certificazione CPF e certificare le organizzazioni per la maturità nella gestione delle vulnerabilità psicologiche;

CONSIDERATO CHE, l'Organizzazione desidera ottenere la certificazione organizzativa nell'ambito dello Schema di Certificazione CPF a uno dei quattro livelli di conformità (Livello 1-4);

CONSIDERATO CHE, l'Organizzazione ha implementato o sta implementando la metodologia CPF e i requisiti CPF-27001:2025;

CONSIDERATO CHE, l'Organismo di Certificazione è disposto a valutare l'implementazione dell'Organizzazione e concedere la certificazione se i requisiti sono soddisfatti;

PERTANTO, in considerazione dei reciproci patti e accordi qui contenuti, le Parti concordano quanto segue:

\section{DEFINIZIONI}

\textbf{1.1 "Certificazione"} significa l'attestazione formale da parte dell'Organismo di Certificazione che l'Organizzazione ha raggiunto uno dei seguenti Livelli di Conformità CPF:
\begin{itemize}
\item Livello 1: Foundation (CPF Score 100-149)
\item Livello 2: Intermediate (CPF Score 70-99)
\item Livello 3: Advanced (CPF Score 40-69)
\item Livello 4: Exemplary (CPF Score 0-39)
\end{itemize}

\textbf{1.2 "CPF Score"} significa il punteggio aggregato di vulnerabilità (range 0-200) dove punteggi più bassi indicano una migliore postura di sicurezza.

\textbf{1.3 "Ambito di Certificazione"} significa le unità organizzative, le sedi e il personale coperti dalla certificazione, come dettagliato nell'Allegato A.

\textbf{1.4 "CPF-27001:2025"} significa lo standard dei requisiti del sistema di gestione CPF.

\textbf{1.5 "Audit di Sorveglianza"} significa l'audit periodico per verificare la conformità continua.

\textbf{1.6 "Non Conformità"} significa il mancato soddisfacimento di un requisito.

\section{AMBITO E LIVELLO DI CERTIFICAZIONE}

\textbf{2.1 Ambito di Certificazione.} La certificazione copre:

\begin{itemize}
\item[] Entità Legale: \underline{\hspace{10cm}}
\item[] Unità di Business: \underline{\hspace{10cm}}
\item[] Sedi: \underline{\hspace{10cm}}
\item[] Personale Totale nell'Ambito: \underline{\hspace{4cm}}
\item[] Esclusioni: \underline{\hspace{10cm}}
\end{itemize}

Ambito dettagliato nell'Allegato A.

\textbf{2.2 Livello di Certificazione Target:}

\begin{itemize}
\item[$\square$] \textbf{Livello 1: Foundation} (CPF Score 100-149)
\item[$\square$] \textbf{Livello 2: Intermediate} (CPF Score 70-99)
\item[$\square$] \textbf{Livello 3: Advanced} (CPF Score 40-69)
\item[$\square$] \textbf{Livello 4: Exemplary} (CPF Score 0-39)
\end{itemize}

\section{PROCESSO DI CERTIFICAZIONE}

\textbf{3.1 Fase di Domanda.} L'Organizzazione deve presentare:
\begin{itemize}
\item Modulo di domanda compilato
\item Report di valutazione CPF valido da Assessor/Auditor certificato
\item Policy CPF approvata dal senior management
\item Organigramma che mostra i ruoli CPF
\item Procedure di protezione della privacy
\item Piani di trattamento dei rischi per gli indicatori Red
\item Evidenza dell'integrazione ISMS
\item Lettera di impegno del management
\item Pagamento della quota di domanda
\end{itemize}

\textbf{3.2 Revisione della Domanda.} Entro 15 giorni lavorativi, l'Organismo di Certificazione deve:
\begin{itemize}
\item Verificare la completezza
\item Verificare il CPF Score e la validità della valutazione
\item Revisionare la documentazione per la conformità al livello target
\item Approvare per l'audit o richiedere informazioni aggiuntive
\item Assegnare un CPF Auditor qualificato
\end{itemize}

\textbf{3.3 Audit di Certificazione.}

\textit{Stage 1 (Revisione Documentale, 1-3 giorni):}
\begin{itemize}
\item Revisione della policy e delle procedure CPF
\item Valutazione della prontezza per lo Stage 2
\item Identificazione delle lacune che richiedono correzione
\end{itemize}

\textit{Stage 2 (Revisione dell'Implementazione, 3-10 giorni):}
\begin{itemize}
\item Verifica del CPF Score attraverso campionamento
\item Revisione della metodologia e delle protezioni della privacy
\item Verifica del trattamento dei rischi
\item Interviste con il management e il personale
\item Revisione delle evidenze per i requisiti del livello target
\item Valutazione dell'integrazione ISMS
\item Valutazione dell'efficacia
\end{itemize}

\textit{Reporting dell'Audit (15 giorni lavorativi):}
\begin{itemize}
\item Riunioni di apertura e chiusura
\item Report di audit scritto
\item Risultati: NC Maggiore, NC Minore, Osservazione, Opportunità
\end{itemize}

\textbf{3.4 Azioni Correttive.} Se ci sono non conformità:
\begin{itemize}
\item L'Organizzazione presenta il piano entro 30 giorni
\item Le NC Maggiori corrette prima della certificazione
\item Le NC Minori correggibili entro 90 giorni dopo
\item Verifica dell'efficacia
\end{itemize}

\textbf{3.5 Decisione di Certificazione.} Entro 15 giorni lavorativi:
\begin{itemize}
\item Concedere al livello appropriato
\item Emettere il certificato e autorizzare l'uso del Marchio
\item Aggiungere al registro pubblico
\item Stabilire il programma di sorveglianza
\item Oppure negare con spiegazione e diritti di appello
\end{itemize}

\section{CONCESSIONE DELLA CERTIFICAZIONE E DIRITTI}

\textbf{4.1 Concessione della Certificazione:}
\begin{itemize}
\item Certificazione del Livello di Conformità CPF
\item Diritto di utilizzare il Marchio di Certificazione
\item Inserimento nel registro pubblico
\item Certificato valido 3 anni
\item Accesso alle risorse
\end{itemize}

\textbf{4.2 Uso del Marchio di Certificazione:}
\begin{itemize}
\item Sito web e materiali di marketing
\item Proposte e presentazioni
\item Sedi degli uffici
\item Firme email
\item Social media
\item Dichiarare: "Organizzazione Certificata CPF - Livello [X]"
\end{itemize}

\textbf{4.3 Restrizioni:}
\begin{itemize}
\item Nessuna modifica al Marchio
\item Non su prodotti/servizi (si applica all'organizzazione)
\item Non per livello superiore a quello certificato
\item Non al di fuori dell'ambito di certificazione
\item Non dopo scadenza/sospensione/revoca
\item Nessun trasferimento o sublicenza
\item Nessuna dichiarazione fuorviante
\end{itemize}

\section{OBBLIGHI}

\textbf{5.1 Mantenimento:}
\begin{itemize}
\item Mantenere la gestione sistematica delle vulnerabilità
\item Continuare l'implementazione CPF-27001:2025
\item Mantenere/migliorare il CPF Score entro il livello
\item Aggiornare i trattamenti dei rischi
\item Mantenere pratiche di preservazione della privacy
\item Fornire risorse adeguate
\end{itemize}

\textbf{5.2 Personale:}
\begin{itemize}
\item Mantenere il CPF Coordinator
\item Livello 2+: Minimo 1 Assessor certificato
\item Livello 3+: Minimo 2 Assessor certificati
\item Livello 4: Team dedicato con Auditor
\item Assicurare il mantenimento CPE
\item Fornire formazione di sensibilizzazione
\end{itemize}

\textbf{5.3 Valutazione e Monitoraggio:}
\begin{itemize}
\item Livello 1: Valutazione annuale
\item Livello 2: Cicli trimestrali
\item Livello 3+: Monitoraggio continuo
\item Utilizzare professionisti certificati
\item Mantenere la documentazione
\item Tracciare i trend
\item Riportare gli indicatori Red secondo i requisiti del livello
\end{itemize}

\textbf{5.4 Revisione del Management:}
\begin{itemize}
\item Livello 1: Annuale
\item Livello 2: Semestrale
\item Livello 3+: Trimestrale
\item Documentare le revisioni con metriche, decisioni, azioni
\end{itemize}

\textbf{5.5 Riduzione degli Incidenti:}
\begin{itemize}
\item Tracciare gli incidenti legati al fattore umano
\item Stabilire la baseline
\item Livello 2: Riduzione del 20\%
\item Livello 3: Riduzione del 40\%
\item Livello 4: Riduzione del 60\%
\item Documentare le evidenze
\end{itemize}

\textbf{5.6 Privacy ed Etica:}
\begin{itemize}
\item Framework di protezione della privacy
\item Mai utilizzare per profilazione individuale
\item Aggregazione minima (10 individui)
\item Livello 3+: Privacy differenziale ($\varepsilon \leq 0.1$)
\item Reporting con ritardo temporale (72 ore)
\item Archiviazione e trasmissione sicure
\item Livello 3-4: Audit esterno annuale sulla privacy
\end{itemize}

\textbf{5.7 Modifiche all'Ambito.} Notificare entro 30 giorni:
\begin{itemize}
\item Cambiamenti organizzativi
\item Espansioni/riduzioni dell'ambito
\item Cambiamenti del personale (>20\%)
\item Cambiamenti del CPF Coordinator
\item Qualsiasi cosa impatti la certificazione
\end{itemize}

\textbf{5.8 Cooperazione:}
\begin{itemize}
\item Concedere accesso per la sorveglianza
\item Rispondere alle richieste tempestivamente
\item Notificare immediatamente: violazioni, aumenti del punteggio, reclami, azioni legali, perdita di personale
\item Implementare azioni correttive
\end{itemize}

\section{SORVEGLIANZA}

\textbf{6.1 Requisiti per Livello:}

\textit{Livello 1:}
\begin{itemize}
\item Sorveglianza annuale da Assessor (1-2 giorni)
\item Revisione del programma e dei risultati
\end{itemize}

\textit{Livello 2:}
\begin{itemize}
\item Biennale da Auditor (2-3 giorni)
\item Desk review trimestrale
\item Verifica della riduzione degli incidenti
\end{itemize}

\textit{Livello 3:}
\begin{itemize}
\item Annuale da Auditor (3-5 giorni)
\item Desk review trimestrale del monitoraggio
\item Verifica annuale dell'audit sulla privacy
\end{itemize}

\textit{Livello 4:}
\begin{itemize}
\item Annuale da Auditor esterno (5-7 giorni)
\item Desk review mensile
\item Audit esterno trimestrale sulla privacy
\item Peer review biennale
\end{itemize}

\textbf{6.2 Processo:}
\begin{itemize}
\item Preavviso di 30 giorni
\item Focus: Trend del Score, metodologia, privacy, revisione del management, incidenti, cambiamenti
\item Risultati documentati
\item Azioni correttive per le NC
\end{itemize}

\textbf{6.3 Risultati:}
\begin{itemize}
\item Nessuna NC: Continuare
\item NC Minori: Piano entro 30 giorni, implementazione entro 90
\item NC Maggiori: Azione immediata, sospensione se non corretto in 90 giorni
\end{itemize}

\textbf{6.4 Monitoraggio del CPF Score:}
\begin{itemize}
\item Miglioramento: Può richiedere upgrade
\item Degradazione fuori range: 90 giorni per ripristinare o downgrade
\item Score >149: Sospensione in attesa di azione correttiva
\end{itemize}

\section{RICERTIFICAZIONE}

\textbf{7.1 Requisito.} Ogni 3 anni.

\textbf{7.2 Processo:}
\begin{itemize}
\item Notifica 180 giorni prima della scadenza
\item Domanda 120 giorni prima
\item Audit di ricertificazione completo
\item Valutazione completa del CPF Score
\item Revisione dei trend triennali
\item Valutazione del miglioramento continuo
\item Audit minimo 60 giorni prima della scadenza
\item Decisione entro 30 giorni
\item Nuovo certificato con date aggiornate
\item Il livello può cambiare in base al punteggio attuale
\end{itemize}

\textbf{7.3 Tempistiche:}
\begin{itemize}
\item Anticipata: Fino a 6 mesi prima (nuovo periodo dalla data effettiva)
\item In ritardo: Ricertificazione completa come nuovo richiedente
\item Nessun periodo di grazia
\end{itemize}

\section{TARIFFE}

\textbf{8.1 Tariffa di Domanda:}

\begin{tabular}{|l|c|}
\hline
1-50 dipendenti & \$500 \\
51-250 & \$1,000 \\
251-1000 & \$1,500 \\
1000+ & \$2,000 \\
\hline
\end{tabular}

Non rimborsabile.

\textbf{8.2 Tariffe di Audit:}

\begin{tabular}{|l|c|c|}
\hline
Dimensione & Stage 1 & Stage 2 \\
\hline
1-50 & \$2,000 & \$4,000 \\
51-250 & \$3,000 & \$7,000 \\
251-1000 & \$5,000 & \$12,000 \\
1000+ & \$8,000 & \$20,000 \\
\hline
\end{tabular}

Complesso/multi-sito: Aggiuntivo \$1,500/giorno

\textbf{8.3 Tariffa di Certificazione:}

\begin{tabular}{|l|c|}
\hline
1-50 & \$1,000 \\
51-250 & \$2,000 \\
251-1000 & \$3,500 \\
1000+ & \$5,000 \\
\hline
\end{tabular}

\textbf{8.4 Sorveglianza Annuale:}

\begin{tabular}{|l|c|}
\hline
Livello 1 & 30\% dell'audit iniziale \\
Livello 2 & 40\% (biennale) \\
Livello 3 & 50\% \\
Livello 4 & 60\% \\
\hline
\end{tabular}

\textbf{8.5 Ricertificazione:}
\begin{itemize}
\item Audit: 75\% dell'iniziale
\item Tariffa: Stessa dell'iniziale
\end{itemize}

\textbf{8.6 Altro:}
\begin{itemize}
\item Espansione dell'ambito: \$1,000-\$5,000
\item Follow-up per NC maggiori: \$1,500/giorno
\item Upgrade di livello: \$2,000-\$8,000
\item Accelerato: Sovrattassa del 25\%
\item Viaggio: Costi effettivi
\end{itemize}

\textbf{8.7 Pagamento:}
\begin{itemize}
\item Domanda: Con la presentazione
\item Stage 1: Prima dell'audit
\item Stage 2: Prima dell'audit
\item Certificazione: Al momento della decisione
\item Sorveglianza: 30 giorni prima
\item Tutte le tariffe in USD
\item In ritardo: Interesse mensile dell'1.5\%
\item Servizi sospesi se >60 giorni in arretrato
\end{itemize}

\section{SOSPENSIONE E REVOCA}

\textbf{9.1 Motivi di Sospensione:}
\begin{itemize}
\item Score fuori range
\item NC Maggiore non corretta (90 giorni)
\item Mancato completamento della sorveglianza
\item Mancato pagamento delle tariffe
\item Violazione della privacy
\item Perdita di personale chiave
\item Cambiamenti organizzativi maggiori
\item Uso improprio del Marchio
\end{itemize}

\textbf{9.2 Processo di Sospensione:}
\begin{itemize}
\item Avviso scritto con i motivi
\item Restrizione immediata sul nuovo uso del Marchio
\item Registro: "Sospeso"
\item Usi esistenti: Aggiungere "Certificazione Sospesa"
\item Massimo 180 giorni
\item Piano di rimedio (30 giorni)
\item Audit di verifica può essere richiesto
\item Reintegrazione al momento del rimedio
\item Revoca se non rimediato
\end{itemize}

\textbf{9.3 Motivi di Revoca:}
\begin{itemize}
\item Mancato rimedio (180 giorni)
\item Violazioni gravi della privacy
\item Frode/dichiarazione falsa/falsificazione
\item Violazioni sistematiche di CPF-27001
\item Profilazione individuale
\item Violazione materiale
\item Uso improprio persistente del Marchio
\item Rifiuto di cooperare
\item Insolvenza/fallimento
\end{itemize}

\textbf{9.4 Processo di Revoca:}
\begin{itemize}
\item Avviso scritto con i motivi
\item 30 giorni per rispondere
\item Revisione da parte di comitato indipendente
\item Decisione entro 45 giorni
\item Se revocato: Cessazione immediata, rimozione dal registro, avviso pubblico, restituzione del certificato, nessun rimborso, divieto di riapplicazione per 2 anni
\item Diritto di appello
\end{itemize}

\textbf{9.5 Ritiro Volontario:}
\begin{itemize}
\item Preavviso di 30 giorni
\item Cessazione immediata
\item Restituzione del certificato
\item Nessun rimborso
\item Può riapplicare in qualsiasi momento
\end{itemize}

\section{APPELLI}

\textbf{10.1 Diritto di Appello:}
\begin{itemize}
\item Diniego della certificazione
\item Determinazione del livello
\item Sospensione
\item Revoca
\item Downgrade
\item Controversie su NC Maggiori
\end{itemize}

\textbf{10.2 Processo:}
\begin{itemize}
\item Scritto entro 30 giorni
\item Tariffa: \$500
\item Motivi ed evidenze
\item Pannello indipendente
\item Decisione entro 45 giorni
\item Opzioni: Confermare/Modificare/Annullare/Rinviare
\item Tariffa rimborsata se successo
\item Finale e vincolante
\end{itemize}

\section{RISERVATEZZA}

\textbf{11.1 Riservatezza dell'OC:}
\begin{itemize}
\item Mantenere la riservatezza di: Dati di valutazione, score, documenti interni, informazioni aziendali, metodi di privacy, risultati
\item Limitare l'accesso al team di audit
\item Non divulgare eccetto: Informazioni del registro pubblico, a CPF3, agli organismi di accreditamento, come richiesto dalla legge
\item Il personale firma accordi di riservatezza
\end{itemize}

\textbf{11.2 Protezione dei Dati:}
\begin{itemize}
\item Conformità con GDPR/CCPA
\item Implementare misure di sicurezza
\item Elaborare solo per la certificazione
\item Notificare le violazioni (24 ore)
\item Cooperare nella risposta alle violazioni
\end{itemize}

\textbf{11.3 Conservazione:}
\begin{itemize}
\item Registri: 7 anni dopo scadenza/revoca
\item Report di audit: 7 anni
\item Appelli/reclami: 10 anni
\item Distruzione sicura
\end{itemize}

\section{LIMITAZIONE DI RESPONSABILITÀ}

\textbf{12.1 Esclusione di Responsabilità.} NESSUNA GARANZIA RIGUARDO RISULTATI DI BUSINESS, PREVENZIONE DEGLI INCIDENTI, CONFORMITÀ NORMATIVA O MIGLIORAMENTI ASSICURATIVI.

\textbf{12.2 Limitazione.} NESSUNA RESPONSABILITÀ PER DANNI INDIRETTI, CONSEQUENZIALI, SPECIALI O PUNITIVI.

\textbf{12.3 Limite Massimo.} RESPONSABILITÀ TOTALE NON SUPERIORE ALLE TARIFFE PAGATE NEI 12 MESI PRECEDENTI IL RECLAMO.

\textbf{12.4 Eccezioni:} Negligenza grave, violazioni della riservatezza, violazioni della protezione dei dati, reclami non permessi di limitare per legge.

\section{INDENNIZZO}

\textbf{13.1 Da parte dell'Organizzazione:} Da reclami derivanti da uso improprio del Marchio, dichiarazioni false, violazioni della privacy, informazioni false, reclami di terze parti.

\textbf{13.2 Da parte dell'OC:} Da violazione della riservatezza, negligenza nell'audit, violazioni dei dati.

\section{DISPOSIZIONI GENERALI}

\textbf{14.1 Legge Applicabile.} [Giurisdizione]

\textbf{14.2 Controversie.} Negoziazione, mediazione, poi arbitrato.

\textbf{14.3 Accordo Completo.} Questo Accordo e gli Allegati.

\textbf{14.4 Modifica.} L'OC può modificare CPF-27001 (preavviso di 180 giorni).

\textbf{14.5 Cessione.} L'Organizzazione non può cedere; l'OC può per trasferimento di attività.

\textbf{14.6 Forza Maggiore.} Nessuna responsabilità per eventi al di fuori del controllo.

\textbf{14.7 Comunicazioni.} Scritte agli indirizzi dichiarati.

\textbf{14.8 Separabilità.} Disposizioni invalide riformate.

\textbf{14.9 Sopravvivenza.} Le Sezioni 10, 12, 13, 14 sopravvivono.

\section*{FIRME}

\textbf{ORGANISMO DI CERTIFICAZIONE:}

Da: \underline{\hspace{6cm}} Data: \underline{\hspace{3cm}}

Nome: \underline{\hspace{6cm}} Titolo: \underline{\hspace{6cm}}

\vspace{2em}

\textbf{ORGANIZZAZIONE:}

Da: \underline{\hspace{6cm}} Data: \underline{\hspace{3cm}}

Nome: \underline{\hspace{6cm}} Titolo: \underline{\hspace{6cm}}

\newpage

\section*{ALLEGATO A: AMBITO DI CERTIFICAZIONE}

Entità Legale: \underline{\hspace{12cm}}

Unità di Business: \underline{\hspace{12cm}}

Sedi: \underline{\hspace{12cm}}

Personale Totale: \underline{\hspace{4cm}}

Esclusioni: \underline{\hspace{12cm}}

Giustificazione: \underline{\hspace{12cm}}

Approvato da:

OC: \underline{\hspace{5cm}} Data: \underline{\hspace{3cm}}

Org: \underline{\hspace{5cm}} Data: \underline{\hspace{3cm}}

\end{document}
