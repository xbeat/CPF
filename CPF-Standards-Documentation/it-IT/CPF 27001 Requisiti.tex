\documentclass[11pt,a4paper]{article}

% Pacchetti
\usepackage[utf8]{inputenc}
\usepackage[italian]{babel}
\usepackage[margin=2.5cm]{geometry}
\usepackage{amsmath}
\usepackage{booktabs}
\usepackage{hyperref}
\usepackage{fancyhdr}

% Stile pagina
\pagestyle{fancy}
\fancyhf{}
\renewcommand{\headrulewidth}{0.4pt}
\fancyhead[L]{CPF-27001:2025}
\fancyhead[R]{Requisiti}
\fancyfoot[C]{\thepage}

% Spacing
\setlength{\parindent}{0pt}
\setlength{\parskip}{0.5em}

% Hyperref setup
\hypersetup{
    colorlinks=true,
    linkcolor=blue,
    citecolor=blue,
    urlcolor=blue,
    pdftitle={CPF-27001:2025 Requisiti},
    pdfauthor={Giuseppe Canale, CISSP},
}

% Titolo
\title{\textbf{CPF-27001:2025}\\
\large Sistema di Gestione della Vulnerabilità Psicologica\\
Requisiti}
\author{Giuseppe Canale, CISSP\\
\small Ricercatore Indipendente\\
\small g.canale@cpf3.org}
\date{Gennaio 2025}

\begin{document}

\maketitle

\begin{abstract}
Questo documento specifica i requisiti per stabilire, implementare, mantenere e migliorare continuamente un Sistema di Gestione della Vulnerabilità Psicologica (PVMS) all'interno delle organizzazioni. CPF-27001:2025 affronta il gap critico nei framework di cybersecurity fornendo requisiti sistematici per identificare e mitigare le vulnerabilità psicologiche pre-cognitive che contribuiscono all'82-85\% degli incidenti di sicurezza. A differenza degli standard tradizionali di security awareness che si concentrano sul processo decisionale consapevole, CPF-27001 stabilisce requisiti per valutare i processi inconsci, le dinamiche di gruppo e gli stati affettivi che abilitano il social engineering e le violazioni legate al fattore umano. Lo standard è applicabile a tutte le organizzazioni indipendentemente dal tipo, dimensione o natura, ed è progettato per integrarsi perfettamente con ISO/IEC 27001:2022 e NIST Cybersecurity Framework 2.0.

\textbf{Parole chiave:} vulnerabilità psicologica, cybersecurity, fattori umani, ISO 27001, gestione della sicurezza, valutazione pre-cognitiva
\end{abstract}

\tableofcontents
\newpage

\section{Introduzione}

\subsection{Background e Contesto}

Nonostante la crescita esponenziale degli investimenti in cybersecurity che superano i 150 miliardi di dollari annui, le violazioni riuscite continuano ad aumentare, con i fattori umani che contribuiscono all'82-85\% degli incidenti secondo il Verizon Data Breach Investigations Report. Questo fallimento persistente rivela un gap fondamentale negli attuali framework di sicurezza: mentre le vulnerabilità tecniche ricevono attenzione sistematica attraverso standard come ISO/IEC 27001:2022 e NIST Cybersecurity Framework 2.0, le vulnerabilità psicologiche rimangono non affrontate.

La ricerca nelle neuroscienze dimostra che le decisioni rilevanti per la sicurezza avvengono 300-500 millisecondi prima della consapevolezza cosciente, con il sistema di rilevamento delle minacce dell'amigdala che inizia le risposte prima che la corteccia prefrontale attivi il pensiero razionale. Questo processamento pre-cognitivo, combinato con le dinamiche di gruppo inconsce identificate da Bion, Klein e Jung, crea vulnerabilità sistematiche che nessuna quantità di formazione sulla security awareness a livello conscio può affrontare.

I framework di sicurezza tradizionali assumono implicitamente modelli di attore razionale dove gli individui, quando informati dei rischi, modificano il comportamento di conseguenza. Questa assunzione non tiene conto di:

\begin{itemize}
\item \textbf{Processi pre-cognitivi} che determinano le decisioni prima della consapevolezza cosciente
\item \textbf{Dinamiche di gruppo inconsce} che sovrastano il giudizio individuale sotto stress
\item \textbf{Stati affettivi} che bypassano la valutazione razionale della sicurezza
\item \textbf{Sovraccarico cognitivo} che forza il ricorso a euristiche sfruttabili
\item \textbf{Compliance basata sull'autorità} che innesca risposte automatiche alla gerarchia percepita
\end{itemize}

CPF-27001:2025 affronta questi gap stabilendo requisiti per la valutazione sistematica e la mitigazione delle vulnerabilità psicologiche, permettendo alle organizzazioni di raggiungere posture di sicurezza predittive che prevengono gli incidenti legati al fattore umano prima che si verifichino.

\subsection{Relazione con Altri Standard}

CPF-27001:2025 è progettato per complementare e migliorare i framework di sicurezza esistenti piuttosto che sostituirli. Lo standard si integra con:

\textbf{ISO/IEC 27001:2022}: CPF-27001 affronta la Clausola 7.2 (Competenza) e la Clausola 7.3 (Consapevolezza) fornendo metodi sistematici per valutare i fattori psicologici che influenzano il comportamento di sicurezza.

\textbf{ISO/IEC 27002:2022}: Mentre ISO/IEC 27002 fornisce guida all'implementazione dei controlli di sicurezza, non affronta i fattori psicologici che determinano l'efficacia dei controlli.

\textbf{NIST Cybersecurity Framework 2.0}: CPF-27001 si mappa direttamente alle funzioni del NIST CSF 2.0 fornendo il livello di intelligenza psicologica che migliora l'efficacia di ogni funzione.

\subsection{Struttura di Questo Documento}

Questo documento segue la struttura delle Direttive ISO/IEC con clausole numerate che specificano i requisiti. I requisiti sono espressi usando linguaggio normativo: ``shall'' indica requisiti obbligatori, ``should'' indica raccomandazioni, e ``may'' indica permessi.

\section{Ambito}

\subsection{Generale}

Questo documento specifica i requisiti per stabilire, implementare, mantenere e migliorare continuamente un Sistema di Gestione della Vulnerabilità Psicologica (PVMS) nel contesto dell'organizzazione. I requisiti specificati in CPF-27001:2025 sono generici e applicabili a tutte le organizzazioni, indipendentemente dal tipo, dimensione o natura.

\subsection{Applicazione}

CPF-27001:2025 deve essere applicato dalle organizzazioni che richiedono una gestione sistematica dei rischi di sicurezza legati al fattore umano e che cercano di integrare la valutazione delle vulnerabilità psicologiche con i framework di sicurezza esistenti.

\subsection{Esclusioni}

CPF-27001:2025 non affronta la valutazione delle vulnerabilità tecniche, l'architettura della sicurezza di rete, i controlli crittografici, le misure di sicurezza fisica, la valutazione psicologica clinica individuale o la gestione delle prestazioni dei dipendenti.

\section{Riferimenti Normativi}

\textbf{ISO/IEC 27001:2022}, Sistemi di gestione della sicurezza delle informazioni — Requisiti

\textbf{ISO/IEC 27002:2022}, Codice di pratica per i controlli di sicurezza delle informazioni

\textbf{NIST Cybersecurity Framework 2.0}

\textbf{Milgram, S. (1974)}, Obedience to Authority

\textbf{Bion, W. R. (1961)}, Experiences in Groups

\textbf{Klein, M. (1946)}, Notes on some schizoid mechanisms

\textbf{Kahneman, D. (2011)}, Thinking, Fast and Slow

\section{Termini e Definizioni}

\subsection{Termini Specifici CPF}

\textbf{vulnerabilità pre-cognitiva}: Debolezza psicologica che opera al di sotto della consapevolezza cosciente e che abilita lo sfruttamento della sicurezza prima che avvenga la valutazione razionale.

\textbf{sistema di gestione della vulnerabilità psicologica (PVMS)}: Parte del sistema di gestione per stabilire, implementare, operare, monitorare, riesaminare, mantenere e migliorare la sicurezza psicologica.

\textbf{schema OFTLISRV}: Metodologia di implementazione sistematica che comprende Osservabili, Fonti di Dati, Temporalità, Logica di Rilevamento, Interdipendenze, Soglie, Risposte e Validazione.

\textbf{stato convergente}: Condizione in cui multiple vulnerabilità psicologiche si allineano simultaneamente, creando una probabilità di violazione esponenzialmente aumentata.

\textbf{Authority Resilience Quotient (ARQ)}: Capacità misurata di mantenere scetticismo appropriato verso le affermazioni di autorità durante il processo decisionale rilevante per la sicurezza.

\textbf{basic assumption dependency (baD)}: Stato inconscio di gruppo caratterizzato dalla ricerca di protezione onnipotente e dall'abdicazione della responsabilità personale di sicurezza.

\textbf{basic assumption fight-flight (baF)}: Stato inconscio di gruppo caratterizzato dalla percezione delle minacce come nemici esterni che richiedono difesa aggressiva o completo evitamento.

\textbf{basic assumption pairing (baP)}: Stato inconscio di gruppo caratterizzato dalla speranza in una futura soluzione messianica piuttosto che affrontare le vulnerabilità attuali.

\textbf{behavioral risk indicator (BRI)}: Metrica quantificabile derivata dal comportamento osservabile che indica il livello di vulnerabilità psicologica.

\textbf{sistema di scoring ternario}: Metodologia di valutazione che utilizza una classificazione a tre stati (Verde/Giallo/Rosso) corrispondenti a livelli di vulnerabilità minimale, moderata e critica.

\textbf{differential privacy}: Framework matematico che assicura che la presenza o assenza dei dati di qualsiasi individuo cambia le probabilità di output al massimo di $e^\varepsilon$ dove $\varepsilon$ rappresenta il budget di privacy.

\textbf{unità minima di aggregazione}: Dimensione minima del gruppo per la quale i dati di valutazione psicologica possono essere riportati, stabilita in dieci individui per prevenire la profilazione individuale.

\textbf{ritardo temporale}: Intervallo di tempo minimo tra la raccolta dei dati e la reportistica, stabilito in 72 ore per prevenire la sorveglianza in tempo reale.

\subsection{Termini Psicologici}

\textbf{processamento di Sistema 1}: Processamento cognitivo veloce, automatico, inconscio che opera attraverso il riconoscimento di pattern e la risposta emotiva.

\textbf{processamento di Sistema 2}: Processamento cognitivo lento, deliberato, conscio che richiede risorse e tempo significativi.

\textbf{amygdala hijack}: Stato neurologico dove il sistema di rilevamento delle minacce dell'amigdala sovrasta il processamento razionale della corteccia prefrontale.

\textbf{carico cognitivo}: Quantità totale di sforzo mentale utilizzato nella memoria di lavoro.

\textbf{splitting}: Meccanismo di difesa primitivo dove il panorama della sicurezza è inconsciamente diviso in oggetti completamente buoni e completamente cattivi.

\textbf{proiezione}: Attribuzione inconscia delle proprie caratteristiche negate su oggetti esterni.

\textbf{transfert}: Reindirizzamento inconscio di sentimenti e atteggiamenti da relazioni passate verso figure o sistemi di autorità di sicurezza presenti.

\textbf{groupthink}: Fenomeno psicologico dove il desiderio di armonia impedisce la valutazione critica.

\textbf{social proof}: Tendenza a conformarsi al comportamento altrui, specialmente in condizioni di incertezza.

\textbf{reciprocità}: Obbligo di restituire favori che gli attaccanti sfruttano.

\subsection{Acronimi}

\textbf{CPF}: Cybersecurity Psychology Framework

\textbf{PVMS}: Sistema di Gestione della Vulnerabilità Psicologica

\textbf{ARQ}: Authority Resilience Quotient

\textbf{baD/baF/baP}: Basic Assumptions (Dependency, Fight-Flight, Pairing)

\textbf{BRI}: Behavioral Risk Indicator

\textbf{ISMS}: Information Security Management System

\section{Contesto dell'Organizzazione}

\subsection{Comprendere l'Organizzazione e il Suo Contesto}

L'organizzazione deve determinare le questioni esterne e interne rilevanti per il suo scopo e che influenzano la sua capacità di raggiungere i risultati attesi del suo sistema di gestione della vulnerabilità psicologica.

L'organizzazione deve determinare i fattori psicologici specifici della cultura organizzativa che influenzano il comportamento di sicurezza, le minacce di social engineering specifiche del settore, i requisiti normativi e i pattern storici degli incidenti di sicurezza legati al fattore umano.

\subsection{Comprendere i Bisogni e le Aspettative delle Parti Interessate}

L'organizzazione deve determinare le parti interessate rilevanti per il PVMS e i loro requisiti, inclusi dipendenti, management, clienti, regolatori, fornitori di assicurazioni, partner e auditor.

\subsection{Determinare l'Ambito del PVMS}

L'organizzazione deve determinare i confini e l'applicabilità del PVMS per stabilire il suo ambito, considerando le questioni esterne e interne, i requisiti delle parti interessate e le unità organizzative coperte.

\subsection{Sistema di Gestione della Vulnerabilità Psicologica}

L'organizzazione deve stabilire, implementare, mantenere e migliorare continuamente un sistema di gestione della vulnerabilità psicologica in conformità con i requisiti di questo documento.

\section{Leadership}

\subsection{Leadership e Impegno}

L'alta direzione deve dimostrare leadership e impegno rispetto al PVMS assicurando che la politica e gli obiettivi siano stabiliti, le risorse siano disponibili e l'importanza di una gestione efficace della vulnerabilità psicologica sia comunicata.

\subsection{Politica}

L'alta direzione deve stabilire una politica CPF che sia appropriata allo scopo dell'organizzazione, includa l'impegno per la valutazione sistematica della vulnerabilità psicologica e la protezione della privacy, e fornisca un framework per stabilire gli obiettivi.

\subsection{Ruoli, Responsabilità e Autorità Organizzative}

L'alta direzione deve assicurare che le responsabilità e le autorità per i ruoli rilevanti siano assegnate e comunicate, inclusi il Coordinatore CPF, il Privacy Officer, gli Specialisti di Valutazione e i Coordinatori di Risposta.

\section{Pianificazione}

\subsection{Azioni per Affrontare Rischi e Opportunità}

\subsubsection{Generale}

L'organizzazione deve determinare i rischi e le opportunità necessarie per assicurare che il PVMS raggiunga i risultati attesi, prevenga effetti indesiderati e raggiunga il miglioramento continuo.

\subsubsection{Valutazione della Vulnerabilità Psicologica}

L'organizzazione deve stabilire processi per la valutazione della vulnerabilità psicologica che valutino le vulnerabilità attraverso tutti i dieci domini CPF, utilizzino 100 indicatori, impieghino metodologie che preservano la privacy, operino su unità minime di aggregazione di dieci individui, implementino differential privacy con $\varepsilon = 0.1$ e mantengano ritardi temporali di minimo 72 ore.

\subsubsection{Trattamento del Rischio Psicologico}

L'organizzazione deve definire e applicare processi per il trattamento del rischio psicologico, selezionando opzioni appropriate per modificare, ritenere, evitare o condividere i rischi.

\subsection{Obiettivi CPF e Pianificazione}

L'organizzazione deve stabilire obiettivi CPF misurabili a funzioni e livelli rilevanti, come ridurre i conteggi degli indicatori Gialli/Rossi, diminuire l'indice di convergenza e ridurre gli incidenti di sicurezza legati al fattore umano.

\subsection{Pianificazione dei Cambiamenti}

Quando l'organizzazione determina la necessità di cambiamenti al PVMS, i cambiamenti devono essere effettuati in modo pianificato considerando lo scopo, l'integrità, le risorse e le protezioni della privacy.

\section{Supporto}

\subsection{Risorse}

L'organizzazione deve determinare e fornire le risorse necessarie per lo stabilimento, l'implementazione, il mantenimento e il miglioramento continuo del PVMS, inclusi personale, infrastruttura tecnologica, strumenti di valutazione e risorse finanziarie.

\subsection{Competenza}

L'organizzazione deve determinare la competenza necessaria delle persone che influenzano le prestazioni del PVMS e assicurare che le persone siano competenti sulla base di educazione, formazione o esperienza appropriate.

Le competenze del Coordinatore CPF includono comprensione dei principi di cybersecurity, conoscenza della teoria psicologica, familiarità con i processi pre-cognitivi e comprensione delle metodologie che preservano la privacy.

Le competenze dello Specialista di Valutazione includono formazione formale in psicologia o scienze comportamentali, comprensione dei concetti psicoanalitici, conoscenza dei bias cognitivi e familiarità con i metodi di data science.

\subsection{Consapevolezza}

L'organizzazione deve assicurare che le persone siano consapevoli della politica CPF, del loro contributo all'efficacia del PVMS, delle protezioni della privacy e che le vulnerabilità psicologiche sono caratteristiche umane normali, non fallimenti individuali.

\subsection{Comunicazione}

L'organizzazione deve determinare la necessità di comunicazioni interne ed esterne rilevanti per il PVMS, incluso cosa comunicare, quando, con chi e come.

\subsection{Informazioni Documentate}

Il PVMS dell'organizzazione deve includere informazioni documentate richieste dal CPF-27001 e determinate necessarie per l'efficacia, inclusi politica CPF, ambito, metodologia di valutazione, procedure di privacy, piani di trattamento del rischio e risultati degli audit.

\section{Operazione}

\subsection{Pianificazione e Controllo Operativo}

L'organizzazione deve pianificare, implementare e controllare i processi necessari per soddisfare i requisiti del PVMS, inclusi cicli di valutazione regolari, monitoraggio continuo, raccolta dati che preserva la privacy, implementazione del trattamento del rischio e integrazione con le operazioni di sicurezza.

\subsection{Valutazione della Vulnerabilità Psicologica}

\subsubsection{Generale}

L'organizzazione deve definire e applicare un processo di valutazione della vulnerabilità psicologica per l'identificazione sistematica delle vulnerabilità attraverso i domini CPF, che avvenga a intervalli pianificati con metodologie validate che mantengono le protezioni della privacy.

\subsubsection{Processo di Valutazione}

La valutazione della vulnerabilità psicologica deve valutare 100 indicatori attraverso 10 domini:

\textbf{Dominio 1: Vulnerabilità Basate sull'Autorità} - Compliance acritica, diffusione della responsabilità, suscettibilità all'impersonificazione dell'autorità, bypass della sicurezza per i superiori, compliance basata sulla paura, effetti del gradiente di autorità, deferenza all'autorità tecnica, normalizzazione delle eccezioni esecutive, social proof basata sull'autorità, escalation dell'autorità in crisi.

\textbf{Dominio 2: Vulnerabilità Temporali} - Bypass indotto dall'urgenza, degradazione cognitiva da pressione temporale, accettazione del rischio guidata dalle scadenze, present bias, sconto iperbolico, pattern di esaurimento temporale, finestre di vulnerabilità basate sull'ora del giorno, mancanze durante weekend/festivi, sfruttamento del cambio turno, pressione della coerenza temporale.

\textbf{Dominio 3: Vulnerabilità dell'Influenza Sociale} - Sfruttamento della reciprocità, trappole di escalation dell'impegno, manipolazione del social proof, override della fiducia basato sul gradimento, decisioni guidate dalla scarsità, sfruttamento del principio di unità, compliance alla pressione dei pari, conformità a norme insicure, minacce all'identità sociale, conflitti di gestione della reputazione.

\textbf{Dominio 4: Vulnerabilità Affettive} - Paralisi decisionale basata sulla paura, assunzione di rischi indotta dalla rabbia, trasferimento della fiducia ai sistemi, attaccamento ai sistemi legacy, nascondimento della sicurezza basato sulla vergogna, overcompliance guidata dal senso di colpa, errori innescati dall'ansia, negligenza correlata alla depressione, disattenzione indotta dall'euforia, effetti di contagio emotivo.

\textbf{Dominio 5: Vulnerabilità del Sovraccarico Cognitivo} - Desensibilizzazione da alert fatigue, errori da decision fatigue, paralisi da sovraccarico informativo, degradazione da multitasking, vulnerabilità del context switching, tunneling cognitivo, overflow della memoria di lavoro, effetti di residuo attenzionale, errori indotti dalla complessità, confusione del modello mentale.

\textbf{Dominio 6: Vulnerabilità delle Dinamiche di Gruppo} - Punti ciechi di sicurezza da groupthink, fenomeni di risky shift, diffusione della responsabilità, social loafing, effetto bystander, assunzioni di gruppo di dipendenza, posture di sicurezza fight-flight, fantasie di speranza del pairing, splitting organizzativo, meccanismi di difesa collettivi.

\textbf{Dominio 7: Vulnerabilità della Risposta allo Stress} - Compromissione da stress acuto, burnout da stress cronico, aggressione della risposta fight, evitamento della risposta flight, paralisi della risposta freeze, overcompliance della risposta fawn, visione a tunnel indotta dallo stress, memoria compromessa dal cortisolo, cascate di contagio dello stress, vulnerabilità del periodo di recupero.

\textbf{Dominio 8: Vulnerabilità dei Processi Inconsci} - Proiezione dell'ombra sugli attaccanti, identificazione inconscia con le minacce, pattern di coazione a ripetere, transfert verso figure di autorità, punti ciechi del controtransfert, interferenza dei meccanismi di difesa, confusione dell'equazione simbolica, trigger di attivazione archetipica, pattern dell'inconscio collettivo, logica onirica negli spazi digitali.

\textbf{Dominio 9: Vulnerabilità dei Bias Specifici dell'AI} - Antropomorfizzazione dei sistemi AI, override del bias di automazione, paradosso dell'avversione agli algoritmi, trasferimento di autorità all'AI, effetti della uncanny valley, fiducia nell'opacità del machine learning, accettazione delle allucinazioni dell'AI, disfunzione del team umano-AI, manipolazione emotiva dell'AI, cecità all'equità algoritmica.

\textbf{Dominio 10: Stati Convergenti Critici} - Condizioni di tempesta perfetta, trigger di fallimento a cascata, vulnerabilità dei punti di ribaltamento, allineamento del formaggio svizzero, cecità al cigno nero, negazione del rinoceronte grigio, catastrofe della complessità, imprevedibilità dell'emergenza, fallimenti dell'accoppiamento di sistema, gap di sicurezza da isteresi.

Per ogni indicatore, la valutazione deve produrre scoring ternario: Verde (0) per vulnerabilità minimale, Giallo (1) per vulnerabilità moderata che richiede monitoraggio, Rosso (2) per vulnerabilità critica che richiede intervento immediato.

\subsubsection{Misure di Preservazione della Privacy}

Tutte le attività di valutazione devono mantenere protezioni della privacy incluse unità minima di aggregazione di dieci individui, differential privacy con $\varepsilon \leq 0.1$, ritardo temporale di 72 ore, analisi basata sui ruoli, minimizzazione dei dati, controlli di accesso, limiti di conservazione e divieto di uso secondario per la valutazione delle prestazioni.

\subsection{Trattamento del Rischio Psicologico}

\subsubsection{Generale}

L'organizzazione deve implementare un piano di trattamento del rischio che affronti le vulnerabilità psicologiche identificate attraverso la valutazione, riconoscendo che le vulnerabilità sono questioni organizzative sistemiche, non fallimenti individuali.

\subsubsection{Protocolli di Risposta}

L'organizzazione deve stabilire protocolli di risposta graduati: lo stato Verde continua il monitoraggio standard, lo stato Giallo aumenta il monitoraggio e implementa interventi preventivi, lo stato Rosso innesca l'escalation immediata e il trattamento di emergenza, la convergenza Critica attiva le procedure di risposta di emergenza.

\subsubsection{Monitoraggio Continuo}

L'organizzazione deve implementare il monitoraggio continuo degli indicatori critici di vulnerabilità psicologica integrato con le operazioni di sicurezza, inclusi monitoraggio in tempo reale, integrazione SIEM, alerting automatizzato e correlazione con il monitoraggio tecnico.

\section{Valutazione delle Prestazioni}

\subsection{Monitoraggio, Misurazione, Analisi e Valutazione}

L'organizzazione deve valutare le prestazioni e l'efficacia del PVMS determinando cosa deve essere monitorato (indicatori, efficacia del trattamento del rischio, prestazioni dei processi), metodi per risultati validi, tempistiche e parti responsabili.

Gli indicatori chiave di prestazione includono numero di indicatori in ogni stato, analisi dei trend, valori dell'indice di convergenza, tassi di incidenti legati al fattore umano, tassi di compliance alle politiche, tempi di risposta e efficacia del trattamento del rischio.

\subsection{Audit Interno}

L'organizzazione deve condurre audit interni a intervalli pianificati per fornire informazioni su se il PVMS è conforme ai requisiti ed è effettivamente implementato e mantenuto.

L'ambito dell'audit deve valutare la conformità della metodologia di valutazione, l'efficacia delle protezioni della privacy, l'adeguatezza della competenza, l'implementazione del trattamento del rischio, l'integrazione con l'ISMS e l'evidenza del miglioramento continuo.

\subsection{Riesame della Direzione}

L'alta direzione deve riesaminare il PVMS a intervalli pianificati per assicurare la continua idoneità, adeguatezza ed efficacia. Gli input del riesame includono stato delle azioni precedenti, cambiamenti nelle questioni, feedback sulle prestazioni, risultati degli audit, risultati della valutazione del rischio e opportunità di miglioramento. Gli output del riesame includono decisioni sui miglioramenti, cambiamenti al PVMS e necessità di risorse.

\section{Miglioramento}

\subsection{Non Conformità e Azione Correttiva}

Quando si verifica una non conformità, l'organizzazione deve reagire per controllarla e correggerla, valutare la necessità di azioni per eliminare le cause, implementare qualsiasi azione necessaria, riesaminare l'efficacia e apportare cambiamenti al PVMS se necessario.

Le non conformità comuni includono mancato mantenimento dell'unità minima di aggregazione, dati di valutazione usati per la valutazione delle prestazioni, protezioni della privacy inadeguate, valutazione che non copre i domini applicabili, competenza insufficiente e mancanza di integrazione con l'ISMS.

\subsection{Miglioramento Continuo}

L'organizzazione deve migliorare continuamente l'idoneità, l'adeguatezza e l'efficacia del PVMS attraverso il raffinamento regolare delle metodologie, il miglioramento delle protezioni della privacy, il miglioramento dell'integrazione con la sicurezza tecnica, lo sviluppo di interventi efficaci e l'espansione dell'ambito di valutazione.

\subsection{Aggiornamenti del Framework}

L'organizzazione deve stabilire un processo per l'aggiornamento degli indicatori CPF e della metodologia di valutazione per affrontare nuove vulnerabilità, cambiamenti nelle tecniche di attacco, progressi nella ricerca psicologica e evoluzione tecnologica.

Gli aggiornamenti del framework devono essere riesaminati attraverso la gestione del cambiamento, mantenere la retrocompatibilità dove fattibile, essere validati prima dell'implementazione, essere documentati con le motivazioni e essere comunicati agli stakeholder.

\section*{Bibliografia}

\textbf{CPF-27002:2025}, Gestione della Vulnerabilità Psicologica — Codice di Pratica

\textbf{Canale, G. (2025)}, The Cybersecurity Psychology Framework. SSRN Electronic Journal.

\textbf{Verizon (2024)}, Data Breach Investigations Report

\textbf{IBM Security (2023)}, Cost of a Data Breach Report

\end{document}
