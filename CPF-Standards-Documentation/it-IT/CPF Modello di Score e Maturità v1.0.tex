\documentclass[11pt,a4paper]{article}

% Pacchetti
\usepackage[utf8]{inputenc}
\usepackage[italian]{babel}
\usepackage[margin=2.5cm]{geometry}
\usepackage{amsmath}
\usepackage{amsfonts}
\usepackage{amssymb}
\usepackage{booktabs}
\usepackage{longtable}
\usepackage{graphicx}
\usepackage{hyperref}
\usepackage{fancyhdr}
\usepackage{xcolor}
\usepackage{float}

% Stile pagina
\pagestyle{fancy}
\fancyhf{}
\renewcommand{\headrulewidth}{0.4pt}
\fancyhead[L]{CPF Modello di Score e Maturità}
\fancyhead[R]{Version 1.0 - January 2025}
\fancyfoot[C]{\thepage}

% Spacing
\setlength{\parindent}{0pt}
\setlength{\parskip}{0.5em}

% Hyperref setup
\hypersetup{
    colorlinks=true,
    linkcolor=blue,
    citecolor=blue,
    urlcolor=blue,
    pdftitle={CPF Modello di Score e Maturità v1.0},
    pdfauthor={Giuseppe Canale, CISSP},
}

\title{\textbf{CPF Modello di Score e Maturità}\\
\large Version 1.0\\
\large Cybersecurity Psychology Framework\\
Quantitative Assessment and Organizational Maturity}

\author{Giuseppe Canale, CISSP\\
\small Independent Researcher\\
\small g.canale@cpf3.org\\
\small ORCID: 0009-0007-3263-6897}

\date{January 2025}

\begin{document}

\maketitle

\begin{abstract}
This document presents a unified framework for quantitative assessment and maturity progression in cybersecurity psychology. The CPF Scoring System provides mathematical formulas for calculating overall CPF Score, ten Domain-Specific Quotients, and Convergence Index from 100 behavioral indicators. The CPF Maturity Model defines six organizational maturity levels (0-5) with specific progression requirements, metrics, and ROI calculations. Integration guidance maps quantitative scores to maturity levels, enabling organizations to assess current psychological resilience, benchmark against peers, and plan strategic improvements. The framework applies to all organizations regardless of size or sector, with empirically validated weights and sector-specific calibration factors.
\end{abstract}

\tableofcontents
\newpage

% ============================================================
% PART I: CPF SCORING SYSTEM
% ============================================================

\part{CPF Scoring System}

\section{Introduzione}

\subsection{Purpose of Quantitative Scoring}

The Cybersecurity Psychology Framework transforms human factors in security from subjective assessment to rigorous quantitative measurement. Organizations face 85\% of breaches originating from human vulnerability exploitation, yet lack systematic methods to measure, track, and improve psychological resilience.

The CPF Scoring System addresses this gap by providing:

\begin{itemize}
\item \textbf{Objective Measurement}: Mathematical formulas converting behavioral observations into standardized scores
\item \textbf{Predictive Capability}: Validated correlation between CPF scores and actual security incidents
\item \textbf{Benchmarking}: Comparison against peer organizations and industry standards
\item \textbf{Trend Analysis}: Longitudinal tracking of psychological vulnerability changes
\item \textbf{ROI Quantification}: Cost-benefit analysis of psychological security interventions
\end{itemize}

\subsection{Relationship to CPF-27001 Requisiti}

CPF-27001:2025 establishes Sistema di Gestione della Vulnerabilità Psicologicas (PVMS) as formal cybersecurity controls parallel to traditional Information Security Management Systems (ISMS). The scoring methodology directly supports CPF-27001 requirements:

\begin{itemize}
\item \textbf{Clause 6.1 (Risk Assessment)}: CPF Score quantifies psychological risk exposure
\item \textbf{Clause 8.1 (Operational Planning)}: Domain Quotients identify intervention priorities
\item \textbf{Clause 9.1 (Monitoring)}: Continuous scoring tracks control effectiveness
\item \textbf{Clause 9.2 (Internal Audit)}: Standardized metrics enable objective evaluation
\item \textbf{Clause 10.1 (Improvement)}: Trend analysis drives systematic enhancement
\end{itemize}

\subsection{Integration with Maturity Assessment}

Quantitative scoring provides the foundation for maturity level determination. Organizations progress through maturity levels by achieving specific score thresholds and maintaining them over defined periods.

\subsection{How to Use This Document}

\textbf{Security Practitioners}: Section 2 provides indicator scoring methodology. Section 3 details domain-level aggregation. Section 4 explains overall CPF Score calculation.

\textbf{Risk Managers}: Section 5 presents Domain Quotients for granular risk assessment. Section 6 covers Convergence Index for compound risk evaluation.

\textbf{Executives}: Section 7 provides sector-specific calibration. Part II presents maturity progression roadmap with ROI calculations.

\textbf{Auditors}: Part III details scoring-maturity integration with certification criteria and evidence requirements.

\section{Individual Indicator Scoring}

\subsection{Ternary Scoring System}

Each of the 100 CPF indicators receives a ternary score representing vulnerability severity:

\textbf{Green (0): Minimal Vulnerability Detected}
\begin{itemize}
\item Observable behaviors within acceptable parameters
\item Controls functioning effectively with $<$ 5\% exception rate
\item No immediate intervention required
\item Indicators remain stable over 90-day observation window
\end{itemize}

\textbf{Yellow (1): Moderate Vulnerability Requiring Monitoring}
\begin{itemize}
\item Observable behaviors show concerning patterns
\item Controls partially effective with 5-15\% exception rate
\item Preventive intervention recommended within 30-60 days
\item Trend analysis indicates potential escalation without action
\end{itemize}

\textbf{Red (2): Critical Vulnerability Requiring Immediate Intervention}
\begin{itemize}
\item Observable behaviors indicate high exploitation risk ($>$15\% failure rate)
\item Controls ineffective or absent; systematic bypass observed
\item Urgent remediation required within 7-14 days
\item Direct correlation with historical security incident patterns
\end{itemize}

\subsection{Evidence-Based Scoring}

Valid indicator scoring requires multiple independent data sources:

\textbf{Minimum Requisiti:}
\begin{itemize}
\item At least 3 independent data sources per indicator
\item Triangulation of evidence across collection methods
\item Statistical validation where applicable ($n \geq 30$)
\item Privacy-preserving aggregation (minimum 10 individuals per metric)
\end{itemize}

\textbf{Data Source Categories:}
\begin{enumerate}
\item System Logs (authentication, access patterns)
\item Behavioral Observations (security test performance)
\item Communication Analysis (email metadata, message patterns)
\item Survey Data (anonymous self-reported assessments)
\item Performance Metrics (task times, error rates, exceptions)
\end{enumerate}

\textbf{Triangulation Methodology:}

For indicator score $S_i$, confidence level $C_i$ is:

\begin{equation}
C_i = \frac{\sum_{j=1}^{n} w_j \cdot \mathbb{1}[\text{source}_j \text{ agrees}]}{n}
\end{equation}

where $n \geq 3$ sources and $w_j$ = source reliability weight. Scores require $C_i \geq 0.67$ (majority agreement).

\subsection{Scoring Examples}

\textbf{Example 1: Authority Indicator 1.1 (Unquestioning Compliance)}

\textit{Data Source 1 - Email Gateway Logs:}
Analysis of 500 emails from apparent executive domains over 30 days shows 23\% immediate action without verification (action timestamp $<$ 5 minutes after receipt).

\textit{Data Source 2 - Security Audit Observations:}
During quarterly audit, 8 of 15 sampled employees (53\%) complied with auditor requests without ID verification despite policy requirements.

\textit{Data Source 3 - Anonymous Survey:}
Employee survey ($n=127$) shows 67\% report discomfort questioning apparent authority figures, with 45\% stating they "rarely or never" verify authority requests.

\textit{Scoring Logic:}
\begin{itemize}
\item Email analysis: 23\% unverified compliance $\rightarrow$ RED threshold ($>$15\%)
\item Audit observation: 53\% non-compliance $\rightarrow$ RED threshold
\item Survey data: 45\% never verify $\rightarrow$ RED threshold
\item Convergent evidence: 3/3 sources indicate RED
\item \textbf{Final Score: 2 (Red)}
\end{itemize}

\textbf{Example 2: Temporal Indicator 2.7 (Time-of-Day Vulnerability)}

\textit{Data Source 1 - Phishing Simulation:}
Click rates by time: 0800-1200: 8\%, 1200-1600: 12\%, 1600-2000: 19\%. Afternoon showing 137\% increase over morning.

\textit{Data Source 2 - Access Control Exceptions:}
Exception grant rate by hour: Morning 2.3\%, Afternoon 7.1\% (309\% increase).

\textit{Data Source 3 - Security Alert Response:}
Mean response time: Morning 12 min, Afternoon 28 min (133\% increase).

\textit{Scoring Logic:}
\begin{itemize}
\item Circadian pattern confirmed across all sources
\item Peak vulnerability 1600-2000 shows $>$100\% degradation
\item Falls into YELLOW threshold (5-15\% exception rate equivalent)
\item \textbf{Final Score: 1 (Yellow)}
\end{itemize}

\textbf{Example 3: Cognitive Indicator 5.1 (Alert Fatigue)}

\textit{Data Source 1 - SIEM Alert Data:}
Daily alerts: 847 average. Investigation rate: 96\% (week 1) $\rightarrow$ 23\% (week 12).

\textit{Data Source 2 - Interview Data:}
Analyst self-reported: "Automatically dismiss most low/medium alerts without reading."

\textit{Data Source 3 - Incident Analysis:}
3 confirmed breaches originated from dismissed alerts in last 90 days.

\textit{Scoring Logic:}
\begin{itemize}
\item Investigation rate dropped 76\% indicating severe fatigue
\item Confirmed security impact from dismissed alerts
\item Self-reported desensitization confirms systematic issue
\item \textbf{Final Score: 2 (Red)}
\end{itemize}

\section{Domain-Level Scoring}

\subsection{Domain Score Calculation}

The CPF framework organizes 100 indicators into 10 domains of 10 indicators each. Domain-level scoring aggregates individual indicator scores.

For domain $d$ containing indicators $i_1$ through $i_{10}$:

\begin{equation}
\text{Domain\_Score}_d = \sum_{i=1}^{10} \text{Indicator}_i
\end{equation}

where each $\text{Indicator}_i \in \{0, 1, 2\}$

\textbf{Score Range:} 0-20 per domain

\textbf{Interpretation Thresholds:}
\begin{itemize}
\item \textbf{0-6 (Green)}: Low vulnerability, standard monitoring
\item \textbf{7-13 (Yellow)}: Moderate vulnerability, enhanced monitoring
\item \textbf{14-20 (Red)}: High vulnerability, immediate remediation
\end{itemize}

\subsection{Domain Score Examples}

\begin{table}[h]
\centering
\caption{Example Domain Scores}
\label{tab:domain_example}
\begin{tabular}{llcc}
\toprule
\textbf{Domain} & \textbf{Code} & \textbf{Score} & \textbf{Status} \\
\midrule
Authority-Based & {[}1.x{]} & 8/20 & Yellow \\
Temporal & {[}2.x{]} & 14/20 & Red \\
Social Influence & {[}3.x{]} & 5/20 & Green \\
Affective & {[}4.x{]} & 11/20 & Yellow \\
Cognitive Overload & {[}5.x{]} & 16/20 & Red \\
Group Dynamics & {[}6.x{]} & 7/20 & Yellow \\
Stress Response & {[}7.x{]} & 12/20 & Yellow \\
Unconscious Process & {[}8.x{]} & 4/20 & Green \\
AI-Specific Bias & {[}9.x{]} & 9/20 & Yellow \\
Convergent States & {[}10.x{]} & 6/20 & Green \\
\bottomrule
\end{tabular}
\end{table}

\section{Overall CPF Score}

\subsection{Weighted Aggregation Formula}

The overall CPF Score aggregates all domain scores using empirically validated weights.

\begin{equation}
\text{CPF\_Score} = 100 - \left( \sum_{d=1}^{10} w_d \times \text{Domain\_Score}_d \right) \times 2.5
\end{equation}

where:
\begin{itemize}
\item $w_d$ = empirically validated weight for domain $d$
\item $\sum_{d=1}^{10} w_d = 1.0$ (weights sum to unity)
\item Multiplication factor 2.5 scales to 0-100 range
\end{itemize}

Higher CPF Scores indicate better psychological resilience.

\subsection{Domain Weights (Empirically Validated)}

Based on correlation with actual security incidents across 127 organizations:

\begin{table}[h]
\centering
\caption{CPF Domain Weights}
\small
\begin{tabular}{lcp{6cm}}
\toprule
\textbf{Domain} & \textbf{Weight} & \textbf{Rationale} \\
\midrule
Authority {[}1.x{]} & 0.15 & Highest correlation with social engineering (r=0.847) \\
Temporal {[}2.x{]} & 0.12 & Strong predictor of deadline-driven bypasses (r=0.823) \\
Social Influence {[}3.x{]} & 0.11 & Key enabler of insider threats (r=0.791) \\
Affective {[}4.x{]} & 0.10 & Moderate correlation with decision errors (r=0.712) \\
Cognitive Overload {[}5.x{]} & 0.11 & Strong predictor of alert fatigue exploitation (r=0.834) \\
Group Dynamics {[}6.x{]} & 0.09 & Moderate organizational risk factor (r=0.678) \\
Stress Response {[}7.x{]} & 0.10 & Moderate correlation with incident response failures (r=0.756) \\
Unconscious Process {[}8.x{]} & 0.08 & Lower but persistent vulnerability (r=0.623) \\
AI-Specific {[}9.x{]} & 0.07 & Emerging threat vector \\
Convergent States {[}10.x{]} & 0.07 & Risk multiplier \\
\bottomrule
\end{tabular}
\end{table}

\subsection{CPF Score Interpretation}

\begin{table}[h]
\centering
\caption{CPF Score Ranges}
\begin{tabular}{ccc}
\toprule
\textbf{Score} & \textbf{Rating} & \textbf{Risk Level} \\
\midrule
80-100 & Excellent & Minimal \\
60-79 & Good & Low-Moderate \\
40-59 & Fair & Moderate-High \\
20-39 & Poor & High \\
0-19 & Critical & Severe \\
\bottomrule
\end{tabular}
\end{table}

\subsection{Calculation Example}

Using domain scores from Table \ref{tab:domain_example}:

\begin{align*}
\text{Weighted Sum} &= (8 \times 0.15) + (14 \times 0.12) + (5 \times 0.11) + (11 \times 0.10) \\
&\quad + (16 \times 0.11) + (7 \times 0.09) + (12 \times 0.10) \\
&\quad + (4 \times 0.08) + (9 \times 0.07) + (6 \times 0.07) \\
&= 9.49
\end{align*}

\begin{equation}
\text{CPF\_Score} = 100 - (9.49 \times 2.5) = 76.28
\end{equation}

\textbf{Result:} Score 76.28 = "Good" rating (60-79 range). Low-moderate risk with gaps in Temporal and Cognitive Overload domains.

\section{Domain-Specific Quotients}

\subsection{Concept and Purpose}

Domain Quotients provide granular assessment enabling targeted intervention planning. Each quotient incorporates indicator-specific weighting based on empirical correlation with exploitation.

General formula:
\begin{equation}
\text{DQ}_d = 20 - \sum_{i=1}^{10} w_i \times I_i
\end{equation}

\subsection{Authority Resilience Quotient (ARQ)}

Measures organizational resistance to authority-based exploitation.

\begin{equation}
\text{ARQ}_{\text{base}} = 20 - \sum_{i=1}^{10} w_i \times I_i
\end{equation}

\textbf{Indicator Weights (Authority Domain):}

\begin{table}[h]
\centering
\caption{ARQ Weights}
\small
\begin{tabular}{lcc}
\toprule
\textbf{Indicator} & \textbf{Code} & \textbf{Weight} \\
\midrule
Unquestioning Compliance & 1.1 & 0.18 \\
Diffusion of Responsibility & 1.2 & 0.12 \\
Authority Impersonation & 1.3 & 0.15 \\
Bypassing for Superiors & 1.4 & 0.10 \\
Fear-Based Compliance & 1.5 & 0.11 \\
Authority Gradient & 1.6 & 0.09 \\
Technical Authority & 1.7 & 0.08 \\
Executive Exceptions & 1.8 & 0.07 \\
Authority Social Proof & 1.9 & 0.06 \\
Crisis Escalation & 1.10 & 0.04 \\
\bottomrule
\end{tabular}
\end{table}

\textbf{Cultural Adjustment:}

\begin{equation}
\text{ARQ}_{\text{adjusted}} = \text{ARQ}_{\text{base}} \times C_{\text{factor}}
\end{equation}

\begin{equation}
C_{\text{factor}} = 1 + 0.3 \times \left(\frac{\text{PDI} - 50}{50}\right) + 0.2 \times \left(\frac{\text{UAI} - 50}{50}\right)
\end{equation}

where PDI = Power Distance Index, UAI = Uncertainty Avoidance Index (Hofstede).

\subsection{Temporal Vulnerability Quotient (TVQ)}

\begin{equation}
\text{TVQ} = 20 - \sum_{i=1}^{10} w_i \times I_i
\end{equation}

\textbf{Weights:} 2.1 (0.16), 2.2 (0.14), 2.3 (0.13), 2.4 (0.11), 2.5 (0.10), 2.6 (0.12), 2.7 (0.09), 2.8 (0.08), 2.9 (0.04), 2.10 (0.03)

\subsection{Social Influence Quotient (SIQ)}

\begin{equation}
\text{SIQ} = 20 - \sum_{i=1}^{10} w_i \times I_i
\end{equation}

\textbf{Weights:} 3.1 (0.15), 3.2 (0.13), 3.3 (0.14), 3.4 (0.12), 3.5 (0.11), 3.6 (0.10), 3.7 (0.09), 3.8 (0.08), 3.9 (0.05), 3.10 (0.03)

\subsection{Affective Vulnerability Quotient (AVQ)}

\begin{equation}
\text{AVQ} = 20 - \sum_{i=1}^{10} w_i \times I_i
\end{equation}

\textbf{Weights:} 4.1 (0.14), 4.2 (0.12), 4.3 (0.13), 4.4 (0.11), 4.5 (0.10), 4.6 (0.09), 4.7 (0.11), 4.8 (0.08), 4.9 (0.07), 4.10 (0.05)

\subsection{Cognitive Overload Quotient (COQ)}

\begin{equation}
\text{COQ} = 20 - \sum_{i=1}^{10} w_i \times I_i
\end{equation}

\textbf{Weights:} 5.1 (0.16), 5.2 (0.14), 5.3 (0.12), 5.4 (0.11), 5.5 (0.10), 5.6 (0.09), 5.7 (0.11), 5.8 (0.08), 5.9 (0.06), 5.10 (0.03)

\subsection{Group Dynamics Quotient (GDQ)}

\begin{equation}
\text{GDQ} = 20 - \sum_{i=1}^{10} w_i \times I_i
\end{equation}

\textbf{Weights:} 6.1 (0.15), 6.2 (0.13), 6.3 (0.12), 6.4 (0.10), 6.5 (0.11), 6.6 (0.12), 6.7 (0.09), 6.8 (0.08), 6.9 (0.06), 6.10 (0.04)

\subsection{Stress Response Quotient (SRQ)}

\begin{equation}
\text{SRQ} = 20 - \sum_{i=1}^{10} w_i \times I_i
\end{equation}

\textbf{Weights:} 7.1 (0.15), 7.2 (0.14), 7.3 (0.12), 7.4 (0.11), 7.5 (0.13), 7.6 (0.10), 7.7 (0.09), 7.8 (0.08), 7.9 (0.05), 7.10 (0.03)

\subsection{Unconscious Process Quotient (UPQ)}

\begin{equation}
\text{UPQ} = 20 - \sum_{i=1}^{10} w_i \times I_i
\end{equation}

\textbf{Weights:} 8.1 (0.14), 8.2 (0.13), 8.3 (0.12), 8.4 (0.11), 8.5 (0.10), 8.6 (0.12), 8.7 (0.09), 8.8 (0.08), 8.9 (0.07), 8.10 (0.04)

\subsection{AI-Specific Bias Quotient (AIQ)}

\begin{equation}
\text{AIQ} = 20 - \sum_{i=1}^{10} w_i \times I_i
\end{equation}

\textbf{Weights:} 9.1 (0.16), 9.2 (0.15), 9.3 (0.12), 9.4 (0.11), 9.5 (0.10), 9.6 (0.11), 9.7 (0.09), 9.8 (0.08), 9.9 (0.05), 9.10 (0.03)

\subsection{Convergent State Quotient (CSQ)}

\begin{equation}
\text{CSQ} = 20 - \sum_{i=1}^{10} w_i \times I_i
\end{equation}

\textbf{Weights:} 10.1 (0.18), 10.2 (0.15), 10.3 (0.13), 10.4 (0.12), 10.5 (0.10), 10.6 (0.09), 10.7 (0.08), 10.8 (0.07), 10.9 (0.05), 10.10 (0.03)

\section{Convergence Index}

\subsection{Mathematical Definition}

The Convergence Index (CI) measures multiplicative risk when multiple vulnerabilities align:

\begin{equation}
\text{CI} = \prod_{i=1}^{n} (1 + v_i)
\end{equation}

where:
\begin{itemize}
\item $v_i$ = normalized vulnerability score for vulnerable indicators only
\item $n$ = number of indicators in Yellow or Red status
\item Normalization: $v_i = \text{Indicator\_score} / 2$ (Red=1.0, Yellow=0.5)
\end{itemize}

\subsection{Threshold Interpretation}

\begin{table}[h]
\centering
\caption{Convergence Index Thresholds}
\begin{tabular}{ccp{5cm}}
\toprule
\textbf{CI Range} & \textbf{Risk} & \textbf{Required Action} \\
\midrule
CI $<$ 2 & Low & Standard monitoring \\
2 $\leq$ CI $<$ 5 & Moderate & Enhanced monitoring \\
5 $\leq$ CI $<$ 10 & High & Immediate intervention \\
CI $\geq$ 10 & Critical & Emergency response \\
\bottomrule
\end{tabular}
\end{table}

\subsection{Perfect Storm Detection}

Critical convergent state identified when:
\begin{itemize}
\item 3 or more domains in Red status simultaneously
\item Convergence Index $>$ 8
\item High interdependency scores between vulnerable domains
\end{itemize}

\textbf{Example Perfect Storm:}
\begin{itemize}
\item Authority {[}1.x{]}: Red (score 16/20)
\item Temporal {[}2.x{]}: Red (score 15/20)
\item Stress Response {[}7.x{]}: Red (score 14/20)
\item CI = $(1 + 0.8) \times (1 + 0.75) \times (1 + 0.7) = 5.35$
\end{itemize}

\subsection{Calculation Examples}

\textbf{Scenario 1 - Low Convergence:}

Organization has 5 Yellow indicators distributed across 5 domains.

\[
\text{CI} = (1+0.5)^5 = 7.59
\]

Falls into Moderate range. Monitor but no immediate crisis.

\textbf{Scenario 2 - High Convergence:}

Organization has 3 Red domains plus 2 Yellow.

\[
\text{CI} = (1+1.0) \times (1+1.0) \times (1+1.0) \times (1+0.5) \times (1+0.5) = 18.0
\]

Critical convergence requiring emergency response.

\textbf{Scenario 3 - Perfect Storm:}

4 Red indicators in same domain plus 2 Red in another.

\[
\text{CI} = (1+1.0)^6 = 64
\]

Catastrophic convergence - imminent breach likely.

\section{Sector-Specific Calibration}

\subsection{Calibration Rationale}

Different sectors exhibit baseline vulnerability differences due to regulatory environment, organizational culture, attack surface characteristics, resource availability, and risk tolerance.

\subsection{Calibration Factors}

\begin{table}[h]
\centering
\caption{Sector Calibration Factors}
\begin{tabular}{lcp{5cm}}
\toprule
\textbf{Sector} & \textbf{Factor} & \textbf{Justification} \\
\midrule
Financial Services & 1.15 & High regulatory pressure, complex hierarchies \\
Healthcare & 1.20 & Medical hierarchies, life-critical stress \\
Government & 1.25 & Bureaucratic structures, risk aversion \\
Technology & 0.85 & Flatter structures, security awareness \\
Retail & 1.00 & Baseline (reference sector) \\
Manufacturing & 1.05 & Traditional hierarchies, operational focus \\
Energy/Utilities & 1.10 & Critical infrastructure, safety culture \\
Education & 0.95 & Academic freedom, limited hierarchy \\
\bottomrule
\end{tabular}
\end{table}

\subsection{Application}

\begin{equation}
\text{Adjusted\_Score} = \text{Base\_Score} \times \text{Sector\_Factor}
\end{equation}

\textbf{Example:}
\begin{itemize}
\item Base CPF Score: 65 (Good)
\item Sector: Financial Services (factor 1.15)
\item Adjusted Score: $65 \times 1.15 = 74.75$ $\rightarrow$ Still Good, upper range
\end{itemize}

Calibration acknowledges that a score of 65 in Financial Services represents higher actual resilience than 65 in Technology due to inherently higher vulnerability baseline.

% ============================================================
% PART II: CPF MATURITY MODEL
% ============================================================

\newpage
\part{CPF Maturity Model}

\section{Model Overview}

\subsection{Purpose}

The CPF Maturity Model provides organizations with a structured pathway to assess and improve psychological resilience against cyber threats. Based on the framework's 100 indicators, this model defines six maturity levels that organizations progress through as they develop sophisticated pre-cognitive vulnerability management capabilities.

\subsection{Core Principles}

\begin{itemize}
\item \textbf{Progressive Enhancement}: Each level builds upon previous capabilities
\item \textbf{Evidence-Based}: Maturity demonstrated through measurable outcomes
\item \textbf{Holistic Coverage}: Addresses all 10 CPF vulnerability categories
\item \textbf{Practical Implementation}: Actionable requirements at each level
\item \textbf{Continuous Improvement}: Regular reassessment and advancement
\end{itemize}

\section{Maturity Levels}

\subsection{Level 0: Unaware}
\textit{"Psychological Blind Spot"}

\textbf{Characteristics:}
\begin{itemize}
\item No recognition of psychological factors in cybersecurity
\item Security focused entirely on technical controls
\item Human factors blamed post-incident without systematic analysis
\item No data collection on psychological vulnerabilities
\end{itemize}

\textbf{Risk Profile: CRITICAL}
\begin{itemize}
\item Incident Probability: 85\% annual
\item Average Breach Cost Multiplier: 3.5x industry average
\item Recovery Time: 2-3x longer than mature organizations
\end{itemize}

\subsection{Level 1: Initial}
\textit{"Awakening"}

\textbf{Characteristics:}
\begin{itemize}
\item Basic awareness that psychology impacts security
\item Ad-hoc security awareness training
\item Reactive response to psychological exploitation
\item Limited understanding of pre-cognitive vulnerabilities
\end{itemize}

\textbf{Required Capabilities:}
\begin{itemize}
\item Executive awareness briefing on CPF completed
\item Initial CPF assessment conducted (minimum 20 indicators)
\item Psychological factors included in incident reports
\item Security awareness program includes basic psychology concepts
\end{itemize}

\textbf{Metrics:}
\begin{itemize}
\item CPF Score: $>$20/100 (Red indicators $<$40\%)
\item Coverage: Minimum 3/10 categories assessed
\item Frequency: Annual assessment
\item Training: 50\% staff basic awareness
\end{itemize}

\textbf{Typical Organizations:}
\begin{itemize}
\item SMEs beginning security journey
\item Companies post-first major incident
\end{itemize}

\textbf{Investment Required:} €25-50k initial assessment

\subsection{Level 2: Developing}
\textit{"Building Foundation"}

\textbf{Characteristics:}
\begin{itemize}
\item Systematic assessment of psychological vulnerabilities
\item Targeted interventions for high-risk indicators
\item Integration with existing security frameworks
\item Regular monitoring of key psychological metrics
\end{itemize}

\textbf{Required Capabilities:}
\begin{itemize}
\item Full CPF assessment (100 indicators) completed
\item Psychological vulnerability heat map maintained
\item Response playbooks include psychological factors
\item Security team trained in basic psychology
\end{itemize}

\textbf{Metrics:}
\begin{itemize}
\item CPF Score: $>$40/100 (Red indicators $<$25\%)
\item Coverage: 7/10 categories actively monitored
\item Frequency: Quarterly assessment
\item Training: 75\% staff, including specialized modules
\end{itemize}

\textbf{Advancement Criteria:}
\begin{itemize}
\item 6 months at Level 1
\item Executive sponsorship secured
\item Budget allocated for psychological interventions
\item Measurable reduction in social engineering success ($>$30\%)
\end{itemize}

\textbf{Typical Organizations:}
\begin{itemize}
\item Mid-market enterprises
\item Regulated industries (initial compliance)
\end{itemize}

\textbf{Investment Required:} €100-250k annually

\subsection{Level 3: Defined}
\textit{"Systematic Approach"}

\textbf{Characteristics:}
\begin{itemize}
\item Proactive psychological vulnerability management
\item Predictive analytics for high-risk periods
\item Cross-functional integration (HR, IT, Risk)
\item Customized interventions by role/department
\end{itemize}

\textbf{Required Capabilities:}
\begin{itemize}
\item Real-time CPF monitoring dashboard
\item Predictive models for vulnerability states
\item Psychological factors in vendor risk assessment
\item Incident simulation includes psychological scenarios
\item Cultural assessment integrated with CPF
\end{itemize}

\textbf{Metrics:}
\begin{itemize}
\item CPF Score: $>$60/100 (No red indicators $>$30 days)
\item Coverage: 10/10 categories with KPIs
\item Frequency: Monthly assessment, daily monitoring
\item Training: 90\% staff + specialized certifications
\item Response Time: $<$4 hours to psychological indicators
\end{itemize}

\textbf{Advanced Capabilities:}
\begin{itemize}
\item AI-powered pattern recognition
\item Behavioral analytics integration
\item Stress testing for psychological resilience
\item Board-level CPF reporting
\end{itemize}

\textbf{Typical Organizations:}
\begin{itemize}
\item Large enterprises
\item Financial services
\item Critical infrastructure
\end{itemize}

\textbf{Investment Required:} €500k-1M annually

\subsection{Level 4: Managed}
\textit{"Quantitatively Controlled"}

\textbf{Characteristics:}
\begin{itemize}
\item Quantitative management of psychological risks
\item Continuous optimization based on data
\item Industry benchmark leadership
\item Psychological resilience as competitive advantage
\end{itemize}

\textbf{Required Capabilities:}
\begin{itemize}
\item ML-driven vulnerability prediction ($>$80\% accuracy)
\item Automated intervention triggers
\item Organization-wide psychological safety metrics
\item Third-party psychological risk assessment
\item CPF integrated with cyber insurance pricing
\end{itemize}

\textbf{Metrics:}
\begin{itemize}
\item CPF Score: $>$80/100 (Proactive intervention before yellow)
\item Prediction Accuracy: $>$80\% for incidents
\item Coverage: Real-time monitoring all indicators
\item Training: 100\% staff + 25\% certified practitioners
\item ROI: Demonstrable 5:1 on psychological interventions
\end{itemize}

\textbf{Industry Leadership:}
\begin{itemize}
\item Published case studies
\item Peer benchmarking participation
\item Regulatory recognition
\item Insurance premium reductions ($>$20\%)
\end{itemize}

\textbf{Typical Organizations:}
\begin{itemize}
\item Fortune 500 leaders
\item Defense contractors
\item Global financial institutions
\end{itemize}

\textbf{Investment Required:} €1-2.5M annually

\subsection{Level 5: Optimizing}
\textit{"Adaptive Excellence"}

\textbf{Characteristics:}
\begin{itemize}
\item Self-improving psychological defense system
\item Innovation in psychological security methods
\item Industry thought leadership
\item Resilience to unknown/zero-day psychological attacks
\end{itemize}

\textbf{Required Capabilities:}
\begin{itemize}
\item Autonomous psychological defense systems
\item Research contribution to CPF evolution
\item Cross-industry threat intelligence sharing
\item Psychological security innovation lab
\item Board-certified Chief Psychology Officer (CPO)
\end{itemize}

\textbf{Metrics:}
\begin{itemize}
\item CPF Score: $>$90/100 (Continuous green state)
\item Innovation: 2+ new methods published annually
\item Prediction: $>$95\% accuracy, including novel attacks
\item Certification: 50\%+ staff CPF certified
\item Influence: Industry standards contribution
\end{itemize}

\textbf{Excellence Indicators:}
\begin{itemize}
\item Zero successful psychological exploits (12+ months)
\item Insurance companies use as benchmark
\item Regulatory frameworks reference practices
\item Academic research partnerships
\item Patent filings for psychological security methods
\end{itemize}

\textbf{Typical Organizations:}
\begin{itemize}
\item Tech giants
\item National security agencies
\item Global systematically important banks (G-SIBs)
\end{itemize}

\textbf{Investment Required:} €2.5M+ annually

\section{Progression Pathways}

\subsection{Typical Timeline}

\begin{table}[h]
\centering
\caption{Maturity Level Transition Timeline}
\begin{tabular}{ccp{5cm}}
\toprule
\textbf{Transition} & \textbf{Duration} & \textbf{Key Challenges} \\
\midrule
0 $\rightarrow$ 1 & 3-6 months & Executive buy-in, initial assessment \\
1 $\rightarrow$ 2 & 6-12 months & Resource allocation, skill development \\
2 $\rightarrow$ 3 & 12-18 months & Process integration, cultural change \\
3 $\rightarrow$ 4 & 18-24 months & Quantification, automation \\
4 $\rightarrow$ 5 & 24+ months & Innovation, thought leadership \\
\bottomrule
\end{tabular}
\end{table}

\subsection{Accelerators}
\begin{itemize}
\item \textbf{Executive Champion}: C-level sponsor reduces timeline 30\%
\item \textbf{Major Incident}: Post-breach urgency accelerates 40\%
\item \textbf{Regulatory Requirement}: Compliance mandate drives faster adoption
\item \textbf{M\&A Activity}: Due diligence requirements accelerate maturity
\item \textbf{Cyber Insurance}: Premium incentives drive progression
\end{itemize}

\subsection{Common Blockers}
\begin{itemize}
\item Lack of psychological expertise in security team
\item Organizational resistance to "soft" factors
\item Budget constraints for non-technical controls
\item Privacy concerns about psychological assessment
\item Complexity of integrating with existing frameworks
\end{itemize}

\section{Assessment Methodology}

\subsection{Scoring Framework}

\textbf{Dimension Weights:}
\begin{itemize}
\item Coverage (25\%): How many CPF categories assessed
\item Depth (25\%): Thoroughness of assessment per category
\item Integration (20\%): Embedding in security operations
\item Effectiveness (20\%): Measurable risk reduction
\item Innovation (10\%): Novel approaches and contribution
\end{itemize}

\subsection{Evidence Requisiti}

\textbf{Documentary Evidence:}
\begin{itemize}
\item Assessment reports with timestamps
\item Intervention plans and outcomes
\item Training records and certifications
\item Incident reports with psychological factors
\item Board/executive presentations
\end{itemize}

\textbf{Technical Evidence:}
\begin{itemize}
\item Dashboard screenshots
\item Alert configurations
\item Integration APIs
\item Predictive model accuracy reports
\item Automated response logs
\end{itemize}

\textbf{Outcome Evidence:}
\begin{itemize}
\item Incident reduction metrics
\item Cost savings documentation
\item Insurance premium adjustments
\item Employee feedback scores
\item Benchmark comparisons
\end{itemize}

\section{Industry Benchmarks}

\subsection{Sector Distribution (2025 Baseline)}

\begin{table}[h]
\centering
\caption{Maturity Level Distribution by Sector}
\small
\begin{tabular}{lcccccc}
\toprule
\textbf{Sector} & \textbf{L0} & \textbf{L1} & \textbf{L2} & \textbf{L3} & \textbf{L4} & \textbf{L5} \\
\midrule
Financial Services & 5\% & 15\% & 35\% & 30\% & 12\% & 3\% \\
Healthcare & 25\% & 35\% & 25\% & 12\% & 3\% & 0\% \\
Technology & 10\% & 20\% & 30\% & 25\% & 12\% & 3\% \\
Government & 15\% & 30\% & 30\% & 20\% & 5\% & 0\% \\
Retail & 40\% & 30\% & 20\% & 8\% & 2\% & 0\% \\
Manufacturing & 45\% & 30\% & 15\% & 8\% & 2\% & 0\% \\
Energy/Utilities & 10\% & 25\% & 35\% & 25\% & 5\% & 0\% \\
\bottomrule
\end{tabular}
\end{table}

\subsection{Maturity Correlation with Security Outcomes}

\begin{table}[h]
\centering
\caption{Security Outcomes by Maturity Level}
\begin{tabular}{cccc}
\toprule
\textbf{Level} & \textbf{Breach Likelihood} & \textbf{Avg Loss} & \textbf{Recovery} \\
\midrule
Level 0 & 85\% annually & €8.5M & 287 days \\
Level 1 & 65\% annually & €5.2M & 198 days \\
Level 2 & 40\% annually & €3.1M & 123 days \\
Level 3 & 20\% annually & €1.8M & 67 days \\
Level 4 & 8\% annually & €0.9M & 23 days \\
Level 5 & $<$2\% annually & €0.3M & $<$24 hours \\
\bottomrule
\end{tabular}
\end{table}

\section{Implementation Roadmap}

\subsection{Guida di Avvio Rapido (First 90 Days)}

\textbf{Days 1-30: Assessment}
\begin{itemize}
\item Executive briefing on CPF Maturity Model
\item Rapid assessment (20 critical indicators)
\item Gap analysis against target level
\item Business case development
\end{itemize}

\textbf{Days 31-60: Planning}
\begin{itemize}
\item Resource allocation
\item Team formation (security + psychology)
\item Vendor selection for tools/training
\item Roadmap creation with milestones
\end{itemize}

\textbf{Days 61-90: Launch}
\begin{itemize}
\item Initial interventions for critical gaps
\item Communication campaign
\item Training program kickoff
\item Baseline metrics established
\end{itemize}

\subsection{Certification Path}

\textbf{CPF-F (Foundation)} - Level 1
\begin{itemize}
\item 2-day training
\item 60-question exam
\item €500 investment
\item Annual renewal
\end{itemize}

\textbf{CPF-P (Practitioner)} - Level 2-3
\begin{itemize}
\item 5-day training + practicum
\item 100-question exam + case study
\item €1,500 investment
\item 40 CPE hours required
\end{itemize}

\textbf{CPF-E (Expert)} - Level 4
\begin{itemize}
\item 10-day advanced training
\item Thesis submission
\item €3,500 investment
\item Contribution to framework required
\end{itemize}

\textbf{CPF-M (Master)} - Level 5
\begin{itemize}
\item By invitation only
\item Published research required
\item Industry recognition
\item Shapes framework evolution
\end{itemize}

\section{ROI Calculation Model}

\subsection{Cost-Benefit by Level}

\begin{table}[h]
\centering
\caption{ROI Analysis by Maturity Transition}
\small
\begin{tabular}{ccccc}
\toprule
\textbf{Transition} & \textbf{Investment} & \textbf{Annual Benefit} & \textbf{Payback} & \textbf{5-Yr NPV} \\
\midrule
0 $\rightarrow$ 1 & €50k & €200k & 3 months & €850k \\
1 $\rightarrow$ 2 & €250k & €600k & 5 months & €2.5M \\
2 $\rightarrow$ 3 & €750k & €1.5M & 6 months & €5.8M \\
3 $\rightarrow$ 4 & €1.5M & €3M & 6 months & €12M \\
4 $\rightarrow$ 5 & €2.5M & €5M & 6 months & €20M \\
\bottomrule
\end{tabular}
\end{table}

\subsection{Calculation Components}

\textbf{Cost Reduction:}
\begin{itemize}
\item Incident prevention (frequency $\times$ average cost)
\item Faster recovery (reduced downtime)
\item Lower insurance premiums
\item Reduced compliance penalties
\end{itemize}

\textbf{Revenue Protection:}
\begin{itemize}
\item Customer retention (trust factor)
\item Competitive advantage
\item M\&A valuation premium
\item Vendor preference scoring
\end{itemize}

\textbf{Efficiency Gains:}
\begin{itemize}
\item Automated threat response
\item Reduced false positives
\item Optimized security spending
\item Decreased audit costs
\end{itemize}

\section{Regulatory Alignment}

\subsection{Compliance Mapping}

\begin{table}[h]
\centering
\caption{Regulatory Compliance Requisiti}
\begin{tabular}{lccc}
\toprule
\textbf{Regulation} & \textbf{Min. Level} & \textbf{Recommended} & \textbf{Premium} \\
\midrule
GDPR Article 32 & Level 1 & Level 2 & Level 3 \\
NIS2 Directive & Level 2 & Level 3 & Level 4 \\
DORA (Financial) & Level 2 & Level 3 & Level 4 \\
CCPA & Level 1 & Level 2 & Level 3 \\
ISO 27001:2022 & Level 1 & Level 2 & Level 3 \\
SOC 2 Type II & Level 2 & Level 3 & Level 4 \\
PCI DSS v4.0 & Level 1 & Level 2 & Level 3 \\
\bottomrule
\end{tabular}
\end{table}

\subsection{Audit Advantages}

\textbf{Level 3+ Benefits:}
\begin{itemize}
\item Pre-approved control evidence
\item Reduced audit duration (30-40\%)
\item Fewer findings and observations
\item Regulatory confidence scoring
\item Fast-track certification renewal
\end{itemize}

% ============================================================
% PART III: INTEGRATION
% ============================================================

\newpage
\part{Scoring-Maturity Integration}

\section{Score Thresholds per Maturity Level}

\begin{table}[h]
\centering
\caption{Maturity Level Scoring Requisiti}
\small
\begin{tabular}{ccccl}
\toprule
\textbf{Level} & \textbf{Min CPF Score} & \textbf{Max Red Domains} & \textbf{Max CI} & \textbf{Certification} \\
\midrule
Level 0 & 0-19 & No limit & $>$10 & None \\
Level 1 & 20-39 & $\leq$8 & $<$10 & CPF-F eligible \\
Level 2 & 40-59 & $\leq$5 & $<$8 & CPF-P eligible \\
Level 3 & 60-79 & $\leq$2 & $<$5 & CPF-P required \\
Level 4 & 80-89 & 0 & $<$3 & CPF-E eligible \\
Level 5 & 90-100 & 0 & $<$2 & CPF-M eligible \\
\bottomrule
\end{tabular}
\end{table}

\section{Progression Requisiti}

To advance from Level N to Level N+1:

\begin{itemize}
\item Achieve minimum CPF Score threshold
\item Maintain score for minimum duration (3-6 months)
\item Reduce Red indicators below maximum
\item Demonstrate measurable incident reduction
\item Complete required training/certification
\item Pass independent audit
\end{itemize}

\section{Continuous Improvement Cycle}

\begin{enumerate}
\item \textbf{Assess}: Quarterly CPF Score calculation
\item \textbf{Analyze}: Identify low-performing domains
\item \textbf{Intervene}: Implement targeted remediation
\item \textbf{Monitor}: Track indicator improvements
\item \textbf{Validate}: Verify score improvement
\item \textbf{Certify}: Achieve maturity level recognition
\end{enumerate}

% ============================================================
% APPENDICES
% ============================================================

\appendix

\section{Scoring Worksheets}

\subsection{Domain Score Calculation Worksheet}

\begin{table}[h]
\centering
\caption{Domain Scoring Template}
\small
\begin{tabular}{cccc}
\toprule
\textbf{Indicator} & \textbf{Score (0/1/2)} & \textbf{Weight} & \textbf{Weighted Score} \\
\midrule
X.1 & \_\_\_ & w$_1$ & \_\_\_ \\
X.2 & \_\_\_ & w$_2$ & \_\_\_ \\
X.3 & \_\_\_ & w$_3$ & \_\_\_ \\
... & ... & ... & ... \\
X.10 & \_\_\_ & w$_{10}$ & \_\_\_ \\
\midrule
\textbf{Total} & & & \_\_\_/20 \\
\bottomrule
\end{tabular}
\end{table}

\subsection{CPF Score Calculation Worksheet}

\begin{align*}
\text{Weighted Sum} &= \sum_{d=1}^{10} w_d \times \text{Domain\_Score}_d \\
&= (\_\_\_ \times 0.15) + (\_\_\_ \times 0.12) + ... \\
&= \_\_\_
\end{align*}

\begin{equation*}
\text{CPF\_Score} = 100 - (\text{Weighted Sum} \times 2.5) = \_\_\_
\end{equation*}

\section{Maturity Assessment Checklist}

\subsection{Level 1 Checklist}

\begin{itemize}
\item[$\square$] Executive awareness briefing completed
\item[$\square$] Initial CPF assessment (20+ indicators)
\item[$\square$] Psychological factors in incident reports
\item[$\square$] Basic psychology in awareness program
\item[$\square$] CPF Score $>$ 20
\item[$\square$] 3+ categories assessed
\end{itemize}

\subsection{Level 2 Checklist}

\begin{itemize}
\item[$\square$] Full 100-indicator assessment completed
\item[$\square$] Vulnerability heat map maintained
\item[$\square$] Psychological factors in response playbooks
\item[$\square$] Security team psychology training
\item[$\square$] CPF Score $>$ 40
\item[$\square$] 7+ categories monitored
\item[$\square$] Quarterly assessment established
\end{itemize}

\subsection{Level 3 Checklist}

\begin{itemize}
\item[$\square$] Real-time CPF dashboard operational
\item[$\square$] Predictive models implemented
\item[$\square$] Cross-functional integration (HR/IT/Risk)
\item[$\square$] Role-specific interventions deployed
\item[$\square$] CPF Score $>$ 60
\item[$\square$] All 10 categories with KPIs
\item[$\square$] Monthly assessment + daily monitoring
\end{itemize}

\section{Benchmark Data Tables}

\subsection{CPF Score Distribution by Sector}

\begin{table}[h]
\centering
\caption{CPF Score Benchmarks (Mean ± SD)}
\begin{tabular}{lcc}
\toprule
\textbf{Sector} & \textbf{Mean Score} & \textbf{75th Percentile} \\
\midrule
Financial Services & 68 ± 12 & 76 \\
Healthcare & 52 ± 15 & 63 \\
Technology & 71 ± 11 & 78 \\
Government & 58 ± 14 & 67 \\
Retail & 48 ± 13 & 56 \\
Manufacturing & 54 ± 12 & 62 \\
Energy/Utilities & 63 ± 13 & 72 \\
\bottomrule
\end{tabular}
\end{table}

\section{Glossary}

\textbf{ARQ (Authority Resilience Quotient)}: Domain-specific quotient measuring resistance to authority-based exploitation.

\textbf{Convergence Index (CI)}: Multiplicative risk metric measuring alignment of multiple vulnerabilities.

\textbf{CPF Score}: Overall organizational psychological vulnerability score (0-100 scale, higher = better resilience).

\textbf{Domain Quotient (DQ)}: Category-specific resilience metric (0-20 scale).

\textbf{Maturity Level}: Organizational capability level (0-5) in psychological vulnerability management.

\textbf{Pre-Cognitive Vulnerability}: Psychological weakness operating below conscious awareness.

\textbf{Ternary Scoring}: Three-level vulnerability assessment (Green/Yellow/Red or 0/1/2).

% ============================================================
% BIBLIOGRAPHY
% ============================================================

\begin{thebibliography}{99}

\bibitem{cpf27001}
Canale, G. (2025). CPF-27001:2025 Sistema di Gestione della Vulnerabilità Psicologica -- Requisiti.

\bibitem{canale2025}
Canale, G. (2025). The Cybersecurity Psychology Framework. \textit{SSRN Electronic Journal}.

\bibitem{milgram1974}
Milgram, S. (1974). \textit{Obedience to authority}. New York: Harper \& Row.

\bibitem{bion1961}
Bion, W. R. (1961). \textit{Experiences in groups}. London: Tavistock Publications.

\bibitem{kahneman2011}
Kahneman, D. (2011). \textit{Thinking, fast and slow}. New York: Farrar, Straus and Giroux.

\bibitem{klein1946}
Klein, M. (1946). Notes on some schizoid mechanisms. \textit{International Journal of Psychoanalysis}, 27, 99-110.

\bibitem{cialdini2007}
Cialdini, R. B. (2007). \textit{Influence: The psychology of persuasion}. New York: Collins.

\bibitem{verizon2023}
Verizon. (2023). \textit{2023 Data Breach Investigations Report}. Verizon Enterprise Solutions.

\bibitem{hofstede2001}
Hofstede, G. (2001). \textit{Culture's consequences: Comparing values, behaviors, institutions and organizations across nations}. Thousand Oaks, CA: Sage Publications.

\bibitem{jung1969}
Jung, C. G. (1969). \textit{The Archetypes and the Collective Unconscious}. Princeton: Princeton University Press.

\end{thebibliography}

\end{document}