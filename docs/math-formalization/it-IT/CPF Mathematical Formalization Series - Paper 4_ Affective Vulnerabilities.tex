\documentclass[11pt,a4paper]{article}
\usepackage{times}
\usepackage{amsmath}
\usepackage{amssymb}
\usepackage{amsfonts}
\usepackage[margin=1in]{geometry}
\usepackage{algorithm}
\usepackage{algorithmic}
\usepackage{hyperref}
\usepackage{booktabs}
\usepackage{graphicx}
\usepackage{cite}
\usepackage{array}

% Setup hyperref
\hypersetup{
    colorlinks=true,
    linkcolor=blue,
    citecolor=blue,
    urlcolor=blue,
    pdftitle={CPF Mathematical Formalization Series - Paper 4},
    pdfauthor={Giuseppe Canale},
}

\title{CPF Mathematical Formalization Series - Paper 4:\\Affective Vulnerabilities: Modelli di Stato Emotivo e Compromissione delle Decisioni di Sicurezza}

\author{
    Giuseppe Canale, CISSP\\
    Independent Researcher\\
    \texttt{g.canale@cpf3.org}\\
    ORCID: 0009-0007-3263-6897
}

\date{\today}

\begin{document}

\maketitle

\begin{abstract}
Presentiamo la formalizzazione matematica completa degli indicatori della Categoria 4 del Cybersecurity Psychology Framework (CPF): Vulnerabilità Affettive. Ciascuno dei dieci indicatori (4.1-4.10) è definito matematicamente attraverso modelli di interazione emozione-cognizione, dinamiche dello stato affettivo e funzioni di compromissione decisionale. La formalizzazione integra la teoria delle relazioni oggettuali di Klein, il framework dell'attaccamento di Bowlby e la neuroscienza affettiva contemporanea per quantificare come gli stati emotivi compromettono sistematicamente il processo decisionale di sicurezza. Forniamo algoritmi espliciti per il monitoraggio dello stato emotivo in tempo reale, la valutazione della qualità decisionale e il rilevamento dei bias affettivi. Questo lavoro stabilisce il fondamento matematico per comprendere come le emozioni creano vulnerabilità di sicurezza prevedibili che persistono nonostante i controlli tecnici e la consapevolezza cosciente.
\end{abstract}

\textbf{Keywords:} Applied Mathematics, Interdisciplinary Psychology, Computational Statistics, Mathematical Modeling, Cybersecurity Research

\section{Introduzione e Contesto CPF}

Il Cybersecurity Psychology Framework (CPF) riconosce che le decisioni di sicurezza avvengono all'interno di contesti emotivi che alterano fondamentalmente la valutazione del rischio e le scelte comportamentali \cite{canale2024cpf}. Mentre le Categorie 1-3 hanno esaminato le vulnerabilità di autorità, temporali e sociali, la Categoria 4 affronta come gli stati affettivi creano bias sistematici che gli avversari possono prevedere e sfruttare.

I modelli di sicurezza tradizionali presumono decisori emotivamente neutrali che valutano i rischi razionalmente. Tuttavia, la ricerca neuroscientifica dimostra che le emozioni non sono ostacoli al pensiero razionale ma componenti integrali di tutti i processi decisionali \cite{damasio1994}. La corteccia prefrontale ventromediale, critica per le decisioni rilevanti per la sicurezza, riceve input diretti dai centri di elaborazione emotiva, rendendo la valutazione della sicurezza puramente razionale neurologicamente impossibile.

Le vulnerabilità affettive differiscono dalle altre vulnerabilità psicologiche nelle loro dinamiche temporali e base fisiologica. A differenza dei bias cognitivi che rispondono all'intervento logico, gli stati emotivi coinvolgono cambiamenti neurochimici che persistono indipendentemente dalla consapevolezza cosciente. Questo crea vulnerabilità che rimangono attive anche quando gli individui riconoscono il loro stato emotivo e tentano la correzione razionale.

I modelli matematici presentati integrano tre framework teorici complementari: (1) la teoria psicoanalitica delle relazioni oggettuali per comprendere il transfert emotivo verso sistemi e minacce, (2) la teoria dell'attaccamento per modellare le relazioni di fiducia con l'infrastruttura di sicurezza, e (3) la neuroscienza affettiva per quantificare le interazioni emozione-cognizione. Questo approccio multi-teorico garantisce una copertura completa dei meccanismi di vulnerabilità affettiva.

\section{Fondamento Teorico: Psicologia Affettiva e Sicurezza}

Le vulnerabilità affettive emergono dall'architettura fondamentale dell'interazione emozione-cognizione umana \cite{ledoux2000}. Il sistema limbico elabora le informazioni sulle minacce 200-300ms prima della consapevolezza cosciente, creando risposte emotive che influenzano l'analisi razionale successiva. Questa precedenza temporale rende le vulnerabilità emotive particolarmente pericolose nei contesti di sicurezza dove sono richieste decisioni rapide.

La teoria delle relazioni oggettuali di Klein \cite{klein1946} fornisce intuizioni cruciali su come gli individui sviluppano relazioni emotive con concetti di sicurezza astratti. I sistemi di sicurezza, le politiche e le minacce diventano oggetti interiorizzati che evocano risposte emotive complesse inclusa l'idealizzazione, l'ansia di persecuzione e le fantasie riparative. Queste relazioni emotive inconsce creano pattern prevedibili di comportamento di sicurezza che persistono attraverso i cambiamenti di politica consci.

La teoria dell'attaccamento di Bowlby \cite{bowlby1969} spiega come i pattern relazionali precoci influenzano la fiducia nell'infrastruttura di sicurezza. L'attaccamento sicuro promuove una valutazione equilibrata del rischio, mentre gli stili di attaccamento insicuro creano distorsioni sistematiche: l'attaccamento ansioso porta a un'eccessiva dipendenza dai sistemi di sicurezza, l'attaccamento evitante promuove l'indipendenza di sicurezza a costo della collaborazione, e l'attaccamento disorganizzato crea comportamenti di sicurezza inconsistenti.

La neuroscienza affettiva contemporanea rivela che le emozioni servono come marcatori somatici che guidano il processo decisionale attraverso il riconoscimento rapido di pattern \cite{damasio1994}. Nei contesti di sicurezza, queste scorciatoie emotive spesso riflettono modelli di minaccia obsoleti o generalizzazioni inappropriate dall'esperienza personale al rischio organizzativo.

La formalizzazione matematica cattura questi meccanismi attraverso dinamiche dello stato emotivo, funzioni di fiducia mediate dall'attaccamento e modelli di compromissione decisionale che tengono conto delle influenze emotive sia consce che inconsce sul comportamento di sicurezza.

\section{Formalizzazione Matematica}

\subsection{Framework di Rilevamento Affettivo Universale}

Ogni indicatore di vulnerabilità affettiva impiega la funzione di rilevamento unificata con ponderazione dello stato emotivo:

\begin{equation}
D_i(t) = w_1 \cdot R_i(t) + w_2 \cdot A_i(t) + w_3 \cdot E_i(t) + w_4 \cdot T_i(t)
\end{equation}

dove $D_i(t)$ rappresenta il punteggio di rilevamento, $R_i(t)$ denota il rilevamento basato su regole, $A_i(t)$ rappresenta il punteggio di anomalia, $E_i(t)$ rappresenta il fattore di stato emotivo, e $T_i(t)$ rappresenta le dinamiche emotive temporali.

L'evoluzione dello stato emotivo incorpora sia risposte immediate che momento emotivo:

\begin{equation}
E_i(t) = \alpha \cdot Emotion_i(t) + \beta \cdot E_i(t-1) + \gamma \cdot \sum_{j} Emotional\_Contagion_{j,i}(t)
\end{equation}

dove $\gamma$ cattura gli effetti di contagio emotivo da altri individui nella rete.

\subsection{Indicatore 4.1: Paralisi Decisionale Basata sulla Paura}

\textbf{Definizione:} Compromissione decisionale di sicurezza attraverso una risposta di paura travolgente che porta all'inazione o all'azione ritardata.

\textbf{Modello Matematico:}

Funzione di intensità della paura che incorpora percezione della minaccia e valutazione della vulnerabilità:
\begin{equation}
F_i(t) = P_{threat}(t) \cdot V_{vulnerability}(t) \cdot \frac{1}{C_{control}(t) + \epsilon}
\end{equation}

dove $P_{threat}$ rappresenta la probabilità di minaccia percepita, $V_{vulnerability}$ misura la vulnerabilità personale percepita, e $C_{control}$ rappresenta il controllo percepito sui risultati.

\textbf{Modello di Paralisi Decisionale:}
\begin{equation}
DP(F,T,O) = \frac{F^{\alpha}}{F^{\alpha} + \beta^{\alpha}} \cdot \left(1 - \frac{1}{1 + e^{-\gamma(T - T_0)}}\right) \cdot \frac{1}{1 + O}
\end{equation}

dove $F$ è l'intensità della paura, $T$ è la pressione temporale, $T_0$ è la soglia di paralisi, e $O$ è il numero di opzioni.

\textbf{Funzione di Ritardo d'Azione:}
\begin{equation}
AD(t) = \tau_0 \cdot e^{\delta \cdot F(t)} \cdot \left(1 + \zeta \cdot Uncertainty(t)\right)
\end{equation}

dove $\tau_0$ è il tempo decisionale di base e $\delta, \zeta$ sono parametri di scala.

\textbf{Funzione di Rilevamento:}
\begin{equation}
D_{4.1}(t) = \begin{cases}
1 & \text{se } DP > 0.7 \text{ e } AD > 2 \cdot AD_{baseline} \\
0 & \text{altrimenti}
\end{cases}
\end{equation}

\subsection{Indicatore 4.2: Assunzione di Rischi Indotta dalla Rabbia}

\textbf{Definizione:} Aumento della tolleranza al rischio e decisioni di sicurezza impulsive guidate da rabbia o frustrazione.

\textbf{Modello Matematico:}

Valutazione del rischio modulata dalla rabbia:
\begin{equation}
R_{perceived}(A) = R_{objective} \cdot (1 - \alpha \cdot A) + \beta \cdot A \cdot R_{reward\_focus}
\end{equation}

dove $A$ rappresenta l'intensità della rabbia, $\alpha$ controlla la minimizzazione del rischio, e $\beta$ controlla l'amplificazione della ricompensa.

\textbf{Modello di Potenziamento dell'Impulsività:}
\begin{equation}
I(A,T) = I_0 \cdot \left(1 + \gamma \cdot A\right) \cdot e^{-\delta \cdot T}
\end{equation}

dove $I_0$ è l'impulsività di base, $T$ è il tempo di riflessione, e $\gamma, \delta$ sono fattori di scala.

\textbf{Funzione di Deplezione Regolatoria:}
\begin{equation}
RD(t) = RD_0 \cdot e^{-\lambda \cdot \int_0^t Anger(\tau) d\tau}
\end{equation}

rappresentando la diminuzione della capacità di autoregolazione sotto rabbia sostenuta.

\textbf{Probabilità di Assunzione di Rischi:}
\begin{equation}
P_{risk}(A,RD) = \frac{A^{\mu} \cdot (1 - RD)^{\nu}}{1 + A^{\mu} \cdot (1 - RD)^{\nu}}
\end{equation}

\textbf{Algoritmo di Rilevamento:}
\begin{equation}
D_{4.2}(t) = \begin{cases}
1 & \text{se } P_{risk} > 0.6 \text{ e } I > I_{threshold} \\
0 & \text{altrimenti}
\end{cases}
\end{equation}

\subsection{Indicatore 4.3: Transfert di Fiducia verso i Sistemi}

\textbf{Definizione:} Relazioni di fiducia emotiva inappropriate con i sistemi di sicurezza basate su pattern di relazioni umane.

\textbf{Modello Matematico:}

Funzione di fiducia basata sull'attaccamento:
\begin{equation}
T_{system}(A,R,S) = w_1 \cdot A_{security} + w_2 \cdot R_{reliability} + w_3 \cdot S_{similarity} + w_4 \cdot A \cdot S
\end{equation}

dove $A_{security}$ è la sicurezza dell'attaccamento, $R_{reliability}$ è l'affidabilità del sistema, $S_{similarity}$ è la somiglianza umana, e $w_4$ cattura gli effetti di interazione.

\textbf{Errore di Calibrazione della Fiducia:}
\begin{equation}
TCE = |T_{system} - T_{appropriate}|
\end{equation}

\textbf{Indice di Antropomorfizzazione:}
\begin{equation}
AI = \sum_{attributes} w_{attr} \cdot \frac{Human\_Attribution_{attr}}{System\_Capability_{attr} + \epsilon}
\end{equation}

\textbf{Modello di Rilevamento del Transfert:}
\begin{equation}
TD(H,S) = \frac{\sum_{patterns} Similarity(H_{pattern}, S_{interaction})}{\sum_{patterns} 1}
\end{equation}

dove i pattern includono dinamiche di dipendenza, idealizzazione e controllo.

\textbf{Framework di Rilevamento:}
\begin{equation}
R_{4.3}(t) = \begin{cases}
1 & \text{se } TCE > 0.5 \text{ e } AI > 0.7 \\
0 & \text{altrimenti}
\end{cases}
\end{equation}

\subsection{Indicatore 4.4: Attaccamento ai Sistemi Legacy}

\textbf{Definizione:} Resistenza emotiva agli aggiornamenti di sicurezza basata su relazioni di attaccamento con sistemi familiari.

\textbf{Modello Matematico:}

Forza dell'attaccamento legacy:
\begin{equation}
LA(F,T,S) = \alpha \cdot F_{familiarity}^{\beta} + \gamma \cdot T_{time\_invested} + \delta \cdot S_{success\_history}
\end{equation}

dove familiarità, investimento di tempo e storia di successo creano legami emotivi.

\textbf{Funzione di Resistenza al Cambiamento:}
\begin{equation}
CR(LA,U) = \frac{LA^{\eta}}{LA^{\eta} + (U \cdot Necessity)^{\eta}}
\end{equation}

dove $U$ rappresenta la comprensione dei benefici del cambiamento e $\eta$ controlla la ripidità della curva di resistenza.

\textbf{Avversione alla Perdita nel Cambiamento di Sistema:}
\begin{equation}
LAV = \lambda \cdot (Functionality_{lost} + Familiarity_{lost}) - Improvements_{gained}
\end{equation}

con coefficiente di avversione alla perdita $\lambda > 1$.

\textbf{Modello di Ritardo della Modernizzazione:}
\begin{equation}
MD(t) = \int_0^t CR(\tau) \cdot Security\_Gap\_Increase(\tau) d\tau
\end{equation}

\textbf{Funzione di Rilevamento:}
\begin{equation}
D_{4.4}(t) = \begin{cases}
1 & \text{se } CR > 0.8 \text{ e } MD > MD_{critical} \\
0 & \text{altrimenti}
\end{cases}
\end{equation}

\subsection{Indicatore 4.5: Occultamento di Sicurezza Basato sulla Vergogna}

\textbf{Definizione:} Occultamento di incidenti o vulnerabilità di sicurezza dovuti alla vergogna, che porta a ritardi nella risposta e aumento dei danni.

\textbf{Modello Matematico:}

Funzione di intensità della vergogna:
\begin{equation}
S_i(E,R,V) = \alpha \cdot E_{exposure} \cdot R_{responsibility} + \beta \cdot V_{vulnerability\_feeling} + \gamma \cdot C_{competence\_threat}
\end{equation}

dove esposizione, responsabilità e minaccia alla competenza contribuiscono all'intensità della vergogna.

\textbf{Modello di Inibizione della Divulgazione:}
\begin{equation}
DI(S,T) = \frac{S^{\delta}}{S^{\delta} + (Support_{available} \cdot T_{trust})^{\delta}}
\end{equation}

dove il supporto disponibile e la fiducia interpersonale moderano gli effetti della vergogna.

\textbf{Probabilità di Occultamento dell'Incidente:}
\begin{equation}
P_{conceal}(S,C,T) = \sigma\left(\epsilon \cdot S - \zeta \cdot C_{consequences\_known} - \eta \cdot T_{time\_pressure}\right)
\end{equation}

\textbf{Fattore di Amplificazione del Danno:}
\begin{equation}
DAF(t) = 1 + \theta \cdot \int_0^t P_{conceal}(\tau) \cdot Threat\_Active(\tau) d\tau
\end{equation}

\textbf{Algoritmo di Rilevamento:}
\begin{equation}
D_{4.5}(t) = \begin{cases}
1 & \text{se } DI > 0.6 \text{ e } DAF > 1.5 \\
0 & \text{altrimenti}
\end{cases}
\end{equation}

\subsection{Indicatore 4.6: Iperconformità Guidata dal Senso di Colpa}

\textbf{Definizione:} Comportamenti di sicurezza eccessivi guidati dal senso di colpa, che portano a un'allocazione inefficiente delle risorse e potenziale teatro di sicurezza.

\textbf{Modello Matematico:}

Funzione di conformità mediata dal senso di colpa:
\begin{equation}
GC(G,P) = Baseline_{compliance} + \alpha \cdot G^{\beta} \cdot P_{past\_failures}
\end{equation}

dove l'intensità del senso di colpa e le esperienze di fallimenti passati guidano l'iperconformità.

\textbf{Indice di Disallocazione delle Risorse:}
\begin{equation}
RMI = \frac{\sum_{measures} Effort_{measure} \cdot (1 - Effectiveness_{measure})}{\sum_{measures} Effort_{measure}}
\end{equation}

\textbf{Rilevamento del Teatro di Sicurezza:}
\begin{equation}
ST = \frac{Visible\_Security\_Measures}{Effective\_Security\_Measures}
\end{equation}

\textbf{Modello di Perpetuazione del Senso di Colpa:}
\begin{equation}
\frac{dG}{dt} = -\kappa \cdot G + \lambda \cdot Trigger_{events} + \mu \cdot Overcompliance\_Failure
\end{equation}

\textbf{Framework di Rilevamento:}
\begin{equation}
R_{4.6}(t) = \begin{cases}
1 & \text{se } RMI > 0.4 \text{ e } ST > 1.8 \\
0 & \text{altrimenti}
\end{cases}
\end{equation}

\subsection{Indicatore 4.7: Errori Scatenati dall'Ansia}

\textbf{Definizione:} Aumento dei tassi di errore nelle procedure di sicurezza dovuto alla compromissione cognitiva indotta dall'ansia.

\textbf{Modello Matematico:}

Relazione ansia-prestazione (curva a U invertita):
\begin{equation}
Performance(A) = P_{max} \cdot e^{-\frac{(A - A_{optimal})^2}{2\sigma^2}}
\end{equation}

dove $A_{optimal}$ rappresenta il livello di ansia ottimale e $\sigma$ controlla la larghezza della curva.

\textbf{Potenziamento del Tasso di Errore:}
\begin{equation}
ER(A) = ER_{baseline} \cdot \left(1 + \alpha \cdot max(0, A - A_{threshold})\right)
\end{equation}

\textbf{Effetto di Restringimento dell'Attenzione:}
\begin{equation}
AN(A) = Attention_{baseline} \cdot e^{-\beta \cdot A}
\end{equation}

\textbf{Compromissione della Memoria di Lavoro:}
\begin{equation}
WMI(A) = WM_{capacity} \cdot \frac{1}{1 + \gamma \cdot A}
\end{equation}

\textbf{Funzione di Probabilità di Errore:}
\begin{equation}
P_{mistake}(A,C) = \frac{ER(A) \cdot (1 - AN(A)) \cdot (1 - WMI(A))}{C_{complexity} + \epsilon}
\end{equation}

\textbf{Algoritmo di Rilevamento:}
\begin{equation}
D_{4.7}(t) = \begin{cases}
1 & \text{se } P_{mistake} > 0.3 \text{ e } A > A_{threshold} \\
0 & \text{altrimenti}
\end{cases}
\end{equation}

\subsection{Indicatore 4.8: Negligenza Correlata alla Depressione}

\textbf{Definizione:} Ridotta vigilanza di sicurezza e comportamenti di manutenzione dovuti all'apatia e alla compromissione cognitiva indotte dalla depressione.

\textbf{Modello Matematico:}

Impatto della depressione sulla motivazione di sicurezza:
\begin{equation}
SM(D) = SM_{baseline} \cdot e^{-\alpha \cdot D} \cdot (1 - \beta \cdot Anhedonia)
\end{equation}

dove $D$ rappresenta la gravità della depressione e l'anedonia misura la ridotta sensibilità alla ricompensa.

\textbf{Declino del Comportamento di Manutenzione:}
\begin{equation}
MB(D,T) = MB_0 \cdot e^{-\gamma \cdot D \cdot T} \cdot \frac{1}{1 + \delta \cdot Fatigue}
\end{equation}

\textbf{Modello di Degradazione della Vigilanza:}
\begin{equation}
VD(t) = \int_0^t Depression(\tau) \cdot Cognitive\_Load(\tau) \cdot e^{-\lambda(t-\tau)} d\tau
\end{equation}

\textbf{Smussamento della Percezione del Rischio:}
\begin{equation}
RPB(D) = \frac{Risk_{perception}}{1 + \epsilon \cdot D \cdot Emotional\_Numbing}
\end{equation}

\textbf{Funzione di Rilevamento:}
\begin{equation}
D_{4.8}(t) = \begin{cases}
1 & \text{se } MB < 0.5 \cdot MB_0 \text{ e } RPB < 0.7 \\
0 & \text{altrimenti}
\end{cases}
\end{equation}

\subsection{Indicatore 4.9: Disattenzione Indotta dall'Euforia}

\textbf{Definizione:} Ridotta percezione del rischio e coscienziosità della sicurezza durante stati emotivi positivi.

\textbf{Modello Matematico:}

Percezione del rischio modulata dall'euforia:
\begin{equation}
RP(E) = RP_{baseline} \cdot \frac{1}{1 + \alpha \cdot E^{\beta}} + \gamma \cdot Optimism\_Bias
\end{equation}

dove $E$ rappresenta l'intensità dell'euforia e il bias di ottimismo riduce la salienza della minaccia.

\textbf{Potenziamento della Motivazione di Approccio:}
\begin{equation}
AM(E) = AM_{baseline} + \delta \cdot E \cdot (1 - Inhibitory\_Control)
\end{equation}

\textbf{Scorciatoie delle Procedure di Sicurezza:}
\begin{equation}
SPS(E,T) = \frac{E^{\epsilon}}{E^{\epsilon} + T_{time\_pressure}^{\epsilon}} \cdot Convenience_{factor}
\end{equation}

\textbf{Funzione di Eccesso di Fiducia:}
\begin{equation}
OC(E,S) = Confidence_{baseline} + \zeta \cdot E - \eta \cdot S_{skill\_level}
\end{equation}

\textbf{Algoritmo di Rilevamento:}
\begin{equation}
D_{4.9}(t) = \begin{cases}
1 & \text{se } SPS > 0.5 \text{ e } OC > OC_{threshold} \\
0 & \text{altrimenti}
\end{cases}
\end{equation}

\subsection{Indicatore 4.10: Effetti di Contagio Emotivo}

\textbf{Definizione:} Diffusione degli stati emotivi attraverso le reti organizzative, amplificando le vulnerabilità emotive individuali.

\textbf{Modello Matematico:}

Propagazione del contagio emotivo:
\begin{equation}
\frac{dE_i}{dt} = -\alpha_i \cdot E_i + \sum_{j \in N(i)} \beta_{ij} \cdot f(E_j - E_i) + \xi_i(t)
\end{equation}

dove $f(x) = tanh(\gamma \cdot x)$ rappresenta la funzione di contagio e $\xi_i(t)$ è il rumore emotivo individuale.

\textbf{Stato Emotivo di Rete:}
\begin{equation}
NES(t) = \sum_{i} w_i \cdot E_i(t)
\end{equation}

con pesi $w_i$ basati sulla centralità di rete e sull'influenza.

\textbf{Fattore di Amplificazione del Contagio:}
\begin{equation}
CAF = \frac{Variance(E_{group}) - Variance(E_{individual})}{Variance(E_{individual})}
\end{equation}

\textbf{Probabilità di Cascata Emotiva:}
\begin{equation}
ECP(t) = \Phi\left(\frac{NES(t) - \mu_{threshold}}{\sigma_{noise}}\right)
\end{equation}

dove $\Phi$ è la distribuzione normale cumulativa.

\textbf{Framework di Rilevamento:}
\begin{equation}
D_{4.10}(t) = \begin{cases}
1 & \text{se } CAF > 2.0 \text{ e } ECP > 0.8 \\
0 & \text{altrimenti}
\end{cases}
\end{equation}

\section{Matrice di Interdipendenza}

Gli indicatori di vulnerabilità affettiva mostrano interdipendenze complesse catturate attraverso la matrice di correlazione $\mathbf{R}_{4}$:

\begin{equation}
\mathbf{R}_4 = \begin{pmatrix}
1.00 & -0.45 & 0.35 & 0.30 & 0.60 & 0.25 & 0.75 & 0.40 & -0.55 & 0.50 \\
-0.45 & 1.00 & -0.20 & -0.25 & -0.40 & -0.30 & -0.35 & -0.50 & 0.40 & 0.30 \\
0.35 & -0.20 & 1.00 & 0.70 & 0.45 & 0.40 & 0.30 & 0.25 & -0.30 & 0.35 \\
0.30 & -0.25 & 0.70 & 1.00 & 0.55 & 0.35 & 0.25 & 0.45 & -0.35 & 0.40 \\
0.60 & -0.40 & 0.45 & 0.55 & 1.00 & 0.50 & 0.65 & 0.35 & -0.45 & 0.55 \\
0.25 & -0.30 & 0.40 & 0.35 & 0.50 & 1.00 & 0.30 & 0.25 & -0.25 & 0.35 \\
0.75 & -0.35 & 0.30 & 0.25 & 0.65 & 0.30 & 1.00 & 0.45 & -0.40 & 0.60 \\
0.40 & -0.50 & 0.25 & 0.45 & 0.35 & 0.25 & 0.45 & 1.00 & -0.65 & 0.30 \\
-0.55 & 0.40 & -0.30 & -0.35 & -0.45 & -0.25 & -0.40 & -0.65 & 1.00 & -0.35 \\
0.50 & 0.30 & 0.35 & 0.40 & 0.55 & 0.35 & 0.60 & 0.30 & -0.35 & 1.00
\end{pmatrix}
\end{equation}

Le interdipendenze chiave includono:
\begin{itemize}
\item Forte correlazione (0.75) tra Paralisi da Paura (4.1) ed Errori da Ansia (4.7)
\item Alta correlazione (0.70) tra Transfert di Fiducia (4.3) e Attaccamento Legacy (4.4)
\item Forte correlazione negativa (-0.65) tra Negligenza da Depressione (4.8) e Disattenzione da Euforia (4.9)
\item Significativa correlazione (0.65) tra Occultamento da Vergogna (4.5) ed Errori da Ansia (4.7)
\end{itemize}

\textbf{Dipendenze Inter-Categoria:}
Relazioni critiche con le vulnerabilità di Autorità (Categoria 1), Temporali (Categoria 2) e Sociali (Categoria 3):
\begin{itemize}
\item $R_{1.5,4.1} = 0.80$: Conformità basata sulla paura correla fortemente con la paralisi da paura
\item $R_{2.3,4.9} = 0.70$: Pressione della scadenza correla con disattenzione indotta dall'euforia
\item $R_{3.9,4.5} = 0.75$: Minacce all'identità sociale correlano fortemente con l'occultamento basato sulla vergogna
\item $R_{1.1,4.3} = 0.65$: Conformità all'autorità correla con il transfert di fiducia al sistema
\end{itemize}

\section{Algoritmi di Implementazione}

\begin{algorithm}
\caption{Valutazione della Vulnerabilità Affettiva}
\begin{algorithmic}[1]
\STATE Inizializza i parametri emotivi $\boldsymbol{\alpha}, \boldsymbol{\beta}, \boldsymbol{\gamma}$
\STATE Carica i profili emotivi di base e i pattern di attaccamento
\FOR{ogni passo temporale $t$}
    \STATE Estrai il contesto emotivo: physiological\_markers(t), behavioral\_patterns(t)
    \STATE Calcola le dinamiche dello stato emotivo per ogni individuo
    \FOR{ogni indicatore $i \in \{4.1, 4.2, \ldots, 4.10\}$}
        \STATE Calcola le metriche di intensità emotiva $E_i(t)$
        \STATE Calcola il rilevamento basato su regole $R_i(t)$
        \STATE Calcola il punteggio di anomalia ponderato dall'emozione $A_i(t)$
        \STATE Valuta le dinamiche emotive temporali $T_i(t)$
        \STATE Calcola il punteggio di rilevamento $D_i(t)$
        \STATE Applica il modello di propagazione del contagio emotivo
        \STATE Aggiorna gli stati delle relazioni di attaccamento
    \ENDFOR
    \STATE Calcola le correzioni di interdipendenza usando $\mathbf{R}_4$
    \STATE Applica le correlazioni inter-categoria con Categorie 1-3
    \STATE Genera avvisi consapevoli dell'emozione con previsioni di contagio
    \STATE Aggiorna i modelli di baseline emotiva
    \STATE Registra i risultati per la raffinazione dei pattern affettivi
\ENDFOR
\end{algorithmic}
\end{algorithm}

\begin{algorithm}
\caption{Analisi della Rete di Contagio Emotivo}
\begin{algorithmic}[1]
\STATE Input: Rete di comunicazione $G(V,E)$, stati emotivi $\mathbf{E}(t)$
\STATE Inizializza i parametri di contagio $\beta_{ij}$, tassi di decadimento $\alpha_i$
\FOR{ogni passo temporale $t$}
    \FOR{ogni individuo $i \in V$}
        \STATE Calcola l'influenza emotiva dai vicini
        \STATE Calcola la suscettibilità emotiva individuale
        \STATE Applica l'equazione differenziale di contagio
        \STATE Aggiorna lo stato emotivo $E_i(t+1)$
    \ENDFOR
    \STATE Calcola i momenti emotivi di rete (media, varianza)
    \STATE Rileva l'inizio di cascate emotive
    \STATE Identifica i super-diffusori emotivi
    \STATE Prevedi la traiettoria di contagio
    \STATE Genera avvisi precoci per crisi emotive
    \STATE Aggiorna i parametri di contagio di rete
\ENDFOR
\STATE Restituisci la mappa di vulnerabilità emotiva con previsioni di propagazione
\end{algorithmic}
\end{algorithm}

\section{Framework di Validazione}

La validazione della vulnerabilità affettiva richiede metriche specializzate che tengono conto delle dinamiche emotive e delle differenze individuali:

\textbf{Metriche di Classificazione dello Stato Emotivo:}
\begin{align}
Precision_{emotional} &= \frac{\sum_{states} |TP_{state}| \cdot w_{state}}{\sum_{states} |TP_{state} + FP_{state}| \cdot w_{state}} \\
Recall_{emotional} &= \frac{\sum_{states} |TP_{state}| \cdot w_{state}}{\sum_{states} |TP_{state} + FN_{state}| \cdot w_{state}}
\end{align}

dove $w_{state}$ fornisce la ponderazione di importanza per diversi stati emotivi.

\textbf{Validazione delle Dinamiche Emotive Temporali:}
Errore Assoluto Medio per la previsione dello stato emotivo:
\begin{equation}
MAE_{emotion} = \frac{1}{T} \sum_{t=1}^{T} |Emotion_{predicted}(t) - Emotion_{actual}(t)|
\end{equation}

\textbf{Accuratezza della Propagazione del Contagio:}
Previsione della diffusione emotiva a livello di rete:
\begin{equation}
CPA = 1 - \frac{|Predicted\_Affected \triangle Actual\_Affected|}{|Actual\_Affected|}
\end{equation}

\textbf{Correlazione della Compromissione Decisionale:}
Correlazione di Pearson tra stato emotivo e qualità decisionale:
\begin{equation}
r_{emotion,decision} = \frac{\sum_{i}(E_i - \bar{E})(D_i - \bar{D})}{\sqrt{\sum_{i}(E_i - \bar{E})^2 \sum_{i}(D_i - \bar{D})^2}}
\end{equation}

\textbf{Validazione del Pattern di Attaccamento:}
Accuratezza della classificazione per i comportamenti di fiducia basati sull'attaccamento:
\begin{equation}
APA = \frac{Correct_{attachment\_classifications}}{Total_{attachment\_assessments}}
\end{equation}

\textbf{Allarme Precoce di Cascata Emotiva:}
Accuratezza della previsione per eventi di contagio emotivo:
\begin{equation}
ECEW = \frac{TP_{cascade\_predictions}}{TP_{cascade\_predictions} + FN_{cascade\_predictions}}
\end{equation}

Metriche di validazione target: Precision > 0.85, Recall > 0.80, MAE < 0.15, CPA > 0.75, ECEW > 0.70.

\section{Conclusione}

Questa formalizzazione matematica delle vulnerabilità affettive fornisce un framework completo per comprendere come gli stati emotivi compromettono sistematicamente il processo decisionale di sicurezza. L'integrazione della teoria psicoanalitica delle relazioni oggettuali, della teoria dell'attaccamento e della neuroscienza affettiva crea un fondamento robusto per prevedere e mitigare le debolezze di sicurezza basate sulle emozioni.

La matrice di interdipendenza rivela pattern critici: le emozioni negative (paura, ansia, vergogna, depressione) si raggruppano insieme e si amplificano a vicenda, mentre le emozioni positive (euforia) creano pattern di vulnerabilità distinti. Le forti correlazioni tra le vulnerabilità affettive e altre categorie CPF dimostrano che le emozioni agiscono sia come fonti di vulnerabilità dirette che come amplificatori per gli effetti di autorità, temporali e di influenza sociale.

I modelli matematici catturano le dinamiche non lineari dell'interazione emozione-cognizione, incluso il momento emotivo, gli effetti di contagio e le relazioni di fiducia mediate dall'attaccamento. Questo consente la previsione degli stati di vulnerabilità emotiva prima che si manifestino in incidenti di sicurezza osservabili, supportando strategie di sicurezza proattive piuttosto che reattive.

Gli algoritmi di implementazione forniscono il monitoraggio dello stato emotivo in tempo reale con modellazione del contagio consapevole della rete. Le organizzazioni possono anticipare le cascate di vulnerabilità emotiva basate sui pattern di comunicazione, gli indicatori di stress e le dinamiche delle relazioni di attaccamento, consentendo interventi mirati prima che si verifichi una degradazione diffusa della sicurezza.

I framework di validazione tengono conto delle sfide uniche della misurazione degli stati emotivi e delle loro implicazioni di sicurezza, incluse le differenze individuali, le dinamiche temporali e gli effetti di propagazione di rete. Le metriche proposte garantiscono sia rigore statistico che applicabilità pratica per i contesti di sicurezza organizzativi.

Il lavoro futuro estenderà questo approccio di modellazione affettiva al sovraccarico cognitivo (Categoria 5) e alle dinamiche di gruppo (Categoria 6), con particolare attenzione agli effetti di interazione emozione-cognizione e agli stati emotivi collettivi. Il fondamento matematico qui stabilito consente strategie di sicurezza emotiva basate sull'evidenza che lavorano con le realtà psicologiche umane piuttosto che contro di esse.

La categoria di vulnerabilità affettiva dimostra che la sicurezza è fondamentalmente un fenomeno emotivo oltre che cognitivo. Le emozioni non sono ostacoli al comportamento di sicurezza razionale ma componenti integrali che devono essere comprese e tenute in considerazione nella progettazione dei sistemi di sicurezza. Formalizzando matematicamente queste dinamiche emotive, consentiamo approcci di sicurezza che sfruttano piuttosto che combattono le risposte emotive umane, creando risultati di sicurezza più resilienti e sostenibili.

\begin{thebibliography}{9}

\bibitem{canale2024cpf}
Canale, G. (2024). The Cybersecurity Psychology Framework: A Pre-Cognitive Vulnerability Assessment Model Integrating Psychoanalytic and Cognitive Sciences. \textit{Preprint}.

\bibitem{damasio1994}
Damasio, A. R. (1994). \textit{Descartes' Error: Emotion, Reason, and the Human Brain}. New York: Putnam.

\bibitem{ledoux2000}
LeDoux, J. (2000). Emotion circuits in the brain. \textit{Annual Review of Neuroscience}, 23, 155-184.

\bibitem{klein1946}
Klein, M. (1946). Notes on some schizoid mechanisms. \textit{International Journal of Psychoanalysis}, 27, 99-110.

\bibitem{bowlby1969}
Bowlby, J. (1969). \textit{Attachment and Loss: Vol. 1. Attachment}. New York: Basic Books.

\end{thebibliography}

\end{document}
