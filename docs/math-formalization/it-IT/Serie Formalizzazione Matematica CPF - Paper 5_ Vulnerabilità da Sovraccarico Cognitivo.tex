\documentclass[11pt,a4paper]{article}
\usepackage{times}
\usepackage{amsmath}
\usepackage{amssymb}
\usepackage{amsfonts}
\usepackage[margin=1in]{geometry}
\usepackage{algorithm}
\usepackage{algorithmic}
\usepackage{hyperref}
\usepackage{booktabs}
\usepackage{graphicx}
\usepackage{cite}
\usepackage{array}

% Setup hyperref
\hypersetup{
    colorlinks=true,
    linkcolor=blue,
    citecolor=blue,
    urlcolor=blue,
    pdftitle={CPF Mathematical Formalization Series - Paper 5},
    pdfauthor={Giuseppe Canale},
}

\title{CPF Mathematical Formalization Series - Paper 5:\\Cognitive Overload Vulnerabilities: Modelli Matematici e Algoritmi di Rilevamento}

\author{
    Giuseppe Canale, CISSP\\
    Independent Researcher\\
    \texttt{g.canale@cpf3.org}\\
    ORCID: 0009-0007-3263-6897
}

\date{\today}

\begin{document}

\maketitle

\begin{abstract}
Presentiamo la formalizzazione matematica completa degli indicatori della Categoria 5 del Cybersecurity Psychology Framework (CPF): Vulnerabilità di Sovraccarico Cognitivo. Ciascuno dei dieci indicatori (5.1-5.10) è rigorosamente definito attraverso funzioni di rilevamento che combinano metriche della teoria dell'informazione, misurazione del carico cognitivo e analisi del carico di lavoro. La formalizzazione consente un'implementazione sistematica in diversi contesti organizzativi mantenendo il fondamento teorico nella ricerca sulla capacità cognitiva di Miller e nella teoria dell'attenzione contemporanea. Forniamo algoritmi espliciti per il rilevamento in tempo reale, matrici di interdipendenza per l'analisi di correlazione e metriche di validazione per la calibrazione continua. Questo lavoro stabilisce il fondamento matematico per l'operazionalizzazione delle vulnerabilità psicologiche basate sul sovraccarico cognitivo nei contesti di cybersecurity.
\end{abstract}

\textbf{Keywords:} Applied Mathematics, Interdisciplinary Psychology, Computational Statistics, Mathematical Modeling, Cybersecurity Research

\section{Introduzione e Contesto CPF}

Il Cybersecurity Psychology Framework (CPF) rappresenta un cambio di paradigma dalla consapevolezza di sicurezza reattiva alla valutazione predittiva delle vulnerabilità attraverso la modellazione dello stato psicologico \cite{canale2024cpf}. A differenza dei framework di sicurezza tradizionali che affrontano i controlli tecnici, il CPF identifica sistematicamente le vulnerabilità psicologiche pre-cognitive che creano punti ciechi di sicurezza sistematici.

L'architettura CPF comprende 100 indicatori organizzati in una matrice 10×10, ciascuno fondato su ricerca psicologica consolidata. Il framework impiega un sistema di valutazione ternario (Verde/Giallo/Rosso) mantenendo una rigorosa protezione della privacy attraverso l'analisi comportamentale aggregata piuttosto che la profilazione individuale.

Questa serie di articoli fornisce la formalizzazione matematica completa per ogni categoria CPF, consentendo un'implementazione e validazione rigorose. Ogni indicatore riceve funzioni di rilevamento esplicite, modellazione delle interdipendenze e specifiche algoritmiche. L'approccio matematico serve due scopi: garantire implementazioni riproducibili tra le organizzazioni e stabilire il CPF come metodologia scientificamente rigorosa adatta per la revisione tra pari e la standardizzazione.

La Categoria 5 si concentra sulle vulnerabilità di sovraccarico cognitivo, attingendo principalmente dalla ricerca pionieristica di Miller sulle limitazioni della capacità cognitiva \cite{miller1956} e dalla successiva ricerca di psicologia cognitiva sull'attenzione e l'elaborazione delle informazioni \cite{kahneman1973}. Queste vulnerabilità sfruttano le risorse cognitive limitate degli esseri umani, creando debolezze di sicurezza sistematiche quando le richieste di informazioni superano la capacità di elaborazione.

\section{Fondamento Teorico: Teoria del Carico Cognitivo}

Le vulnerabilità di sovraccarico cognitivo emergono dalle limitazioni fondamentali della capacità di elaborazione delle informazioni umane. Il lavoro seminale di Miller \cite{miller1956} ha dimostrato che gli esseri umani possono elaborare efficacemente circa $7 \pm 2$ unità di informazione discrete simultaneamente. La ricerca contemporanea ha raffinato questa comprensione, rivelando molteplici colli di bottiglia nell'architettura cognitiva \cite{baddeley1992}.

Il sistema cognitivo opera con tre vincoli primari: (1) la capacità della memoria di lavoro limita l'elaborazione di informazioni concorrenti, (2) i meccanismi di attenzione filtrano e danno priorità ai flussi di informazioni, e (3) il controllo esecutivo gestisce l'allocazione delle risorse tra richieste concorrenti \cite{miyake2000}. Quando le informazioni rilevanti per la sicurezza superano questi limiti di capacità, emergono vulnerabilità sistematiche attraverso pattern di degradazione prevedibili.

La ricerca dimostra che il sovraccarico cognitivo si manifesta attraverso firme comportamentali misurabili: aumento della latenza di risposta, tassi di errore elevati, qualità decisionale ridotta e attenzione ristretta \cite{sweller1988}. Queste firme forniscono indicatori osservabili per la modellazione matematica e i sistemi di rilevamento automatizzato.

I modelli matematici qui presentati catturano le dinamiche del carico cognitivo attraverso misure information-theoretic, analisi del carico di lavoro e pattern di distribuzione dell'attenzione. Ogni indicatore impiega approcci di rilevamento complementari: (1) calcoli dell'entropia dell'informazione per la valutazione della complessità, (2) analisi temporale per il rilevamento dei pattern di carico di lavoro, e (3) modellazione del tasso di errore per l'identificazione dell'overflow di capacità.

\section{Formalizzazione Matematica}

\subsection{Framework di Rilevamento Universale}

Ogni indicatore di sovraccarico cognitivo impiega la funzione di rilevamento unificata:

\begin{equation}
D_i(t) = w_1 \cdot I_i(t) + w_2 \cdot W_i(t) + w_3 \cdot E_i(t)
\end{equation}

dove $D_i(t)$ rappresenta il punteggio di rilevamento per l'indicatore $i$ al tempo $t$, $I_i(t)$ denota la misura del carico informativo, $W_i(t)$ rappresenta la metrica del carico di lavoro, e $E_i(t)$ rappresenta l'indicatore del tasso di errore. I pesi $w_1, w_2, w_3$ sommano all'unità e sono calibrati attraverso le baseline organizzative.

L'evoluzione temporale segue lo smoothing esponenziale con decadimento cognitivo:

\begin{equation}
T_i(t) = \alpha \cdot D_i(t) + (1-\alpha) \cdot T_i(t-1) \cdot e^{-\lambda \Delta t}
\end{equation}

dove $\alpha = e^{-\Delta t/\tau}$ fornisce lo smoothing temporale e $\lambda$ rappresenta il tasso di recupero cognitivo.

\subsection{Indicatore 5.1: Desensibilizzazione da Fatica di Avviso}

\textbf{Definizione:} Riduzione sistematica della reattività agli avvisi dovuta a volume eccessivo di notifiche di sicurezza.

\textbf{Modello Matematico:}

L'indice di fatica di avviso utilizzando la teoria dell'informazione:
\begin{equation}
AFI(t) = 1 - \frac{H(Response|Alert)}{H(Response)}
\end{equation}

dove $H(Response|Alert)$ rappresenta l'entropia condizionale delle risposte dati gli avvisi, e $H(Response)$ rappresenta l'entropia di risposta di base.

\textbf{Funzione di Desensibilizzazione:}
\begin{equation}
S(t) = S_0 \cdot e^{-\beta \int_0^t N_{alerts}(\tau) d\tau}
\end{equation}

dove $S_0$ è la sensibilità iniziale, $\beta$ è il tasso di desensibilizzazione, e $N_{alerts}(\tau)$ è il tasso di avvisi al tempo $\tau$.

\textbf{Soglia di Rilevamento:}
\begin{equation}
R_{5.1}(t) = \begin{cases}
1 & \text{se } \frac{R_{investigated}(t)}{R_{presented}(t)} < \theta_{response} \\
0 & \text{altrimenti}
\end{cases}
\end{equation}

dove $\theta_{response} = 0.3$ rappresenta la soglia critica del rapporto di risposta.

\textbf{Analisi del Pattern Temporale:}
\begin{equation}
P_{fatigue}(t) = \frac{d}{dt}\left(\frac{N_{false\_positive}(t)}{N_{total}(t)}\right)
\end{equation}

Quando $P_{fatigue}(t) > 0$, indica un aumento del tasso di dismissione dei falsi positivi caratteristico della fatica di avviso.

\subsection{Indicatore 5.2: Errori da Fatica Decisionale}

\textbf{Definizione:} Qualità decisionale degradata da richiesta cognitiva eccessiva in scelte sequenziali.

\textbf{Modello Matematico:}

La funzione di fatica decisionale che segue il decadimento a legge di potenza:
\begin{equation}
Q(n) = Q_0 \cdot n^{-\alpha}
\end{equation}

dove $Q(n)$ rappresenta la qualità decisionale dopo $n$ decisioni, $Q_0$ è la qualità iniziale, e $\alpha > 0$ è l'esponente di fatica.

\textbf{Modello di Deplezione Cognitiva:}
\begin{equation}
C(t) = C_{max} - \int_0^t E_{cognitive}(\tau) \cdot e^{-\lambda(t-\tau)} d\tau
\end{equation}

dove $C(t)$ rappresenta la capacità cognitiva disponibile, $E_{cognitive}(\tau)$ è l'impiego dello sforzo cognitivo, e $\lambda$ è il tasso di recupero.

\textbf{Correlazione del Tasso di Errore:}
\begin{equation}
E_{rate}(t) = E_{baseline} \cdot \left(1 + \gamma \cdot \max\left(0, \frac{C_{required}(t) - C(t)}{C_{max}}\right)\right)
\end{equation}

dove $\gamma$ rappresenta il fattore di amplificazione degli errori quando la capacità richiesta supera la capacità disponibile.

\textbf{Funzione di Rilevamento:}
\begin{equation}
D_{5.2}(t) = \frac{E_{rate}(t) - E_{baseline}}{E_{baseline}} \cdot \frac{N_{decisions}(t)}{N_{baseline}}
\end{equation}

\subsection{Indicatore 5.3: Paralisi da Sovraccarico Informativo}

\textbf{Definizione:} Paralisi decisionale risultante da volume di informazioni eccessivo che supera la capacità di elaborazione.

\textbf{Modello Matematico:}

Misura del sovraccarico di entropia dell'informazione:
\begin{equation}
H_{overload}(t) = \sum_{i} p_i(t) \log_2\left(\frac{1}{p_i(t)}\right) - H_{capacity}
\end{equation}

dove $p_i(t)$ rappresenta la distribuzione di probabilità sulle fonti di informazione e $H_{capacity}$ è la capacità di elaborazione individuale.

\textbf{Funzione di Soglia di Paralisi:}
\begin{equation}
P_{paralysis}(I) = \frac{1}{1 + e^{-k(I - I_0)}}
\end{equation}

dove $I$ rappresenta il carico informativo, $I_0$ è la soglia di paralisi, e $k$ controlla la ripidità della transizione.

\textbf{Modello del Tempo di Elaborazione:}
\begin{equation}
T_{process}(I) = T_0 \cdot \left(1 + \frac{I}{I_{capacity}}\right)^{\beta}
\end{equation}

con scalamento superlineare quando l'informazione supera la capacità.

\textbf{Criterio di Rilevamento:}
\begin{equation}
R_{5.3}(t) = \begin{cases}
1 & \text{se } H_{overload}(t) > 0 \text{ e } T_{process} > 3 \cdot T_0 \\
0 & \text{altrimenti}
\end{cases}
\end{equation}

\subsection{Indicatore 5.4: Degradazione da Multitasking}

\textbf{Definizione:} Degradazione delle prestazioni quando si gestiscono simultaneamente più compiti rilevanti per la sicurezza.

\textbf{Modello Matematico:}

La funzione di efficienza del multitasking:
\begin{equation}
E_{multi}(n) = \frac{1}{n} \cdot \left(1 - \frac{(n-1) \cdot S_{cost}}{1 + S_{cost}}\right)
\end{equation}

dove $n$ rappresenta il numero di compiti concorrenti e $S_{cost}$ è il parametro del costo di switching.

\textbf{Matrice di Interferenza dei Compiti:}
\begin{equation}
I_{jk} = \rho \cdot \frac{R_{shared}(j,k)}{R_{total}(j) \cdot R_{total}(k)}
\end{equation}

dove $\rho$ è il coefficiente di interferenza e $R_{shared}(j,k)$ rappresenta le risorse cognitive condivise tra i compiti $j$ e $k$.

\textbf{Degradazione delle Prestazioni:}
\begin{equation}
P_{degraded}(t) = 1 - \prod_{j=1}^{n} \left(1 - \sum_{k \neq j} I_{jk} \cdot A_k(t)\right)
\end{equation}

dove $A_k(t)$ rappresenta il livello di attivazione del compito $k$.

\textbf{Funzione di Rilevamento:}
\begin{equation}
D_{5.4}(t) = \frac{P_{degraded}(t)}{P_{threshold}} \cdot \frac{N_{concurrent}(t)}{N_{optimal}}
\end{equation}

dove $N_{optimal} = 3$ basato sulla ricerca empirica \cite{rubinstein2001}.

\subsection{Indicatore 5.5: Vulnerabilità del Cambio di Contesto}

\textbf{Definizione:} Lacune di sicurezza durante le transizioni cognitive tra diversi contesti operativi.

\textbf{Modello Matematico:}

Funzione del costo di cambio di contesto:
\begin{equation}
C_{switch}(t) = c_0 \cdot \sum_{i} I_{context}(i,t) \cdot e^{-\lambda t_i}
\end{equation}

dove $c_0$ è il costo base di switching, $I_{context}(i,t)$ indica il cambiamento di contesto, e $t_i$ è il tempo dal cambio.

\textbf{Modello della Finestra di Vulnerabilità:}
\begin{equation}
V_{window}(t) = V_{max} \cdot e^{-\alpha t} \cdot \sin^2\left(\frac{\pi t}{T_{cycle}}\right)
\end{equation}

rappresentando una vulnerabilità elevata immediatamente dopo i cambi di contesto.

\textbf{Probabilità di Errore Durante i Cambi:}
\begin{equation}
P_{error}(t) = P_{baseline} \cdot \left(1 + \beta \cdot V_{window}(t)\right)
\end{equation}

\textbf{Soglia di Rilevamento:}
\begin{equation}
R_{5.5}(t) = \begin{cases}
1 & \text{se } N_{switches}(t) > N_{threshold} \text{ e } P_{error} > 2 \cdot P_{baseline} \\
0 & \text{altrimenti}
\end{cases}
\end{equation}

\subsection{Indicatore 5.6: Tunneling Cognitivo}

\textbf{Definizione:} Focus attentivo ristretto che porta a trascurare gli indicatori di sicurezza periferici.

\textbf{Modello Matematico:}

Funzione di larghezza del tunnel dell'attenzione:
\begin{equation}
W_{attention}(L) = W_0 \cdot e^{-\gamma L}
\end{equation}

dove $W_0$ è la larghezza attentiva di base, $L$ è il livello di carico cognitivo, e $\gamma$ è il coefficiente di tunneling.

\textbf{Probabilità di Rilevamento Periferico:}
\begin{equation}
P_{peripheral}(t) = P_{max} \cdot \frac{W_{attention}(L(t))}{W_{max}}
\end{equation}

\textbf{Indice di Tunneling:}
\begin{equation}
TI(t) = 1 - \frac{N_{peripheral\_detected}(t)}{N_{peripheral\_presented}(t)}
\end{equation}

\textbf{Correlazione del Carico di Lavoro:}
\begin{equation}
L(t) = \alpha \cdot \frac{N_{primary\_tasks}(t)}{N_{baseline}} + \beta \cdot Urgency(t)
\end{equation}

\textbf{Funzione di Rilevamento:}
\begin{equation}
D_{5.6}(t) = TI(t) \cdot \frac{L(t)}{L_{threshold}}
\end{equation}

\subsection{Indicatore 5.7: Overflow della Memoria di Lavoro}

\textbf{Definizione:} Fallimenti cognitivi quando i requisiti di informazioni concorrenti superano la capacità della memoria di lavoro.

\textbf{Modello Matematico:}

Calcolo del carico della memoria di lavoro:
\begin{equation}
WM_{load}(t) = \sum_{i=1}^{n} w_i \cdot C_i(t) \cdot e^{-\lambda t_i}
\end{equation}

dove $w_i$ è il peso di importanza, $C_i(t)$ è la complessità dell'elemento $i$, e $t_i$ è il tempo in memoria.

\textbf{Rilevamento dell'Overflow di Capacità:}
\begin{equation}
O_{overflow}(t) = \max\left(0, \frac{WM_{load}(t) - WM_{capacity}}{WM_{capacity}}\right)
\end{equation}

\textbf{Modello di Probabilità di Errore:}
\begin{equation}
P_{error}(t) = P_{base} \cdot \left(1 + \delta \cdot O_{overflow}(t)^2\right)
\end{equation}

con scalamento quadratico per riflettere la rapida degradazione oltre la capacità.

\textbf{Implementazione della Regola 7±2 di Miller:}
\begin{equation}
WM_{capacity} = 7 \pm 2 \cdot \sigma_{individual}
\end{equation}

dove $\sigma_{individual}$ tiene conto della variazione individuale.

\textbf{Criterio di Rilevamento:}
\begin{equation}
R_{5.7}(t) = \begin{cases}
1 & \text{se } O_{overflow}(t) > 0.2 \\
0 & \text{altrimenti}
\end{cases}
\end{equation}

\subsection{Indicatore 5.8: Effetti di Residuo Attentivo}

\textbf{Definizione:} Interferenza cognitiva persistente da compiti precedenti che influenza i giudizi di sicurezza correnti.

\textbf{Modello Matematico:}

Funzione di decadimento del residuo attentivo:
\begin{equation}
R_{residue}(t) = R_0 \cdot e^{-\mu t} + \sum_{i} A_i \cdot e^{-\mu (t - t_i)}
\end{equation}

dove $R_0$ è il residuo iniziale, $\mu$ è il tasso di decadimento, e $A_i$ rappresenta il residuo del compito $i$.

\textbf{Misura di Interferenza:}
\begin{equation}
I_{interference}(t) = \int_{-\infty}^{t} R_{residue}(\tau) \cdot S_{similarity}(current, past(\tau)) \cdot d\tau
\end{equation}

dove $S_{similarity}$ misura la similarità semantica tra compiti correnti e passati.

\textbf{Impatto sulle Prestazioni:}
\begin{equation}
P_{impact}(t) = \frac{I_{interference}(t)}{I_{baseline}} - 1
\end{equation}

\textbf{Funzione di Rilevamento:}
\begin{equation}
D_{5.8}(t) = \max\left(0, P_{impact}(t)\right) \cdot \frac{N_{transitions}(t)}{N_{normal}}
\end{equation}

\subsection{Indicatore 5.9: Errori Indotti dalla Complessità}

\textbf{Definizione:} Errori sistematici quando la complessità dei compiti di sicurezza supera la capacità di elaborazione cognitiva.

\textbf{Modello Matematico:}

Misura della complessità utilizzando la complessità ciclomatica:
\begin{equation}
CC(T) = E - N + 2P
\end{equation}

dove $E$ sono gli archi (punti decisionali), $N$ sono i nodi (passi di processo), e $P$ sono le componenti connesse.

\textbf{Indice di Complessità Cognitiva:}
\begin{equation}
CCI(T) = \alpha \cdot CC(T) + \beta \cdot D_{depth}(T) + \gamma \cdot I_{interactions}(T)
\end{equation}

dove $D_{depth}$ è la profondità di nidificazione e $I_{interactions}$ è la complessità dell'interfaccia.

\textbf{Funzione di Probabilità di Errore:}
\begin{equation}
P_{error}(CCI) = P_{min} + (P_{max} - P_{min}) \cdot \frac{CCI^n}{CCI^n + K^n}
\end{equation}

seguendo l'equazione di Hill con legame cooperativo.

\textbf{Degradazione delle Prestazioni:}
\begin{equation}
D_{performance}(t) = 1 - \frac{1}{1 + e^{\kappa(CCI(t) - CCI_{threshold})}}
\end{equation}

\textbf{Soglia di Rilevamento:}
\begin{equation}
R_{5.9}(t) = \begin{cases}
1 & \text{se } CCI(t) > CCI_{threshold} \text{ e } P_{error} > 0.1 \\
0 & \text{altrimenti}
\end{cases}
\end{equation}

\subsection{Indicatore 5.10: Confusione del Modello Mentale}

\textbf{Definizione:} Errori derivanti da modelli cognitivi incorretti o conflittuali dei sistemi di sicurezza.

\textbf{Modello Matematico:}

Misura di coerenza del modello mentale:
\begin{equation}
C_{consistency}(t) = 1 - \frac{1}{n} \sum_{i=1}^{n} \frac{|M_i(t) - M_{correct}|}{M_{range}}
\end{equation}

dove $M_i(t)$ è l'elemento del modello mentale individuale e $M_{correct}$ è il modello corretto.

\textbf{Rilevamento del Conflitto di Modello:}
\begin{equation}
M_{conflict}(t) = \sum_{i,j} w_{ij} \cdot |M_i(t) - M_j(t)| \cdot I_{related}(i,j)
\end{equation}

dove $I_{related}(i,j)$ indica la relazione concettuale tra elementi.

\textbf{Entropia di Confusione:}
\begin{equation}
H_{confusion}(t) = -\sum_k p_k(t) \log_2 p_k(t)
\end{equation}

dove $p_k(t)$ rappresenta la distribuzione di probabilità sulle interpretazioni del modello concorrenti.

\textbf{Errore di Previsione Comportamentale:}
\begin{equation}
E_{prediction}(t) = \frac{1}{n} \sum_{i=1}^{n} |Action_{predicted}(i,t) - Action_{actual}(i,t)|
\end{equation}

\textbf{Funzione di Rilevamento:}
\begin{equation}
D_{5.10}(t) = \frac{H_{confusion}(t)}{H_{max}} + \frac{E_{prediction}(t)}{E_{threshold}}
\end{equation}

\section{Matrice di Interdipendenza}

Gli indicatori di sovraccarico cognitivo mostrano interdipendenze significative catturate attraverso la matrice di correlazione $\mathbf{R}_{5}$:

\begin{equation}
\mathbf{R}_5 = \begin{pmatrix}
1.00 & 0.75 & 0.60 & 0.55 & 0.50 & 0.45 & 0.70 & 0.65 & 0.40 & 0.55 \\
0.75 & 1.00 & 0.70 & 0.60 & 0.55 & 0.40 & 0.65 & 0.50 & 0.45 & 0.50 \\
0.60 & 0.70 & 1.00 & 0.50 & 0.45 & 0.35 & 0.55 & 0.40 & 0.60 & 0.80 \\
0.55 & 0.60 & 0.50 & 1.00 & 0.85 & 0.30 & 0.45 & 0.75 & 0.35 & 0.40 \\
0.50 & 0.55 & 0.45 & 0.85 & 1.00 & 0.25 & 0.40 & 0.80 & 0.30 & 0.35 \\
0.45 & 0.40 & 0.35 & 0.30 & 0.25 & 1.00 & 0.70 & 0.35 & 0.50 & 0.45 \\
0.70 & 0.65 & 0.55 & 0.45 & 0.40 & 0.70 & 1.00 & 0.60 & 0.55 & 0.65 \\
0.65 & 0.50 & 0.40 & 0.75 & 0.80 & 0.35 & 0.60 & 1.00 & 0.45 & 0.50 \\
0.40 & 0.45 & 0.60 & 0.35 & 0.30 & 0.50 & 0.55 & 0.45 & 1.00 & 0.65 \\
0.55 & 0.50 & 0.80 & 0.40 & 0.35 & 0.45 & 0.65 & 0.50 & 0.65 & 1.00
\end{pmatrix}
\end{equation}

Le interdipendenze chiave includono:
\begin{itemize}
\item Forte correlazione (0.85) tra Degradazione da Multitasking (5.4) e Cambio di Contesto (5.5)
\item Alta correlazione (0.80) tra Cambio di Contesto (5.5) e Residuo Attentivo (5.8)
\item Forte correlazione (0.80) tra Sovraccarico Informativo (5.3) e Confusione del Modello Mentale (5.10)
\item Significativa correlazione (0.75) tra Fatica di Avviso (5.1) e Fatica Decisionale (5.2)
\item Alta correlazione (0.70) tra Fatica di Avviso (5.1) e Overflow della Memoria di Lavoro (5.7)
\end{itemize}

\section{Algoritmi di Implementazione}

\begin{algorithm}
\caption{Valutazione del Sovraccarico Cognitivo}
\begin{algorithmic}[1]
\STATE Inizializza i parametri di baseline $\boldsymbol{\mu}, \boldsymbol{\Sigma}, \boldsymbol{w}$
\STATE Imposta i limiti di capacità cognitiva $WM_{capacity}, H_{capacity}, CCI_{threshold}$
\FOR{ogni passo temporale $t$}
    \STATE Raccogli la telemetria del carico cognitivo $\mathbf{x}(t)$
    \STATE Misura la complessità dei compiti $CCI(t)$, entropia dell'informazione $H(t)$
    \STATE Calcola il carico di lavoro corrente $N_{concurrent}(t)$, $N_{switches}(t)$
    \FOR{ogni indicatore $i \in \{5.1, 5.2, \ldots, 5.10\}$}
        \STATE Calcola il carico informativo $I_i(t)$
        \STATE Calcola la metrica del carico di lavoro $W_i(t)$
        \STATE Calcola il tasso di errore $E_i(t)$
        \STATE Calcola $D_i(t) = w_1 I_i(t) + w_2 W_i(t) + w_3 E_i(t)$
        \STATE Aggiorna lo stato temporale con decadimento $T_i(t) = \alpha \cdot D_i(t) + (1-\alpha) \cdot T_i(t-1) \cdot e^{-\lambda \Delta t}$
    \ENDFOR
    \STATE Calcola le correzioni di interdipendenza usando $\mathbf{R}_5$
    \STATE Genera avvisi di sovraccarico basati su soglie di capacità
    \STATE Aggiorna le baseline cognitive con l'apprendimento organizzativo
    \STATE Registra i risultati per l'ottimizzazione della capacità
\ENDFOR
\end{algorithmic}
\end{algorithm}

\section{Framework di Validazione}

Ogni indicatore subisce una validazione continua attraverso molteplici metriche:

\textbf{Metriche del Carico Cognitivo:}
\begin{align}
Capacity\_Utilization &= \frac{Current\_Load}{Maximum\_Capacity} \\
Overload\_Duration &= \int_{overload} dt \\
Recovery\_Rate &= \frac{dCapacity}{dt}|_{recovery}
\end{align}

\textbf{Correlazione delle Prestazioni:}
\begin{equation}
\rho_{performance} = \frac{Cov(Overload, ErrorRate)}{\sigma_{Overload} \cdot \sigma_{ErrorRate}}
\end{equation}

\textbf{Validazione Information-Theoretic:}
\begin{equation}
MI(Load, Performance) = \int \int p(l,p) \log \frac{p(l,p)}{p(l)p(p)} \, dl \, dp
\end{equation}

\textbf{Validazione Temporale:}
Modellazione del recupero cognitivo utilizzando il recupero esponenziale:
\begin{equation}
R(t) = R_{max} \cdot (1 - e^{-\mu t})
\end{equation}

La ricalibrazione si attiva quando le stime di capacità deviano di $>15\%$.

\textbf{Protocollo di Validazione Incrociata:}
La modellazione delle differenze individuali tiene conto della variazione della capacità cognitiva:
\begin{equation}
CV_{individual} = \sqrt{\frac{1}{n-1} \sum_{i=1}^{n} (Capacity_i - \bar{Capacity})^2}
\end{equation}

\section{Conclusione}

Questa formalizzazione matematica delle vulnerabilità di sovraccarico cognitivo fornisce un fondamento rigoroso per l'implementazione della Categoria 5 del CPF. Ogni indicatore riceve funzioni di rilevamento esplicite che combinano teoria dell'informazione, misurazione del carico cognitivo e analisi del carico di lavoro mantenendo l'efficienza computazionale per l'operazione in tempo reale.

La matrice di interdipendenza cattura importanti correlazioni tra i fenomeni di sovraccarico cognitivo, consentendo un rilevamento potenziato attraverso l'analisi multivariata. Gli algoritmi di implementazione forniscono una guida chiara per l'integrazione del sistema, mentre i framework di validazione garantiscono un'accuratezza sostenuta attraverso profili cognitivi diversi.

Il lavoro futuro estenderà questo approccio matematico alle restanti categorie CPF, creando una specifica formale completa per la valutazione delle vulnerabilità psicologiche nei contesti di cybersecurity. Il rigore matematico consente la ricerca riproducibile, implementazioni standardizzate e validazione obiettiva dell'efficacia del framework CPF.

La categoria delle vulnerabilità di sovraccarico cognitivo serve come fondamento per comprendere come le limitazioni dell'elaborazione delle informazioni creano punti ciechi di sicurezza sistematici. Formalizzando matematicamente questi meccanismi cognitivi, consentiamo il rilevamento e la mitigazione automatizzati delle vulnerabilità che storicamente sono state affrontate solo attraverso approcci soggettivi di gestione del carico di lavoro.

\begin{thebibliography}{9}

\bibitem{canale2024cpf}
Canale, G. (2024). The Cybersecurity Psychology Framework: A Pre-Cognitive Vulnerability Assessment Model Integrating Psychoanalytic and Cognitive Sciences. \textit{Preprint}.

\bibitem{miller1956}
Miller, G. A. (1956). The magical number seven, plus or minus two: Some limits on our capacity for processing information. \textit{Psychological Review}, 63(2), 81-97.

\bibitem{kahneman1973}
Kahneman, D. (1973). \textit{Attention and Effort}. Englewood Cliffs, NJ: Prentice Hall.

\bibitem{baddeley1992}
Baddeley, A. (1992). Working memory. \textit{Science}, 255(5044), 556-559.

\bibitem{miyake2000}
Miyake, A., Friedman, N. P., Emerson, M. J., Witzki, A. H., Howerter, A., \& Wager, T. D. (2000). The unity and diversity of executive functions and their contributions to complex "frontal lobe" tasks. \textit{Cognitive Psychology}, 41(1), 49-100.

\bibitem{sweller1988}
Sweller, J. (1988). Cognitive load during problem solving: Effects on learning. \textit{Cognitive Science}, 12(2), 257-285.

\bibitem{rubinstein2001}
Rubinstein, J. S., Meyer, D. E., \& Evans, J. E. (2001). Executive control of cognitive processes in task switching. \textit{Journal of Experimental Psychology: Human Perception and Performance}, 27(4), 763-797.

\end{thebibliography}

\end{document}
