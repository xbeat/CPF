\documentclass[11pt,a4paper]{article}
\usepackage{times}
\usepackage{amsmath}
\usepackage{amssymb}
\usepackage{amsfonts}
\usepackage[margin=1in]{geometry}
\usepackage{algorithm}
\usepackage{algorithmic}
\usepackage{hyperref}
\usepackage{booktabs}
\usepackage{graphicx}
\usepackage{cite}
\usepackage{array}

% Setup hyperref
\hypersetup{
    colorlinks=true,
    linkcolor=blue,
    citecolor=blue,
    urlcolor=blue,
    pdftitle={CPF Mathematical Formalization Series - Paper 2},
    pdfauthor={Giuseppe Canale},
}

\title{CPF Mathematical Formalization Series - Paper 2:\\Temporal Vulnerabilities: Modelli di Pressione Temporale e Carico Cognitivo Temporale}

\author{
    Giuseppe Canale, CISSP\\
    Independent Researcher\\
    \texttt{g.canale@cpf3.org}\\
    ORCID: 0009-0007-3263-6897
}

\date{\today}

\begin{document}

\maketitle

\begin{abstract}
Presentiamo la formalizzazione matematica completa degli indicatori della Categoria 2 del Cybersecurity Psychology Framework (CPF): Vulnerabilità Temporali. Ciascuno dei dieci indicatori (2.1-2.10) è definito matematicamente attraverso funzioni di rilevamento dipendenti dal tempo che incorporano modellazione dell'urgenza, valutazione del carico cognitivo temporale e analisi del ritmo circadiano. La formalizzazione attinge dalla teoria del prospetto di Kahneman e Tversky e dalla ricerca sull'attualizzazione temporale per quantificare come la pressione temporale degrada sistematicamente la qualità delle decisioni di sicurezza. Forniamo algoritmi espliciti per la valutazione delle vulnerabilità temporali in tempo reale, matrici di interdipendenza che catturano gli effetti a cascata temporali e metriche di validazione per la calibrazione del cronotipo organizzativo. Questo lavoro stabilisce il fondamento matematico per l'operazionalizzazione delle vulnerabilità psicologiche basate sul tempo che creano finestre prevedibili di debolezza di sicurezza.
\end{abstract}

\textbf{Keywords:} Applied Mathematics, Interdisciplinary Psychology, Computational Statistics, Mathematical Modeling, Cybersecurity Research

\section{Introduzione e Contesto CPF}

Il Cybersecurity Psychology Framework (CPF) affronta il divario critico tra le realtà psicologiche umane e le strategie di difesa della cybersecurity \cite{canale2024cpf}. Mentre la Categoria 1 si è concentrata sulle vulnerabilità basate sull'autorità, la Categoria 2 esamina come i fattori temporali degradano sistematicamente l'efficacia della sicurezza attraverso meccanismi psicologici prevedibili.

Le vulnerabilità temporali rappresentano uno degli aspetti più quantificabili della debolezza di sicurezza umana. A differenza delle dinamiche di autorità che variano in base alla cultura organizzativa, gli effetti temporali seguono pattern universali radicati nella psicologia evoluzionistica e nella biologia circadiana. Gli attaccanti sfruttano costantemente questi pattern attraverso social engineering guidato da scadenze, attacchi di fine giornata e manipolazione della pressione temporale.

Questo articolo fornisce la formalizzazione matematica completa per tutti i dieci indicatori di vulnerabilità temporale, consentendo il rilevamento e la previsione sistematici delle debolezze di sicurezza basate sul tempo. Ogni indicatore riceve funzioni di rilevamento esplicite che catturano la relazione non lineare tra pressione temporale e qualità delle decisioni di sicurezza.

I modelli matematici integrano tre approcci complementari: (1) funzioni di attualizzazione iperbolica per la valutazione delle minacce future, (2) modellazione del carico cognitivo per gli effetti della pressione temporale, e (3) analisi circadiana per la variazione temporale delle prestazioni. Questo approccio multifaccettato garantisce una copertura completa dei meccanismi di vulnerabilità temporale.

\section{Fondamento Teorico: Psicologia Temporale}

Le vulnerabilità temporali emergono dall'intersezione della teoria del prospetto \cite{kahneman1979}, della teoria del carico cognitivo \cite{sweller1988} e della cronobiologia \cite{roenneberg2012}. La cognizione temporale umana mostra bias sistematici che creano vulnerabilità di sicurezza prevedibili quando sfruttate dagli avversari.

Il meccanismo fondamentale coinvolge l'attualizzazione iperbolica, dove le ricompense immediate ricevono una ponderazione sproporzionata rispetto alle conseguenze ritardate \cite{ainslie2001}. Nei contesti di sicurezza, questo si manifesta come bias del presente nella valutazione del rischio: le esigenze operative immediate costantemente prevalgono sulle conseguenze di sicurezza future quando la pressione temporale aumenta.

La ricerca dimostra che le prestazioni cognitive seguono pattern circadiani prevedibili, con prestazioni di picco che si verificano 2-4 ore dopo il risveglio e degradazione significativa durante i punti bassi circadiani \cite{schmidt2007}. La qualità delle decisioni di sicurezza correla fortemente con questi ritmi biologici, creando finestre di vulnerabilità sistematiche.

I modelli matematici presentati catturano questi meccanismi attraverso funzioni di ponderazione temporale, modificatori del carico cognitivo e curve di prestazione circadiane. Ogni indicatore quantifica aspetti specifici della vulnerabilità temporale mantenendo l'efficienza computazionale per il monitoraggio in tempo reale.

\section{Formalizzazione Matematica}

\subsection{Framework di Rilevamento Temporale Universale}

Ogni indicatore di vulnerabilità temporale impiega la funzione di rilevamento unificata con ponderazione temporale:

\begin{equation}
D_i(t) = w_1 \cdot R_i(t) + w_2 \cdot A_i(t) + w_3 \cdot T_i(t) + w_4 \cdot C_i(t)
\end{equation}

dove $D_i(t)$ rappresenta il punteggio di rilevamento, $R_i(t)$ denota il rilevamento basato su regole, $A_i(t)$ rappresenta il punteggio di anomalia, $T_i(t)$ rappresenta l'indicatore di pressione temporale, e $C_i(t)$ rappresenta il modificatore di prestazione circadiana.

L'evoluzione temporale incorpora effetti di momento:

\begin{equation}
S_i(t) = \alpha \cdot D_i(t) + \beta \cdot S_i(t-1) + \gamma \cdot \frac{dD_i}{dt}
\end{equation}

dove $\gamma$ cattura gli effetti di velocità nell'evoluzione della vulnerabilità temporale.

\subsection{Indicatore 2.1: Aggiramento di Sicurezza Indotto dall'Urgenza}

\textbf{Definizione:} Elusione dei controlli di sicurezza sotto pressione temporale percepita per rispettare le scadenze.

\textbf{Modello Matematico:}

La funzione di accelerazione dell'urgenza:
\begin{equation}
U_i(t) = \frac{\Delta t_{normal} - \Delta t_{urgent}}{\Delta t_{normal}}
\end{equation}

dove $\Delta t_{normal}$ rappresenta il tempo di completamento del compito di base e $\Delta t_{urgent}$ rappresenta il tempo di completamento accelerato sotto pressione.

\textbf{Modello di Probabilità di Aggiramento:}
\begin{equation}
P_{bypass}(U,D) = \frac{1}{1 + e^{-\beta(U \cdot \alpha + D \cdot \gamma - \theta)}}
\end{equation}

dove $U$ è il livello di urgenza, $D$ è la prossimità della scadenza, e $\theta$ è la soglia decisionale.

\textbf{Funzione di Rilevamento:}
\begin{equation}
R_{2.1}(t) = \begin{cases}
1 & \text{se } U_i(t) > 0.5 \text{ e } N_{bypass} > \tau_{bypass} \\
0 & \text{altrimenti}
\end{cases}
\end{equation}

\textbf{Modello di Regressione Temporale:}
Utilizzando la regressione di Poisson per la previsione del tasso di aggiramento:
\begin{equation}
\log(\lambda) = \beta_0 + \beta_1 \cdot pressure + \beta_2 \cdot deadline\_proximity + \beta_3 \cdot circadian
\end{equation}

\subsection{Indicatore 2.2: Degradazione Cognitiva da Pressione Temporale}

\textbf{Definizione:} Riduzione sistematica della capacità cognitiva sotto vincoli temporali che porta a errori di sicurezza.

\textbf{Modello Matematico:}

Funzione della capacità cognitiva sotto pressione temporale:
\begin{equation}
C(t,p) = C_0 \cdot e^{-\alpha p(t)} \cdot (1 + \beta \sin(\frac{2\pi(t - \phi)}{24}))
\end{equation}

dove $C_0$ è la capacità di base, $p(t)$ è la pressione temporale, e il termine sinusoidale cattura la variazione circadiana.

\textbf{Modello del Tasso di Errore:}
\begin{equation}
E(t) = E_{baseline} \cdot \left(\frac{C_0}{C(t,p)}\right)^{\gamma}
\end{equation}

\textbf{Rilevamento Attraverso Metriche di Prestazione:}
\begin{equation}
D_{2.2}(t) = \frac{E(t) - \mu_E}{\sigma_E} \cdot \frac{RT(t) - \mu_{RT}}{\sigma_{RT}}
\end{equation}

dove $RT$ rappresenta la degradazione del tempo di risposta sotto pressione.

\subsection{Indicatore 2.3: Accettazione del Rischio Guidata dalla Scadenza}

\textbf{Definizione:} Accettazione di rischi di sicurezza elevati per rispettare le scadenze del progetto.

\textbf{Modello Matematico:}

Funzione di attualizzazione iperbolica per la valutazione del rischio:
\begin{equation}
V(R,D) = \frac{R}{1 + k \cdot D}
\end{equation}

dove $V$ è il valore del rischio percepito, $R$ è la magnitudine del rischio effettivo, $D$ è la distanza della scadenza, e $k$ è il tasso di sconto specifico dell'organizzazione.

\textbf{Soglia di Accettazione del Rischio:}
\begin{equation}
P_{accept}(R,D) = \sigma\left(\frac{V(R,D) - \theta_{risk}}{\sigma_{noise}}\right)
\end{equation}

\textbf{Integrazione della Gestione dei Progetti:}
\begin{equation}
DD_{score}(t) = \frac{\sum_{i} Risk_i \cdot Accepted_i(t)}{\sum_{i} Risk_i \cdot Proposed_i(t)}
\end{equation}

\textbf{Condizione di Rilevamento:}
\begin{equation}
R_{2.3}(t) = \begin{cases}
1 & \text{se } DD_{score}(t) > \mu + 2\sigma \text{ e } D < 7 \text{ giorni} \\
0 & \text{altrimenti}
\end{cases}
\end{equation}

\subsection{Indicatore 2.4: Bias del Presente negli Investimenti di Sicurezza}

\textbf{Definizione:} Sottovalutazione sistematica dei benefici di sicurezza futuri rispetto ai costi immediati.

\textbf{Modello Matematico:}

Coefficiente di bias del presente nei calcoli del ROI di sicurezza:
\begin{equation}
NPV_{biased} = \sum_{t=1}^{T} \frac{B_t}{(1+r)^t \cdot (1+\delta)^t} - C_0
\end{equation}

dove $\delta$ rappresenta il parametro di bias del presente oltre il tasso di sconto standard $r$.

\textbf{Modello di Decisione di Investimento:}
\begin{equation}
P_{invest}(NPV,PB) = \frac{e^{\alpha \cdot NPV_{biased}}}{1 + e^{\alpha \cdot NPV_{biased}}}
\end{equation}

\textbf{Rilevamento del Bias:}
\begin{equation}
PB_{indicator} = \frac{NPV_{standard} - NPV_{biased}}{NPV_{standard}}
\end{equation}

\textbf{Funzione di Soglia:}
\begin{equation}
D_{2.4}(t) = \max(0, PB_{indicator} - \theta_{pb})
\end{equation}

\subsection{Indicatore 2.5: Attualizzazione Iperbolica delle Minacce Future}

\textbf{Definizione:} Ponderazione sproporzionata delle minacce immediate rispetto ai rischi futuri nell'allocazione delle risorse.

\textbf{Modello Matematico:}

Confronto tra attualizzazione iperbolica ed esponenziale:
\begin{align}
V_{exp}(t) &= V_0 \cdot e^{-rt} \\
V_{hyp}(t) &= \frac{V_0}{1 + kt}
\end{align}

\textbf{Misura dell'Inconsistenza di Attualizzazione:}
\begin{equation}
DI(t_1, t_2) = \frac{V_{hyp}(t_1)/V_{hyp}(t_2)}{V_{exp}(t_1)/V_{exp}(t_2)}
\end{equation}

\textbf{Analisi della Prioritizzazione delle Minacce:}
\begin{equation}
TP_{score} = \sum_{i} w_i \cdot \frac{T_{immediate,i}}{T_{future,i}}
\end{equation}

dove $w_i$ rappresenta il peso della minaccia e $T$ rappresenta l'allocazione delle risorse.

\textbf{Algoritmo di Rilevamento:}
\begin{equation}
R_{2.5}(t) = \begin{cases}
1 & \text{se } DI > 1.5 \text{ e } TP_{score} > \tau_{tp} \\
0 & \text{altrimenti}
\end{cases}
\end{equation}

\subsection{Indicatore 2.6: Pattern di Esaurimento Temporale}

\textbf{Definizione:} Degradazione sistematica della vigilanza di sicurezza durante periodi di lavoro prolungati.

\textbf{Modello Matematico:}

Funzione di decadimento della vigilanza:
\begin{equation}
V(t) = V_0 \cdot e^{-\alpha t} + V_{min}
\end{equation}

dove $V_0$ è la vigilanza iniziale, $\alpha$ è il tasso di decadimento, e $V_{min}$ è la vigilanza minima sostenibile.

\textbf{Modulazione Circadiana:}
\begin{equation}
V_{circ}(t) = V(t) \cdot (1 + A \sin(\frac{2\pi(t - \phi)}{24} + \psi))
\end{equation}

dove $A$ è l'ampiezza circadiana, $\phi$ è lo spostamento di fase, e $\psi$ è il cronotipo individuale.

\textbf{Modello di Fatica Cumulativa:}
\begin{equation}
F(t) = \int_0^t W(\tau) \cdot e^{-\lambda(t-\tau)} d\tau
\end{equation}

dove $W(\tau)$ è la funzione del carico di lavoro e $\lambda$ è il tasso di recupero.

\textbf{Soglia di Rilevamento:}
\begin{equation}
D_{2.6}(t) = \frac{F(t)}{F_{threshold}} \cdot \left(1 - \frac{V_{circ}(t)}{V_0}\right)
\end{equation}

\subsection{Indicatore 2.7: Finestre di Vulnerabilità in Base all'Ora del Giorno}

\textbf{Definizione:} Periodi prevedibili di ridotta efficacia di sicurezza basati sui ritmi circadiani.

\textbf{Modello Matematico:}

Funzione di prestazione circadiana:
\begin{equation}
P_{circ}(h) = P_0 + A_1 \cos\left(\frac{2\pi(h - \phi_1)}{24}\right) + A_2 \cos\left(\frac{4\pi(h - \phi_2)}{24}\right)
\end{equation}

dove $h$ è l'ora del giorno, $P_0$ è la prestazione di base, e $A_1, A_2$ sono le ampiezze armoniche.

\textbf{Integrazione del Cronotipo Individuale:}
\begin{equation}
\phi_{individual} = \phi_{population} + \Delta\phi_{chronotype} + \Delta\phi_{age}
\end{equation}

\textbf{Rilevamento della Finestra di Vulnerabilità:}
\begin{equation}
VW(h) = \begin{cases}
1 & \text{se } P_{circ}(h) < P_0 - \sigma_{threshold} \\
0 & \text{altrimenti}
\end{cases}
\end{equation}

\textbf{Metrica Organizzativa Aggregata:}
\begin{equation}
D_{2.7}(t) = \frac{\sum_{i=1}^{N} VW_i(hour(t))}{N} \cdot Incident\_Rate(hour(t))
\end{equation}

\subsection{Indicatore 2.8: Lapsi di Sicurezza nel Weekend/Festività}

\textbf{Definizione:} Ridotta vigilanza di sicurezza durante i periodi non lavorativi che porta a finestre di sfruttamento.

\textbf{Modello Matematico:}

Funzione dell'effetto weekend:
\begin{equation}
WE(d) = \begin{cases}
\beta_{weekend} & \text{se } d \in \{Saturday, Sunday\} \\
\beta_{holiday} & \text{se } d \in Holiday\_Set \\
1.0 & \text{altrimenti}
\end{cases}
\end{equation}

\textbf{Modello dell'Impatto del Personale:}
\begin{equation}
S_{effective}(d,h) = S_{nominal} \cdot WE(d) \cdot Coverage(h)
\end{equation}

\textbf{Probabilità di Successo dell'Attacco:}
\begin{equation}
P_{success}(d,h) = P_{baseline} \cdot \left(\frac{S_{nominal}}{S_{effective}(d,h)}\right)^{\gamma}
\end{equation}

\textbf{Algoritmo di Rilevamento:}
\begin{equation}
R_{2.8}(t) = \begin{cases}
1 & \text{se } WE(day(t)) < 0.7 \text{ e } Incident\_Count > \tau_{weekend} \\
0 & \text{altrimenti}
\end{cases}
\end{equation}

\subsection{Indicatore 2.9: Finestre di Sfruttamento del Cambio Turno}

\textbf{Definizione:} Periodi di vulnerabilità durante le transizioni del personale e i processi di passaggio di consegne.

\textbf{Modello Matematico:}

Funzione di vulnerabilità della transizione di turno:
\begin{equation}
STV(t) = \sum_{i} A_i \cdot e^{-\frac{(t-t_{shift,i})^2}{2\sigma_{transition}^2}}
\end{equation}

dove $t_{shift,i}$ rappresenta i tempi di cambio turno e $A_i$ rappresenta la gravità della transizione.

\textbf{Efficienza del Trasferimento di Informazioni:}
\begin{equation}
ITE(t) = \frac{Information_{received}(t)}{Information_{available}(t)} \cdot Quality\_Factor(t)
\end{equation}

\textbf{Modello del Divario di Copertura:}
\begin{equation}
CG(t) = \max(0, 1 - \frac{Staff_{effective}(t)}{Staff_{required}})
\end{equation}

\textbf{Rilevamento Combinato:}
\begin{equation}
D_{2.9}(t) = STV(t) \cdot (1 - ITE(t)) \cdot CG(t)
\end{equation}

\subsection{Indicatore 2.10: Pressione della Coerenza Temporale}

\textbf{Definizione:} Pressione a mantenere tempi di risposta coerenti che porta a scorciatoie di sicurezza.

\textbf{Modello Matematico:}

Indice di pressione della coerenza:
\begin{equation}
CPI(t) = \frac{Var(Response\_Times)}{Mean(Response\_Times)^2} \cdot Penalty\_Factor
\end{equation}

\textbf{Modello di Probabilità di Scorciatoia:}
\begin{equation}
P_{shortcut}(CPI, deadline) = \sigma(\alpha \cdot CPI + \beta \cdot deadline\_pressure - \theta)
\end{equation}

\textbf{Funzione di Degradazione della Qualità:}
\begin{equation}
QD(t) = Q_0 \cdot (1 - \gamma \cdot P_{shortcut}(t))
\end{equation}

dove $Q_0$ è la qualità di base e $\gamma$ rappresenta la gravità dell'impatto della scorciatoia.

\textbf{Soglia di Rilevamento:}
\begin{equation}
R_{2.10}(t) = \begin{cases}
1 & \text{se } CPI(t) > \tau_{cpi} \text{ e } QD(t) < Q_{min} \\
0 & \text{altrimenti}
\end{cases}
\end{equation}

\section{Matrice di Interdipendenza}

Gli indicatori di vulnerabilità temporale mostrano interdipendenze complesse catturate attraverso la matrice di correlazione $\mathbf{R}_{2}$:

\begin{equation}
\mathbf{R}_2 = \begin{pmatrix}
1.00 & 0.75 & 0.60 & 0.45 & 0.50 & 0.65 & 0.55 & 0.40 & 0.35 & 0.70 \\
0.75 & 1.00 & 0.55 & 0.40 & 0.35 & 0.80 & 0.50 & 0.30 & 0.25 & 0.65 \\
0.60 & 0.55 & 1.00 & 0.70 & 0.75 & 0.45 & 0.35 & 0.40 & 0.30 & 0.50 \\
0.45 & 0.40 & 0.70 & 1.00 & 0.85 & 0.30 & 0.25 & 0.35 & 0.20 & 0.40 \\
0.50 & 0.35 & 0.75 & 0.85 & 1.00 & 0.40 & 0.30 & 0.45 & 0.25 & 0.35 \\
0.65 & 0.80 & 0.45 & 0.30 & 0.40 & 1.00 & 0.75 & 0.60 & 0.55 & 0.50 \\
0.55 & 0.50 & 0.35 & 0.25 & 0.30 & 0.75 & 1.00 & 0.85 & 0.70 & 0.40 \\
0.40 & 0.30 & 0.40 & 0.35 & 0.45 & 0.60 & 0.85 & 1.00 & 0.65 & 0.30 \\
0.35 & 0.25 & 0.30 & 0.20 & 0.25 & 0.55 & 0.70 & 0.65 & 1.00 & 0.45 \\
0.70 & 0.65 & 0.50 & 0.40 & 0.35 & 0.50 & 0.40 & 0.30 & 0.45 & 1.00
\end{pmatrix}
\end{equation}

Le interdipendenze chiave includono:
\begin{itemize}
\item Forte correlazione (0.85) tra Bias del Presente (2.4) e Attualizzazione Iperbolica (2.5)
\item Alta correlazione (0.80) tra Degradazione Cognitiva (2.2) e Pattern di Esaurimento (2.6)
\item Significativa correlazione (0.85) tra Finestre Ora del Giorno (2.7) e Lapsi Weekend (2.8)
\item Moderata correlazione (0.75) tra Aggiramento per Urgenza (2.1) e Degradazione Cognitiva (2.2)
\end{itemize}

\textbf{Dipendenze Inter-Categoria:}
Relazioni critiche con le vulnerabilità di Autorità (Categoria 1):
\begin{itemize}
\item $R_{1.5,2.1} = 0.70$: Conformità basata sulla paura amplificata dalla pressione temporale
\item $R_{1.10,2.3} = 0.65$: L'escalation di crisi aumenta l'accettazione del rischio guidata dalla scadenza
\item $R_{1.1,2.6} = 0.55$: L'esaurimento aumenta la conformità senza domande
\end{itemize}

\section{Algoritmi di Implementazione}

\begin{algorithm}
\caption{Valutazione della Vulnerabilità Temporale}
\begin{algorithmic}[1]
\STATE Inizializza i parametri temporali $\boldsymbol{\alpha}, \boldsymbol{\beta}, \boldsymbol{\gamma}$
\STATE Carica i profili circadiani e i cronotipi organizzativi
\FOR{ogni passo temporale $t$}
    \STATE Estrai il contesto temporale: $hour(t)$, $day(t)$, $workload(t)$
    \STATE Calcola il modificatore di prestazione circadiana $C_{circ}(t)$
    \FOR{ogni indicatore $i \in \{2.1, 2.2, \ldots, 2.10\}$}
        \STATE Calcola la pressione temporale $TP_i(t)$
        \STATE Calcola il rilevamento basato su regole $R_i(t)$
        \STATE Calcola il punteggio di anomalia $A_i(t)$ con ponderazione temporale
        \STATE Valuta la posteriore bayesiana $B_i(t)$
        \STATE Calcola il punteggio di rilevamento $D_i(t)$
        \STATE Applica lo smoothing temporale con momento
        \STATE Aggiorna lo stato temporale $S_i(t)$
    \ENDFOR
    \STATE Calcola le correzioni di interdipendenza usando $\mathbf{R}_2$
    \STATE Applica le correlazioni inter-categoria con la Categoria 1
    \STATE Genera avvisi consapevoli del tempo con stime di tempo alla vulnerabilità
    \STATE Aggiorna i modelli circadiani e di fatica
    \STATE Registra i risultati per la raffinazione del cronotipo
\ENDFOR
\end{algorithmic}
\end{algorithm}

\begin{algorithm}
\caption{Previsione della Vulnerabilità Circadiana}
\begin{algorithmic}[1]
\STATE Input: Tempo corrente $t$, orizzonte di previsione $h$
\STATE Inizializza l'array di previsione della vulnerabilità $V_{forecast}[h]$
\FOR{ogni tempo futuro $t_f = t + \Delta t$ fino a $t + h$}
    \STATE Calcola la prestazione circadiana $P_{circ}(t_f)$
    \STATE Stima il carico di lavoro $W_{est}(t_f)$ dai pattern storici
    \STATE Calcola l'accumulo di fatica $F_{acc}(t_f)$
    \STATE Prevedi la disponibilità del personale $Staff(t_f)$
    \STATE Calcola la vulnerabilità combinata $V(t_f)$
    \STATE Memorizza nell'array di previsione
\ENDFOR
\STATE Identifica le finestre di vulnerabilità di picco
\STATE Genera avvisi preventivi per i periodi ad alto rischio
\STATE Restituisci la timeline della vulnerabilità con intervalli di confidenza
\end{algorithmic}
\end{algorithm}

\section{Framework di Validazione}

La validazione della vulnerabilità temporale richiede metriche specializzate che tengono conto dei fenomeni dipendenti dal tempo:

\textbf{Metriche di Classificazione Temporale:}
\begin{align}
Precision_t &= \frac{\sum_i TP_i \cdot w_t(i)}{\sum_i (TP_i + FP_i) \cdot w_t(i)} \\
Recall_t &= \frac{\sum_i TP_i \cdot w_t(i)}{\sum_i (TP_i + FN_i) \cdot w_t(i)}
\end{align}

dove $w_t(i)$ fornisce la ponderazione temporale basata sul timing del rilevamento.

\textbf{Validazione Circadiana:}
Misura dell'accuratezza di fase:
\begin{equation}
\phi_{error} = \min(|\phi_{predicted} - \phi_{observed}|, 24 - |\phi_{predicted} - \phi_{observed}|)
\end{equation}

\textbf{Analisi della Stabilità Temporale:}
Funzione di autocorrelazione per la coerenza temporale:
\begin{equation}
R(\tau) = \frac{E[(X_t - \mu)(X_{t+\tau} - \mu)]}{\sigma^2}
\end{equation}

\textbf{Accuratezza Predittiva:}
Errore Assoluto Medio per il timing della vulnerabilità:
\begin{equation}
MAE_{temporal} = \frac{1}{n} \sum_{i=1}^{n} |t_{predicted,i} - t_{actual,i}|
\end{equation}

\textbf{Validazione Incrociata con Stratificazione Temporale:}
Garantire che i set di training e test mantengano la rappresentatività temporale:
\begin{equation}
Stratification_{score} = \frac{\text{Var}(temporal\_distribution_{test})}{\text{Var}(temporal\_distribution_{total})}
\end{equation}

Il punteggio di stratificazione target si avvicina a 1.0 per un bilanciamento temporale ottimale.

\textbf{Validazione dell'Adattamento del Cronotipo:}
Misura dell'efficacia della personalizzazione:
\begin{equation}
Adaptation_{gain} = \frac{Accuracy_{personalized} - Accuracy_{generic}}{Accuracy_{generic}}
\end{equation}

\section{Conclusione}

Questa formalizzazione matematica delle vulnerabilità temporali fornisce un fondamento rigoroso per comprendere e rilevare le debolezze di sicurezza basate sul tempo. L'integrazione dell'attualizzazione iperbolica, della modellazione circadiana e della teoria del carico cognitivo crea un framework completo per la valutazione delle vulnerabilità temporali.

La matrice di interdipendenza rivela forti correlazioni tra gli indicatori temporali e effetti significativi inter-categoria con le vulnerabilità basate sull'autorità. Questo dimostra che la pressione temporale agisce come un moltiplicatore di forza per altre vulnerabilità psicologiche, enfatizzando l'importanza critica delle strategie di sicurezza consapevoli del tempo.

Gli algoritmi di implementazione consentono il monitoraggio delle vulnerabilità temporali in tempo reale con capacità predittive. Le organizzazioni possono anticipare le finestre di vulnerabilità basate sui pattern circadiani, le previsioni del carico di lavoro e i modelli di accumulo della fatica, consentendo misure di sicurezza proattive piuttosto che reattive.

Il lavoro futuro estenderà questo approccio di modellazione temporale alle restanti categorie CPF, con particolare attenzione agli effetti di amplificazione temporale sul sovraccarico cognitivo (Categoria 5) e le risposte allo stress (Categoria 7). Il rigore matematico qui stabilito fornisce un fondamento per strategie di sicurezza temporali basate sull'evidenza che tengono conto dei pattern prevedibili della cognizione temporale umana.

La categoria delle vulnerabilità temporali dimostra che la sicurezza non riguarda solo ciò che le persone sanno, ma quando applicano quella conoscenza. Formalizzando matematicamente queste dinamiche temporali, consentiamo sistemi di sicurezza che si adattano alle realtà temporali umane piuttosto che aspettarsi che gli esseri umani mantengano una vigilanza costante in tutti i periodi di tempo.

\begin{thebibliography}{9}

\bibitem{canale2024cpf}
Canale, G. (2024). The Cybersecurity Psychology Framework: A Pre-Cognitive Vulnerability Assessment Model Integrating Psychoanalytic and Cognitive Sciences. \textit{Preprint}.

\bibitem{kahneman1979}
Kahneman, D., \& Tversky, A. (1979). Prospect theory: An analysis of decision under risk. \textit{Econometrica}, 47(2), 263-291.

\bibitem{sweller1988}
Sweller, J. (1988). Cognitive load during problem solving. \textit{Cognitive Science}, 12(2), 257-285.

\bibitem{roenneberg2012}
Roenneberg, T., \& Merrow, M. (2016). The circadian clock and human health. \textit{Current Biology}, 26(10), R432-R443.

\bibitem{ainslie2001}
Ainslie, G. (2001). \textit{Breakdown of Will}. Cambridge University Press.

\bibitem{schmidt2007}
Schmidt, C., Collette, F., Cajochen, C., \& Peigneux, P. (2007). A time to think: Circadian rhythms in human cognition. \textit{Cognitive Neuropsychology}, 24(7), 755-789.

\end{thebibliography}

\end{document}
