\documentclass[11pt,a4paper]{article}
\usepackage[margin=1in]{geometry}
\usepackage{listings}
\usepackage{hyperref}
\usepackage{graphicx}
\usepackage{amsmath}
\usepackage[utf8]{inputenc}
\usepackage{textgreek}

\title{The Cybersecurity Psychology Framework:\\
From Theoretical Model to Operational Implementation\\
\large{A Technical Guide for Pilot Deployment}}

\author{Giuseppe Canale, CISSP\\
\texttt{kaolay@gmail.com}\\
Framework Website: \url{https://cpf3.org}}

\date{\today}

\begin{document}
\maketitle

\begin{abstract}
The Cybersecurity Psychology Framework (CPF) provides a systematic approach to 
predicting security incidents through psychological state assessment. With 85\% 
of breaches involving human factors, traditional technical controls are 
insufficient. This paper presents a practical implementation guide leveraging 
modern LLM capabilities for real-time psychological signal processing. We seek 
collaboration with organizations and research institutions to refine and 
validate the framework through controlled pilots.
\end{abstract}

\section{Introduction: Why Now?}

Three convergent factors make CPF implementation suddenly feasible:

\begin{enumerate}
\item \textbf{LLM Revolution}: Local language models can now process organizational 
communications at scale, extracting psychological indicators in real-time.

\item \textbf{Data Availability}: Modern organizations generate rich behavioral 
data through digital communications, authentication logs, and collaboration tools.

\item \textbf{Breach Economics}: With average breach costs reaching \$4.88M, 
even 20\% reduction justifies significant investment in predictive capabilities.
\end{enumerate}

This is not about proving psychology works in cybersecurity—the evidence is 
overwhelming. This is about engineering practical systems to capture and act 
on psychological signals we know exist.

\section{Core Implementation Architecture}

\subsection{Data Pipeline}

\begin{lstlisting}[language=Python, caption=Core CPF Processing Pipeline]
import numpy as np
from transformers import AutoModelForSequenceClassification
from datetime import datetime, timedelta
import pandas as pd

class CPFAnalyzer:
    def __init__(self, llm_model='microsoft/deberta-v3-small'):
        self.llm = AutoModelForSequenceClassification.from_pretrained(llm_model)
        self.indicators = self.initialize_indicators()
        self.baseline = {}
        
    def analyze_temporal_exhaustion(self, email_df):
        """CPF Indicator 2.6: Temporal Exhaustion Pattern"""
        
        # Response time degradation over day
        email_df['hour'] = pd.to_datetime(email_df['timestamp']).dt.hour
        email_df['response_time'] = email_df['response_timestamp'] - email_df['received_timestamp']
        
        # Calculate degradation curve
        hourly_avg = email_df.groupby('hour')['response_time'].mean()
        degradation = (hourly_avg.iloc[-1] - hourly_avg.iloc[0]) / hourly_avg.iloc[0]
        
        # Threshold: >50% degradation = HIGH risk
        if degradation > 0.5:
            return 2  # Red
        elif degradation > 0.2:
            return 1  # Yellow
        return 0  # Green
        
    def detect_cognitive_overload(self, text_corpus):
        """CPF Indicator 5.3: Cognitive Overload via Linguistic Markers"""
        
        features = []
        for text in text_corpus:
            # Linguistic complexity
            sentences = text.split('.')
            avg_length = np.mean([len(s.split()) for s in sentences])
            
            # Cognitive markers
            confusion_markers = ['confused', 'lost', 'overwhelmed', 'too much']
            marker_density = sum(text.lower().count(m) for m in confusion_markers) / len(text.split())
            
            # LLM sentiment
            sentiment = self.llm(text).logits.argmax().item()
            
            features.append({
                'complexity': avg_length,
                'confusion': marker_density,
                'sentiment': sentiment
            })
            
        # Statistical anomaly detection
        df = pd.DataFrame(features)
        threshold = df['confusion'].mean() + 2 * df['confusion'].std()
        
        current = df['confusion'].iloc[-100:].mean()  # Last 100 messages
        
        return 2 if current > threshold else 0
        
    def calculate_convergence_index(self):
        """Critical state when multiple categories align"""
        
        active_categories = sum(1 for score in self.indicators.values() if score > 1)
        
        if active_categories >= 3:
            # Multiple systems failing = exponential risk
            return np.exp(active_categories / 10)
        return 1.0
\end{lstlisting}

\subsection{Integration Points}

The framework integrates with existing infrastructure without disruption:

\begin{lstlisting}[language=Python, caption=SIEM Integration Example]
class CPFSIEMConnector:
    def __init__(self, siem_api):
        self.siem = siem_api
        self.cpf = CPFAnalyzer()
        
    def process_real_time(self):
        """Process streaming events for psychological indicators"""
        
        for event in self.siem.stream_events():
            if event.type == 'authentication_failure':
                self.update_stress_indicator(event)
                
            elif event.type == 'email_metadata':
                exhaustion = self.cpf.analyze_temporal_exhaustion(event.data)
                if exhaustion > 1:
                    self.siem.create_alert({
                        'severity': 'MEDIUM',
                        'title': 'Temporal Exhaustion Detected',
                        'description': f'Team showing {exhaustion} exhaustion level',
                        'recommendation': 'Increase phishing detection sensitivity'
                    })
                    
            # Convergence check every hour
            if event.timestamp.minute == 0:
                convergence = self.cpf.calculate_convergence_index()
                if convergence > 2:
                    self.trigger_defensive_mode(convergence)
\end{lstlisting}

\section{Measurable Indicators}

\subsection{Tier 1: Immediately Available (Week 1)}

These indicators use data every organization already collects:

\begin{table}[h]
\centering
\begin{tabular}{|l|l|l|l|}
\hline
\textbf{Indicator} & \textbf{Data Source} & \textbf{Metric} & \textbf{Threshold} \\
\hline
Temporal Exhaustion & Email timestamps & Response degradation & >50\% slower \\
Alert Fatigue & SIEM acks & Dismissal rate & >70\% ignored \\
Password Chaos & AD/LDAP & Reset frequency & >2x baseline \\
Ticket Velocity & ITSM & Volume spike & >3σ deviation \\
Auth Failures & Logs & Failure rate & >5x normal \\
\hline
\end{tabular}
\end{table}

\subsection{Tier 2: LLM-Enhanced (Week 2-4)}

Requiring text analysis via local LLM:

\begin{lstlisting}[language=Python, caption=LLM-Based Stress Detection]
def detect_organizational_stress(slack_export, llm_model):
    """
    Analyze Slack/Teams for stress indicators
    Privacy: No message content stored, only scores
    """
    
    stress_indicators = {
        'temporal': ['deadline', 'urgent', 'asap', 'now'],
        'emotional': ['frustrated', 'confused', 'worried', 'stressed'],
        'cognitive': ['cant think', 'too much', 'overwhelming'],
        'social': ['nobody helping', 'on my own', 'where is everyone']
    }
    
    daily_scores = []
    
    for day_messages in slack_export.group_by_day():
        embeddings = llm_model.encode(day_messages)
        
        # Aggregate stress without storing content
        stress_score = llm_model.classify_batch(
            embeddings, 
            labels=['calm', 'normal', 'stressed', 'crisis']
        )
        
        daily_scores.append({
            'date': day_messages.date,
            'stress_mean': np.mean(stress_score),
            'stress_std': np.std(stress_score),
            'message_volume': len(day_messages)
        })
        
    return pd.DataFrame(daily_scores)
\end{lstlisting}

\section{Pilot Protocol}

\subsection{Phase 1: Baseline (Days 1-30)}

\begin{enumerate}
\item Deploy collectors (read-only, no PII storage)
\item Establish organizational baseline patterns
\item Train LLM on organization-specific language
\item Identify top 5 relevant CPF indicators
\end{enumerate}

\subsection{Phase 2: Correlation (Days 31-60)}

\begin{enumerate}
\item Daily CPF scoring
\item Correlate with:
   \begin{itemize}
   \item Security incidents
   \item Help desk tickets
   \item Error rates
   \item Productivity metrics
   \end{itemize}
\item Identify predictive lead time
\end{enumerate}

\subsection{Phase 3: Prediction (Days 61-90)}

\begin{enumerate}
\item Enable predictive alerts
\item Measure:
   \begin{itemize}
   \item True positive rate
   \item False positive rate
   \item Lead time accuracy
   \item SOC team feedback
   \end{itemize}
\item ROI calculation
\end{enumerate}

\section{Privacy and Ethics}

\textbf{Non-negotiable principles:}

\begin{itemize}
\item No individual profiling - minimum 10-person aggregation
\item No message content storage - only statistical scores
\item No punitive use - framework identifies systemic issues, not "problem employees"
\item Full transparency - teams know what's measured and why
\item Opt-out capability - for research/pilot phase
\end{itemize}

\section{Expected Outcomes}

\subsection{Conservative Estimate}
\begin{itemize}
\item 48-hour warning for high-risk periods
\item 20\% reduction in human-factor incidents
\item 30\% faster incident response
\end{itemize}

\subsection{Optimistic Projection}
\begin{itemize}
\item 14-day predictive capability
\item 50\% incident reduction
\item Automated defensive adjustments
\item 3x ROI within first year
\end{itemize}

\section{Call for Collaboration}

We seek three types of partners:

\subsection{Enterprise Pilots}
Organizations with:
\begin{itemize}
\item 500+ employees
\item Mature SOC operations
\item SIEM/SOAR infrastructure
\item Willingness to share anonymized outcomes
\end{itemize}

\subsection{Academic Validation}
Universities/institutions for:
\begin{itemize}
\item Rigorous statistical validation
\item Peer review publication
\item Ethical framework development
\item Cross-cultural validation
\end{itemize}

\subsection{Technology Partners}
SIEM/SOAR vendors for:
\begin{itemize}
\item Native integration development
\item Scaling to enterprise level
\item ML model optimization
\item Commercial deployment
\end{itemize}

\section{Implementation Roadmap}

\begin{table}[h]
\centering
\begin{tabular}{|l|l|l|}
\hline
\textbf{Quarter} & \textbf{Milestone} & \textbf{Deliverable} \\
\hline
Q1 2025 & 5 pilot deployments & Validation data \\
Q2 2025 & Academic publication & Peer-reviewed paper \\
Q3 2025 & V2.0 framework & Refined indicators \\
Q4 2025 & Commercial release & Enterprise product \\
\hline
\end{tabular}
\end{table}

\section{Conclusion}

CPF is not experimental psychology—it's engineering implementation of 
established psychological principles. The convergence of LLMs, behavioral 
data availability, and breach economics makes this the right moment for 
deployment.

We're not asking "does psychology matter in security?" We know it does. 
We're asking "how can we operationalize this knowledge to prevent breaches?"

Join us in answering that question.

\section*{Contact}

\begin{itemize}
\item Email: kaolay@gmail.com
\item Framework: \url{https://cpf3.org}
\item Pilot Application: \url{https://cpf3.org/pilot}
\item GitHub: \url{https://github.com/cpf-framework} [Coming Q1 2025]
\end{itemize}

\appendix

\section{Sample CPF Indicators for Quick Reference}

\begin{enumerate}
\item \textbf{Authority Vulnerabilities}: Executive exception patterns
\item \textbf{Temporal Vulnerabilities}: End-of-day degradation
\item \textbf{Social Influence}: Peer pressure compliance
\item \textbf{Affective States}: Stress/fear indicators
\item \textbf{Cognitive Overload}: Information processing limits
\item \textbf{Group Dynamics}: Collective blind spots
\item \textbf{Stress Response}: Fight/flight/freeze patterns
\item \textbf{Unconscious Process}: Shadow projections
\item \textbf{AI-Specific}: Automation bias
\item \textbf{Convergent States}: Perfect storm conditions
\end{enumerate}

\end{document}