\documentclass[11pt,a4paper]{article}
\usepackage{times}
\usepackage{amsmath}
\usepackage{amssymb}
\usepackage{amsfonts}
\usepackage[margin=1in]{geometry}
\usepackage{algorithm}
\usepackage{algorithmic}
\usepackage{hyperref}
\usepackage{booktabs}
\usepackage{graphicx}
\usepackage{cite}
\usepackage{array}

% Setup hyperref
\hypersetup{
    colorlinks=true,
    linkcolor=blue,
    citecolor=blue,
    urlcolor=blue,
    pdftitle={CPF Mathematical Formalization Series - Paper 3},
    pdfauthor={Giuseppe Canale},
}

\title{CPF Mathematical Formalization Series - Paper 3:\\Social Influence Vulnerabilities: Reciprocity, Commitment, and Social Proof Models}

\author{
    Giuseppe Canale, CISSP\\
    Independent Researcher\\
    \texttt{g.canale@cpf3.org}\\
    ORCID: 0009-0007-3263-6897
}

\date{\today}

\begin{document}

\maketitle

\begin{abstract}
We present the complete mathematical formalization of Category 3 indicators from the Cybersecurity Psychology Framework (CPF): Social Influence Vulnerabilities. Each of the ten indicators (3.1-3.10) is mathematically defined through social network analysis, reciprocity modeling, and commitment escalation functions. The formalization draws from Cialdini's principles of influence, social psychology research, and network theory to quantify how social dynamics systematically create exploitable security vulnerabilities. We provide explicit algorithms for real-time social influence detection, network vulnerability assessment, and escalation modeling. This work establishes the mathematical foundation for operationalizing social manipulation tactics that consistently bypass technical security controls through psychological influence mechanisms.
\end{abstract}

\section{Introduction and CPF Context}

The Cybersecurity Psychology Framework (CPF) addresses the critical gap between human psychological realities and cybersecurity defense strategies \cite{canale2024cpf}. While Categories 1 and 2 focused on authority and temporal vulnerabilities respectively, Category 3 examines how social influence mechanisms systematically compromise security through predictable psychological compliance patterns.

Social influence vulnerabilities represent the most consistently exploited attack vector in modern cybersecurity. Unlike technical vulnerabilities that require specific knowledge or resources, social influence attacks exploit universal human psychological mechanisms that operate across cultures and demographics. Cialdini's six principles of influence \cite{cialdini2007} provide a scientifically validated framework for understanding these mechanisms.

This paper provides complete mathematical formalization for all ten social influence vulnerability indicators, enabling systematic detection and prediction of social manipulation attempts. Each indicator receives explicit detection functions that capture the nonlinear relationship between social pressure and compliance behavior.

The mathematical models integrate three complementary approaches: (1) social network analysis for relationship mapping, (2) game-theoretic models for reciprocity dynamics, and (3) commitment escalation functions for consistency exploitation. This multi-faceted approach ensures comprehensive coverage of social influence vulnerability mechanisms.

\section{Theoretical Foundation: Social Influence Psychology}

Social influence vulnerabilities emerge from the intersection of evolutionary psychology \cite{buss1999}, social network theory \cite{watts1998}, and compliance psychology \cite{kelman1958}. Human social cognition evolved for small-group environments, creating systematic blind spots when applied to modern organizational contexts and digital communications.

The fundamental mechanism involves social compliance heuristics that bypass analytical thinking. These heuristics evolved as adaptive shortcuts for navigating complex social environments but become exploitable vulnerabilities when adversaries deliberately trigger them \cite{cialdini2007}.

Research demonstrates that social influence operates through six primary channels: reciprocity (obligation to return favors), commitment/consistency (pressure to align with previous positions), social proof (tendency to follow others' behavior), liking (preference for agreeable sources), authority (deference to perceived expertise), and scarcity (value attribution to rare opportunities) \cite{cialdini2007}.

The mathematical models presented capture these mechanisms through influence propagation functions, commitment escalation curves, and social proof amplification factors. Each indicator quantifies specific aspects of social vulnerability while maintaining computational efficiency for real-time monitoring.

\section{Mathematical Formalization}

\subsection{Universal Social Influence Detection Framework}

Each social influence vulnerability indicator employs the unified detection function with social network weighting:

\begin{equation}
D_i(t) = w_1 \cdot R_i(t) + w_2 \cdot A_i(t) + w_3 \cdot N_i(t) + w_4 \cdot S_i(t)
\end{equation}

where $D_i(t)$ represents detection score, $R_i(t)$ denotes rule-based detection, $A_i(t)$ represents anomaly score, $N_i(t)$ represents network influence factor, and $S_i(t)$ represents social context modifier.

The social network evolution incorporates influence propagation effects:

\begin{equation}
S_i(t) = \alpha \cdot D_i(t) + \beta \cdot S_i(t-1) + \gamma \cdot \sum_{j \in N(i)} w_{ij} \cdot S_j(t)
\end{equation}

where $\gamma$ captures network influence effects and $w_{ij}$ represents edge weights in the social influence graph.

\subsection{Indicator 3.1: Reciprocity Exploitation}

\textbf{Definition:} Manipulation through creation of artificial obligation relationships leading to security compromise.

\textbf{Mathematical Model:}

The reciprocity imbalance function:
\begin{equation}
R_i(t) = \sum_{j \in N(i)} \frac{Favors_{j \to i}(t) - Favors_{i \to j}(t)}{Favors_{j \to i}(t) + Favors_{i \to j}(t) + \epsilon}
\end{equation}

where $\epsilon$ prevents division by zero and $N(i)$ represents the network neighborhood of individual $i$.

\textbf{Compliance Pressure Model:}
\begin{equation}
P_{comply}(R,T) = \frac{1}{1 + e^{-\beta(R \cdot \alpha + T \cdot \gamma - \theta)}}
\end{equation}

where $R$ is reciprocity imbalance, $T$ is time since favor received, and $\theta$ is the compliance threshold.

\textbf{Detection Function:}
\begin{equation}
D_{3.1}(t) = \begin{cases}
1 & \text{if } R_i(t) > \tau_{recip} \text{ and } Request\_Anomaly > \sigma \\
0 & \text{otherwise}
\end{cases}
\end{equation}

\textbf{Network Reciprocity Analysis:}
Using weighted reciprocity index for organizational networks:
\begin{equation}
WRI = \frac{\sum_{i,j} w_{ij} \cdot \frac{min(x_{ij}, x_{ji})}{max(x_{ij}, x_{ji})}}{\sum_{i,j} w_{ij}}
\end{equation}

where $x_{ij}$ represents interaction frequency and $w_{ij}$ represents relationship weight.

\subsection{Indicator 3.2: Commitment Escalation Traps}

\textbf{Definition:} Progressive commitment increase through small initial agreements leading to major security compromises.

\textbf{Mathematical Model:}

Commitment escalation function:
\begin{equation}
C(n) = C_0 \cdot \prod_{i=1}^{n} (1 + \alpha_i \cdot consistency\_pressure_i)
\end{equation}

where $C_0$ is initial commitment level and $\alpha_i$ represents escalation factor for step $i$.

\textbf{Foot-in-the-Door Effect Model:}
\begin{equation}
P_{accept}(n) = P_0 \cdot \left(\frac{C(n)}{C_0}\right)^{\beta}
\end{equation}

where $\beta$ captures the commitment-compliance relationship strength.

\textbf{Request Sequence Analysis:}
\begin{equation}
ESI = \frac{\sum_{i=1}^{n-1} \log\left(\frac{Risk_{i+1}}{Risk_i}\right)}{n-1}
\end{equation}

where $ESI$ is the Escalation Sequence Index measuring average risk increase per step.

\textbf{Detection Threshold:}
\begin{equation}
R_{3.2}(t) = \begin{cases}
1 & \text{if } ESI > 0.2 \text{ and } Sequence\_Length > 3 \\
0 & \text{otherwise}
\end{cases}
\end{equation}

\subsection{Indicator 3.3: Social Proof Manipulation}

\textbf{Definition:} Exploitation of tendency to follow perceived group behavior in security decisions.

\textbf{Mathematical Model:}

Social proof influence function:
\begin{equation}
SP(p,s) = \frac{p^{\alpha}}{p^{\alpha} + (1-p)^{\alpha}} \cdot s^{\beta}
\end{equation}

where $p$ is proportion claiming compliance, $s$ is perceived similarity to reference group, and $\alpha, \beta$ are scaling parameters.

\textbf{False Consensus Detection:}
\begin{equation}
FC = \frac{Claimed\_Compliance - Actual\_Compliance}{Actual\_Compliance + \epsilon}
\end{equation}

\textbf{Bandwagon Effect Model:}
\begin{equation}
P_{join}(t) = \frac{N_{participants}(t)}{N_{total}} \cdot \frac{1}{1 + e^{-\gamma(t - t_0)}}
\end{equation}

where $\gamma$ controls adoption rate and $t_0$ represents the tipping point.

\textbf{Detection Algorithm:}
\begin{equation}
D_{3.3}(t) = SP(p,s) \cdot FC \cdot \mathbb{I}[Claims\_Verification\_Failed]
\end{equation}

\subsection{Indicator 3.4: Liking-Based Trust Override}

\textbf{Definition:} Security compromise through exploiting preference for agreeable or similar sources.

\textbf{Mathematical Model:}

Liking-based trust function:
\begin{equation}
T_{liking}(similarity, agreeability) = w_1 \cdot S + w_2 \cdot A + w_3 \cdot S \cdot A
\end{equation}

where $S$ represents similarity score, $A$ represents agreeability score, and $w_3$ captures interaction effects.

\textbf{Similarity Exploitation Index:}
\begin{equation}
SEI = \frac{\sum_{traits} |User_{trait} - Attacker_{claimed\_trait}|}{\sum_{traits} User_{trait}}
\end{equation}

\textbf{Trust Override Probability:}
\begin{equation}
P_{override}(T,V) = \sigma\left(\alpha \cdot T_{liking} - \beta \cdot Verification\_Strength + \gamma\right)
\end{equation}

where $V$ represents verification strength and $\sigma$ is the sigmoid function.

\textbf{Detection Framework:}
\begin{equation}
R_{3.4}(t) = \begin{cases}
1 & \text{if } SEI < 0.3 \text{ and } P_{override} > 0.7 \\
0 & \text{otherwise}
\end{cases}
\end{equation}

\subsection{Indicator 3.5: Scarcity-Driven Decisions}

\textbf{Definition:} Security compromise through artificial urgency and limited availability claims.

\textbf{Mathematical Model:}

Scarcity valuation function:
\begin{equation}
V_{perceived} = V_{base} \cdot \left(\frac{1}{1 + Availability}\right)^{\alpha} \cdot (1 + \beta \cdot Urgency)
\end{equation}

where $V_{base}$ is baseline value, and $\alpha, \beta$ control scarcity and urgency sensitivity.

\textbf{Loss Aversion Amplification:}
\begin{equation}
LA = \lambda \cdot Loss_{potential} - Gain_{potential}
\end{equation}

where $\lambda > 1$ represents loss aversion coefficient (typically 2.25).

\textbf{Decision Quality Degradation:}
\begin{equation}
DQ(t) = DQ_0 \cdot e^{-\gamma \cdot Urgency\_Level(t)} \cdot \left(\frac{Time_{available}}{Time_{needed}}\right)^{\delta}
\end{equation}

\textbf{Scarcity Manipulation Detection:}
\begin{equation}
D_{3.5}(t) = \max\left(0, \frac{V_{perceived} - V_{rational}}{V_{rational}} - \tau_{scarcity}\right)
\end{equation}

\subsection{Indicator 3.6: Unity Principle Exploitation}

\textbf{Definition:} Manipulation through appeals to shared identity, common purpose, or group membership.

\textbf{Mathematical Model:}

Unity influence strength:
\begin{equation}
U(identity, purpose) = w_1 \cdot I_{shared} + w_2 \cdot P_{common} + w_3 \cdot \sqrt{I_{shared} \cdot P_{common}}
\end{equation}

where $I_{shared}$ quantifies shared identity strength and $P_{common}$ measures common purpose alignment.

\textbf{In-Group Bias Quantification:}
\begin{equation}
IGB = \frac{Trust_{ingroup} - Trust_{outgroup}}{Trust_{baseline}}
\end{equation}

\textbf{Group Identity Amplification:}
\begin{equation}
GIA(t) = \sum_{markers} w_{marker} \cdot Presence_{marker}(t) \cdot Authenticity_{marker}
\end{equation}

where markers include language, symbols, shared experiences, and cultural references.

\textbf{Detection Function:}
\begin{equation}
R_{3.6}(t) = \begin{cases}
1 & \text{if } U > \tau_{unity} \text{ and } IGB > 0.5 \\
0 & \text{otherwise}
\end{cases}
\end{equation}

\subsection{Indicator 3.7: Peer Pressure Compliance}

\textbf{Definition:} Security compromise through explicit or implicit peer group pressure for conformity.

\textbf{Mathematical Model:}

Peer pressure intensity function:
\begin{equation}
PP(n,d,c) = \frac{n^{\alpha}}{1 + e^{-\beta(c - c_0)}} \cdot \frac{1}{1 + \gamma \cdot d}
\end{equation}

where $n$ is number of peers, $d$ is social distance, $c$ is consensus level, and $c_0$ is consensus threshold.

\textbf{Conformity Probability Model:}
\begin{equation}
P_{conform}(PP, personality) = \frac{PP^{\delta}}{PP^{\delta} + (Resistance_{personal})^{\delta}}
\end{equation}

\textbf{Social Distance Calculation:}
\begin{equation}
SD_{ij} = \sqrt{\sum_{k} w_k \cdot (attribute_{i,k} - attribute_{j,k})^2}
\end{equation}

\textbf{Detection Algorithm:}
\begin{equation}
D_{3.7}(t) = PP(t) \cdot P_{conform}(t) \cdot \mathbb{I}[Behavioral\_Change\_Detected]
\end{equation}

\subsection{Indicator 3.8: Conformity to Insecure Norms}

\textbf{Definition:} Adoption of organizationally prevalent but insecure practices through normative social influence.

\textbf{Mathematical Model:}

Normative influence strength:
\begin{equation}
NI = \frac{\sum_{i} w_i \cdot Norm\_Adherence_i}{\sum_{i} w_i} \cdot Visibility_{factor}
\end{equation}

where $w_i$ represents peer influence weight and visibility captures observability of behavior.

\textbf{Norm Establishment Rate:}
\begin{equation}
\frac{dN}{dt} = \alpha \cdot Adoption\_Rate \cdot (1 - N) - \beta \cdot N
\end{equation}

where $N$ represents norm strength, $\alpha$ is adoption coefficient, and $\beta$ is decay rate.

\textbf{Security Risk Normalization:}
\begin{equation}
SRN(t) = \int_0^t Risk\_Acceptance\_Rate(\tau) \cdot e^{-\lambda(t-\tau)} d\tau
\end{equation}

\textbf{Threshold Detection:}
\begin{equation}
R_{3.8}(t) = \begin{cases}
1 & \text{if } NI > 0.7 \text{ and } SRN > \tau_{norm} \\
0 & \text{otherwise}
\end{cases}
\end{equation}

\subsection{Indicator 3.9: Social Identity Threats}

\textbf{Definition:} Security compromise through threats to individual or group social identity and reputation.

\textbf{Mathematical Model:}

Social identity threat intensity:
\begin{equation}
SIT = \sum_{dimensions} w_d \cdot \frac{|Identity_{current,d} - Identity_{threatened,d}|}{Identity_{current,d} + \epsilon}
\end{equation}

where $d$ indexes identity dimensions (professional, cultural, personal).

\textbf{Defensive Response Model:}
\begin{equation}
DR(SIT, resources) = \frac{SIT^{\alpha}}{1 + e^{-\beta(Resources - R_0)}}
\end{equation}

where $Resources$ includes social capital, reputation, and defensive capabilities.

\textbf{Identity Protection Prioritization:}
\begin{equation}
IPP = \frac{Identity\_Protection\_Effort}{Total\_Available\_Effort}
\end{equation}

\textbf{Vulnerability Assessment:}
\begin{equation}
D_{3.9}(t) = SIT \cdot (1 - DR) \cdot IPP
\end{equation}

\subsection{Indicator 3.10: Reputation Management Conflicts}

\textbf{Definition:} Security compromise arising from conflicts between security requirements and reputation preservation.

\textbf{Mathematical Model:}

Reputation-security trade-off function:
\begin{equation}
RST = \frac{Reputation_{Risk}}{Security_{Risk} + Reputation_{Risk}}
\end{equation}

\textbf{Reputation Capital Model:}
\begin{equation}
RC(t) = RC_0 \cdot e^{-\alpha \cdot Negative_{Events}(t)} + \beta \cdot \int_0^t Positive_{Actions}(\tau) \cdot e^{-\gamma(t-\tau)} d\tau
\end{equation}

\textbf{Conflict Resolution Bias:}
\begin{equation}
CRB = \frac{Decisions_{reputation\_favoring} - Decisions_{security\_favoring}}{Total_{Conflict\_Decisions}}
\end{equation}

\textbf{Detection Function:}
\begin{equation}
R_{3.10}(t) = \begin{cases}
1 & \text{if } RST > 0.6 \text{ and } CRB > 0.3 \\
0 & \text{otherwise}
\end{cases}
\end{equation}

\section{Interdependency Matrix}

The social influence vulnerability indicators exhibit complex interdependencies captured through correlation matrix $\mathbf{R}_{3}$:

\begin{equation}
\mathbf{R}_3 = \begin{pmatrix}
1.00 & 0.65 & 0.45 & 0.70 & 0.40 & 0.55 & 0.60 & 0.50 & 0.35 & 0.45 \\
0.65 & 1.00 & 0.50 & 0.60 & 0.45 & 0.40 & 0.55 & 0.45 & 0.30 & 0.35 \\
0.45 & 0.50 & 1.00 & 0.55 & 0.60 & 0.75 & 0.80 & 0.85 & 0.40 & 0.50 \\
0.70 & 0.60 & 0.55 & 1.00 & 0.35 & 0.65 & 0.50 & 0.45 & 0.55 & 0.60 \\
0.40 & 0.45 & 0.60 & 0.35 & 1.00 & 0.30 & 0.35 & 0.40 & 0.25 & 0.30 \\
0.55 & 0.40 & 0.75 & 0.65 & 0.30 & 1.00 & 0.70 & 0.65 & 0.50 & 0.55 \\
0.60 & 0.55 & 0.80 & 0.50 & 0.35 & 0.70 & 1.00 & 0.85 & 0.45 & 0.50 \\
0.50 & 0.45 & 0.85 & 0.45 & 0.40 & 0.65 & 0.85 & 1.00 & 0.40 & 0.45 \\
0.35 & 0.30 & 0.40 & 0.55 & 0.25 & 0.50 & 0.45 & 0.40 & 1.00 & 0.75 \\
0.45 & 0.35 & 0.50 & 0.60 & 0.30 & 0.55 & 0.50 & 0.45 & 0.75 & 1.00
\end{pmatrix}
\end{equation}

Key interdependencies include:
\begin{itemize}
\item Strong correlation (0.85) between Social Proof (3.3) and Conformity to Norms (3.8)
\item High correlation (0.85) between Peer Pressure (3.7) and Conformity to Norms (3.8)
\item Significant correlation (0.80) between Social Proof (3.3) and Peer Pressure (3.7)
\item Strong correlation (0.75) between Social Identity Threats (3.9) and Reputation Conflicts (3.10)
\end{itemize}

\textbf{Cross-Category Dependencies:}
Critical relationships with Authority vulnerabilities (Category 1) and Temporal vulnerabilities (Category 2):
\begin{itemize}
\item $R_{1.1,3.3} = 0.75$: Authority compliance amplified by social proof
\item $R_{1.9,3.3} = 0.85$: Authority-based social proof directly correlates with general social proof
\item $R_{2.1,3.5} = 0.70$: Urgency bypass correlates with scarcity-driven decisions
\item $R_{1.6,3.9} = 0.65$: Authority gradient effects correlate with social identity threats
\end{itemize}

\section{Implementation Algorithms}

\begin{algorithm}
\caption{Social Influence Vulnerability Assessment}
\begin{algorithmic}[1]
\STATE Initialize social network parameters $\boldsymbol{\alpha}, \boldsymbol{\beta}, \boldsymbol{\gamma}$
\STATE Load organizational social graph and communication patterns
\FOR{each time step $t$}
    \STATE Extract social context: network\_state(t), communication\_flows(t)
    \STATE Calculate social influence propagation coefficients
    \FOR{each indicator $i \in \{3.1, 3.2, \ldots, 3.10\}$}
        \STATE Compute social pressure metrics $SP_i(t)$
        \STATE Calculate rule-based detection $R_i(t)$
        \STATE Compute network-weighted anomaly score $A_i(t)$
        \STATE Evaluate social context modifier $S_i(t)$
        \STATE Calculate detection score $D_i(t)$
        \STATE Apply social influence propagation model
        \STATE Update network influence state $N_i(t)$
    \ENDFOR
    \STATE Compute interdependency corrections using $\mathbf{R}_3$
    \STATE Apply cross-category correlations with Categories 1 and 2
    \STATE Generate influence-aware alerts with propagation predictions
    \STATE Update social network evolution models
    \STATE Log results for influence pattern refinement
\ENDFOR
\end{algorithmic}
\end{algorithm}

\begin{algorithm}
\caption{Social Network Vulnerability Mapping}
\begin{algorithmic}[1]
\STATE Input: Social network $G(V,E)$, influence patterns $P$
\STATE Initialize vulnerability node scores $V_{vuln}[|V|]$
\FOR{each node $v \in V$}
    \STATE Calculate centrality measures: degree, betweenness, closeness
    \STATE Compute social capital score $SC(v)$
    \STATE Assess influence susceptibility $IS(v)$
    \STATE Evaluate network position vulnerability $NPV(v)$
    \STATE Calculate composite vulnerability $V_{vuln}[v] = f(SC, IS, NPV)$
\ENDFOR
\FOR{each edge $(u,v) \in E$}
    \STATE Calculate influence flow capacity $IFC(u,v)$
    \STATE Assess manipulation resistance $MR(u,v)$
    \STATE Compute edge vulnerability $E_{vuln}(u,v)$
\ENDFOR
\STATE Identify critical influence paths using shortest-path algorithms
\STATE Calculate network-wide vulnerability metrics
\STATE Generate targeted mitigation recommendations
\STATE Return vulnerability map with confidence scores
\end{algorithmic}
\end{algorithm}

\section{Validation Framework}

Social influence vulnerability validation requires specialized metrics accounting for network effects and social dynamics:

\textbf{Network-Aware Classification Metrics:}
\begin{align}
Precision_{network} &= \frac{\sum_{clusters} |TP_{cluster}| \cdot w_{cluster}}{\sum_{clusters} |TP_{cluster} + FP_{cluster}| \cdot w_{cluster}} \\
Recall_{network} &= \frac{\sum_{clusters} |TP_{cluster}| \cdot w_{cluster}}{\sum_{clusters} |TP_{cluster} + FN_{cluster}| \cdot w_{cluster}}
\end{align}

where $w_{cluster}$ provides cluster-based weighting for network structure.

\textbf{Influence Propagation Validation:}
Prediction accuracy for influence spread:
\begin{equation}
IPA = 1 - \frac{|Predicted\_Influence\_Set \triangle Actual\_Influence\_Set|}{|Actual\_Influence\_Set|}
\end{equation}

\textbf{Social Proof Accuracy:}
Measuring false consensus detection:
\begin{equation}
SPA = \frac{TP_{false\_consensus}}{TP_{false\_consensus} + FN_{false\_consensus}}
\end{equation}

\textbf{Commitment Escalation Prediction:}
Mean Absolute Error for escalation sequences:
\begin{equation}
MAE_{escalation} = \frac{1}{n} \sum_{i=1}^{n} |Risk_{predicted,i} - Risk_{actual,i}|
\end{equation}

\textbf{Cross-Validation with Social Stratification:}
Ensuring training and test sets preserve social structure:
\begin{equation}
Social\_Preservation = \frac{Network\_Properties_{test}}{Network\_Properties_{total}}
\end{equation}

Target preservation score approaches 1.0 for optimal social structure maintenance.

\textbf{Influence Resistance Validation:}
Measuring intervention effectiveness:
\begin{equation}
Resistance_{gain} = \frac{Vulnerability_{baseline} - Vulnerability_{post\_intervention}}{Vulnerability_{baseline}}
\end{equation}

\section{Conclusion}

This mathematical formalization of social influence vulnerabilities provides a rigorous foundation for understanding and detecting manipulation-based security weaknesses. The integration of network analysis, reciprocity modeling, and commitment escalation theory creates a comprehensive framework for social vulnerability assessment.

The interdependency matrix reveals strong correlations between social influence indicators and significant cross-category effects with authority and temporal vulnerabilities. This demonstrates that social influence acts as a force multiplier for other psychological vulnerabilities, emphasizing the critical importance of social-aware security strategies.

The implementation algorithms enable real-time social influence monitoring with predictive capabilities for influence propagation. Organizations can anticipate vulnerability cascades based on network topology, communication patterns, and social pressure indicators, enabling proactive rather than reactive security measures.

Future work will extend this social modeling approach to the remaining CPF categories, with particular attention to social amplification effects on cognitive overload (Category 5) and group dynamics (Category 6). The mathematical rigor established here provides a foundation for evidence-based social security strategies that account for the predictable patterns of human social cognition.

The social influence category demonstrates that security is fundamentally a social phenomenon, not merely a technical one. By formalizing these social dynamics mathematically, we enable security systems that understand and account for human social realities rather than expecting humans to operate outside their evolved psychological frameworks.

\begin{thebibliography}{9}

\bibitem{canale2024cpf}
Canale, G. (2024). The Cybersecurity Psychology Framework: A Pre-Cognitive Vulnerability Assessment Model Integrating Psychoanalytic and Cognitive Sciences. \textit{Preprint}.

\bibitem{cialdini2007}
Cialdini, R. B. (2007). \textit{Influence: The Psychology of Persuasion}. New York: Harper Business.

\bibitem{buss1999}
Buss, D. M. (1999). \textit{Evolutionary Psychology: The New Science of the Mind}. Boston: Allyn \& Bacon.

\bibitem{watts1998}
Watts, D. J., \& Strogatz, S. H. (1998). Collective dynamics of 'small-world' networks. \textit{Nature}, 393(6684), 440-442.

\bibitem{kelman1958}
Kelman, H. C. (1958). Compliance, identification, and internalization: Three processes of attitude change. \textit{Journal of Conflict Resolution}, 2(1), 51-60.

\end{thebibliography}

\end{document}